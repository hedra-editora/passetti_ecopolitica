%\textbf{e c o p o l í t i c a}%

%\textbf{s u m ár i o}%

%\begin{itemize}
%\item
%  \textbf{proveniências e emergência da ecopolítica}
%\item
%  \textbf{o mundo mudou?}
%\item
%  \textbf{vida outra}
%\end{itemize}%

%``Eu vi a radiação, ela é azul, azul e reverbera.''%

%\begin{itemize}
%\item
%  \textbf{considerações}
%\item
%  \textbf{percursos}
%\item
%  \textbf{o meio e o ambiente}
%\item
%  \textbf{o dispositivo meio ambiente}
%\item
%  \textbf{segurança planetária}
%\end{itemize}%

%planetárias%

%\begin{itemize}
%\item
%  \textbf{o dispositivo diplomático-policial}
%\end{itemize}%

%pacificações%

%ambientes de segurança%

%\begin{itemize}
%\item
%  \textbf{o dispositivo monitoramento}
%\end{itemize}%

%\begin{quote}
%mutações%

%práticas de monitoramento
%\end{quote}%

%\begin{itemize}
%\item
%  \textbf{penalização a céu aberto}
%\end{itemize}%

%\begin{quote}
%cidadão-polícia%

%democracia e polícia%

%nova política e \emph{antipolítica}
%\end{quote}%

%\begin{itemize}
%\item
%  \textbf{direitos e o dispositivo resiliência}
%\item
%  \textbf{ecopolítica: poder em fluxo e política}
%\item
%  \textbf{ecopolítica, planeta e sociedade}
%\end{itemize}%

%mundo%

%diminutas%

%\begin{itemize}
%\item
%  \textbf{bibliografia}
%\item
%  \textbf{agradecimentos}
%\end{itemize}



\chapter{Proveniências e a emergência da ecopolítica}

Um dia no Nu"-Sol (Núcleo e Sociabilidade Libertária, do Programa de
Estudos Pós"-Graduados em Ciências Sociais da \versal{PUC}"-\versal{SP}), olhamo"-nos firmes.
Naquele girar de globos oculares e retinas, piscando suavemente, estava
marcado o começo da elaboração de um projeto que respondesse aos nossos
incômodos libertários. Era preciso produzir um projeto de pesquisa que
falasse do nosso tempo, dos espaços modificados, da intensidade
libertária presente desde 1968, redimensionada em 1999 com o movimento
antiglobalização e que ganharia outras dimensões inéditas. Era 2008.

A ampliação de práticas democráticas sincronizadas com a racionalidade
neoliberal também nos apontava para algo que mudara na biopolítica. As
leituras da obra de Michel Foucault, e mais do que elas, as sugestões do
filósofo"-historiador francês para se perseguir outros percursos,
atiçavam para enfrentar a nova situação planetária na qual a Terra, seus
viventes e o espaço sideral passavam a ser o alvo da gestão
governamental.

Uma nova governamentalidade se disseminava. Estados nacionais se
redimensionavam em União de Estados, os mercados de comércio criados no
pós"-\versal{II} Guerra Mundial se ampliavam, a democracia liberal levava de
roldão o socialismo autoritário ou o submetia, as práticas democráticas
e de participação entraram nas relações econômicas, sociais, culturais,
familiares e pessoais. A Internet apareceu, e a programação eletrônica
por interfaces diplomáticas redimensionou e inovou. E novas
institucionalizações começaram a ocorrer, tendo nas Nações Unidas um
ponto de inflexão e difusão que produziu, em 2000, os Objetivos de
Desenvolvimento do Milênio (\versal{ODM}), atingindo resultados cada vez mais
satisfatórios aos seus propósitos e voltados a encontrar \emph{melhores}
condições de vida para os chamados países subdesenvolvidos ou em
desenvolvimento. O grande encontro conhecido como \emph{Rio 92}
ultrapassava os limites das intenções e da retórica.

Estas constatações nos levaram anos antes a iniciar alguns breves
estudos sobre o que seria a passagem ou a transformação da biopolítica
para a ecopolítica. O governo da espécie (a população) agora não era
mais algo preferencial de políticas de Estado e muito menos do governo
do planeta. Avizinhava"-se a busca das proveniências de uma
governamentalidade compartilhada, sustentável e produtora de uma
governança \emph{global}. Todos estão convocados a participar de
\emph{melhorias} no planeta para o futuro das novas gerações. De certa
maneira, essa governamentalidade depende dos princípios de economia
política revisados pelos neoliberais e que equacionaram a mutação da
força de trabalho em capital humano. Qual maneira? Buscá-la em suas
variedades foi o modo como nos interessamos pelo que há de mais caro à
anarquia e aos anarquismos, a produção de resistências. Neste campo
especial, compreender os redimensionamentos de territórios, gestões de
governos do Estado e das pessoas com suas condutas e contracondutas,
assim como a inquietação e os alertas relativos à energia nuclear de uso
pacífico e violento na segunda metade do século passado, até os atuais,
relativos ao clima. As lutas ecológicas alteraram as relações com o meio
ambiente.

Como situar essas relações de poder que ultrapassavam os consagrados
territórios soberanos, pelas quais a \versal{ONU} absorvia os fluxos de
discussões abertas que ocorriam em \versal{ONG}s, Fundações, Institutos e
Organizações empresarias, que produziam fóruns e conferências,
dinamizando práticas democráticas e consagrando a racionalidade
neoliberal, por meio de regulamentações internacionais? Se a guerra era
a política prolongada por outros meios (Clausewitz), ou se a política
era a guerra prolongada por outros meios (Foucault) --- assim as
conhecemos com o Estado"-nação e as mais diferentes implicações
provocadas por cada uma delas ---, também nos interessava situar como na
produção absorvente de fluxos democráticos se produz a política atual. E
discutir a política contemporânea exige revirar os dispositivos de
segurança e, por conseguinte, a liberdade liberal fundada em sua própria
segurança, as situações de guerra e a de polícia interna.

A entrada decisiva dos componentes computo"-informacionais não só acionou
um regime de comunicação contínua, como tornou mais evidente que as
mudanças que ocorriam levavam a institucionalizações inacabadas, que se
repartiam em outros fluxos, absorviam contracondutas, minimizavam
resistências. Para essa sociedade de controles, como suscintamente
descrevera o filósofo Gilles Deleuze, diferentemente da disciplinar, a
vigilância não era mais suficiente. Essa propagação de controles gerou
monitoramentos não só eletrônicos, mas de condutas por cada indivíduo.
Mas seriam ainda indivíduos, como supõe o liberalismo, ou mesmo a
tradição marxista? Algo nos dizia que Nietzsche tinha acertado em situar
que em nossa sociedade o que há são divíduos. Como são, como se
relacionam, quais práticas produzem e como se ajustam à conduta
democrática de capital humano inovador? E, diante disso, o que
poderíamos esperar das resistências, posto que desde o acontecimento
\emph{1999} foi possível compreender as capturas de lideranças e agentes
de contestação no fluxo de participação democrática? Quais resistências
se prolongariam, o que a \emph{multidão} redireciona, para onde levam as
novas práticas, reconhecidamente derivadas da história política das
práticas anarquistas? E o que mais surpreendia, e permanece
surpreendendo, é como as condutas penalizadoras cresceram. Elas não
exigem mais as instituições austeras de outrora como as prisões, os
manicômios, os internatos, apesar de delas não se apartarem, porém,
encontraram novos fluxos punitivos penalizadores a céu aberto. Monitorar
a tudo e a todos passaria a ser a segurança de cada um para um futuro
melhor?

No meio disso tudo, jamais se tinha vivido a proliferação de direitos
como depois de \emph{1968}. É interessante notar como aquele
acontecimento que produziu a emergência das minorias no plano político,
muitas vezes liberadas de partidos políticos e sindicatos, as
instituições até então capazes de absorver movimentos sociais, também
lhes deu visibilidades sociais, culturais e pessoais (consolidando,
assim, a dimensão das práticas dividuais). Os direitos de minorias, para
nós, durante a elaboração do projeto de pesquisa, para além da
Declaração Universal dos Direitos Humanos de 1948 como um marco de
liberdade liberal, serviam, paradoxalmente, para criticar os regimes
autoritários nos anos 1980, anteriormente consolidados com o imperativo
capitalista, o socialismo autoritário, e proporcionar uma revisão
urgente nas práticas intervencionistas. Para os libertários sempre é
importante discernir entre minoria numérica, voltada para compor com
governos majoritários, e minorias potentes capazes de inventar
liberdades. A pletora de direitos de minorias ajustou a participação de
cada divíduo e deixou claro que nessa sociedade a convocação à
participação se tornou referência para as condutas econômicas, sociais,
culturais e pessoais. Ela traria consigo uma nova figura, a do portador
de direitos?

A programática sustentável que se espraiava pelo planeta e que
mapeáramos durante a primeira década dos anos 2000 exigia detalhamentos,
como situar as forças em lutas e a produção de documentos desse arquivo
planetário \versal{ONU}. E assim caminhamos por esse arquivo imenso e inacabado
no qual sobressaiam as Metas do Milênio para o período 2000-2015, e
perscrutávamos o que viria após sua finalização inacabada, cuja resposta
apareceu, em 2014, com a definição dos Objetivos de Desenvolvimento
Sustentável (\versal{ODS}) quando, pela primeira vez, a \versal{ONU} deixou de demarcar em
seus investimentos a distinção entre desenvolvidos e subdesenvolvidos/em
desenvolvimento. A sustentabilidade passou a ser uma meta para todos os
Estados, povos, etnias, minorias numéricas e divíduos. Não foi de
estranhar o novo acontecimento planetário \emph{2011}, já durante o
primeiro ano da pesquisa, e muito menos os nacionais, em \emph{junho de
2013}. Na Grécia, em 2008, as práticas libertárias os anunciavam.

Ecopolítica é um livro resultante destas inquietações que encontram em
outros pesquisadores e libertários modos distintos de enfrentar a
racionalidade neoliberal, muitas vezes como ideologia. Não se trata de
vê-la como disciplina de conhecimento específica, nem como a utopia após
o antropoceno. Nossa contribuição aos resistentes é a do convite a
cartografias que inventem liberdades, portanto uma \emph{ação direta}.
Diante da insistente naturalização das desigualdades, hoje em dia
pacificadas e unificadas em torno do pluralismo democrático, seguindo o
poeta popular, ``é sempre bom lembrar que um copo vazio está cheio de
ar'', mas que o ar está antes e depois do copo, que é combustão para o
fogo que arde, evapora e cai como chuva, às vezes ácida, que produz
transbordamentos de rios nessa Terra que sempre treme e tremerá,
trazendo terremotos e maremotos, porque seu núcleo vivo e pulsante é
composto do magma, assim como nós somos uma certa ebulição sanguínea e
nervosa.

A genealogia do poder vai ali onde parece que tudo começa com a
grandiloquência dos atos inaugurais da cultura. Tudo começa na criança
e, por isso, em nossa cultura, ela é o objeto primordial de
investimentos em condutas que pretendem forjar identidades, crenças, fé,
a situação do \emph{mundo}, e educá-la a se adequar a limites por meio
de castigos e recompensas. Hoje, nem tanto ou de outra forma, pois dela
também se espera que na família, na escola e na comunidade participe,
seja portadora de direitos inacabados e acredite no futuro
\emph{melhor}. Um jeito democrático, inclusivo, responsável de torná-la
um eficiente capital humano. Se, no passado os lados do triângulo
equilátero herdado da Revolução Francesa (igualdade, liberdade e
fraternidade) orientaram governos e as empolgantes e decepcionantes
revoluções, hoje um quadrado perfeito se forma com o acréscimo do amor.
É preciso amar a todos, ao planeta, aos seus deuses, de preferência
ecumenicamente, aos superiores e aos inferiores, em função de uma
cultura de paz que se consolide, com mais qualidade de vida e
redutores"-protetores de vulnerabilidades. Tudo como quer a produção
capitalista sustentável (e, porque não dizer, as alternativas) e a
liberdade liberal. O ideal de revolução herdado da França do século
\versal{XVIII} cedeu lugar ao ideal democrático estadunidense do mesmo século. O
rodízio capitalista estará preservado. Até quando?

\chapter{O mundo mudou?}

O mundo não é mais o mesmo. Os filhos da geração da \versal{II} Guerra Mundial
ouviram de seus pais a constante surpresa com as mudanças no mundo:
poderíamos desaparecer como espécie depois das bombas sobre Hiroshima e
Nagasaki. Como escapar da morte não mais pela grande guerra, nem por uma
vontade política de governos e cidadãos contra outros povos, como
pretendeu o nazismo?

A guerra e o seu final anunciavam relações mais livres com o sexo, era
cada vez mais visível uma disponibilidade maior do mercado em absorver
as mulheres, e já se falava, entre paredes, da presença homossexual no
interior das famílias.

O mundo mudara mesmo, o socialismo poderia se fazer global e os
democratas se mostravam apreensivos. Vivíamos na América do Sul uma nova
era que imprensa e analistas caracterizavam como populismo, uma certa
conexão entre interesses burgueses e operários em função do
desenvolvimento.

Havia uma obsessão pelos meios para sairmos do subdesenvolvimento. A
crença na escola para todos estava estampada nas casas para que a vida
dos filhos se tornasse menos dura, e fosse possível uma ascensão social
mais elástica pelo acesso dos filhos dos operários ao trabalho
intelectual.

O mundo mudava rápido com a televisão e com a busca de outra vida no
espaço sideral. Ao sair desse mundo terreno para o espaço celestial,
estaríamos infringindo o governo de Deus que olha para todos do alto e
nos vê em qualquer canto? Se o ``homem'' fosse para o espaço ele nos
veria também lá do alto com equipamentos que poucos desconfiavam
existir.

Um dia, em 1961, o astronauta soviético Yuri Gagarin viu a Terra lá do
alto e simplesmente relatou: ``a Terra é azul''. Ele confirmou todas as
evidências da física relativas à incidência dos raios solares sobre a
superfície do planeta. Mostrou que o tom azul celestial tão comum nas
imagens dos santos era mais que a vontade de cor de Deus, que o céu
poderia ser para \emph{todos} e que a escuridão da noite era mais que um
tapete de estrelas ornado pelo luar. Olhar para o firmamento era um
pouco mais do que o caminho para o paraíso. Em 6 de maio de 1970, Ernst
Stuhlinger, diretor de ciência do Centro de Voo Espacial Marshall da
\versal{NASA} respondeu à carta da irmã Mary Jucunda, freira missionária em
Zâmbia, antiga Rodésia do Norte, colônia inglesa, que questionava o
gasto de dinheiro com o projeto de missão tripulada a Marte com tantas
crianças morrendo de fome na África. Stuhlinger enviou com sua resposta
a foto conhecida como ``Nascimento da Terra''. Esta carta ficou tão
famosa que foi publicada pela \versal{NASA} como ``Por que explorar o espaço?''.
Ao endossar os argumentos da missionária, o diretor de ciência sublinha
a importância da ocupação do espaço sideral. Mais do que isso, mostra
uma das proveniências do redimensionamento prioritário do interesse no
planeta: ``há grandes extensões de terras que poderiam ter um
aproveitamento muito melhor com métodos eficazes de controle de
drenagem, uso de fertilizantes, previsão do tempo, estimativas de
fertilidade, plantio programado, seleção do solo, hábitos de plantio,
época de cultivo, inspeção de plantação e planejamento de colheita. A
melhor ferramenta para aprimorar todas essas atividades é, sem dúvida, o
satélite artificial da Terra. (...) Maior produção de alimentos graças a
dados colhidos por satélites e melhor distribuição de alimentos graças a
boas relações internacionais são apenas dois exemplos do profundo
impacto que o programa espacial terá sobre a Terra. Eu gostaria de
mencionar mais dois exemplos: o estímulo ao desenvolvimento tecnológico
e a geração de conhecimento científico. (...) Precisamos de mais
conhecimentos em física e química, em biologia e fisiologia e,
principalmente em medicina para lidar com todos os problemas que ameaçam
a vida humana: fome, doença, contaminação de alimento e água, poluição
do meio ambiente. (...) {[}A Terra{]} será melhor, não porque usaremos
todo o nosso conhecimento tecnológico e científico para melhorar nossas
condições de vida, mas também porque estamos chegando a um apreço muito
mais profundo por nossa Terra, pela vida e pelo homem'' (Usher, 2014:
206-209).

Para um menino e uma menina aqui embaixo olhando para o céu havia apenas
a imagem refletida da nossa bola, a mesma bola de futebol que traduzia
em gols as vitórias de uns artistas subdesenvolvidos. O Brasil era o
campeão mundial na Europa e no ano seguinte serial o bicampeão mundial.

O mundo era a Terra e nele qualquer um poderia ser campeão ao ascender
socialmente no capitalismo ou encontrar a igualdade no socialismo.
Parecia até que tudo se encontrava em sínteses dicotômicas
ultrapassáveis: capitalismo"-socialismo, democracia"-socialismo,
Europa"-Américas, subdesenvolvido"-desenvolvido, céu e Terra. Não, não era
tão simples, mas parecia simples. Bastava ter vontade. Ingenuidade de
crianças e jovens que cresciam num novo mundo de consumo e ideias, de
possíveis liberdades e descrença na guerra, ainda que houvesse a guerra
da Coreia e mais tarde a do Vietnã. Mas quando esta ocorreu o mundo
estava em disputa acirrada: não bastava sonhar com ascensão social, e
seria verdadeiro tudo que nossos avós malucos diziam sobre o socialismo
na \versal{URSS} e na China? O anarquismo era mesmo uma coisa do passado quando
os operários viviam sua infância?

Falava"-se de novas revoluções, uma delas bem pertinho de nós, em Cuba.
Era possível. O chamado populismo era real. O homem no espaço era real e
ainda falavam que chegaríamos à lua e que o país que lá chegasse seria o
vitorioso nesta corrida. Tudo uma corrida, uma tremenda competição
científica, artística, futebolística, todos querendo escola, exigindo
mais e mais do Estado.

Falava"-se que com a \versal{ONU} teríamos definitivamente um guardião da paz e
que a Declaração Universal dos Direitos Humanos de 1948 nos preservaria
de políticas de extinção da espécie, das torturas, traria igualdades
fundamentais e nos reconheceria, finalmente, como membros da espécie
humana com todas as nossas diferenças. Mas isso era uma conversa
episódica nos lares ou mesmo na escola.

O Brasil era rico em riquezas naturais e isso era uma vantagem.
Ensinavam que a Floresta Amazônica era o pulmão do mundo, que o rio
Amazonas não era o mais extenso, mas o mais caudaloso. Porém, ele não
era só do Brasil, nascia a oeste, no Peru, para desaguar no oceano
Atlântico. Ele era e não era brasileiro. A geografia que aprendíamos o
represava no Estado"-nação e ensinavam que este era resultante da fusão
harmônica de todos os povos que aqui habitavam.

Nas aulas de história, apesar de toda eloquência investida nas
descobertas marítimas portuguesas e na ação filantrópica dos jesuítas,
alguns de nós desconfiavam das simples constatações que essa terra
brasileira era habitada por selvagens sem rei, lei e alma. A professora
ficou embaraçada quando alguém perguntou como os índios viveram por
tanto tempo sem rei, lei e alma antes da chegada dos descobridores. Eles
não sabiam o que era o progresso que os brancos trouxeram. Assim nos
educavam para compreender que o progresso vinha de fora, dos mais
desenvolvidos. Então alguém sabia mais de nós que nós mesmos? Assim como
os portugueses e espanhóis sabiam mais dos índios que eles mesmos? Que
mundo estranho era aquele!

A escola nos ensinava a geografia e a história, como amar a língua
portuguesa, nossas riquezas naturais e os governantes que eram tão
inteligentes que conseguiam juntar todos os segmentos sociais crentes no
futuro. O Brasil era o país do futuro. O maior do \emph{nosso}
continente, cercado de povos amigos de língua espanhola que também
civilizaram os índios. E todos os demais povos eram índios integrados
pelo Estado"-nação aos imigrantes europeus.

Os estrangeiros sabiam colonizar, por vezes foram violentos escravizando
índios e depois negros para cá trazidos porque falavam que o índio era
indolente. Ouvíamos até que houvera os antropófagos, que havia pigmeus
na África, que o ``homem'' se deslocara pelas terras desse mundo há
muitos milhões de anos, que cientificamente veio do macaco e que era
filho de Deus. Tudo ficava um pouco confuso. Nos primórdios éramos
filhos de Deus à sua imagem, depois descendíamos evolutivamente do
macaco, uns eram superiores outros selvagens e até havia os bárbaros
como os nórdicos que destruíram o Império Romano, uma cultura decalcada
da Grécia, e que houve a civilização egípcia, muito imponente em
arquitetura e agricultura. Havia uma evolução linear e outra não tão
linear, mas sempre vencia o mais forte e a guerra, mesmo sendo violenta,
trazia o resultado definitivo para que a paz prevalecesse.

O mundo mudava com as guerras. Nossa civilização era a dos vencedores.
Nosso Estado era dos brasileiros, mesmo que a gente pouco entendesse que
o romano era romano em muitos outros territórios. Coisa do passado.
Éramos americanos e nem tanto, porque americano era o habitante dos
Estados Unidos. Tudo o que era moderno, da calça ao eletrodoméstico, e
principalmente o automóvel e o avião, era americano. A música e os
filmes modernos também eram criações da cultura americana. Os franceses
gostavam de livros, os portugueses de \emph{Os Lusíadas}, que eram
coisas longas, que exigiam muito tempo de cada um e não traziam o livre
divertimento moderno do rádio, da televisão e do cinema. Era para o
gosto de quem tinha tempo, era para intelectuais que apreciavam a
história da cultura, para a elite.

Os índios não tinham essa cultura superior, nem os escravos negros, mas
o europeu a tinha, e se estudássemos muito nós teríamos possibilidade de
encontrar um bom trabalho e ter tempo para ler. Ao menos os que passaram
a ter o gosto pelos livros. Era preciso estudar muito para aos 14 anos
ter possibilidade de encontrar um emprego melhor do que o sujo das
fábricas e passar para o turno da noite na escola. Não era raro ouvir do
colega que ele queria trabalhar no banco ou no comércio. Tinha até umas
meninas que diziam querer trabalhar, que achavam a vida das mães delas
muito chata cuidando da casa, e alguns meninos concordavam um pouco com
elas.

Tínhamos que estudar, nos preparar para o trabalho limpo e engravatado
ou com vestido, para constituir família, ter uma mulher ou marido e
outra mãe em casa. Tudo simples assim. Alguns achavam que essa era a
simplificação do que deveria ser a vida futura, normal e melhor do que a
que levávamos. Quem não seguisse esse itinerário seria um operário ou um
puxador de carroça, doméstica ou esposa. Mas para seguirmos esse trajeto
tão claro, limpo e saudável não podíamos ser indisciplinados na escola.

Deveríamos ser obedientes. Muitos colegas eram surrados em suas casas,
às vezes chegavam com marcas e alguns de nós ficávamos revoltados com os
pais deles. Muitas vezes, eram eles que azaravam a aula e a professora
os colocava de castigo. Eles ficavam mais revoltados e ganhavam também
nossa adesão --- de poucos, é verdade. A professora, a diretora, a
funcionária que vigiava o recreio eram nossos alvos. Alguns partiam para
o enfrentamento físico com elas ou mediam forças e atemorizavam os
colegas mais passivos e conformados. E aí virava outra confusão sobre
como era certo ir contra as autoridades, mas que era errado amedrontar
fisicamente os colegas. Bem, isso não tinha solução.

Naquele teatro de punições e lições sobre o mundo e o Brasil, primeiro
como crianças e depois como jovens, aprendemos a ser normais e, de vez
em quando, indisciplinados. Às vezes, abria"-se uma conversa sobre um
parente doente ou louco; que os hospitais eram estranhos porque tinha
ala para ricos e ala para pobres; que cabia às mães levar seus bebês nos
postos de puericultura, limpar a casa e deixar os filhos limpos de banho
tomado e uniforme lavado e passado para irem à escola. Tinha mãe de
colega que era louca, vizinho que era louco, e às vezes diziam que tinha
crianças com doença mental. Os indisciplinados, principalmente os
surrados em casa, eram vistos como anormais. Tinham de ser muito bem
observados para que não virassem pequenos criminosos. Um pequeno furto
na sala de aula, e não foram poucos por desejo de lanche, borracha, uns
trocados... era uma sessão estranha de pressão pela autoacusação ou
delação do infrator. O resultado ia para o Livro da escola, os pais eram
comunicados e o menino ou a menina era visto a partir de então como um
pequeno ladrão ou pequena ladra. Esperava"-se que com isso as boas maçãs
fossem afastadas das podres.

Aos poucos alguns percebiam que a escola era parecida com o que o pai e
mãe contavam da fábrica ou do escritório. A escola também se assemelhava
à casa. Mas a escola procurava se mostrar como o centro de preparação
para os desvios nos lares e na sociedade, o lugar ideal, do certo, de
como aprender a obedecer seguindo a didática, os zelos com o asseio, os
momentos para comer, brincar e voltar a estudar. E continuar estudando
em casa cumprindo todas as lições a serem feitas (no Rio de Janeiro,
lição de casa é dever, e pelo menos para uns de nós essa designação era
muito mais autoritária, mesmo constatando que não trazer a lição pronta
ou o dever de casa cumprido repercutia nas mesmas sanções). Ensinar as
crianças a obedecer e seguir as regras para a boa formação era com que a
escola contribuía para nos fazer desse mundo. Tirava"-nos do mundinho da
casa e da família, da vizinhança, das ruas, dos maus elementos, dos
perigos e das doenças epidêmicas para preparar cada um e a todos para a
vida do trabalho.

Não éramos tão tolos ou ingênuos, porque tínhamos ouvidos, pensávamos, e
nos perguntávamos se todas as escolas só serviam para preparar para o
trabalho futuro no mundo civilizado. Desconfiávamos que houvesse escolas
para quem manda, mesmo que quase todos quisessem ter uma boa formação
para mandar nos subordinados. A escola que ensina a obedecer também
instrui como se deve mandar nos outros. Eram todas mais ou menos iguais.
Vez ou outra entrava nas conversas o assunto universidade. Era um
assunto muito raro, uma coisa distante, feita para quem gostava de ler,
de estudar, ou melhor, quem gostava de ler e estudar e tinha tempo.
Muitos saíam da escola para um trabalho sem registro; meninas voltavam
para casa para ajudar na preparação do jantar ou cuidar dos irmãos mais
novos. E ainda tinha aquele mundão de lições"-deveres a cumprir.

Gostávamos de cantar as músicas do rádio, alguns tinham vitrolas e
discos de música clássica (que muitos achavam chatas e longas),
detestávamos as aulas de canto orfeônico e alguns, entre nós, se
recusavam, em filas duplas dispostas no pátio antes do início das aulas,
a cantar o hino nacional. Tinha os que se emocionavam e os que dublavam
para burlar os olhos sobre nós das professoras e dos bedéis que
comandavam na dianteira cada conjunto de alunos enfileirados até a sala
de aula. Ali tudo recomeçava no ponto anterior, seguindo a sequência
programada de ensino da língua, da aritmética, das ciências e da
história e geografia.

Todos também aprendíamos a amar os símbolos da pátria e a compreender
que tudo em volta era corriqueiro, normal e que qualquer situação
anormal ou perigosa que acontecia no país e no mundo seria solucionada
pela polícia e a justiça, que o governo tinha de saber tratar dos mais
diferentes e difíceis problemas, e que nós deveríamos contribuir para
que a ordem sempre fosse mantida: ordem e progresso como estava na faixa
da bandeira do Brasil que tinha o verde das florestas (sim, da Amazônica
e das matas), o amarelo do ouro mineiro e do sol, o azul do firmamento e
o branco da paz, tudo integrado na bandeira. E como ela era linda!
Ninguém nos ensinava que as cores signficavam outra coisa: o verde era
cor da Casa Real dos Bragança e o amarelo da Casa dos Habsgurgo e, por
isso, a bandeira da República era quase igual à do Império.

O mundo da escola, fora dela e depois do sistema solar e da via Láctea,
tudo por Deus, pelo progresso, pela ordem. Neste mundo também de
indisciplinas, anormais, perigosos, inconformados, revoltados. Sim,
tinha de ser assim aqui e no socialismo, porque há sempre quem manda e
obedece. Tinha o governo que mandava, a professora que governava, os
bedéis que vigiavam, e todos governavam nossas condutas e nós as dos
outros, uma vigilância preventiva, diziam, eficiente para que todos
vivêssemos em paz, respeitássemos a autoridade superior, colaborássemos
contra a guerra. Afinal, o Brasil era um país pacífico!

E também entre nós havia algumas crianças que ouviam o contrário. Mas
todos tinham que ir para a escola, porque simplesmente éramos uns
privilegiados, porque no Brasil, um país subdesenvolvido, não havia
escola para todos, porque na cidade não havia escola para todos os
filhos de migrantes pobres e na zona rural, quase nada de escola.
Parecia que os filhos dos trabalhadores das cidades tinham o mesmo
privilégio dos meninos e meninas ricas. Estávamos já nos desenvolvendo.
Os filhos dos operários já iam mais regularmente para o ginásio, e
alguns ouviram a notícia de que a Terra é azul, que testes nucleares
eram realizados no Oceano Pacífico assim como ocorrera no deserto de
Sonora no México antes dos americanos jogarem as bombas sobre as cidades
japonesas.

Isso era desenvolvimento, mas também um perigo para todos. Afinal, quais
seriam os efeitos dessa palavra macabra: radioatividade? Mas se tratava
o câncer com radioterapia... Havia uma duplicidade nisso tudo. Falava"-se
muito das reservas de petróleo, das disputas pelo carvão e o aço, das
siderurgias, e do Volkswagem, o fusca que seria humanizado pelos
estúdios Disney, e que fora o automóvel saudado pelo Adolf Hitler como
carro de acesso popular a qualquer alemão. A gente sempre sabia umas
coisas estranhas que não estavam na enciclopédia e muito menos na
revista \emph{Seleções}, no \emph{Almanaque Fontoura} ou na escola.
Sabia"-se mais de criptonita que de radioatividade, de um estilo de vida
americano veiculado pelos gibis e pelas revistas femininas de romances
fotografados. Às vezes sabíamos de sexo por meio de livros proibidos e
os inesquecíveis ``catecismos''. Às vezes, notávamos como nossos pais
ficavam apreensivos com a edição extra do Repórter Esso, na \versal{TV} Tupi,
interrompendo a programação ou mesmo uma propaganda. Anunciava que algo
temeroso acontecia no mundo e, por vezes, no país: greve, renúncia do
presidente, problemas políticos, um incêndio em um edifício,
alagamentos, enfim, o extraordinário que era o ordinário em nossas vidas
poupadas pela escola. Não me recordo de uma interrupção destas em jogo
de futebol, mas o futebol pela \versal{TV} era aos domingos ou numa noite de
semana, e os domingos e a noite diziam que eram horas de descanso em
família, dos pecados, das transgressões.

Mesmo que a terra tremesse no Japão, onde era dia, só saberíamos na
manhã do dia seguinte nos jornais que estampavam política, tragédias
provocadas por eventos da natureza, matanças em família, assaltos
inéditos, notícias escandalosas sobre o terceiro sexo, tudo isso nas
bancas de jornal e revistas como na canção cinematográfica de Caetano
Veloso, que estamparia ironicamente tanta alegria"-alegria. Tinha dia
santo, feriados, natal e dia do trabalho como até hoje. E tinha
revoluções, homem na lua, e os Beatles e os Rolling Stones, a \versal{MPB}, o
Cinema Novo nos festivais de cinema europeus, mas ainda tinha o
Mazzaropi, a \versal{TV}, o João Goulart como vice"-presidente em viagem
diplomática pela China comunista, a sua volta sob o parlamentarismo com
Tancredo Neves, o herdeiro da caneta de Getúlio Vargas, as Ligas
Camponesas, o operariado nas ruas, a presença da Revolução Cubana, a
propagação do socialismo, os hippies, undergrounds, beats, palavras em
português e inglês que misturavam ordem e contra"-ordem. Estávamos
mudando mesmo.

Um golpe de Estado no Brasil anunciou o que seria um corretivo (éramos
indisciplinados a tal ponto?) para que a democracia se solidificasse e a
ameaça comunista fosse banida do país, com a ajuda dos americanos.
Diziam que não, mas nas casas de muitos de nós diziam que sim; uns para
se posicionarem contra, outros agradecendo a Deus. Será que estes eram
assuntos para um novo Tribunal de Nuremberg ou o Tribunal de Tóquio?
Não, estes existiram para julgar crimes da \versal{II} Guerra Mundial, mas se
falava que havia nazistas na Argentina e no Paraguai, até mesmo aqui no
Brasil onde houvera os Integralistas. Aquela tal de \versal{ONU} falava muito de
desenvolvimento, de fome no mundo, mas não dizia nada de ditaduras. Não
se metia onde mandavam americanos e soviéticos. Estava localizada em um
belo conjunto arquitetônico, projetado por Oscar Niemeyer, fincado em
Manhattan, em Nova Iorque, e lá dentro havia muita diplomacia e
recomendações sobre isso e aquilo, crianças e seus direitos, mulheres
idem, idem, idem e ibidem.

Mas os pretos não podiam se sentar em qualquer lugar nos ônibus dos \versal{EUA}
até que uma costureira preta se recusou a ceder o lugar para um branco.
Essa era democracia americana em que todos deviam se espelhar? Para que
serviam os direitos humanos da \versal{ONU} se não houvesse mulheres como aquela
Rosa Louise McCauley, conhecida como Rosa Parks, em 1955? Em 13 de maio
de 1958, Jackie Robinson, um jogador preto de beisebol que em 1947
derrubara a proibição de pretos nos times profissionais e era engajado
na luta contra a segregação racial, encaminhou uma carta ao presidente
Dwight D. Eisenhower: ``Assisti à Reunião de Cúpula dos Líderes Negros
ontem, e ouvi o senhor falar que precisamos ter paciência. Tive vontade
de me levantar e dizer: `Ah, não! De novo! Com todo o respeito,
lembro"-lhe que temos sido o povo mais paciente do mundo. Quando o senhor
falou que devemos ter amor"-próprio, perguntei a mim mesmo como
haveríamos de ter amor"-próprio e continuar sendo pacientes diante do
tratamento que temos recebido ao longo dos anos. Dezessete milhões de
negros não podem esperar que o coração dos homens mude'' (Usher,
2014:174). Muito à boca pequena corriam os boatos sobre a vida dura no
socialismo soviético, bem autoritário, como os anarquistas prenunciaram,
e, gradativamente, levados ao público em 1956 por Nikita Kruschev sobre
o governo de Stálin. Não se devia falar disso, que era ínfimo diante dos
crimes capitalistas.

A década de 1950 fez explodir outra bomba nos anos 1960, quando as
crianças filhas da geração pós \versal{II} Guerra Mundial já eram jovens
insatisfeitos com democracias, consumismos, socialismos, ditaduras,
quando de repente aconteceu \emph{1968}. Os jovens escancaravam àquele
\emph{mundo} que ele tinha acabado! Não estavam ali para mostrar os
rumos para um novo mundo, apenas lançavam bombas nos costumes, nas
ideias, no convencional, no conformismo, na padronização, na idiotice,
no sexo timidamente desgovernado, na crença no Estado, no dia seguinte,
no futuro melhor. A escola já não nos continha.

Ano da revolta! Ano de avanço dos reformadores também, ano dos limites à
ameaça da guerra nuclear, ano da emergência da Teologia da Libertação na
igreja católica, com a opção preferencial pelos pobres, na América
Latina, depois do Concílio Vaticano \versal{II} (iniciado em 1961 e concluído em
1965) e da Conferência de Medelin (1968), ano anterior ao da chegada do
homem à lua e do início da colonização do espaço sideral. 1968, ano
também da Primeira Conferência de Direitos Humanos da \versal{ONU}, em Teerã,
sobre o indispensável para a paz e a justiça. 1968, ano dos jovens nas
ruas, das ocupações de universidades e fábricas, e de ultimato aos
negócios entre Estado, sindicatos e burocracia estatal, dos cartazes
provocativos, instigantes que lembravam os dos anarquistas na Revolução
Espanhola, das artes fora dos museus e do governo de elites, de cinema
convulsivo, teatro despojado, em qualquer lugar, de jovens encontrando
novas formas de viver e lutar nas selvas contra os regimes, de ditaduras
recrudescendo com apoio de americanos e soviéticos, de uma revolução
cultural na China com base em um único livro de seu líder e em processos
macabros de delação e de torturas, de anúncio de um possível socialismo
pela via eleitoral que se consolidou no Chile (Salvador Allende, eleito
por uma diferença mínima pelo voto direto, acabou escolhido
indiretamente pelo Congresso, seguindo"-se a Constituição do Chile).

Empresários ladeados de cientistas e interessados se reuniram na Itália
e fundaram o Clube de Roma para equacionar novas medidas para o
desenvolvimento. Quem lê tanta notícia? Quem faz essas notícias? Esse
mundo acabou. Ou, se preferirem, acabará duas décadas depois, em 1989,
com a queda do muro de Berlim, edificado em 1961; entre 1985 e 1991, com
o governo de Mikhail Gorbachev, encerrando o Estado soviético a partir
da \emph{glasnost} (liberdade política) e da \emph{perestroika}
(restruturação econômica); com o Tratado de Maastricht, em 1992, que
consagrou a União Europeia; com o fim da chamada Guerra Fria. Ano de
1989 marcado, também, pelo massacre na Praça da Paz Celestial, situada
na capital da China, em Pequim, quando centenas de estudantes que
protestavam nas ruas foram mortos pelo Estado.

A espécie passará a ser governada de outro modo. Os programas
biopolíticos ajustáveis a qualquer regime deixarão de existir. O alvo
principal dos governos passará a ser o planeta, visando"-se recuperar sua
vida degradada, e a ser conservado em benefício das futuras gerações.
Configura"-se desde o final da \versal{II} Guerra Mundial a emergência da
ecopolítica para a qual todos os regimes possíveis do Estado devem ceder
à democracia que, por sua vez, passa também a ser uma prática social e
cultural plástica. Em lugar do mundo, construção cara à tradição grega,
começa a tomar seu lugar o global como sinônimo de uma uniformidade
econômica capitalista e política democrática alcançável para o planeta.
E, se preferimos, planeta e global são apenas maneiras de sublinhar que
a ecopolítica procurará dar conta não só do governo da espécie humana,
mas dos viventes na Terra e projetados para o espaço sideral.

\chapter{Vida outra}

A vida global dentro e fora do planeta, no sistema solar e na via Láctea
e, para além dela, no universo em expansão passa a ser possível. Não há
mais o vitorioso consagrado pela corrida espacial. Hoje não são apenas
os \versal{EUA} e a Rússia (antiga \versal{URSS}) que \emph{correm} para o espaço. São
todos os Estados que dependem da colonização sideral para controlar suas
agriculturas, reservas naturais e os deslocamentos de populações,
localizar terroristas, comunicar em tempo real, dirigir bombas a alvos
certeiros, fazendo das guerras ou \emph{conflitos} localizados uma nova
forma de serviço que se dispensa de voos tripulados. O controle sideral
sobre o planeta mapeia populações e áreas consideradas vulneráveis,
traça as fronteiras com maior precisão, controla tráficos de mercadorias
ilegais e o tráfego nas cidades, localiza piratarias, funciona com
equipamentos inimagináveis para o amplo policiamento e uma dilatada
segurança.

Tudo deve funcionar para que a Terra seja conservada. É preciso estudar
cientificamente o clima para que medidas de desenvolvimento capitalista
sejam tomadas reduzindo os efeitos nocivos ao planeta combalido. Cada
humano deve aprender a cuidar da natureza porque dela é parte
constitutiva; a natureza não mais é vista como algo disposto à
transformação humana, mas as mudanças operadas devem estar sintonizadas
com o controle dos ecossistemas. As águas, mares e ares devem ser
monitorados para que a vida dos vivos neste planeta seja conservada.
Todos devemos ser responsáveis pelas medidas a serem tomadas, nos
precaver de usos indevidos, educar uns aos outros, e temos, portanto,
uma outra tarefa. Para tal, não devemos ser meros obedientes às regras e
passamos a ter a obrigação de produzir inovações capazes de ser
incorporadas para a melhor gestão da vida como governo do vivo
restaurado e dos viventes dessa Terra.

Houve sim uma mudança radical da racionalidade capitalista desde
\emph{1968}. Encerrou"-se o embate entre a propagação do socialismo após
o final da \versal{II} Guerra Mundial e a compressão capitalista que lançou mão
de medidas de \emph{welfare"-state}, caracterizando um período de forte
presença do Estado na economia, consolidação de monopólios e riquezas,
limites ao mercado e benefícios sociais aos trabalhadores. Para os
socialistas de então estávamos num momento de ampliação do monopólio da
propriedade que poderia redundar na propriedade estatal exclusiva, mesmo
itinerário que soviéticos e chineses buscavam para seus respectivos
domínios sobre seus territórios e os dos vizinhos.

\emph{1968} foi o surpreendente acontecimento que colocou quase tudo em
\emph{xeque}. Seus mobilizadores jovens e os não tão jovens não se
orientavam por um método, uma organização de consciência, um
planejamento estratégico, eles apenas expunham diariamente que nada
daquilo interessava. A reação conservadora não tardou. O Clube de Roma
já esboçava a necessidade de se pensar racionalmente uma nova
configuração capitalista que escapasse da governamentalização do Estado.
Os liberais nos anos 1930, amparados em críticas à racionalidade
capitalista desde o século \versal{XIX}, matutavam como restaurar o mercado, o
\emph{hommo oeconomicus} e sua fobia ao Estado. As discussões sobre
Estado máximo e Estado mínimo expunham os riscos liberais e as vantagens
socialistas. A chamada escola austríaca de economia, cujos grandes
expoentes foram Ludwig von Mises e seu discípulo Friedrich Hayek,
encontrou fora da Europa, principalmente nos \versal{EUA}, repercussões
acentuadas nos estudos de Walter Lippmann, um economista de Harvard que
ganhou notoriedade como jornalista e com dois importantes livros:
\emph{Opinião pública}, de 1922, e principalmente \emph{A boa
sociedade}, de 1937.

Em 1938, em Paris, o Colóquio Walter Lippmann colocaria as bases do
neoliberalismo (termo cunhado pelo ordoliberal Alexander Rustow), da
racionalidade neoliberal adversa ao socialismo e ao convencional
\emph{laissez"-faire}. Neste colóquio estiveram presentes os austríacos e
alguns franceses como Raymond Aron. Anos mais tarde, em 1947, em Mont
Pélier, na Suíça, os \emph{neoliberais} voltaram a se reunir tendo por
referência o livro de Hayek, \emph{O caminho da servidão}, publicado em
1944, e por tema a crítica ao Plano Beveridge, de 1942, proposto pela
Inglaterra --- baseado, também, na Carta do Atântico, firmada um ano
antes por Roosevelt e Churchill, e que passaria a ser um dos documentos
de referência na criação da \versal{ONU} e da Declaração Universal dos Direitos
Humanos de 1948 --- para ser levado adiante após o final da guerra,
visto por eles como o grande entrave ao desenvolvimento capitalista.
Pensar a nova configuração mundial a partir do final da \versal{II} Guerra
Mundial passou a ser uma meta de combate ao socialismo e às intervenções
estatais na economia, e foi um dos seus integrantes, mais uma vez Walter
Lippmann, que criou o termo \emph{guerra fria}, em 1947, para
caracterizar a situação de confronto entre socialistas e democratas. O
pensamento sobre a racionalidade neoliberal estava armado e a presença
destes intelectuais mais que reconhecida. Faltava a hora e a vez. Diante
de \emph{1968}, a reação conservadora tomou a dianteira para organizar a
economia e as condutas dos cidadãos atordoados e carentes de ordem.

Michel Foucault, em \emph{Nascimento da biopolítica}, um curso em
andamento no Collège de France entre 1978 e 1979, publicado apenas em
2004, em um certo momento deixa de lado suas considerações sobre a
biopolítica ao constatar que algo de novo ocorrera na economia mundial e
pelos números seguidos de prêmios Nobel de economia aos chamados
neoliberais. Talvez os chilenos soubessem bem o que se passava, na
medida em que, após o golpe financiado pelo governo estadunidense
capitaneado por Richard Nixon contra o governo Allende, as primeiras
aplicações dessa nova racionalidade ali se instauraram. Afinal, o Chile
era um espaço adequado tendo em vista as medidas estatizantes daquele
governo deposto de perfil socializante. Atento às novas mudanças,
Foucault se dedicou a exibir as premissas do ordoliberalismo alemão e do
liberalismo estadunidense estabelecendo diferenças e conexões entre uma
programática de restauração de mercado e o estilo de vida estadunidense
de crença no mercado. Não se tratava de ostentar distinções ideológicas,
mas de expor a formação de um saber que se propunha revisar o
\emph{laissez"-faire}, principalmente redimensionando a noção de
trabalho.

Foucault deve ter visto suas anotações e considerações serem perdidas
naquela época, seja porque alguns decidiram enquadrá-lo como liberal ou
neoliberal, seja porque desinteressou"-se de biopolítica para se
concentrar, posteriormente, na questão da ética, da política que começa
em cada um, das formas de subjetivação e das subjetividades pelas quais
os súditos se governam. Mas talvez tudo isso seja apenas uma
especulação, porque na obra do fiolósofo"-historiador francês as questões
relativas às condutas, contracondutas, resistências sempre estiveram
presentes e os efeitos da racionalidade neoliberal desenhados naquele
curso de 1978-1979 apenas alertavam para a situação geral das sujeições
e assujeitamentos contemporâneos. Quando Foucault se voltou para a ética
e a estética dos gregos e romanos não estava dando uma guinada em suas
pesquisas, mas abrindo outras possibilidades para se compreender a
história do sujeito no Ocidente. O redimensionamento da noção de
trabalho promovida pelos neoliberais encontrava uma condição histórica
real. Não mais levar adiante a noção de trabalho"-força, mas situar a
força de trabalho como capital humano. O sentido da cooperação tão cara
aos liberais encontraria um ponto fundamental para nova inflexão. Não se
tratava tão somente de considerar o salário como renda, mas de estudar e
investir sobre os caracteres biológicos hereditários desta população (e
aí o aspecto biopolítico), os seus ambientes, sua capacidade cognitiva
impulsionada pela educação e a renda da família, e pela possibilidade de
investimento da família em seus filhos como um empreendimento.
Tratava"-se de estabelecer relações diplomáticas entre o capital e o
capital humano (Passetti, 2013).

O capital humano deve ser um empreendedor do qual depende o capital para
diversificar seus investimentos; o capital humano deve ser cuidado (com
escolas e saúde) pelo Estado para que cada um se realize no mercado,
sendo inovador e tendo a capacidade de ajustar"-se às adversidades. O
capital humano não precisa de demais suportes governamentais; deve
aprender a governar"-se e, como inovador, ser ao mesmo tempo obediente e
criador de novas regras e produtos. O capital humano, para existir,
depende de uma mudança nas relações autoritárias e hierárquicas das
empresas. Ele deve participar da produção, e a produção de mercadorias
passa a ser secundária à produção de produtos. A conformação eletrônica
da economia, enquanto produção e gestão, exige do capital humano que
dedique suas energias intelectuais participando ativamente da vida
econômica, social, cultural e política. Ele será um \emph{hommo
oeconomicus} e ao mesmo tempo um sujeito de direito. ``O hommo
oeconomicus, isto é, aquele que aceita a realidade ou responde
sistematicamente às modificações nas variáveis do meio, esse hommo
oeconomicus aparece justamente como o que é manejável, no que vai
responder sistematicamente a modificações sistemáticas que serão
introduzidas artificialmente no meio. O hommo oeconomicus é aquele que é
iminentemente governável. De parceiro intangível do laissez"-faire, o
hommo oeconomicus aparece agora como correlativo de uma
governamentalidade que vai agir sobre o meio e modificar
sistematicamente as variáveis do meio'' (Foucault, 2008a: 369). Ele se
volta aos interesses particulares desde que estes não impeçam os
coletivos e vice"-versa; participará de modo compartilhado da vida
econômica da empresa, da política, da natureza do planeta e do seu local
de habitação. O capital humano, portanto, não deve ser discriminado, ao
contrário, deve participar da ampliação de direitos e das variadas
diferenças aglutinadas no pluralismo político e no multiculturalismo.
Ele não é mais um indivíduo, pessoa ou sujeito, dele se exige ser um
divíduo ativo, propositivo, protagonista, empoderado.

Foi assim que de repente a noção de biopolítica deixou de ter a
relevância que obteve no governo da espécie humana na sociedade
disciplinar. A biologia trata de conceitos, os fatos observados,
enquanto a física trabalha com simplificações, idealizações e abstrações
a serem resolvidas em equações matemáticas, e a química opera entre os
dois. Não há mais o biológico sem a física e a química. Em 1912,
Larietta Leavit, astrônoma surda, descobriu no Observatório de Harvard
um método para calcular as luminosidades das estrelas (as Cefeidas),
mostrando sua terceira dimensão. Com isso, situou a noção de universo em
expansão e a possiblidade de se calcular seu início no passado. Naquela
época creditava"-se às mulheres a tarefa de medir posições e magnitudes
das estrelas. Entre 1925 e 1938, Otto Loewi permaneceu na Universidade
de Graz, na Alemanha, e identificou os neurotransmissores, inclusive a
acetilcolina (transmissor de redução de intensidades) e a adrenalina, ou
epifedrina (transmissor de aceleração). A produção de um saber
físico"-bioquímico na primeira metade do século passado já antevê os
futuros investimentos tanto no planeta e no universo quanto no humano e
em suas potencialidades inteligentes. No final do século \versal{XIX} já se
conhecia outro micro"-organismo, o vírus. Os trabalhos de Louis Pasteur e
Jules Francois Joubert contribuíram para entender a história natural em
nível microbiano como um novo tipo de ``ecologia do mundo microscópio''
(Lightman, 2015: 196).

Em 1943, Selman Abrahan Wakesman descobriu a estreptomicina, que vencia
a tuberculose e a peste ampliando os antibióticos, depois da descoberta
de Alexander Fleming da penicilina em 1928. Nos anos 1950 virá o estudo
do \versal{DNA}. A saúde biológica agora está relacionada ao meio ambiente,
deslocada das instituições centralizadas de cura e exclusão (hospitais,
manicômios, asilos), e se encontra conectada a centros, postos, espaços
descentralizados. A educação formal está revestida de certificações
agregadas ao capital humano inteligente para um planeta a ser cuidado em
suas adversidades. A política de direitos e punições persegue uma
conduta adequada para a consolidação da paz, e se espera um cidadão com
novos deveres na pletora de direitos. A segurança de cada um, do local
ao global, depende da precisão dos equipamentos e da conduta esperada de
cada cidadão, algo mais que apenas a convencional polícia e as forças
armadas, do que os meros programas de combate às epidemias que possam
atacar o potencial inteligente do capital humano.

Até quando o conceito histórico"-político de biopolítica traçado por
Foucault esperaria por um novo conceito relativo a essa sociedade de
controles sumariamente descrita por Deleuze (1992)? David Lapoujade
afirma que na filosofia de Deleuze interessa os movimentos aberrantes
que ``atravessam a matéria, a vida, o pensamento, a natureza, a
histórica das sociedades'' (2015: 9). Trata"-se de uma filosofia que
apresenta uma lógica irracional (a mais alta potência do pensar) dos
movimentos aberrantes (a mais alta potência do existir); é uma filosofia
que recusa o ordinário, o regular e o legal. Determinar um problema
consiste em estabelecer o próprio fato dos movimentos aberrantes:
questões de fato, do direito e da vida. Os movimentos aberrantes atestam
uma ``vida inorgânica que atravessa os organismos e ameaça sua
integridade; uma vida indiferente aos corpos que atravessa quanto os
sujeitos que transforma'' (Idem: 22). São movimentos que nos arrancam de
nós mesmos. Qual governamentalidade lhe corresponderia, quais novas
institucionalizações poderiam ser descritas?

Foi assim que começou esta pesquisa, pretendendo situar uma nova
situação de governo do planeta no qual a espécie humana está governada e
se governando. Em uma sociedade das inteligências que intercepta a
relação trabalho intelectual/trabalho manual, e na qual somos convocados
a participar como capital humano, como ocorrem as resistências? Como os
Estados se relacionam com elas? Quais seriam as novas relações de poder
na sociedade de controles? E, quando os Estados"-nação se conectam a um
governo planetário orquestrado pela \versal{ONU}, a política ainda seria uma
guerra prolongada por outros meios? A paz cosmopolita kantiana estaria
mesmo em vias de se consolidar? Os terrorismos deixariam de ser
territorializados e ter por alvo o Estado a ser modificado? A que servem
tantos direitos de minorias que apareceram depois de \emph{1968}? Quais
os benefícios da igualdade de direitos e como o direito, como luta
política pela vida, transformou"-se em meio para a vida equilibrada do
capital humano? Como compreender no âmbito das relações diplomáticas
entre capital/capital humano a emergência da \emph{nova política} e da
\emph{antipolítica}? Não foram poucas as questões que foram aparecendo,
algumas encontrando respostas, outras apenas traçando pistas para
situarmos as resistências e suas radicalidades, nosso principal
interesse.

Não se produzem leis, regras, tratados, códigos, enfim, um documento,
sem forças em luta. Esses documentos trazem"-nas em suas linhas e
decisões. Um arquivo traz a massa de coisas ditas em uma cultura, o que
é valorizado, conservado, reutilizado, repetido, transformado. Foucault
se perguntava em 1969: ``como se faz que uma dada época se possa dizer
isso que jamais tenha sido dito?'' (2014: 52). O arquivo é a grande
prática dos discursos, nele estão as condições históricas que tornaram
possíveis a aparição, o funcionamento, a transformação de um discurso.
Portanto, situar em que consistiu a mudança é muito mais relevante do
que buscar a causa de uma coisa dada, e, por conseguinte, faz"-se
necessário marcar séries. O sujeito não é uno, está dividido, não é
soberano, depende de outros, não tem uma origem absoluta, mas uma função
modificável incessantemente.

O \emph{arquivo ecopolítica} procura trazer as modificações e o que não
tenha sido dito, por meio de séries que produzem fluxos ininterruptos e
que se conectam: meio ambiente, direitos, segurança e penalização a céu
aberto. Na história não há espectadores, nem protagonistas. Ela é uma
sucessão de fragmentos, acasos, violências e rupturas: ``só o homem que
cria a história, a saber, aquele que se encontra dentro dela, para ver a
história'' (Foucault, 2011a: 65), mesmo afastado do acontecimento, está
concernido a ele. As resistências e os modos de combate a elas passam a
ser fundamentais. O terror, por sua vez, é ``o mecanismo mais
fundamental da classe dominante para o exercício de seu poder, sua
dominação, sua hipnose e sua tirania'' (Idem: 68). É a obediência cega
exigida que contradiz a revolução, e é a parte constitutiva da revolução
contemporânea. No século \versal{XIX}, ressalta Foucault, ela era desejada pelas
massas. Hoje (ele diz isso em 1976, no Japão), não há mais esse desejo
com ardor, a não ser em uma minoria que recorre ao terrorismo ou a uma
elite extremamente intelectual. Caberia ao intelectual de hoje
restabelecer à imagem da revolução a mesma taxa de desejabilidade do
século \versal{XIX}, restituir a imantação entre arte e ciência.

Foucault fala de si mesmo como Maurice Florence e situa o que é
``estudar a constituição do sujeito como objeto para ele próprio: a
formação dos procedimentos pelos quais o sujeito é levado a se observar,
se analisar, se decifrar e se reconhecer como campo de saber possível.
Trata"-se em suma, da história da `subjetividade', se entendermos essa
palavra como a maneira pela qual o sujeito faz a experiência de si mesmo
em um jogo de verdade, no qual se relaciona consigo mesmo'' (Foucault,
2004a: 236). Falar em resistências, o campo em que se constituem as
sociedades, não é julgar as boas e as más resistências pelo uso ou não
de violências, afinal em certos momentos não há como responder à
violência a não ser com violência (Foucault, 2004b). É voltar"-se sobre
si, inscrito na transhistoricidade do cinismo, e cuidar de si, de modo
diferente do romano e do cristão que instituíram o \emph{livrar"-se de
si} que nos enovelou na relação religião"-direito"-ciência (que, desde o
século \versal{XVII}, distingue"-se de outras formas de conhecimento por responder
a certos critérios que se modificam, e que Foucault chama de episteme).
Cuidar de si, ocupar"-se desde jovem, produz a arte da existência
envolvendo relações interpessoais e institucionais (Sócrates), mas
também trouxe consigo as relações entre corpo sem sofrimento e alma sem
agitação com o auxílio da razão (Sêneca e Marco Aurélio), posto que
segundo Epicteto, em seus \emph{Diálogos}, Zeus deu a tarefa ao humano,
diferenciado dos demais animais para os quais tudo está pronto, de
cuidar de si. Portanto, há diversas ocupações neste trabalho sobre si
que o cristianismo bloqueou por meio do pastorado ou redimensionou o
estado de não"-perturbação indicado por Sêneca, aquele do prazer que se
tem sobre si e que em nada se assemelha a uma força dominada ou ao
exercício sobre uma força prestes a se revoltar. O prazer sexual ainda
era da ordem da força e, portanto, cuidar"-se como gozo é se abster de
desejo e perturbação. Entretanto, mais tarde o cristianismo associará
prazer sexual com o mal.

Diante disso, como operar uma conversão de si? Isso supõe um
deslocamento do olhar do sujeito em direção a si e o retorno do sujeito
sobre si. É preciso uma arte, uma técnica de navegação, a pilotagem,
para romper com a renúncia de si e a promessa de salvação. Foucault
(2004) situará o retorno de si a partir do século \versal{XVI} com Montaigne e no
século \versal{XIX}, entre outros, Stirner, Schopenhauer, Nietzsche, o dandismo e
o pensamento anarquista, posto que ``não há outro ponto, primeiro e
último, de resistência ao poder político senão a relação de si para
consigo'' (Idem: 306). A governamentalidade passa a ser o campo
estratégico das relações de poder no que têm de reversível,
transformável e móvel, pois enquanto a concepção jurídica do sujeito de
direito o cerca pela teoria do poder político, a governamentalidade se
debruça sobre o poder como conjunto de relações reversíveis (relações do
sujeito de si para consigo). Há um sujeito a ser diluído nesta conversão
de si.

Encontraríamos na ecopolítica uma governamentalidade planetária cuja
finalidade seria a de absorver as resistências ou imobilizá-las?
Diferentemente da sociedade disciplinar, as resistências não seriam mais
execradas, punidas, extintas ou expelidas? Quais seriam as novas
dimensões do pastorado? Sob quais condições os dispositivos
diplomático"-militares entrariam em mudanças?

Seguindo com Foucault, como se encontram as relações de
poder−governamentalidade−governo de si e dos outros−relações de si para
consigo na ecopolítica da sociedade de controles?

Em seu derradeiro curso no Collège de France, entre 1983-1984,
intitulado \emph{A coragem da verdade} (Foucault, 2011) envereda pelo
cinismo e enfrenta a relação única e necessária entre ontologia da alma
e estilística da existência como sendo continuidade do \emph{mesmo}.
Portanto, estilística da existência não é colocar em prática a ontologia
da alma. O cinismo lida com o intolerável, a insolência, não se
impressiona com o medo: dizer a verdade e o modo de vida estão
relacionados ao proferir a parrésia, o risco de morte em pronunciar uma
verdade. Trata"-se de uma filosofia para a qual é preciso estar livre de
qualquer vínculo para manifestação do bios, da vida, da existência como
aleturgia (Foucault, 2010), a manifestação da verdade, o escândalo da
verdade: ``com o cinismo temos uma terceira forma da coragem da verdade,
distinta da bravura política, distinta também da ironia socrática. A
coragem cínica da verdade consiste em conseguir fazer condenar,
rejeitar, desprezar, insultar, pelas pessoas a própria manifestação do
que elas admitem ou pretendem admitir no nível dos princípios. Trata"-se
de enfrentar a cólera delas dando a imagem do que, ao mesmo tempo,
admitem e valorizam em pensamento e rejeitam e desprezam em sua própria
vida. É isso o escândalo cínico'' (Foucault, 2011: 205).

O cinismo é transhistórico: não há uma atitude cínica por excelência,
não se trata de uma conduta; está à margem de instituições, leis e
grupos sociais reconhecidos; é o cerne da filosofia, mas dela está
excluído. Foucault, seguindo Dion Crisóstomo, indica três categorias de
filósofos: os que se calam, os professores e os cínicos. Estamos diante
do inevitável. Prosseguir com os cínicos é manter o agonismo do poder, é
romper com uma seleção de critérios científicos em vigência, é
misturar"-se nos modos de vida estranhos e escandalosos, é reconhecer o
que se fala e é ouvido por crianças e pelos \emph{marinheiros} dos
tempos atuais. Os cínicos ensinam a resistir, combater por meio de
modelos, relatos, exemplos, uma ética da \emph{verdadeira vida}. O
cinismo presente na filosofia e dela excluído está em Platão (não
dissimular, não misturar, o reto, o imóvel, o incompatível), na ousadia
política de Sócrates (dizer algo diferente), na veiculação da tradição
socrática a temáticas comuns a outras filosofias como: a preparação para
a vida; ocupar"-se de si mesmo (o cuidado de si é o cuidar de si mesmo);
para ocupar"-se de si deve"-se estudar o que é real e útil para a
existência; tudo estar garantido pelo modo como se vive. E isso tudo
está em Sócrates, nos estóicos, nos epicuristas. Mas nos cínicos está
também \emph{alterar o valor da moeda}: a vida de cão, a \emph{vida
outra} e \emph{não outro mundo}.

A vida pública do cínico é de naturalidade exposta, não dissimulada,
porque a natureza nunca pode ser um mal. Nada de misturas, viver na
indiferença e na pobreza física e material como escolha, ativo e
corajoso para resistir em uma pobreza infinita ou indefinida. Viver o
prazer de ser desonrado pelos demais (o que será mais tarde aproveitado
pelo cristianismo como humildade cristã): é preciso assumir a
animalidade como dever, vida sem corrupção, vida soberana: ``vida nua,
vida mendicante, vida bestial, ou ainda vida de impudor, vida de
despojamento e vida de animalidade: é isso que surge com os cínicos no
limite da filosofia antiga (...). O cinismo aparece em suma como o ponto
de convergência de alguns temas totalmente coerentes, e, ao mesmo tempo,
essa figura da vida outra, da vida desavergonhada, da vida de desonra,
da vida de animalidade, é também o que, para a filosofia antiga, para o
pensamento antigo, a ética e a cultura antiga inteira, também é o mais
difícil de aceitar. O cinismo é portanto essa espécie de careta que a
filosofia faz para si mesma, esse espelho quebrado em que o filósofo é
ao mesmo tempo chamado a ser e a não reconhecer (...); ela é a
consumação da verdadeira vida, mas como exigência de uma vida
radicalmente outra'' (Foucault, 2011: 237-238). Ainda deveremos manter o
cínico no ostracismo?

Qual soberano? Na filosofia antiga, e a imantamos desde então, ser
soberano é ser seu, pertencer a si mesmo, socorrer os outros (amigos ou
alunos), mas para o cínico ele é o rei (rompendo com a filosofia
monarquista de Platão) e são os demais que precisam de sombra; ele é um
antirrei, a seu modo é o homem do meio"-dia nietzcheano, é o monarca sem
ideal, mas de fato, e esconde sua soberania no despojamento: ocupa"-se
dos outros, não tem a missão de legislar ou de governar, e se assemelha
ao médico ao tratar das pessoas. Não precisa de um corpo burocrático nem
de conselheiros como um rei real. Prefere o combate: é útil para a
briga, morde e ataca; investe contra as convenções e as instituições;
batalha por si e pelos outros; ``o cínico é o rei da miséria, um rei de
resistência, um rei de dedicação'' (Idem: 247).

Foucault chega então ao ponto nodal para a emergência da ecopolítica: a
vida militante. O que há em comum entre o cinismo e o militantismo? Como
sabemos, a doutrina cínica desapareceu e suas práticas são
desqualificadas pela filosofia que, segundo Foucault, procura salvar seu
núcleo. Portanto dela resta a \emph{atitude}, a maneira de ser que não
se assemelha ao alegado cinismo contemporâneo como sinônimo de
individualismo (um cinismo negativo), é o que há de mais vivo, a forma
de existência como escândalo. Também não se assemelha ao cristianismo
pelas práticas do ascetismo, mas às práticas políticas dos movimentos
revolucionários do século \versal{XIX}, porque ali não havia somente projeto
político, mas antes uma forma de vida vivida. Foucault a chama de
\emph{militantismo}: ``a vida revolucionária, a vida como atividade
revolucionária teve esses três aspectos; a socialidade secreta, a
organização instituída e, depois, o testemunho pela vida (testemunho da
verdadeira vida pela vida)'' (Ibidem: 162). Todavia, se em alguns
momentos de sua obra Foucault parece conhecer claramente o anarquismo do
século \versal{XIX}, talvez menos por preguiça e mais para atiçar, sublinha a
necessidade de estudos sobre esse militantismo, antes deste se conformar
em partidos e agências revolucionárias que se arvoram em organizar
platonicamente a verdadeira vida. Os anarquismos desde o século \versal{XIX}
traçam seus percursos sintonizando a vida vivida com sua utopia, ou
melhor, nos termos de Foucault, produzem suas heterotopias no presente,
de modo similar ao que ele compreende a vida como manifestação da arte:
o artista como condição da obra de arte, antiplatônica pela irrupção do
de baixo, embaixo, em uma relação antiaristotélica com a cultura como
recusa, rejeição de toda forma adquirida. ``Neste ocidente que inventou
tantas verdades diversas e moldou artes de existência tão múltiplas, o
cinismo não para de lembrar o seguinte: que muito pouca verdade é
indispensável para quem quer viver verdadeiramente e que muito pouca
vida é necessária quando se é verdadeiramente apegado à verdade''
(Ibidem:166).

A vida militante reverte na soberania do \emph{bíos philosophicos} em
resistência combativa. Não é proselitismo de seita de pequeno número de
privilegiados. Nos termos deleuzeanos, trata"-se da vida das
\emph{minorias potentes}. Uma militância em \emph{meio aberto} que não
exige uma paideia, mas recorre, se necessário, aos meios violentos e
drásticos, segundo Foucualt. Está em jogo mudar e não alcançar a vida
feliz. Um pouco ou muito de \emph{1968} encontra"-se aí.

Os desdobramentos repercutem nas condutas físicas e corporais dos
cínicos como zeladores de seus próprios pensamentos. E como tal não se
encontram congelados naquela ocasião, a não ser pela arbitrariedade dos
teóricos. A mudança em sua conduta provocada pela atitude cínica no
governo de si ressoa nas relações com os outros, e não tem por tarefa
mostrar o que é mundo em sua verdade. Trata"-se do exercício da parrésia,
de proferir a verdade com o risco de vida, sem proteção institucional ou
de um superior. Entretanto, o ascetismo, do paganismo ao cristianismo,
propagou"-se recomendando uma alimentação reduzida, e o cristianismo
voltou"-se para o outro mundo e não para o mundo \emph{outro},
alinhando"-se à metafísica platônica, e difundindo o princípio da
obediência ao outro como parrésia cristã (o tema da parrésia"-confiança
foi substituído pelo da obediência trêmula a Deus gerando a desconfiança
em si). Onde há obediência não pode haver parrésia.

Numa sociedade como a de hoje, em que prepondera o ecumenismo e o
terrorismo religioso, os obstáculos ao exercício da parrésia são
enormes, fazendo com que as resistências possam ser compreendidas como
assimiláveis, como contracondutas que não são capazes de combate aberto,
metamorfoseadas em agenciadoras de negociações de conflitos.

Montar o \emph{arquivo ecopolítica} demandou a busca pelas proveniências
seguindo a genealogia do poder, encontrando seus ínfimos começos no
século passado que situam as mudanças relativas à noção de \emph{meio},
os condicionantes que levaram à pletora de direitos a partir da segunda
metade do século \versal{XX}, os fluxos de segurança modificáveis relativos à
soberania interna e externa, às variações acerca da penalização.
Declarações, tratados, códigos, programáticas irradiadas pela \versal{ONU}
acompanhadas de participação cada vez mais atuante da sociedade civil
organizada e de repercussões nos Estados"-Nação foram estudadas segundo
as implicações da racionalidade neoliberal. Essas transformações
acompanhadas neste arquivo sinalizaram para a emergência de novos
dispositivos relativos aos fluxos que explicitam as novas condutas sob a
ecopolítica, o governo do planeta. Chegamos a um momento em que
\emph{todos} devem ser democratas, agora ou no futuro, como a utopia
dessa sociedade fundada no compartilhamento de interesses. A tradicional
divisão das forças políticas segundo a soberania em direita, centro e
esquerda parece não dar mais conta da vida política governada pelas
programáticas. A vida absorvida e dimensionada pela informática e a
computação fundada em protocolos diplomáticos reitera a prática da
negociação e da diplomacia dos negócios econômicos, culturais, sociais e
políticos.

Montar esse \emph{arquivo ecopolítica} em transformação passou a ser uma
tarefa mais tranquila a partir do documento \emph{Carta da Terra} e
principalmente dos Objetivos de Desenvolvimento do Milênio, projetado
para o período 2000-2015, dos quais definitivamente emerge o modo de
desenvolvimento aspirado pelo capitalismo atual fundado na
sustentabilidade e que redundou em um novo programa para o período
2016-2030, intitulado Objetivos de Desenvolvimento Sustentável. Agora,
não mais a distinção entre países desenvolvidos e em desenvolvimento. Ao
contrário, todos estão convocados a participar no desenvolvimento
sustentável, sem tarefas específicas aos mais fortes e aos menos fortes,
mas levando adiante o compartilhamento.

Buscar a governamentalidade planetária na ecopolítica redundou em
revisar as variações da biopolítica e os modos pelos quais o governo do
planeta lida com os governos da espécie. Neste percurso acidentado, de
forças que se transmutam e de novas forças que se constituem,
enveredamos pelas modulações de condutas que se apresentam, que lutam ou
disputam, por subjetividades que se refazem ou são inventadas diante das
subjetivações.

Uma nota ainda é necessária. Como intrometer Deleuze nesta perspectiva
metodológica que ladeia Foucault? A resposta breve estaria no seu breve
escrito ``Post"-scriptum das sociedades de controle'' amplamente
conhecido, estudado, dissecado desde sua publicação. Deleuze naquele
escrito procurava marcar algumas das transformações ocorridas na
sociedade disciplinar que mudava como Foucault salientara, e ia direto a
uma série de mutações ocorridas nas instituições disciplinares e na vida
econômica que ainda se pautava na produção de miséria. Os \versal{ODM} de certo
modo respondem à gestão da miséria delimitando índices a serem
alcançados em favor da \emph{erradicação da miséria}, dos direitos das
mulheres, da educação em geral, da redução da mortalidade infantil, da
saúde das mães, do combate a epidemias, e introduzem a sustentabilidade
ambiental e as novas parcerias para o desenvolvimento. Havia ali
programas de biopolítica, mas principalmente pelos últimos objetivos uma
nova visão sobre os problemas do planeta e que repercutirão depois nos
\versal{ODS}. Tais documentos também sinalizavam para novas subjetivações.

\section{``Eu vi a radiação, ela é azul, azul e reverbera''}

Vinte cinco anos após o voo de Gagarin, na mesma Rússia, na fronteira
entre a Ucrânia com a Bielorússia, ocorre em Tchernóbil um inesperado
acidente nuclear. Não havia a guerra, ou melhor, somente ainda uma
convencional guerra fria, numa época em que o uso da energia atômica era
controlado e destinado ao uso pacífico. Os acordos celebrados durante a
\emph{détente} pareciam poupar o planeta de uma hecatombe. Os
ecologistas alertavam para o controle pouco eficiente das usinas
nucleares e o fato de haver pouca atenção com os efeitos anteriores dos
testes nucleares. Porém, tudo caminhava sob o monitoramento de
burocratas. A Copa do Mundo de futebol começaria no México em 31 de
maio.

Na madrugada de 26 de fevereiro, o reator e o prédio do quarto bloco da
Central Elétrica Atômica de Tchernóbil explodiu. Atingiu o nível máximo,
segundo a Agência Internacional de Energia Atômica (\versal{AIEA}) ---
formalmente vinculada à \versal{ONU} desde 1969, depois de negociações e
regulamentações sobre o uso pacífico da energia atômica que se
estenderam desde 1953 até sua fundação, em 1957. Os únicos eventos em
perigo máximo, similares a Tchernóbil aconteceram em Three Miles Island,
na Pensilvania, \versal{EUA}, em 28 de março de 1979, e depois em Fukushima, no
Japão, em 11 de março de 2011, após o tsunami provocado por um maremoto.
Para além do noticiário jornalístico revestido de pronunciamentos sobre
autoridades governamentais, internacionais e cientistas, procura"-se
descrever os esforços dos Estados e das parcerias internacionais,
enfatizar recomendações, enfim, fora do alcance secreto dos efeitos,
ficam os segredos da devastação pouco ou quase nada registrados,
guardados pelas populações que foram alvo destes acidentes. E por isso
mesmo, somos tomados pelos relatos científicos sobre acidentes,
programas internacionais de ajuda às chamadas vítimas, e o silêncio é
revestido com precauções.

Sabemos muito pouco ou quase nada dos efeitos de tais acidentes
ocorridos e das precauções a respeito de futuros eventuais imprevistos.
Afinal, a ameaça à espécie pela guerra nuclear parece estar
definitivamente afastada. Ameaças, entretanto, não são descartadas, seja
pelo bioterrorismo, disseminando vírus e bactérias, seja na rotineira
vida hospitalar com as infecções hospitalares. Dos acidentes nucleares
às hospitalizações, o ar encontra"-se cada vez mais contaminado. Se em
1961, com Gagarin, a espécie redescobriu o espaço sideral produzindo a
possibilidade de se constatar que a Terra é azul, e ao ocupar esta
imensidão se descobriu de novo a Terra azul, e de lá se pode monitorar
economias, conflitos, insurreições e terrorismos, além de acontecimentos
naturais como furacões, ciclones, tempestades, algo mudou
definitivamente no clima do planeta e isso não se deu sob a regra geral
do antropoceno. Se as perspectivas de sustentabilidade para a vida do
planeta passam pelo equacionamento possível das igualdades formais,
estas estão subordinadas à perpetuação das desigualdades
socioeconômicas, ainda que se pretenda melhorar as precariedades. As
condições climáticas e de saúde estão governadas por invisibilidades
equacionadas por programáticas planetárias que orientam Estados e a vida
de cada um. Uma governamentalidade planetária voltada para as melhorias
da Terra e as conquistas siderais ecoa subjetividades que creem no bom
governo dos homens ilustres compartilhado por cada um. A
responsabilidade se generaliza e o meio sustentável é meio e meta.

Hiroshima e Nagasaki legaram o \emph{hibakusi}, aqueles que só podem
casar entre si, as inúmeras notícias sobre câncer e queimaduras e
devastações. Mas isso foi resultante do efeito democrático final para
dar fim à guerra no oriente. No ocidente, com a investida da \versal{URSS} sobre
o nazismo a partir da vitória em Stalingrado, a Europa entrava em
período de paz, com a investida dos países aliados e a divisão do
``mundo'' pelas duas maiores potências vitoriosas. A Europa encontrou a
paz por meio da investida fulminante dos soviéticos, e a Ásia, pelo
mortal final da guerra com as bombas atômicas.

A Terra é azul. É assim que a vemos do espaço sideral, de fora para
dentro. Zóia Danívolovna Bruk, inspetora do Serviço de Proteção da
Natureza, sublinha em depoimento a Svetlana Aleksiévitch que todos
estavam assustados e preparados para a guerra atômica e não para o que
ocorreu em Tchernóbil. ``Surgiu na cidade, não se sabe de onde, uma
mulher maluca. Andava pelo mercado e dizia: `Eu vi a radiação. Ela é
azul, azul e reverbera''' (Aleksiévitch, 2016: 258); segundo o fotógrafo
Viktor Latun: ``Chegavam rumores de que era um fogo extraterrestre, que
nem era fogo, mas uma luz. Uma reverberação. Uma aurora. Não de um azul
qualquer, mas de um azulado celestial. E que a fumaça não era fumaça''
(Ibidem: 296); Kátia K. disse: ``a fumaça da central não era negra, nem
amarela, era azul. De tom azulado'' (Ibidem: 147-148). A fumaça é azul;
não dá para fechar os olhos e não notar o que sai de dentro para fora
neste planeta.

Muitos apreciam o Prêmio Nobel de Literatura. E mesmo os que lhe dão
pouco valor acabam dedicando"-lhe atenções, sejam os admiradores de
Albert Camus, que não compreendem porque ele aceitou a láurea, sejam os
de Jean Paul Sartre, que entendem o significado virtuoso da rejeição com
a diferença de que Sartre pertencia às camadas burguesas de Paris e
Camus era um \emph{pied"-noir}; um tinha onde se escorar materialmente
(leia"-se propriedades), além do financiamento soviético desatinado a ele
e sua esposa, e outro tinha apenas seu trabalho, sua literatura. Mas se
o Nobel serve para algo é para nos alertar sobre como vai a ordem
econômica, a científica e, por vezes, a literária. Pouco ou quase nada
se conhecia a respeito da subjetividade da população de Tchernóbil que
passou pelo \emph{acidente}. E isso importa. Aquela população camponesa
religiosa, assim como os funcionários, os soldados, os liquidadores, os
inspetores, os oficiais militares, os milhares de anônimos e as
crianças, viviam ainda crentes naquele país stalinista, no comunismo,
nas palavras ocas do ex"-chefe da \versal{KGB}: ``Gorbatchóv surgiu só depois das
festas de maio e disse: `Não se preocupem, camaradas, a situação está
sob controle. Houve um incêndio, um simples incêndio. Não é nada
grave... Lá, as pessoas continuam vivendo e trabalhando''' (Ibidem:
229). Enquanto isso em Tchernóbil, desde o acidente, todos obedeciam aos
comandos, sob as herdadas táticas de guerra aplicadas sobre a evacuação
de população agora repaginadas para efeitos de catástrofe ecológica, de
controle pacífico da população a ser atendida, algo que mais tarde
repercutiria nas intervenções pacifistas promovidas pela a \versal{ONU} como a
\versal{MINUSTAH} e no controle ao tráfico de drogas nas \versal{UPP}s do Rio de Janeiro,
posteriormente a Medellín. A guerra ensinou outras táticas de controle
local das populações \emph{sob risco}, assim como redimensionou os
funcionamentos dos campos de concentração, porém, expôs cruelmente como
os soviéticos engavam a si mesmos.

O historiador Aleksandr Reválski situa: ``Eu sou historiador. Antes, me
dediquei aos assuntos linguísticos, à filosofia da língua. Nós pensamos
com a língua, mas a língua também nos pensa. Aos dezoito anos, talvez um
pouco antes, quando comecei na ler as obras em \emph{samizdat}\footnote{``Edições
  independentes, reproduzidas à mão ou em máquina de escrever, de obras
  proibidas na União Soviética. Alimentava a dissidência, às margens das
  publicações oficiais'' (Aleksiévitch, 2016: 266. Nota da tradutora).}
e vi se revelarem para mim autores como Chamálov e Soljenítsin, eu
imediatamente compreendi que toda a minha infância e a infância dos
amigos da minha rua estavam impregnadas da mentalidade dos campos de
concentração --- e isso apesar de ter crescido numa família de
intelectuais (o meu bisavô foi sacerdote, o meu pai, professor na
universidade de Petersburg). Inclusive todo o léxico da minha infância
saía da linguagem dos prisioneiros. Para nós, crianças, era normal
chamar o nosso pai de \emph{pakhán} e nossa mãe de \emph{makhana}
{[}jargão carcerário para ``chefe''{]}. `Para um cu ardiloso, uma pica
com rosca', isso eu aprendi aos nove anos. Sim, nem uma palavra civil.
Até os jogos, os ditos e as adivinhações vinham dos ambientes dos
campos. Porque os presos não constituíam um mundo à parte, que só
existia na cadeia, longe de nós. Tudo isso estava ao nosso lado. Como
escreveu Akhmátova {[}poeta russa{]}: `meio país encarcerava e meio país
estava encarcerado'. Penso que nossa mentalidade carcerária devia
inevitavelmente se chocar com a cultura. Com a civilização, com os
ciclótrones'' (Ibidem: 266-267). Ludmila Dimítrevna Poliánskaia,
professora rural, constata: ``Estamos fechados na zona. Não habitamos
mais. Vivemos num gulag. O gulag de Tchernóbil'' (Ibidem: 282). O
operador de Câmera cinematográfica Sergei Gúrin relata que filmou
pessoas que passaram por campos de concentração e que elas não queriam
falar, não queriam saber de rememorar a guerra, ``fogem daquilo que
descobriram ali sobre o homem. Daquilo que veio à tona do seu interior,
de debaixo da pele'' (Ibidem: 159), como em Tchernóbil; diz que mostrou
os filmes de Tchernóbil para algumas crianças e uma delas, ruborizada e
tímida, perguntou sobre porque não ajudaram os animais. Ele não soube
responder, mas acrescentou: ``a nossa arte só trata do sofrimento e do
amor humano, não de tudo que é vivo. Só do homem. Não nos rebaixamos até
os animais e as plantas'' (Ibidem: 161). Não havia como evacuar
minhocas, vermes, pardais e pombas. Segundo Anatóli Chimánki:
``apareceram os primeiros cachorros"-lobos, nascidos de lobas e cachorros
fugidos para o bosque. São maiores que os lobos, não param diante de
sinalizações, não temem a luz nem o homem, não respondem à \emph{vaba}
(grito de caçadores que imita a chamada dos lobos). E também os gatos
selvagens se reúnem em grupos e já não têm medo do homem. A memória da
obediência ao homem desapareceu. A fronteira entre o real o irreal está
se apagando...'' (Ibidem:179).

Não é difícil concordar com o professor da Universidade Estatal de
Gómel: ``os nossos escritores continuam a escrever sobre a guerra, sobre
os campos de trabalho stalinistas, mas calam sobre Tchenóbil. Há talvez
um, dois livros e acabou"-se. Você acha que é mera causalidade? O
acontecimento ainda está à margem da cultura. É um trauma da cultura. E
a nossa única resposta é o silêncio. Fechamos os olhos como crianças
pequenas e acreditamos que assim nos escondemos, que o horror não nos
alcançará'' (Ibidem: 130). Mas as crianças desde cedo sabem o que é
alopecia porque ficaram sem pelos, cabelos, sobrancelhas, cílios. Diz o
médico Arkádi Pávlovitch Bogdankévitch: ``Já se acostumaram. Mas na
nossa aldeia só temos uma escola primária, e as crianças que passam para
o quinto ano, têm de tomar ônibus para ir a outra escola, a dez
quilômetros. Choram, não querem ir. Lá outras crianças vão rir delas''
(Ibidem: 163).

O livro de Aleksiévitch, publicado em 2013, impele cada um a algo mais
que pensar sobre os horrores do acontecimento e do que nos esquecemos,
por vezes, propositalmente, mas que no cotidiano é simples lembrança
perdida ou apenas desconhecimento. Ao tratarmos da ecopolítica,
procuramos permanecer atentos a esses trágicos, esquecidos ou arquivados
registros, herdados desde Hiroshima e Nagazáki, Auschwitz e todos os
demais campos de concentração, para buscar nestes baixos começos a
emergência disso que muitos gostam de chamar nova vida, nova política,
novo desenvolvimento, novas subjetividades, onde cada vez mais parece
pretender anunciar o fim das resistências, ou ainda sua arbitrária
substituição ou tentativa de apagamento pelas práticas de resiliência.

A Terra é azul, a radiação é azul!

\textbf{\\
}

\chapter{Considerações}

O começo de uma pesquisa, de um livro, de um curso muitas vezes pretende
delimitar um ponto central a ser mostrado como imantação de causalidades
e irradiação de efeitos. Somos educados para ver assim e voltar ao
passado para envolver o presente ou remontar o passado a partir das
tensões atuais e trazer desdobramentos. Estabelecemos os momentos
primordiais orientados por saberes amplamente reconhecidos socialmente
na comunidade científica, no Estado, na situação geral da sociedade, em
que se pressentem reformas necessárias, correções de rotas, impulsões
para o futuro. Estamos marcados pelo presente na reconstrução real ou
ideal e nas projeções funcionais ou transformadoras para o futuro
próximo ou distante. São essas, em poucas palavras, as condições da
produção de saberes humanistas, reconhecendo"-se ou não os poderes que os
fazem emergir, suas especificidades, suas predisposições na expressão do
regime de produção da verdade.

Desde o final do século passado o termo biopolítica passou a compor
conceitualmente análises marxistas elaboradas principalmente por Antonio
Negri e Michael Hardt, a habitar as reflexões filosóficas de Giorgio
Agamben, e como uma nascente de rio escondida, talvez sob uma moita,
passou a ganhar a dimensão de luta em oposição ao que seria o governo do
biopoder de Estado. Por sua vez, liberais estadunidenses começaram a
investir mais reiteradamente na noção de governamentalidade.

Na mesma ocasião essas leituras provocaram uma inquietação relacionada à
produção deste conceito na história"-política traçada por Michel Foucault
relacionada com os acontecimentos do final do século \versal{XVIII} até sua
declaração macabra no nazismo alemão. Suas pesquisas sobre o corpo
marcado de histórias pelos governos da soberania e das disciplinas o
encaminharam para os estudos sobre a biopolítica, o governo da espécie
como pastorado moderno. O corpo passou a ser alvo da eugenia com base
nos caracteres hereditários e na sua construção destes modificados pelo
ambiente. O amplo debate sobre as políticas sob esta orientação,
respectivamente, os estudos de Mandel e Lamarck, atingiu variados
propósitos. A presença de Lamarck orientou as políticas nazistas e do
socialismo soviético; as de Mandel aos poucos governaram um dos
requisitos da racionalidade neoliberal relativo à herança hereditária do
capital humano. O corpo era o alvo e a meta do governo do Estado até o
final da \versal{II} Guerra Mundial.

As análises de Foucault são sempre inacabadas, não no sentido da
formalidade, mas da própria história do corpo por ela atravessada. Ainda
que suas diversas pesquisas não devam ser consideradas como desvios das
originais, apresentam mudanças de rotas que delineiam outros percursos
para o acontecimento principal em sua obra que é o sujeito no ocidente.
É como se Foucault deixasse em aberto espaços móveis para outras
investigações sobre as sujeições.

O sujeito em Foucault não adquire a dimensão de condutor, tampouco se
opõe ao objeto e muito menos é objeto do conhecimento em uma relação
dialética. Há um grande esforço em sua obra para abrir perspectivas,
pois não há o lugar privilegiado do poder considerado em relações
estratégicas de controle pelas inúmeras forças. O inacabado aparece por
dentro e por fora: na produção conceitual histórica sem buscar apelo
transcendente ou totalizador e na dinâmica dos acontecimentos
surpreendentes e, portanto, dos singulares que os compõem.

A biopolítica e o biopoder não são conceitos muito precisos no sentido
da oposição formal de um ao outro, de governo de Estado e de
resistências na sociedade civil. Se pudermos considerar o biopoder como
a articulação de dispositivos governamentais de Estado na gestão da
população, a biopolítica trata da dinâmica desta relação justamente pela
impossibilidade de se opor Estado e sociedade civil, seja como recurso
teórico, ou mesmo como oposição de interesses reais. Mesmo não deixando
de lado que os liberais não desconsideram essa íntima relação, porque
ela fortalece a presença imperativa do Estado; que a suposta separação
indica as possibilidades da ocupação do Estado para uma reforma geral e
justa das desigualdades sociais capitalistas sob o manto liberal; ou
ainda, que esta familiar relação sustenta o revestimento liberal e o
socialista, como defendem os anarquistas, e que produz lutas pela sua
dissolução; enfim, por qualquer um desses ângulos, está em jogo a
situação geral da sociedade civil e os modos pelos quais produzem"-se
políticas e antipolíticas.

Opor biopoder e biopolítica é desconsiderar a relação contínua e tensa
das lutas em direção à efetivação de uma forma de sociedade. Deixando"-se
também de lado a questão relativa à sociedade quanto às suas relações de
complementariedade, inferioridade ou preponderância em relação ao Estado
em sua continuidade ou abolição, a emergência da biopolítica e do
biopoder está relacionada às institucionalizações do Estado"-nação, às
formas de gestão governamental do Estado, às técnicas de governo da
própria sociedade civil, às práticas discursivas e não"-discursivas
produzidas. A distinção é meramente formal, e foi pouco esclarecida por
Foucault, mas faz sobressair a biopolítica por ser conceito capaz de
articular tecnologias de poder e governo do soberano e entre os súditos.

``A biopolítica como regulação da população é uma política de Estado que
também não prescinde das diversas práticas da sociedade civil que deram
conta da produção de um corpo saudável, mesmo sob as condições de
desigualdades, algumas vezes amenizadas como efeitos do sindicalismo e
da ameaça revolucionária. Pela biopolítica se pretendia governar os
corpos vivos, a população, instituindo que a vida de cada um dependia da
política. A biopolítica se constitui, portanto, tendo por alvo
totalizante o corpo"-espécie (população) e funciona articulada com os
poderes disciplinares individualizantes (utilidade e docilidade),
atrelando o conjunto e o individual, e intimamente relacionado aos
dispositivos de segurança. A biopolítica compõe a série população --
processos biológicos -- regulações e regulamentações relacionadas ao
corpo"-espécie como gestão calculista da vida; as disciplinas, por sua
vez, estão vinculadas ao corpo"-máquina como administração dos corpos e
estão compostas na série corpo -- disciplina -- instituições. São
tecnologias políticas que visam normalizações. A soberania sobre o
território se exerce combinando leis e normas disciplinares tendo em
vista o governo seguro do conjunto da população em seus espaços. Para se
modificar a conduta da espécie humana se atua sobre o meio,
preferencialmente urbano, em que ela habita'' (Passetti, 2013: 3-4).

O governo não se restringe ao exercício dos regimes políticos no Estado
e de suas incursões para equacionar níveis de desigualdades expressos
estatística ou ideologicamente por meio de políticas compensatórias. A
produção de subjetividades, por conseguinte, é fundamental para a
economia, e principalmente na contemporânea, esteja ela sob o regime do
mercado e seus planejamentos, ou do Estado e suas planificações. A mesma
produção de subjetividades é capaz de evocar, instaurar ou procurar
suprimir subjetivações que se encontram escondidas, no ostracismo, em
gestação, em formatações ou simplesmente abolidas no regime de produção
de verdades.

O poder não oprime subjetividades, delas precisa de modo positivo,
porém, as relações de poder produzem subjetivações outras que estão
relacionadas a práticas de liberdade, expõem singularidades, constituem
resistências que atravessam, nas sociedades de disciplinas, as redes de
poder e ultrapassam a conformação reflexiva do amor à sujeição. A
produção de subjetividades situa os efeitos das sujeições e estas não se
reduzem aos mecanismos repressivos do poder institucionalizados no
Estado ou ao revestimento ideológico a ser ultrapassado pela organização
coletiva da consciência.

Do mesmo modo, quando se busca a relação com as subjetivações não
estamos restritos a uma consciência adquirida em busca de ser
concretizada, mas voltados para a efetivação de práticas de liberdade
capazes de desestabilizar as certezas subjetivas e de inibir
desdobramentos nas práticas governamentais do soberano e entre os
súditos. O sujeito sob sujeições é simultaneamente governado e
governante de si, e é nestas condições que ele pode repensar seu próprio
governo, encontrar outras vazões, produzir o escândalo e o insuportável,
ou seja, o intempestivo. As subjetividades produzidas e atualizadas são
simultaneamente maneiras pelas quais cada um se governa para obedecer ao
soberano, amar as sujeições, empalidecer e corar para a dinâmica da vida
biológica, espiritual, produtiva. As formas de subjetivações escancaram
modos de confrontos, revoltas, o incansável ato de demolição, o
exercício de vida livre e, por conseguinte, são essas subjetivações que
reviram as subjetividades constituídas, que tanto dão forma ao indivíduo
perigoso, anormal, cidadão, revolucionário, quanto rompem com as
chamadas ``ideias"-fixas''.

As subjetivações, invariavelmente, colocam a questão da ética como
tradução pessoal da moral em cada um, mas é também aí, neste momento
exato no qual emerge a política, que se enunciam antipolíticas. Não se
trata apenas de algo pessoal, psicológico ou individual que se passa no
decurso do processo normativo das distribuições fixas e móveis, mas da
produção de singularidades que ultrapassam qualquer síntese de
consciência ou escolha livre e autônoma da razão moderna. São
resultantes de tensões que não podem ser capturadas constantemente, por
estarem em transitoriedade, serem desterritorializantes, imanentes e,
portanto, nômades. Não se é apenas governado do alto para baixo, mas se
é governado também de baixo para cima: somos irremediavelmente
governados por nós mesmos e pela nossa inserção nas lutas, segundo
nossos quereres, e livres na recomposição das vontades de poder, que
pretendem reiterar o soberano (povo, classe, sociedade). A subjetividade
que nos governa, em seu limite, sinaliza para uma falta ou para o que
devemos recolocar no lugar de algo a ser substituído.

A ética não é um instante subjetivo por si só; ela é efeito de como
lidamos com a supremacia da moral e de como cada um pode surpreender as
condutas esperadas. Trata"-se, então, de éticas de singularidades que
enfrentam os ditames da moral com o rompimento explícito a respeito do
que venha a ser o diverso diante da restauração do uno.

A biopolítica é produção de verdades sobre o humano vivo sob a forma de
população a ter sua vida prolongada sob as reponsabilidades de governos
de Estado e é parte constitutiva do indivíduo governado articulado às
variadas maneiras de gestão da vida, supondo condutas que vão do
conformismo ao revolucionarismo, de um lado, e de outro lado, à revolta
e ao inominável. A biopolítica é um efeito do modo através do qual a
economia política metamorfoseou o povo em população e o modo pelo qual a
política econômica procurou transformar o proletariado em agente
econômico.

Em suma, as subjetividades contemporâneas produzidas pela biopolítica
não procedem simplesmente do governo do Estado, dependem da disposição
dos indivíduos na sociedade civil e produzem resistências singulares às
relações de poder e às suas tecnologias. As subjetivações, por sua vez,
também estão a serviço da construção de subjetividades, isto é, da
formatação dessa intimidade com os processos normativos das disciplinas
acoplados aos procedimentos legais que dão em especiais presenças como
cidadãos, religiosos, cientistas, trabalhadores, pensadores, policiais,
políticos, mendigos, possíveis populações dispostas a intervenções
governamentais nas suas sociabilidades, nas consciências a serem
reformadas, por meio dos saberes da estatística, dos projetos, das
políticas compensatórias. A subjetividade, enfim, depende dos modos mais
ou menos intensos de produção de subjetivação na ordem, da mesma maneira
que esta é incapaz de abarcar as singularidades que não se disponham a
ser assimiladas ou capturadas e estejam desvencilhadas das esperanças
transcendentais.

A biopolítica, essa nova disposição dos governos sobre a vida, o seu
meio físico, a longevidade, a saúde, a educação, as assistências,
articulados pelo saber estatístico e pelo jogo político institucional,
também absorve as disposições filantrópicas e solidárias na sociedade
voltadas para suprir a necessidade de conter contestações sociais gerais
e específicas assim como pestes, epidemias, vícios, efeitos de
cataclismos, trânsito de mercadorias e de pessoas pelas cidades.

Trata"-se, portanto, de lidar com as práticas de biopolítica desde o
final do século \versal{XVIII} e encontrar sua pertinência nos dias atuais,
independentemente de uma possível oposição formal ao biopoder. Ao Estado
cabe, modernamente, encontrar compensações e punições necessárias para
manter seu poder diante dos perigos e dos \emph{perigosos}, e o faz de
maneira a obter um \emph{quantum} de consenso que o legitima e legaliza
suas condutas. Em nada diferem as práticas entre os súditos na sociedade
civil. Entretanto, os conceitos genealógicos do poder não encontram uma
via pavimentada destinada a um terminal, como os saberes transcendentes
costumam construir para além da linguagem metafórica que lhes é comum.
As preocupações relacionadas com as populações são correlatas à
conservação da segurança do território interno e diante de ameaças
externas, à estabilidade política do soberano, aos efeitos de fenômenos
surpreendentes da natureza, à preservação da circulação das mercadorias.
A biopolítica estava atenta para as circulações de pessoas, que, como as
mercadorias, entravam e saíam legal ou ilegalmente do território a ser
assegurado. A biopolítica é uma tecnologia de governo voltada para
garantir a liberdade liberal e, antes de tudo, a do Estado, e, nesse
sentido, também a do Estado socialista. A biopolítica está diretamente
relacionada à segurança interna e externa, aos perigos de insurreições e
revoluções internas e de guerras entre nações.

Os desdobramentos da política do nacional"-socialismo alemão já eram
transterritoriais, negociando deportações, confinando agrupamentos em
guetos e campos de concentração nos espaços próprios do seu Estado e em
territórios anexados; ampliava a já conhecida circulação de grupos
étnicos em busca de refúgios seguros diante de ameaças de extinção, não
mais pelo acolhedor exílio, mas agora pela inovação em confinamentos
governados pelo Estado com a participação dos próprios presos na gestão
de cada unidade de encarceramento em situações próximas do escravagismo.
Não se tratava mais de libertar escravos pelos efeitos coloniais, lutas
que se avolumaram desde os cativeiros imperiais romanos, mas agora
segundo um critério biopolítico relativo à dominância racial e os
explícitos ressentimentos com grupos acusados de levarem um povo mítico
à bancarrota, acrescidos de perigosos e anormais à própria dinâmica
capitalista de intervenção estatal. O capitalismo democrático ocidental
somente se interpôs às práticas nazistas (de Estado e sociedade) quando
a \versal{II} Guerra Mundial caminhava para seu fim (o mesmo deve"-se dizer do
socialismo soviético, um pouco antes, a partir da invasão do exército
alemão na \versal{URSS}).

O limite da biopolítica, como o conhecemos, redimensionou quem era o
cidadão livre e definiu, a priori, os perigosos e anormais a serem
retirados de circulação regular. Tratou"-se de uma ação política e
policial com a adesão e consentimento do conjunto de cidadãos alemães
que se consideravam superiores e que articulou, de modo estratégico, as
populações confinadas a colaborarem na gestão de sua escravidão e morte
em função da sua esperança de libertação futura e perpetuação racial.
Não se tratava mais do direito de causar a vida e devolver à morte, mas
do direito de definir quem devia viver e morrer. A biopolítica encontrou
seu limite, não como anomia ou perversão, mas como modo de governar pelo
racismo de Estado. Não se tratava mais dos elementos nodais da soberania
em obter obediência por medo ou esperança, gerando consenso por
consentimento, adesão, convencimento e omissão. Estavam em jogo os
mecanismos de obtenção de consenso sob o governo do medo e da morte.

Por sua vez, a dilatação da Rússia, inicialmente, e depois como \versal{URSS}, em
direção aos Estados do Ocidente durante a reversão da extensão alemã
durante a \versal{II} Guerra Mundial, também deixou claro que o governo da
população interna estava a cargo de uma dominância que funcionava pelo
mesmo dispositivo de produção do medo. Em ambos os casos não se buscou a
obediência pela esperança, mas sua afirmação como gestão da vida
organizada de cima para baixo por meio da adesão, temerária ou não, mas
amplamente consensual, de baixo para cima pela população
\emph{escolhida}. De ambos os lados, subjetividades se formaram,
paradoxalmente: uma na aversão ao socialismo e a outra no combate ao
capitalismo burguês. Em ambas, a grande \emph{massa} dispunha"-se a seus
governantes, preferencialmente a partir das referências ditadas pelos
seus líderes (respectivamente, Hitler e Stalin). Diferenciava estas duas
formas similares de governo biopolítico dos demais fascismos o uso
direto do confinamento em campos de concentração (a Alemanha os deixava
expostos, submetia"-se a vistorias internacionais como as da Cruz
Vermelha, que lhes eram favoráveis; a \versal{URSS}, fechada aos democratas
europeus, escondia deliberadamente os \emph{gulags} e contava com uma
opinião mundial socialista a acobertar seu sistema de reformatação de
consciências).

A biopolítica funcionava aceleradamente articulando Estados e sociedades
civis com base em subjetividades que tinham por função extirpar demais
possibilidades de subjetivações. A biopolítica expressou nesse seu
limite a aversão repressiva e mortal contra as resistências voltadas
contra o Estado ou mesmo o seu governo. Porém, encontrou um lugar
reformado para continuar existindo quando, após algumas décadas
seguintes ao final da \versal{II} Guerra Mundial, a democracia passou a ser a
prática recomendada de gestão da população em escala mundial. No
ambiente planetário, foi a partir desse momento que a noção de
biopolítica se viu catapultada, agora, em modos de resistir das
populações. O conceito de Foucault, elaborado ao final dos anos 1970 e
de pouca visibilidade até os anos 1990, agora se presta a novos
dimensionamentos que curiosamente atraem liberais e marxistas,
pretendendo saltar o limite fronteiriço do racismo, seja pelas
providenciais medidas multiculturalistas sob a forma de direitos, seja
por acomodar a biopolítica como política das \emph{multidões} frente ao
biopoder. Mas nenhuma delas enfrenta de modo direto os dispositivos de
exceção intrínsecos à democracia liberal ou social que fazem a
continuidade do Estado como categoria do entendimento. A filosofia
kantiana governa, pelo sim e pelo talvez.

Dizem os liberais que as ditaduras têm vida curta. Porém, são longas as
jornadas para quem a elas está submetido. As ditaduras são formas de
regimes do Estado para os quais as forças relutantes das sociedades
tendem a se organizar e obter apoios exteriores, capitaneados pelas
direções democrático"-representativas. As forças democráticas dirigem as
lutas contra as ditaduras, incorporam as demais no mesmo diapasão, e
depois de derrubado o governo ditatorial elas se desvencilham, aos
poucos, dos demais parceiros na empreitada. O anarquista Errico
Malatesta, de modo simples e direto, mostrou a grandiosidade desta
prática e a monstruosidade da sua institucionalização. As forças
adversárias e inimigas dos totalitarismos são muitas, mas,
posteriormente, são seletivamente contempladas, capturadas, assimiladas,
combatidas, expurgadas ou aniquiladas por meio do ostracismo. As demais
são reconduzidas, paulatinamente, à condição anterior de indivíduos,
grupos ou classes perigosas para as quais são destinadas as medidas
policiais e penais\footnote{Bonaventura Durruti, revolucionário
  anarquista espanhol, em entrevista a Pierre van Passen, em outubro de
  1936, situa com clareza o fascismo, a reação burguesa e a prática
  anarquista: ``Nenhum governo da Terra combaterá o fascismo até a
  morte. Quando a burguesia perceve que o poder lhe está escapando das
  mãos, recorre ao fascismo para se afirmar. Há muito tempo o governo
  liberal da Espanha já poderia ter reduzido o poder dos fascistas à
  impotência. Ao invés disso, vacilou. Manobrou e procurou ganhar tempo.
  Mesmo hoje, ainda existem pessoas no nosso próprio governo que desejam
  tratar os golpistas com luvas de pelica. Nunca se sabe o que poderá
  acontecer, não é verdade? Talvez um dia o nosso governo irá precisar
  dos militares rebeldes para destruir o movimento operário. (...) Se a
  revolução vencer, é claro que a burguesia não vai se dar assim, sem
  mais, por derrotada. Somos anarco"-sindicalistas. Lutamos pela
  revolução e sabemos o que queremos. Para nós significa muito pouco que
  em algum lugar da Terra exista uma União Soviética, na medida em que
  Stálin comprou a paz e a tranquilidade desta União Soviética vendendo
  trabalhadores alemães e chineses à barbárie fascista. Queremos fazer a
  Revolução aqui na Espanha não depois da próxima guerra europeia, mas
  agora, neste momento. Com nossa Revolução, damos mais dor de cabeça a
  Hitler e Mussolini do que o Exército Vermelho. Com nosso exemplo
  mostramos à classe trabalhadora alemã e italiana como se deve tratar o
  fascismo'' (Pierre van Passen. ``Entrevista''. In: \emph{Toronto Daily
  Star}, Toronto, 26 de outubro de 1936, apud Enzensberger, 1987: 186).}.

Quando a subjetivação produz subjetividades que se refazem e renovam
procurando obter efeitos universais intercalados de juízos, maneiras de
viver, expectativas imediatas e futuras em conformidade, ela evidencia
sua força. Porém, a produção de subjetivação é resultante do agonismo
das lutas que marca uma determinação ou dominância, e, invariavelmente,
envolto em batalhas constantes. Quando as subjetividades realizam suas
variadas expressões capazes de serem imantadas a um universal, temos o
efeito liberal esperado; quando o universal se configura extraordinário
como no socialismo, a tensão entre a subjetivação em vigência e as
demais em busca de existência tende a colidir. A grande vantagem obtida
pela subjetivação liberal é a de produzir muitas subjetividades como
especificidades possíveis de serem contempladas, diferentemente da
socialista. Mas como a vida moderna e contemporânea não se instituiu de
forma dicotômica, os alvos de assimilação e combate das subjetivações
variam, pois se imiscuem propositalmente dentro delas. Assim, não é de
estranhar que o nazismo tenha produzido subjetividades que se acoplaram
com rapidez tanto ao capitalismo quanto ao socialismo, e que populações
dentro e fora da Alemanha se sentiram tão contempladas pela biopolítica
nazista, como Estados liberais e socialistas que com ela compactuavam.
Um interceptador a \emph{posteriori}: como as práticas democráticas
funcionaram como restaurador, agregando práticas cotidianas e
institucionais de controle da subjetividade nazista à normalidade
democrática? A produção de subjetivação democrática"-liberal tende a
educar para a inclusão institucional, ao mesmo tempo em que defende seus
excessos protegidos como esporádicos momentos de crise. Porém, as
subjetivações capitalistas e socialistas não se bastam ou simplesmente
se opõem pela maior ou menor capacidade de governar subjetividades
conhecidas e abertura ou não para novas. As subjetivações são
imprevistas, surpreendentes, audaciosas na história"-política.

O pluralismo democrático conta somente com os que se assemelham e refaz,
renova e revigora a subjetividade ao recolocar em funcionamento os
dispositivos de segurança e direitos em torno da democracia como regime
imperfeito e como o melhor dos regimes. As vias desobstruídas estão
novamente aptas a ser trafegadas e é disso que o liberalismo precisa
para fazer circular pessoas, mercadorias e produtos por terra, mar e ar.
Por sua vez, o vaivém socialista extensionista, se não teve problemas
para fechar pactos de não"-agressão com o nacional"-socialismo, esmagou"-o,
até tomar Berlim nos combates militares finais da \versal{II} Guerra Mundial, na
mesma velocidade com que acordou durante sua invasão pelas tropas do
nacional"-socialismo. Desse modo compreende"-se como os liberais nunca se
opuseram radicalmente ao nacional"-socialismo e aos fascismos, esses
regimes de morte que não apresentavam riscos ao capitalismo, pois seu
inimigo principal se chamava socialismo, seu meio, o Estado, e sua
finalidade, a lucratividade em tempos de crise. E a justificativa
socialista soviética também caminhou por esses trilhos paralelos.
Tratou"-se, tão somente, de esperar o momento adequado na guerra para
democratas e socialistas, como aliados, recolocarem as coisas em uma
\emph{nova ordem mundial}, quando se formou o Conselho de Segurança da
Organização das Nações Unidas (\versal{ONU}), simultaneamente ao encerramento dos
tribunais de Nuremberg e Tóquio. Enquanto isso, os democratas liberais
contaram com o nazismo para não só refrear a propagação do ideal
socialista, como para produzir evidências com certas semelhanças
totalitárias, acomodados na distinção formal entre nazismo e fascismo
como governo de efeitos parlamentares recalcitrantes e socialismo como
governo de partido único. E, por fim, encontrar o novo trajeto para
combater o chamado intervencionismo estatal que tanto importunara os
liberais da primeira metade do século \versal{XX}, seja pela ameaça direta da
revolução socialista, e em menor escala dos anarquismos, ou ainda pela
necessidade de governamentalizar ainda mais o Estado, como no \emph{New
Deal}, para capturar as positividades operárias organizadas em
sindicatos e produzir políticas compensatórias.

O apogeu macabro da biopolítica coincidiu, também, com a produção do
saber liberal redimensionado como o verdadeiro liberalismo, segundo
Ludwig von Mises, um de seus patronos, ou mesmo o neoliberalismo com
suas diferentes gestações, como traçou Foucault em \emph{Nascimento da
biopolítica}, desvencilhando"-se da argumentação ideológica para situar a
emergência da racionalidade neoliberal. A governamentalização do Estado,
levada adiante desde o último cartel do século \versal{XIX}, situava as
preocupações com respostas diretas, precisas e contínuas dos governos
liberais, seja pela instauração da Comuna de Paris em 1871, seja pelos
variados movimentos de contestação operária, reformistas, sindicais,
revolucionários, terroristas e anarquistas de então. A biopolítica fazia
o necessário e suficiente para a segurança na circulação de mercadorias,
mas não para a circulação de pessoas. Não conseguia produzir
equacionamentos capazes de conter os levantes, os protestos, as
experimentações e as contestações regulares de trabalhadores nas mesmas
cidades. Havia segmentos desta sociedade que não se dispunham a pleitear
filiação ou aderir à biopolítica e produziam uma subjetivação radical.

Da Europa aos Estados Unidos da América, incluindo o restante do
continente americano e as indisposições latentes descolonizadoras no
Oriente, tudo isso mostrava que as técnicas biopolíticas de governo
funcionavam para produzir reformas nas condições urbanas das cidades;
introduziam modificações na arquitetura e no urbanismo, procurando
facilitar o trânsito de equipamentos de defesa contra possíveis
perturbações sociais, e as organizações filantrópicas tinham presenças
cada vez mais marcantes em solo estadunidense: causar a vida
seequiparava ao devolver à morte. Causar a vida produtiva e
politicamente docilizada moveu também para produzir os milhões de mortos
combatentes desempenhando nas fileiras das forças armadas a sagacidade
de suas estratégias em nome da pátria, do Estado, de certo modo de viver
disposto para a morte de um \emph{em nome de}, que era também em nome de
cada um. As positividades do poder cresciam e decresciam. E por isso
mesmo o nazismo não se intimidou em declarar \emph{quem devia morrer}
para que os corpos eugênicos vivessem; os socialistas se voltaram para
recuperar consciências degeneradas na miséria dos \emph{gulags}; e do
mesmo modo os estadunidenses construíram seus campos de concentração
para garantir \emph{quem devia viver} mesmo procedente de \emph{raças}
em guerra, mas ajustados à sua \emph{terra}, ao seu ambiente.

A guerra declarada, mesmo dispondo de regulamentações sobre o governo de
inimigos capturados, é antes de tudo uma forma rápida e produtiva de
fabricar mortes para que outros vivam. E capitalismo e socialismo
encontravam"-se nessa situação durante o período da \versal{II} Guerra Mundial:
suas vidas dependiam de mortes justificáveis. A filantropia e os
direitos sociais funcionam no mundo democrático capitalista no mesmo
diapasão dos direitos desiguais do socialismo: serviam para a demarcação
dos vivos e para deixar morrer pelo desemprego, falta de ocupação e pela
gestão dos corpos para a guerra: aos que retornavam, algumas medalhas de
heróis, uma situação de busca de emprego ou convalescença contínua,
reencontro trágico com famílias e como refazer estas relações valoradas
sobremaneira no soldado em meio ao campo de guerra. Isso aos vencedores,
pois aos derrotados destinavam"-se escombros arquitetônicos, isolamentos,
fome, desespero, humilhação, a espera de um gesto filantrópico.

Ao mesmo tempo em que o resultado da guerra opôs duas grandes potências
(\versal{EUA} e \versal{URSS}) em certa posição equilibrada, a indústria bélica e as
agências de inteligência em ambos os Estados se diversificaram em função
de um direito de matar como jamais se conhecera. Depois das bombas de
Hiroshima e Nagasaki, que encerraram a guerra contra o Japão, as duas
potências passaram a lutar pelo domínio da ameaça de morte planeta
adentro. Não se tratava mais dos campos de concentração e extermínio do
nazismo governado por crueldades inimagináveis, condenados formalmente
na recém"-inaugurada \versal{ONU}, mas estava em jogo uma maneira sutil de ameaçar
com a morte imediata enormes populações, seus governantes e/ou
investidores pela disputa imperial, um assunto ao qual o Conselho de
Segurança da \versal{ONU} deveria se ater. As subjetivações continham agora uma
meta de guerra final governada pela disputa imperial entre blocos
arranjados depois do fim da \versal{II} Guerra Mundial. E, obviamente, as
subjetividades assim se comportavam, porque cada um trazia dentro de si
as lembranças da guerra e abominava reprisar tamanho banho de sangue.
Valeria a pena uma batalha final? Os inimigos, agora, estavam dispostos
ao \emph{seu} lado, quase impessoais. Era preciso deles se livrar,
delatando"-os como espiões e/ou conspiradores para inquéritos que
deveriam destiná-los à prisão, à tortura ou morte. A ameaça de morte
pelas potências imperiais passou a governar a vida de cada um. A \versal{URSS} já
fizera isso desde o final dos anos 1920, alijando inimigos e adversários
do socialismo soviético.

As forças políticas de contestações radicais, como os anarquistas,
produziam subjetivações múltiplas e se associavam de modo contundente.
Os filhos de operários, diante do descaso estatal com crianças
abandonadas, mendigas, prostituídas e serviçais que circulavam pelas
ruas das cidades como gatunos de toda sorte, e que abalavam valores e
idealizações de costumes familiares burgueses, eram destinados a
internatos e reformatórios ou capturados como informantes ou futuros
policiais. Entretanto, essas forças políticas e sociais se mostravam
capazes de inventar solidariedades variadas para enfrentar situações de
greve, doenças, preconceitos, segregações e a produção desse exército de
reserva de poder recrutado entre seus similares. A cidade limpa e sã
precisava separar e condenar indesejáveis da mesma maneira que confinava
os produtivos aos espaços disciplinares, cujos espaços eram governados
por normalizações permanentes. Estas forças libertárias sabiam também se
associar para contestar e inventar outros espaços de liberdade, em suas
escolas, sindicatos, no interior das fábricas, nas paralizações de
greves.

Se a fábrica disciplinar, mais do que formar o operário na técnica de
produção, produzia subjetividades por meio de contratos e normas, também
ali fermentavam subjetivações por meio de contestação voltada para a
dinâmica da sabotagem, da apropriação de máquinas e do modo fazer que
apartava o trabalho manual do trabalho intelectual. Se os operários
tiveram que aprender a ler escrever nas suas \emph{longas noites
proletárias} depois da jornada de trabalho, eles também somaram os
diversos socialismos da primeira metade do século \versal{XIX} para enunciar em
certo momento a \emph{unidade das ilegalidades populares}, como
sublinhara Michel Foucault no encerramento de \emph{Vigiar e punir}.
Sabiam que a política era a guerra civil prolongada por outros meios,
como definira o filósofo e historiador"-político em \emph{A sociedade
punitiva}, assim como, em \emph{Em defesa da sociedade} e em \emph{A
vontade de saber} inverteria a equação de Clausewitz, com o domínio da
conformação biopolítica, para afirmar que a política é a guerra
prolongada por outros meios.

A política governamental, o biopoder, as compensações políticas contra
as classes perigosas precisavam encontrar mecanismos biopolíticos de
enfrentamentos com a movimentação operária tendo em vista pacificá-la.
Precisava atrair os sindicatos, cada vez mais próximos dos partidos
revolucionários ou do anarcossindicalismo, para compor na gestão do
Estado com as representações políticas, empresariais e a burocracia
\emph{pública}. Deveria articular os assistencialismos sociais em
fábricas e bairros, os benemerentes em saúde hospitalar e asilar laicos
e religiosos, os vigilantes da moral agrupados em variadas associações
voltadas ao combate aos vícios, compondo com os disseminadores de
virtudes como esportistas de espírito olímpico, competidores livres e,
principalmente, um sistema educacional escolar universal funcionando
como regulador dos chamados desvios na família e do convívio no ambiente
degenerado.

Era preciso \emph{governamentalizar o Estado} com escolas, modo pelo
qual a subjetividade capitalista ― da ordem, disciplina, obediência e
formação técnica ― constituiria a continuidade higienista necessária e
suficiente para motivar subjetividades: limpar as cidades, os corpos dos
operários, os espaços urbanos, fazer parecer limpo como sinônimo de
integrado e explicitar que os demais estavam no limite da marginalidade,
apartados em habitações coletivas, vagando pelas ruas, ensandecidos ou
simplesmente famélicos, mortos e quase mortos, pois nessa sociedade o
vivo era o comprovadamente produtivo até mesmo no ínfimo comércio. Era
importante, para além da filantropia, fazer do Estado o agente capaz de
produzir políticas compensatórias capazes de higienizar as classes
perigosas, como o liberal Max Weber recomendava no início do século \versal{XX}.
A biopolítica ampliava e solidificava seus princípios de saúde, gestão
do espaço urbano e rural, e escolarização no sentido de constituir
subjetividades concordes capazes de estancar, simultaneamente, a luta
contra a exploração e a dominação e redimensioná-la em disputa por
ocupações, empregos, vagas, direitos. Ao monopólio legítimo da violência
e da gramática o Estado agregou o da educação e da saúde social.

Entretanto, não só as vivências da Comuna de Paris, em 1871, os
episódios terroristas anarquistas contra burgueses, príncipes e reis,
mas também a disposição crescente para o combate dentro dos ilegalismos
aos quais foram regularmente destinados levaram ao acontecimento
Revolução Russa. Surpreendendo as determinações teóricas da crítica da
economia política que avistava a revolução socialista como passo adiante
ao período da transformação do capitalismo de livre"-concorrência em
monopolista, esta revolução expôs as mais variadas práticas de
anarquistas, socialistas e comunistas acopladas ou não à pacífica tomada
do Estado em um país com baixo desenvolvimento das forças produtivas. O
governo czarista caiu impossibilitado com sua vontade de poder.
Assistiu"-se, posteriormente, ao rápido movimento que transformou a
iminente abolição do Estado em sua ocupação, pela força interna dos
bolchevistas na luta contra os \emph{inimigos} \emph{burgueses} da
revolução, aos efeitos da pressão internacional após o final da I Guerra
Mundial, às impossibilidades de conjugar probabilidades, principalmente
as derivadas da situação real da maioria camponesa, e, enfim, à
\emph{necessidade} de institucionalização de um Exército Vermelho. As
variáveis justificadoras da necessidade da centralização bolchevista são
muitas, e a contenção de revoltas entre os alijados, também funcionou
aos moldes da luta contra tiranias, juntando forças e depois
seletivamente delas se livrando ou abençoando. Com o redimensionamento
econômico proporcionado pela planificação com a Nova Política Econômica,
em 1921, a revolução russa passou a ser bolchevista e governou o Estado
por meio da vanguarda do proletariado, o partido da revolução.

Nas primeiras décadas do século passado, a miserável vida do campesinato
espanhol, somada às condições de exploração industrial do proletariado
urbano (também recrutado na zona rural) no oeste da Espanha, levou à
queda do rei, aos governos provisórios e à república quando o
enfrentamento se tornava mais evidente. Muitos analistas consideram que
os anarquistas nunca deveriam ter aberto mão de seu princípio de recusa
às eleições porque a anarquia encontrara entre camponeses, operários e
marginais uma penetração incomensurável. Tratava"-se de uma subjetivação
própria e múltipla que aderiu, mesmo que circunstancialmente, aos
efeitos de outras subjetivações e se enfraqueceu. Desconsiderando as
conquistas de direitos obtidos na República ou mesmo os efeitos de
experimentações livre da gestão governamental, em pouco tempo a situação
de guerra civil real deu os contornos definitivos de uma política como
guerra civil prolongada por outros meios cujo resultado foi o apoio das
duas potências emergentes (\versal{EUA} e \versal{URSS}) à destruição do movimento
anarquista (e, em menor escala, da dissidência trotskista) que
repercutiu em vitória do fascismo franquista. Um pouco de biopolítica
ocorreu no governo republicano mesclando sociais"-democratas e
anarquistas em torno da vida na cidade e de sua limpeza; todavia, os
camponeses e os operários sabiam como autogerir suas produções,
transportes e sistemas de comunicação de modo a dispensarem"-se de
Estado, industriais, latifundiários e Igreja. Mas foi menos a
biopolítica e mais a institucionalização de variadas proveniências na
política, com os anarquistas esquivando"-se de suas práticas
antipolíticas e de suas subjetivações para acomodarem novas
subjetividades, que levou ao fortalecimento, mais uma vez, como na
Comuna de Paris, dos investimentos externos, agora de capitalistas e
soviéticos. A chamada \emph{guerra civil espanhola} expõe de modo
trágico os efeitos da força das subjetivações estatistas na
governamentalização do Estado.

Os limites da biopolítica ou sua baixa repercussão expuseram as
possibilidades de revolução, da mesma maneira que proporcionaram as
condições para fascismos e o nazismo. A biopolítica mostrou"-se incapaz
de dar conta de \emph{causar a vida} em uma proporção superior possível
a \emph{devolver à morte}. É por isso que, paradoxalmente, situam"-se
seus dois limites, ambos autoritários: de um lado, o nazismo e o
fascismo, e de outro lado, o socialismo estatal (o racismo de Estado
sobre o corpo e sobre a consciência). Os três regimes acenavam com
esperança aos miseráveis, aos trabalhadores ocupados, aos desempregados,
aos doentes, analfabetos, mulheres e crianças. Entretanto, a reviravolta
positiva da biopolítica ocorreu com o \emph{New Deal}, nos Estados
Unidos da América, precavendo"-se da conturbação socialista acentuada e
coerente com o estilo de vida liberal estadunidense, investindo em
saúde, escolas, relações solidificadas com sindicatos, manutenção do
congresso, filantropias e intervenções pontuais na economia. A seu modo
o racismo de Estado nos \versal{EUA} esteve mais saturado que os outros três, mas
levou adiante sua cultura liberal fundadora, enquanto os Estados
europeus viviam dos efeitos contornados pelos reis, parlamentos
enfraquecidos, movimentos operários intensos em que o liberalismo era
mais um conjunto de ideias e estilos circunstancialmente capitalistas
que propriamente uma cultura nacional. Será com o Plano Marshall que a
biopolítica estadunidense redimensionará a europeia em
\emph{welfare"-state}.

Para todos esses Estados, a \versal{II} Guerra Mundial foi necessária e
suficiente. Para todos esses Estados a biopolítica precisava encontrar
outros meios para ampliar o causar a vida. E a morte na guerra, em
combates, em guetos, campos de concentração e extermínio, nas ruas pelas
milícias fascistas, pelas traições e trapaças policiais instaurou a
\emph{população} como meta de governo do Estado, sociedade civil e
indivíduos a ser equacionada pelo saber neoliberal, livre da
possibilidade política de \emph{povo} ou \emph{proletariado}. Havia
ainda um \emph{mundo} em busca de desenvolvimento para o qual regimes
autoritários ainda eram imprescindíveis na luta contra o inimigo
socialista imperial. Porém, a disputa pelo governo das subjetividades
capitalísticas e socialistas ainda se encontrava em aberto. No Brasil
não foi diferente, o Estado Novo instituiu as medidas de saúde,
educação, acertos com o operariado apartado do anarquismo e do comunismo
e emparedados, estabelecendo taticamente relações próximas com o \versal{PCB} e
seu líder, Luíz Carlos Prestes, para deflagrar a abertura de direitos
sociais, combinando, inicialmente, práticas fascistas e, posteriormente,
estadunidenses liberais.

A criação da \versal{ONU} produziu o documento mais importante para a reviravolta
esperada. Em 1948, é assinada a \emph{Declaração Universal dos Direitos
Humanos} (\versal{DUDH}) pelos países capitalistas, liberais ou não. Estava claro
para o ocidente que condenar a \emph{shoah} desencadeada pelo nazismo
era finalmente obter o acesso à humanidade, à propriedade e à democracia
compondo um tripé que funcionaria no âmbito da \emph{paz} e da segurança
para minar o socialismo soviético. A \versal{URSS} se recusou a assiná-lo
juntamente com seus \emph{parceiros}, seja porque a \versal{DUDH} se fundamentava
no direito individual à propriedade, seja porque seu reconhecimento
explicitaria, cedo ou tarde, a existência dos \emph{gulags.} Neste
documento, diferentemente dos julgamentos nos tribunais de guerra, a
assinatura conjunta era impossível. Os julgamentos de Nuremberg e
Tóquio, levados adiante pelos aliados, inovaram, pois pela primeira vez
pessoas consideradas titulares do poder político foram responsabilizadas
pela guerra e não o Estado e a população como um todo, como aconteceu
até o final da I Guerra Mundial, quando os derrotados receberam toda a
culpa pela guerra perdida. Essa mudança permitiu recompor a soberania
alemã e japonesa, associando"-a aos aliados vencedores, sem cortar linhas
de crédito e sem isolar esses países dos foros internacionais e do
capitalismo que se globalizava.

O final da \versal{II} Guerra Mundial, definitivamente, redimensionou a segurança
dos Estados, a liberdade liberal, a liberdade \emph{proletária}, os
problemas de defesa, a iminência da morte e o governo do medo. \versal{EUA} e
\versal{URSS}, cada um a seu modo, dispuseram"-se no mapa do domínio global ao
final da referida guerra, respectivamente em retração e extenção.

A biopolítica, o governo da vida da população, passaria por uma
metamorfose, uma transfiguração? A biopolítica não fora capaz de levar
adiante o direito de causar a vida e devolver à morte, na qual o homem
era o \emph{animal} de cuja política dependia sua vida.
Democrático"-representativa ou estatizante, a política seria capaz de
produzir \emph{mais} compensatórios suficientes? Seria essa a sua
utopia?

Houve uma reviravolta após a \versal{II} Guerra Mundial. A biopolítica, o governo
da espécie conformada em população, chegara ao seu limite. A esse
perímetro corresponderiam, também, outros limites ao corpo individual
sob o comando das disciplinas? A relação entre o governo do indivíduo
pelas disciplinas e da espécie pela biopolítica entrariam em
metamorfoses?

A inteligência, gradativamente, foi ocupando o lugar de alvo principal
dos Estados e da produção. Não se extraia mais somente utilidade
econômica preponderantemente física do corpo individual organizado pela
produção intelectual. Sua respectiva docilidade esperada também deixava
de obter o efeito esperado. Os efeitos individualizantes e totalizantes
do poder travaram. O dispositivo panóptico, fundamental para a
vigilância dos corpos produtivos, em gestação para a produção ou
recolhidos em asilos e prisões, já não era suficiente, como sinalizava a
eficiência das agências de inteligência para além da função de
espionagem. O pastorado moderno realizado pela biopolítica
encontrar"-se"-á com uma nova conformação? Não será mais suficiente e
seguro vigiar e punir os corpos em espaços disciplinares descontínuos,
mas disponíveis aos encontros no espaço público para contestar,
protestar, questionar mais que políticas, as formas de vida, de
produção, governos e Estados?

Diante das inúmeras disputas de matizes variáveis, indo de manifestações
em favor da natureza sob os efeitos atômicos aos direitos civis e de
minorias, que enunciavam limites da democracia liberal e do socialismo,
o final da guerra decompunha a população em conjuntos móveis da espécie
a serem governados com compensações ou repressões às múltiplas
populações que se desdobravam. Não as chamaremos de \emph{multidão},
pois as conexões não se dão no âmbito de singularidades que se opõem ao
\emph{molar}, mas pela produção simultânea de condutas e contracondutas.
Estas últimas podem ser apreendidas em suas singularidades quando
afirmam anticondutas, assim como a permanência de atitudes antipolíticas
às quais estão relacionadas em sua transhistoricidade.

O conjunto estrutural do saber formado por capitalistas e trabalhadores,
ou por empresários, empregados, elites, grupos de status, sindicatos,
não satisfazia mais às localizações e identidades vigiadas. As mulheres
não suportavam mais os limites do mercado de emprego e/ou o confinamento
à vida doméstica. O sexo solto implodia as sexualidades governadas.
Novas subjetivações ganhavam presença nas ruas, nas mídias, nas
fábricas, nos escritórios, bares, nas escolas e universidades.

A economia política e a planificação socialista se viram em apuros. As
contraposições se expressavam com regularidade exigindo reformas,
revoluções, e não cessavam as revoltas. Os costumes estavam em xeque, o
tradicionalismo político"-parlamentar era insatisfatório, as políticas
compensatórias obtinham baixa repercussão (sempre se deseja mais, há
sempre uma \emph{falta} a ser suprida), queria"-se mais do
\emph{welfare"-state}, dos excessos e escassez no consumo, da falta de
empregos, de universidades ensimesmadas, da escolarização restrita, dos
programas de saúde precários, enfim, por mais que o capitalismo se
retorcesse para produzir consenso, também se via diante de uma situação
favorável, ainda que em certa medida houvesse o desfavorável, e que ia
da manifestação por direitos ao terrorismo e à guerrilha, manifestações
contra a guerra, aversões à energia atômica, a nova ameaça de morte pela
extinção do planeta, e ao inominável. O mesmo se passava com a \versal{URSS},
principalmente, em seus espaços anexados. Estavam postas as condições
para a emergência de um acontecimento em âmbito planetário.

O investimento em inteligência na espionagem, na classificação da
população (como atesta o uso dos primeiros computadores com fichas da
\versal{IBM} pelo nacional"-socialismo para controlar as populações de campos de
concentração), nos armamentos letais (como os mísseis V1 e V2 do
nazismo, as bombas despejadas contra Hiroshima e Nagasaki e as usinas
atômicas), ou mesmo a reconfiguração aguardada da economia política
liberal sobre o trabalho transformando"-o em capital humano, algo se
movia para redimensionar as relações de governo sobre os corpos. A
linguagem computo"-informacional gradualmente se espraiaria da órbita da
indústria sideral e de defesa para as relações de produção em geral.

Foi no interior do acontecimento \emph{1968} que os empresários
começaram a repensar seus planejamentos para o desenvolvimento e
repercutiram as críticas longamente esboçadas pelos liberais desde o
século \versal{XIX}, e formuladas às vésperas do final da \versal{II} Guerra Mundial por
um grupo de pensadores que avaliara o redimensionamento do trabalho em
capital humano e encontrara a chave para a sua racionalidade neoliberal.
O acontecimento \emph{1968} expõe as diversas forças em luta, coloca o
agonismo em uma dimensão que enuncia o insuportável, mobiliza os
enunciados, enfim, estabelece uma situação de tensão cultural, social e
política, para além dos equacionamentos possíveis de contenção acionados
sucessivamente por intervenções econômicas. Os liberais se viram diante
da necessidade de restaurar o mercado e denunciar as falácias do
intervencionismo econômico. Isso lhes era adequado em duas dimensões:
agenciar empregabilidades e, ao mesmo tempo, interceptar o socialismo em
extenção. Os princípios democráticos e seu fundamento na \versal{DUDH} eram
taticamente importantes para justificar financiamentos de ditaduras nas
Américas contra as insurreições de mote socialista, ao mesmo tempo em
que funcionavam como pressão à distensão nas ditaduras europeias ainda
em vigência, para justificar intervenções na Ásia e África, e colidir
com o socialismo.

A Europa deixava, definitivamente, de ser o centro capitalista, e com a
dominância estadunidense a produção da verdade capitalista e democrática
passou a funcionar como difusora de fluxos de renovação, mas,
curiosamente, não de mão única como no passado, agora aparecia a \versal{ONU}
como parceira agregadora e agência de atração dos Estados por meio de
uma inquietação elevada ao patamar de legitimidade: as condições de
conservação do planeta.

Se o socialismo no pós"-guerra se extendera e o capitalismo passava por
um refluxo constatável, incluindo as lutas pela descolonização na África
e Ásia, e as mobilizações populares nas Américas, diante do
acontecimento \emph{1968} era preciso rever a economia e a política. A
guerra convencional chegava a um esgotamento com o embate estadunidense
no Vietnã que repercutia desfavoravelmente em seu próprio interior. A
economia política liberal necessitava se atualizar e somente uma nova
adequação poderia gerar um ganho para a sua própria reforma. Não bastava
denunciar o voluntarismo no acontecimento \emph{1968}, e muito menos
caracterizá-lo como expressão da crise com incapacidade organizativa.
Neste acontecimento não se tratava de \emph{a priori} promover uma
revolução, mas dar passagem para as revoltas, cujos efeitos poderiam ser
surpreendentes nas relações agônicas de poder. Os múltiplos
contraposicionamentos vieram às ruas, às mídias, às fábricas, escolas,
universidades, enunciando o insuportável. Era uma luta pela vida, e como
qualquer luta por direitos é antes de tudo luta pela vida, tratava"-se de
compreender o caudaloso fluxo. Nem a clássica economia política nem a
crítica da economia política, tampouco o ressurgimento dos anarquismos,
tidos como cadáveres após o massacre na Revolução Espanhola, eram
capazes de organização, seja pelos seus próprios limites
histórico"-políticos escancarados naquele momento, seja pela expectativa
dos anarquistas em fazer acontecer uma mudança no seu decorrer.

Sabemos que os contraposicionamentos não só podem colocar em perigo os
costumes convencionais, a economia e a política, como também podem
contribuir para seus redimensionamentos ou implosões. As revoluções
atestam isso, assim como os ciclos de reformas governamentais pontuais
ou amplas.

Os efeitos do acontecimento \emph{1968} produziram os mais variados
fluxos que colocaram novos problemas aos capitalistas. Era inadiável dar
uma resposta imediata a tudo isso, ou melhor, mais do que reprimir,
produzir uma situação nova capaz de contrapor também ao inédito um
efeito eficiente e eficaz à difusa manifestação por direitos. Era
imperativo reconhecer direitos civis e de minorias como consagrava a
\versal{DUDH}. Era necessário ser democrático. Dessa maneira, não seria
imperativa uma reforma político"-partidária imediata, mas em breve curso.
Do mesmo modo que era necessário manter temporariamente regimes
autoritários, tão estratégicos nesse momento como foram o nazismo e o
fascismo no entre guerras, precisava acontecer algo que modulasse as
condutas em favor dos direitos. Se devia ter mais mercado, as empresas
precisariam passar por uma reforma em sua estrutura pela qual o trabalho
intelectual preponderaria e o trabalho mecânico seria rebaixado. Era
forçoso rever os financiamentos, a gestão governamental, reavivar o
mercado com serviços, produtos e direitos. Era necessário que a
democracia invadisse as relações empresariais para conter as negociações
com os sindicatos. Era o tempo para redimensionar a burocracia estatal.
Era preciso outro \emph{salto} para o desenvolvimento. Tudo isso e muito
mais exigia uma redefinição dos meios de produzir para gerar
lucratividades e ao mesmo tempo levar ao trabalhador sua nova condição
de empreendedor de si e parceiro nos negócios. O mercado tinha que
mostrar que era o exclusivo espaço para a competitividade reconhecida de
cada um com méritos e prêmios, não mais de modo tópico, mas como
\emph{universalização}. A complementaridade entre capital e trabalho,
entre empresário e trabalhador empreendedor devia alcançar um nível de
estabilidade. Portanto, nada melhor que incentivar a capacidade de
inovação do trabalhador como capital humano. E se este ainda estivesse
em uma condição de incapacidade constatável, contar com a colaboração do
Estado construindo escolas, universidades e programas de saúde para
equipar os potenciais capitais humanos de seus déficits hereditários e
de cultura.

A reversão conservadora liberal, governada pela racionalidade
neoliberal, encontrou seu novo costume, abandonando as relações no eixo
para as relações em fluxos. A democracia representativa passou a agregar
a participação organizada de modo sistêmico e modulável, procurando
ajustar no Estado e na sociedade civil a organização para o acesso à
informação, à comunicação e ao conhecimento colaborativo, com efeitos
imediatos de aplicabilidades na empresa, nos partidos políticos, nas
burocracias, na produção de mercadorias e produtos. O mercado se
apresentou como o espaço seguro para empregos e direitos, e
consequentemente sustentado em novas regulamentações legais e
normativas. O mercado passa a ser o espaço para a formação de uma
sociabilidade democrática que ancorará o regime político e o
\emph{público} se reveste de mercadologias: tudo deve girar em torno de
empregos e ocupações pelos mais qualificados, a partir de sua formação
real ou como investimento em capital humano. Sem colaboração não se
compartilha a segurança de estar empregado.

A democracia funciona para bem ajustar empresários e empreendedores de
si e estes entre seus pares gestando uma relação de amizade inédita no
capitalismo em função de sempre \emph{melhorar} cada um em sua adequada
e conjunta condição. Transferido para o âmbito de sua vida social isso
repercutirá na gestão das \emph{comunidades} em função do interesse
individual jamais ser sobreposto pelo coletivo e ao mesmo tempo
dimensionar o interesse coletivo como aquele que não intercepta, mas
promove o empreendedor. Ao lado de empresas na economia, o social será
habitado pelas mais variadas formas de organizações, de \versal{ONG}s a Fundações
e Institutos, na organização das heterodoxas populações (mulheres,
pobres, negros, deficientes, homossexuais, idosos, crianças e jovens,
aposentados, desempregados, refugiados...), assim como serão acionadas
as mais variadas parcerias entre a sociedade civil organizada, os
Estados, os fóruns internacionais e as organizações internacionais.

Ainda são vigiados os espaços disciplinares da escola, de certas
fábricas, nas forças armadas, nos hospitais... Mas agora,
principalmente, monitora"-se a inteligência do capital humano, a saúde
medicinal do corpo, o trânsito pela \emph{urbe}, a natureza, o espaço
sideral habitado por satélites, e onde está o \emph{inimigo}. A vida
passou a ser monitorada, e quando se diz monitorada se diz também
constantemente avaliada por meio de índices que orientam políticas
governamentais em âmbito planetário sobre a persistência da
racionalidade neoliberal. A mudança nada repentina produzida pela
racionalidade neoliberal ajustaria as passagens de regimes autoritários
para democracias, assim como modularia conivências com ditaduras aos
moldes socialistas como na China, porque era preciso manter a produção
de mercadorias, a exploração de recursos energéticos e criar produtos
por meio de vultosos financiamentos. Não se tratava mais somente de
investimentos de capitais multinacionais e respectivos financiamentos
com a adesão dos Estados"-nação no emaranhado jogo dos monopólios e seus
blocos privilegiados nos governos. Tratava"-se de democratizar o governo
dos monopólios. Era imperativo que tudo isso funcionasse em favor do
juízo democrático.

O percurso acidentado entre capitalismo e socialismo durante a chamada
Guerra Fria obteve o ponto de reversão crucial a partir da
\emph{détente} estabelecida desde a crise dos mísseis em Cuba e que
encontrou na Conferência de Helsinki, em 1975 (Nuenlist, 2005), seu
momento promissor com o reconhecimento por todos (incluindo, agora, a
\versal{URSS} e os Estados a ela perfilados) das garantias dos direitos humanos.
Obra do acaso, da arrogância de Brezhnev (ou Brejnev), da astúcia
diplomática ocidental? Pouco importa. A partir de então, e dinamizados
em escala internacional pelo governo Jimmy Carter, nos \versal{EUA}, os direitos
humanos passaram a funcionar também como uma plataforma de
desestabilização dos regimes totalitários (e, consequentemente, do
socialismo) e dos autoritários. O efeito internacional desta conferência
repercutiu nas intromissões do Vaticano, nos desdobramentos da
intervenção militar com apoio soviético na Polônia em dezembro de 1981,
e na chamada Revolução de Veludo na Tchecoslováquia, marcando o
rompimento, em 1989, com o socialismo autoritário, por meio de
mobilizações pacíficas (mesmo que outras violentas ainda ocorressem,
como no caso da dissolução da Iugoslávia) que passaram a reconduzir
Estados europeus acoplados à \versal{URSS} para a democracia e o capitalismo, do
mesmo modo que regimes autoritários na América Latina encontraram, nesta
década, os princípios de distensões gradativas para a democracia.

Seja no âmbito internacional, seja nos nacionais, a partir da derrocada
socialista soviética em 1991, e do Tratado de Maastricht, em fevereiro
de 1992, fundando a Europa federativa, a reversão neoliberal acompanhada
de democracia entra em fase de consolidação e extenção no oriente, na
América Latina e África (guardadas as ressalvas estratégicas e táticas
com a China). O principal efeito se deu com a criação pela Assembleia
das Nações Unidas, em 20 de dezembro de 1993, do Escritório do Alto
Comissário das Nações Unidas para os Direitos Humanos, com a função de
proteger e difundir os direitos humanos, supervisionando o Comitê dos
Direitos Humanos, sediado em Genebra. Por fora e no combate interno, os
direitos humanos foram se consolidando como a questão principal para a
efetivação democrática e capitalista, orientando, inclusive, Missões de
Paz que procuram conter eclosões como as ocorridas com o genocídio em
Ruanda, em 1994, e o conflito Darfur"-Sudão, entre 2003-2006.

Os direitos conquistados comporão uma pletora organizada por meio de
\versal{ONG}s, fundações e institutos com a função regulamentada de consulta e
participação nos diversos fóruns preparativos para o grande encontro
sobre desenvolvimento e meio ambiente na cidade do Rio de Janeiro, em
1992, com a participação de chefes de Estados, sob o comando da \versal{ONU}.

A partir desse momento consolidava"-se não somente uma nova gestão da
vida no \emph{ambiente} do planeta, assim como se espraiava o
\emph{dispositivo monitoramento} como meio principal para o governo das
condutas. A segurança e os direitos da chamada sociedade civil
organizada e dos Estados em âmbito planetário situavam uma nova
governamentalização em curso. Esta estava em jogo a partir das novas
relações com a natureza a ser conservada por meio da reponsabilidade
compartilhada por empresários, Estados e sociedade civil organizada,
concretizando a importância do \emph{dispositivo meio ambiente}.

A proteção e a difusão dos direitos humanos contam com a colaboração
presencial de \versal{ONG}s, conectando um fluxo de segurança que articula
Estados, o cotidiano e o humano tornando compreensíveis os
desdobramentos da segurança relacionada ao ambiente, ao clima, à
alimentação, à defesa, à gestão de conflitos internos. Uma segurança
para a qual não basta apenas o dispositivo diplomático"-militar. O
monitoramento produz um policiamento estendido, não mais apreendido
apenas como agente repressivo, mas recuperando sua proveniência
médico"-administrativa (Foucault, 2003) e a redimensionando faz da
conduta do \emph{divíduo} um policiamento constante, aos demais
cidadãos, comunidades, sociedade, suas instituições e Estados,
redimensionando a função do pastorado governamental da biopolítica.
Pastor de si e dos outros, o divíduo é capaz de fortalecer o governo
entre os próprios súditos e fortificar a soberania. O dispositivo
diplomático"-militar se metamorfoseia em dispositivo
militar"-diplomático"-policial. Uma nova conduta a ser construída por
\emph{todos} requer, portanto, que cada um seja adaptável às
circunstâncias e às adversidades, seja resiliente.

A partir da noção de \emph{resiliência}\footnote{``Resiliência significa
  a capacidade de `resistir a' ou de `ressurgir de' um choque. A
  resiliência de uma comunidade, relativa aos possíveis eventos que
  resultem de uma ameaça, se determina pelo grau de recursos necessários
  com que conta esta comunidade e sua capacidade de organizar"-se tanto
  antes como durante os momentos urgentes'' (Global Assessment Report on
  Disaster Risk Reduction 2011: Revealing Risk, Redefining. Development.
  Estratégias internacionais para redução de desastres das Nações
  Unidas, \versal{ONU}, 2009: 20-21). ``Resiliência (capacidade de recuperação)
  resulta da correta adaptação produzida por governos, empresas,
  sociedade civil, organizações, famílias e particulares com grande
  capacidade adaptativa. A vulnerabilidade é o oposto de capacidade
  adaptativa'' (\versal{ONU}-\versal{HABITAT} --- Informe Mundial sobre Asentamientos
  Humanos, 2011: 49).}, com definições variadas, todavia vinculadas à
redução de \emph{vulnerabilidades}, constitui"-se o contínuo que vai da
educação de crianças e jovens ao governo das cidades e da conservação do
meio ambiente ao \emph{ambiente} \emph{transterritorial}. A
\emph{resiliência} redimensiona a segurança, os sistemas protetivos
integrais e funciona para conter, gerenciar e monitorar \emph{estados de
violência}. As práticas de \emph{resiliência} pretendem ser o ponto de
clivagem à contenção de resistências, mas também à prática principal, a
de inibição de resistências. A \emph{resiliência}, ao pressupor
adaptabilidade atingível, indica quem não for resiliente como o agente
da recusa à diferença; entretanto, como toda relação de poder supõe
resistências (não confundidas com reação), os resistentes passam a fazer
parte do \emph{déficit} com o qual a sociedade deve saber tratar, em
condições análogas ao do chamado crime.

A ampliação gradual da noção de \emph{ambiente}\footnote{Diante da
  etimologia da palavra ambiente, suas proveniências no latim e inglês
  abarcando ambos, \emph{entorno, o que circula, sinônimo de âmbito},
  \emph{proximidade} a \emph{ambição}; do francês, relacionando
  \emph{favores, cortesias} e \emph{bajulações}; chegando a
  \emph{ambiências} (no francês e inglês) própria a atmosferas
  relacionadas a arranjo entre elementos, maneira pela qual se lida com
  arredores, subúrbios, como no alemão \emph{Umgebung} (Partridge,
  2006).} ultrapassa a de \emph{natureza} e estende"-se para o rural em
função do discurso ecológico. Houve o deslocamento da noção de
\emph{ambiente} fundada em condições específicas de populações
problemáticas, como situava a Escola de Chicago, para a de vida urbana
de cidades sustentáveis, que supõe garantias de segurança a todos com
possibilidades de participação. Ao mesmo tempo ocorre a introdução da
noção de \emph{ambiente internacional}\footnote{``Em meio a um ambiente
  internacional em constante mudança, a \versal{UNESCO} sempre buscou trazer
  soluções práticas para os desafios específicos apresentados por cada
  momento histórico. Com sua capacidade de promover o diálogo e a
  criatividade, a diversidade cultural se mostra como condição essencial
  para a paz e para o desenvolvimento sustentável''. (Unesco,
  \emph{Convenção sobre a proteção e promoção da diversidade das
  expressões culturais,} ratificada pelo Brasil Dec. Legislativo
  485/2006, p. 21).} como ultrapassagem das demarcações restritas entre
Estados ou de interfaces com a sociedade civil próprias das Relações
Internacionais.

Os governos regulados e regulamentados enfrentam contínuas manifestações
de protestos no \emph{ambiente}, como se presenciou mais intensamente
durante o ano de 2011. A deposição do ditador da Tunísia e as
reviravoltas no Norte da África ocorreram em função da supressão das
tiranias e da necessidade de construção contínua da democracia,
salvaguardada pelo ambiente internacional. Ao mesmo tempo, a morte de
Osama bin Laden pelo governo estadunidense mostrou outra possibilidade
de contenção do terrorismo fundamentalista como grande ameaça da década,
e o presidente dos Estados Unidos da América enfatizou, em seu discurso
vitorioso, o triunfo da América \emph{resiliente.} Os protestos na
Grécia, Espanha, Bolívia e Chile, entre outros, colocaram em questão os
efeitos esperados de \emph{resiliências} enquanto situações adaptáveis
diante da educação e cultura resilientes que se instituiram.

As práticas de \emph{resiliência} situaram as transformações
institucionais, governamentalidades e principalmente as insipientes
resistências mais radicais na conformação da \emph{cidadania digital}
(Shirky, 2011). Mas principalmente, como sublinha Saul Newman (2011),
evidenciaram"-se pertinentes nas resistências que se cosntruiram desde as
práticas anarquistas de Seattle, em 1999, até o \emph{Occupy Wall
Street}, em 2011. Da mesma maneira, a noção de insurreição, usada para
qualificar tanto protestos nos países árabes quanto as ações de grupos
clandestinos contra a ocupação anglo"-americana no Iraque ou as
atividades de guerrilhas na Colômbia, aparece como constitutiva da
reforma da ordem, sem o aspecto revolucionário esperado na sociedade
disciplinar.

A relação entre \emph{resiliência} e \emph{ambiente internacional}
conecta as políticas de combate ao risco e a instituição da
\emph{cultura de paz}. A \emph{resiliência} abrange, também, tanto
efeitos climáticos e demais incidentes naturais como maneiras de
prevenir e proteger áreas urbanas e rurais, suas populações e economias.

Um \emph{ambiente internacional resiliente} começa a se configurar
vinculado ao \emph{desenvolvimento sustentável} e à \emph{cultura de
paz}, fundindo os princípios da prevenção e da precaução, atingindo
áreas, pessoas, águas e ares. Trata"-se de um \emph{ambiente} em fluxo,
que deve passar por uma redefinição gradativa como resultante dos
esforços conectados entre organizações internacionais, governos
nacionais e sociedades civis acopladas a eles, produzindo um
monitoramento eficiente e eficaz com a participação de cada indivíduo,
que também se desdobra pelas interfaces como \emph{divíduos} e seus
direitos inacabados (os anormais devem ceder lugar aos resilientes: o
que excluía agora inclui; o que produzia perigo, agora é passível de
gerenciamento pelo controle dos transtornos, e assim, evitam"-se as
\emph{perturbações}), como cidadão e polícia da vida segura. Neste
sentido, a democratização dos costumes produz a politização de cada um
em função do exercício de um pastorado que ultrapassa os limites da
biopolítica: o pastorado configurado pelo monitoramento de cada um por
cada um e de todos para cada um procura incorporar a noção de
\emph{resistência} em interface com a de \emph{resiliência} e
\emph{risco} --- capacidade de suportar os efeitos dos incidentes e
retornar à condição adaptável em um ambiente \emph{seguro}. O capital
humano, enfim, passa a ser apreendido enquanto configuração inacabada
pelo \emph{empreendedorismo} individual, como empreendedorismo de si,
empresarial e governamental fortalecedores da \emph{segurança} do
\emph{ambiente internacional}.

A noção de \emph{campo de concentração a céu aberto} adquire contornos
mais definidos pela capacidade plástica e elástica da
\emph{resiliência}, geralmente associada à figura do elástico por ter
contornos definidos de fronteiras em dilatação e contração ao estado
original, demarcação contínua e móvel do território, governo da
comunidade e dimensões do \emph{ambiente} compondo adaptabilidades
diante de adversidades, com a ajuda de \emph{fatores protetivos} capazes
de conter resistências, inibi"-las ou fazê-las assumir as condições de
risco diante das \emph{vulnerabilidades} a serem contidas.

A ecopolítica é noção analítica que procura encontrar genealogicamente
uma generalização histórico"-política relacionada ao presente, sem a
pretensão a ser universal. Da mesma maneira que a biopolítica diz
respeito ao governo totalizador do poder no corpo"-espécie relacionado ao
seu aspecto individualizante no corpo do indivíduo útil e dócil da
sociedade disciplinar, estamos diante de uma metamorfose na qual a
ecopolítica, como governo do planeta no aspecto totalizador se relaciona
com o governo das inteligências do corpo convocado regularmente a
participar. É a racionalidade neoliberal que baliza essa produção
incessante que tem por finalidade inibir resistências nesta sociedade de
controles em que vivemos. A ecopolítica se dinamiza por direitos
inacabados, seguranças redimensionadas, monitoramentos contínuos, em
favor de um ambiente que conserve a natureza de modo sustentável e para
o qual se recomenda a conduta resiliente.

A análise genealógica nos remete aos \emph{baixos começos} para nos
mostrar emergências que produzem subjetivações inéditas e
surpreendentes, talvez por se ater a uma série de saberes insurrecionais
propositalmente esquecidos. Ela também nos move a um interessado arquivo
que se move em função de uma ordem sustentável que se consolida, que
procura blindar o insuportável e o inominável, destinada ao
\emph{melhor} futuro para as próximas gerações. Âmbito do real e da
utopia capitalista para o qual as mobilizações, desde o movimento
antiglobalização em 1999, produz contracondutas assimiláveis e outras
singulares que fermentam a antipolítica.

Será que a política ainda é guerra prolongada por outros meios, que o
poder se produz em redes, que a Terra encontrará finalmente num futuro
não tão distante um exoplaneta similar? Algumas respostas transitórias e
transáveis serão esboçadas no decorrer desse livro, segundo as
inquietações, adendos, complementos possíveis de cada leitor interessado
nesse arquivo, nesse acontecimento e que essas urgências possam
redundar. É por demais incisivo enfrentar a morte, principalmente a sua
e a da sociedade. Até quando permanecerá a crença na vida eterna e no
rejuvenescimento perpétuo?

\chapter{Percursos}

Um cão amarrado em estribo embaixo de uma carroça segue o caminho entre
as rodas. Certo dia, o cão é libertado do estribo; ele circula, olha e
estanca. A carroça parte e o cão a segue. Esta cena emblemática de
\emph{Viridiana}, de Luis Buñuel, cineasta abertamente libertário,
escancara o cruel jogo da acomodação à obediência. O cão acostumado a
seguir a carroça, depois de libertado permanece com a mesma conduta. Foi
adestrado a obedecer aos trajetos da carroça, acompanhar seu dono a
qualquer destino, conformar"-se com o andar manso da diligência. Obedece,
segue o curso, pouco se importando com a libertação que o outro
promoveu. Assujeitado, mais do que obediente, ele segue e seguirá seu
caminho conhecido.

A liberdade é mais do que um ato generoso, humanista, filantrópico,
legal ou consciente. Ela supõe uma prática que liberta também de ser
escravo de si. O indivíduo na massa é similar. O indivíduo \emph{livre}
sob as amarras visíveis ou invisíveis segue os condutores, seu pastor,
como segue o rebanho animal ou humano. As libertações promovidas pelos
condutores de consciência dispuseram os mesmos humanos a perseguir
caminhos traçados para que eles os seguissem, e os seguiram. Nisso
liberais e socialistas se assemelham. Pouco importa a declaração verbal
ou escrita do direito em lei ou norma, a obediência ao superior é a sua
condição principal, seu assujeitamento segundo o qual ele mesmo se
coloca para seguir mirando o superior, é sua condição de vida, de uma
existência funérea.

\begin{enumerate}
\def\labelenumi{\arabic{enumi}.}
\item
  \textbf{Jacques Prévert: ``Cena da vida dos antílopes''}
\end{enumerate}

\emph{Na África, existem muitos antílopes; eles são bichos encantadores
e muito rápidos na corrida.}

\emph{Os habitantes da África são os homens negros, mas há também homens
brancos: são os que estão de passagem, vão fazer negócios e precisam de
ajuda dos negros; mas os negros gostam mais de dançar que de construírem
estradas ou ferrovias, é um trabalho duro que volta e meia causa a morte
deles.}

\emph{Quando os brancos chegam, os negros costumam fugir, os brancos os
caçam e os negros são obrigados a fazer ferrovia ou estrada, e os
brancos os chamam de ``trabalhadores voluntários''.}

\emph{E aqueles que não dá para prender, porque estão longe demais, ou
porque o laço é curto demais, ou porque eles correm rápido demais, são
atacados com fuzis, e é por isso que às vezes uma bala perdida na
montanha mata um pobre antílope adormecido.}

\emph{Então, é uma alegria para os brancos e também para os negros,
porque os negros em geral se alimentam mal, todo mundo desce de volta
para aldeia gritando:}

\emph{``Nós matamos um antílope'', e tocam muita música.}

\emph{Os homens negros batem nos tambores e acendem umas fogueironas, os
homens brancos os assistem dançando, e no dia seguinte escrevem para os
amigos: ``Tinha um tamborzão extremamente bem executado!''.}

\emph{No alto, na montanha, os pais e os amigos do antílope se olham sem
dizer nada... Eles sentem que alguma coisa aconteceu...}

\emph{O Sol se põe e cada bicho se pergunta, sem se atrever a levantar a
voz, para deixar os outros preocupados: ``Onde será que ela foi? Ela
disse que estaria de volta às nove... para o jantar!''.}

\emph{Uma antílope, sobre uma pedra, olha a aldeia, bem longe, lá
embaixo, no vale: é uma aldeia bem pequena, mas tem muita luz e cantoria
e gritos... uma festa com uma fogueira.}

\emph{Uma fogueira para os homens, a antílope entendeu; ela sai de cima
da pedra, vai encontrar os outros e diz: }

\emph{``Não vale a pena esperar, podemos jantar sem ela...''}

\emph{Então todos os outros antílopes se sentam à mesa, mas ninguém está
com fome. É uma refeição bem triste} (Prevert, 2007).

\begin{enumerate}
\def\labelenumi{\arabic{enumi}.}
\setcounter{enumi}{1}
\item
  \textbf{hikikomori:}
\end{enumerate}

\begin{quote}
\emph{autistas informáticos, jovens japoneses com completa retração
social} (Vila"-Matas, 2011: 36).
\end{quote}

A pesquisa com base na informação eletrônica pode fazer do pesquisador
um \emph{hakikomori}, isolado e crente no vasto, modificável,
constantemente revisto e ampliado \emph{mundo} dos sites, \emph{wikis},
postagens, documentos oficias e denúncias. Prender"-se aí, no constante
monitoramento de seu material de pesquisa, é associar ao pesquisador,
além da característica de \emph{hikikomori}, a de policial, ao monitorar
idas e vindas, ora constatando os efeitos positivos dos programas de
governamentalidade, ora deslocando"-se para o oposto, denunciando os
efeitos negativos do gerenciamento da vida, recoberto de saudosismo de
um iluminismo perdido. Entretanto, ao revirarem, ainda, as informações
sobre o regime do inquérito, estes pesquisadores nada mais fazem que
metamorfosear as pesquisas em humanidades, como as conhecemos,
principalmente, no século passado. Não se trata de acoplar informações,
delimitar tempos de investigação, moderadamente constituídos e
reatualizados, sob a égide da melancolia ou do êxtase.

Não estamos mais diante das inquietações de Kant sobre \emph{quem
somos}, mas estamos diante do que estamos nos transformando (Foucault,
2000) e ao mesmo tempo \emph{contra o que somos} (Foucault, 1995),
atualizando a questão de Spinoza, \emph{o que estamos fazendo de nós
mesmos?}, e o alerta de Deleuze (1992) sobre como os jovens descobrirão
na pele as novas maneiras de resistir nesta sociedade de controle que se
configura.

A \emph{ecopolítica} procura responder a algumas destas novas
institucionalizações, não como disciplina acadêmica ou componente da
gestão do governo, mas como prática de governo do planeta nos tempos de
transformação (de si, dos outros e do planeta no universo), como
política de governo transterritorial.

O uso estratégico de \emph{biopolítica} como resistência de
singularidades ao \emph{biopoder}, apenas provoca dicotomia teórica
reposta em termos de Estado"-sociedade civil. Não dá conta das novas
conformações, e o conceito de Foucault, próprio da sociedade
disciplinar, ganha, em intérpretes marxistas, um novo vazio como
designação universal das singularidades, compondo com a captura pelos
liberais e afins em função das possibilidades de consolidar o
\emph{futuro melhor}. A pretensão em ambos os casos é a de atualizar o
conceito, e desta forma, realizam seus usos políticos, para os quais
Foucault sempre permaneceu alerta, ainda que tivesse o cuidado de
enfatizar que seus conceitos estavam relacionados à história do presente
das sociedades disciplinares que se transformavam.

O menos estranho nisso tudo é que a apropriação do conceito por teorias
soberanas ressalta o incômodo produzido pelas análises de Michel
Foucault ao instabilizar a soberania, as teorias e as práticas nas
previsíveis disputas políticas. De certo modo, liberais e marxistas
produzem, pela incorporação do conceito de biopolítica, a aceitabilidade
tardia de Foucault na ciência política pelo esvaziamento do conceito em
mão dupla: em práticas redutoras do tamanho do Estado ― e neste sentido
deveriam aceitar a análise de Foucault sobre a racionalidade neoliberal
afirmada pela teoria do capital humano e os correlatos déficits de
criminalidade com os quais a sociedade deve seletivamente conviver ―, ou
em resistências que aguardam condutores ― e com isso deveriam abandonar
a crítica ideológica ao neoliberalismo e enfrentar as artimanhas em que
a \emph{esquerda} se viu conectada desde então.

Desvencilhados do enquadramento ideológico em neoliberalismo, os
liberais procuram situar uma nova etapa do capitalismo como
\emph{desenvolvimento sustentável}, proposição dimensionada no interior
da racionalidade neoliberal por meio de programas de déficits elaborados
em comissões, comitês, organizações \emph{internacionais}. Não se trata
mais de intervir em saúde e educação para estabelecer novos patamares de
cultura política, mas de produzir conexões neste sentido. Por sua vez, a
nostalgia do \emph{welfare"-state} como estratégia de \emph{esquerda}
nada mais faz do que evidenciar as práticas necessárias do Estado"-Nação
para viabilizar as recomendações internacionais. De ambos os lados, em
defesa do \emph{desenvolvimento sustentável}, do ecosocialismo, ou da
ecologia como humanização e estratégia política de contestação, ambos
inscrevem"-se como atuantes forças na configuração da ecopolítica atual.

Regidos por configurações de poder soberano e de formatação
jurídico"-política, liberais e marxistas parecem ter se encontrado no
mesmo fluxo, ainda que, no plano ideológico, conservadores e
revolucionários articulem discursos de reviravoltas imediatas: os
primeiros, assustando"-se com qualquer protesto, exigindo menos impostos,
novos investimentos em segurança e defesa militar; os demais, aguardando
a qualquer momento o fato revolucionário com base em críticas
ideológicas ao capital e crendo em mobilizações de protestos como
antessala insurrecional da revolução, constituindo o comum, mas também
sem prescindir de segurança, ainda que revestida de seguridade social e
ecológica. Ambos operam no interior da cultura do castigo.

A reviravolta nas pretensões capitalistas com base no
\emph{desenvolvimento sustentável} evidencia que o alvo não é
administrar a condição de pobreza, mas elevar os indicadores econômicos,
auferir índices de felicidade e de desenvolvimento humano (saúde,
educação e cultura), disseminar uma cultura da paz, educar pessoas e
práticas para a \emph{resiliência}, instituir uma \emph{economia verde},
encontrando certa \emph{qualidade de vida} com redutores de
vulnerabilidades e gerando condições compartilhadas para uma \emph{vida
melhor} de pessoas, ares, mares, relevos, florestas, enfim, do
\emph{ambiente}. Trata"-se de um investimento na \emph{ocupação} e manejo
de inteligências, em participações, conexões, variadas identidades,
assentamento de direitos, segurança e securitizações, conservação do
planeta que requer, antes de tudo, \emph{moderação}, e meio encontrado
para tal condição encontra"-se na vida \emph{resiliente.}

\begin{enumerate}
\def\labelenumi{\arabic{enumi}.}
\setcounter{enumi}{2}
\item
  \textbf{iluministas}
\end{enumerate}

Há um componente subjetivo importante orientado pela ética e de que não
se deve prescindir. E, nesta pesquisa, a ética como estética da
existência remete às resistências diante da configuração do trabalhador
em empreendedor, ou seja, em capital humano e, por conseguinte, zeloso
pela formação de crianças e jovens, diante do propósito da
\emph{erradicação da pobreza}. Cabe ao pesquisador caminhar pela
genealogia do poder, cauteloso com o isolamento, e atento ao que este
lhe propicia no percurso metodológico.

Não se deixa de lançar mão de livros e artigos impressos, sabendo que
hoje em dia valoriza"-se a produção intelectual por meio de artigos em
publicações eletrônicas, segundo \emph{rankings} elaborados por comitês
de pesquisadores, em agências estatais ou não, a serem reconhecidos por
\emph{todos}. O ramo das publicações científicas passa por novas
exigências que configuram o atual \emph{estado da arte}, envolvendo
citações e reconhecimento pelo ranqueamento científico que situam as
relações entre a ciência e a política como campos distintos e compostos
de critérios muito bem delimitados. Os aproveitamentos das conquistas
científicas dizem respeito, portanto, às suas utilidades na política, na
economia ou na sociedade, visto não haver ciência que desconheça a
política desde a constatação de Heidegger, para quem a ciência e a
universidade passaram a ser conduzidas pela política, até sua ampliação
em institutos, fundações e até mesmo \versal{ONG}s e \emph{think tanks}. Trata"-se
menos de uma ciência que instrumentaliza governos e comissões
internacionais pelas ações dos \emph{policy makers}, mas da
institucionalização desta relação: tudo é científico e tudo é político,
com suas variadas conexões de aplicabilidades.

Segundo Foucault, a ciência política emerge como saber de Estado e,
diríamos, não só como estatística ou suporte para a biopolítica em suas
relações com a soberania ou com as disciplinas na gestão governamental,
mas por meio dos efeitos positivos de poder que não se encontram somente
no governo dos vivos, na utilidade e docilidade obtidas de cada um
produtivo ou na utopia democrática. A ciência política, hoje, vai um
pouco mais longe, define o horizonte democrático sob uma conformação
elástica que abarca território, população e culturas em dilatação e
novos campos e fluxos. Ela também se mostra \emph{resiliente},
adaptando"-se e configurando possibilidades que não estão mais sob o
regime exclusivo do constitucionalismo governamental, mas que avançam
pelos espaços de superfície e do ar, ampliando o que seria participação
com representação política no governo do Estado e colaborando com a
governança global. Não há mais, como nunca houve, segundo Foucault, a
separação Estado"-sociedade civil, nem mesmo uma formal distinção
nacional entre Estados e gestão da vida no planeta. O planeta urbanizado
e, por conseguinte, policiado, exige um pouco mais de governos,
governamentalidades que repõem a situação de \emph{ingovernável}, menos
pelo fim da política e mais por novos inícios, como se a profanação
necessária e suficiente já tivesse ocorrido (Agamben, 2007, 2009).

A ciência política encontra"-se num fluxo em que é cada vez mais difícil
acomodar as rubricas formais das pesquisas (teoria, políticas públicas,
governos, regimes políticos, relações internacionais etc.), segundo os
critérios ainda disciplinares ou ainda sob os pouco claros
discernimentos sobre interdisciplinaridades, ou seja, composições
disciplinares. O pesquisador que entra no fluxo eletrônico de
disponibilização de documentos, práticas, convocações, programas de
gestão, enfim, no caudaloso e espetacular \emph{mundo} das comunicações
eletrônicas constantes e moduláveis, por várias razões não se encontra
apartado de uma política, assim como não estiveram os que trabalharam
com resultados de experimentos e suas versões para dinamizar a economia,
prescrições médicas e de saúde, equipamentos de defesa militar e
segurança, formas de regimes, dietas e dietética, educação e
escolarização. No governo soberano do Estado, o governo dos vivos, sua
biopolítica e seus efeitos extremos de racismo de Estado, conforme
sinalizou Foucault (1999), jamais funcionaram sem ciências. Isto não
configura uma condição ideológica, mas, como bem salientou o filósofo e
historiador, deu"-se em decorrência das relações saber"-poder e, para além
destas, pela produção de verdades. Neste âmbito, não há, portanto,
distinção possível entre sociedade civil e Estado, apartada ou separada,
senão por artifícios conceituais próprios da teoria da soberania e de
sua conformação, segundo a definição descendente do poder por meio do
edifício jurídico"-político, com ênfase nos aspectos repressivos como
dever e salvaguarda diante da legitimidade.

\begin{enumerate}
\def\labelenumi{\arabic{enumi}.}
\setcounter{enumi}{3}
\item
  \textbf{a massa analfabeta}
\end{enumerate}

\emph{Como diz um amigo, tudo acabou ou está acabando. Não resta outra
{coisa} senão uma grande massa analfabeta criada deliberadamente pelo
poder, uma espécie de multidão amorfa que mergulha todos em uma
mediocridade geral. Existe um imenso mal"-entendido} (Vila"-Matas, 2011:
171).

A constatação do importante editor falido que pretende reencontrar"-se
com o espaço de James Joyce, em Dublin, e seu mundo burguês --- habitado
por escritores, suas convenções, confrarias, citações intermináveis por
dentro da literatura ---, expressa a situação do mal"-entendido e das
vaidades literárias.

Todavia, ele mesmo, o editor, um \emph{hakikomori} apanhado pelo Google,
sabe localizar o jeito de tantas palavras, informações e comunicações
entre a ``massa de analfabetos criada deliberadamente pelo poder''. Aqui
a multidão permanece informe, como o lado obscuro da multidão de
singularidades vista por Negri e Hardt (2004), a partir da informação
eletrônica e as maneiras de convocação. Abre"-se, entretanto, o espaço
móvel para compreender os ``protestos'' próprios do ano de 2011, menos
como efêmero de resistentes e mais como elasticidade das contestações
com base no que o anarquismo colocava como ``política radical'' (Newman,
2011). Absorção do anarquismo ou efeito de sua estagnação? Nova forma
democrática de protestar e acomodar institucionalidades? Seja como for,
não mais \emph{massa}.

Teriam os pesquisadores e politólogos se acomodado aos limites do
``comentário'', restritos em sua capacidade de pensar? Trata"-se de muita
compaixão? Vila"-Matas nos interpela nas páginas de seu livro (178 e
sgs.). Enfim, o editor Riba passa à carta de Flaubert para Julia Piera,
que a lê com variante própria. Estão no cemitério de Dublin, na capela,
os escritores e sua oradora fúnebre:

\emph{Tudo isso me dá náuseas. Em medicina a literatura parece uma
grande empresa de micróbios. É esse o cheiro que as pessoas têm, mais
que tudo! Estão sempre a exclamar como Sr. Policarpo: `Ah, Deus meu! Em
que século me fizeste nascer!' e de seguir tapando os ouvidos, como
fazia o santo homem quando encontrava diante de si uma proposta
indecorosa. Enfim. Chegará um tempo em que o mundo todo terá se
transformado em um homem de negócios e num imbecil (então, graças a
Deus, eu já terei morrido). Pior passarão nossos sobrinhos. As gerações
futuras serão de uma tremenda burrice e grosseria}. (Vila"-Matas, 2011:
238).

Diante de Flaubert, Samuel Beckett, também dublinense:

\emph{Pessoalmente, não tenho nada contra os cemitérios, passeio neles
com prazer, com mais prazer do que em outros lugares, talvez, quando sou
obrigado a sair. O cheiro dos cadáveres, que sinto nitidamente sob o
cheiro da relva e do humo, não me desagrada. Talvez um pouco doce
demais, um pouco estonteante, mas como é preferível ao dos vivos, das
axilas, dos pés, das bundas, dos prepúcios cerosos e dos óvulos
desapontados. (...) Por mais que eles se lavem, os vivos, por mais que
se perfumem, eles fedem} (Beckett, 2004: 2).

Diante de Flaubert, a carta de Coubert na Comuna de Paris:

\emph{As administrações anteriores que governaram a França quase
destruíram a arte ao protegê-la e ao suprimir sua espontaneidade. Essa
abordagem feudal, sustentada por um governo despótico e discricionário,
não produziu nada além de arte aristocrática e teocrática, justamente o
oposto das tendências modernas, de nossas necessidades, de nossa
filosofia, e da revelação do homem manifestando sua individualidade e
sua independência física e moral. Hoje, numa época em que a democracia
deve reger todas as coisas, seria ilógico a arte, que conduz o mundo,
ficar para trás na revolução que está ocorrendo agora na França. (...)
Não há dúvidas que o governo não deve tomar a dianteira em questões
públicas, pois não é capaz de carregar em seu interior o espírito de uma
nação; consequentemente, qualquer proteção será em si mesma prejudicial.
As academias e o Instituto, que apenas promovem a arte convencional e
banal, para que sejam julgados por seus integrantes, opõem"-se necessária
e sistematicamente a novas criações da mente humana e infligem a morte
de mártires em todos os homens inventivos e talentosos, em detrimento de
uma nação e para a glória de uma tradição e doutrina estéreis} (Courbet,
2009: 123-124).

O editor falido de Vila"-Matas vai à busca da grande joia da literatura,
Joyce"-Ulisses, refazendo este percurso, ainda que perca sua mulher para
o zen budismo, acomodada em uma nova configuração adequada aos nossos
tempos. Riba permanece \emph{pirandellianamente} como uma personagem à
procura do autor, esse outro personagem da modernidade, como analisou
Michel Foucault.

Perdido no tempo perdido \emph{proustiano}, revira memórias e afirma
comentários pessimistas sobre o presente. Eis um intelectual deste tempo
cuja perda do espaço inventado pela arte parece difícil de ser
recuperada.

Em \emph{Dublinesca}, de Enrique Vila"-Matas, busca"-se a literatura
preciosa na construção da linguagem. Ricardo Piglia, em \emph{Respiração
artificial}, delineia outra maneira do exercício da citação invadir,
pelo acaso, o efeito na massa medíocre anunciada por Kafka, como veremos
adiante. Mas também fica a questão: como ultrapassar a relação
elite"-massa, vanguarda, tiranias e revolução?

A literatura informa revirando filosofias e a história. Produz por um
efeito de linguagem a diversidade dos possíveis diante do que se
configura como filosofia e história. Esta literatura contemporânea
produzida por escritores"-professores remete o leitor para a discussão de
momentos e instantes que não podem ser apreendidos pelo discurso
acadêmico, com suas regras fixas de excelência, ao apartar o subjetivo
das objetividades e constituir um discurso rígido. Pouco importa a
citação dos literatos ou a apreensão circunstancial de um conhecimento
relacionado à literatura ou mesmo suas sugestões mais livres. Há um
intervalo estrondoso produzido pela literatura contemporânea que diz
respeito à análise propriamente dita a partir de situações e personagens
em condição extraordinária. E esta não se resume a um instante no
acontecimento histórico; mais precisamente, remete o leitor e a análise
para os baixos começos tanto da continuidade das hierarquias dos saberes
e da subordinação das práticas não"-discursivas, quanto dos encontros
inusitados com pequenas e anônimas personagens que pensam a vida e o
governo da vida.

A massa disforme aguardando um líder se fraciona em variados
agrupamentos capturados pela participação, direitos, administração do
local, vínculos exteriores e se torna esperançosa, dispõe"-se a ser
ocupada, clama por empregos, quer colaborar com a crise de sua vida e a
do capitalismo. Convence"-se, aos poucos, de que trabalho e segurança
sustentam suas especificidades culturais, e adere a todas as formas de
julgamentos, recriando e ampliando o regime dos castigos. Não mais
punição pela vigilância, mas pelos monitoramentos que ultrapassam a
vigilância delimitada a espaços de confinamentos, não mais apenas a
sanção normalizadora que deve repor a ordem articulada ao princípio da
prevenção geral, mas uma sanção aos normais para evitarem os erros,
conectada ao princípio da precaução. Não mais \emph{massa} analfabeta,
mas alfabetizada com vocabulário restrito e satisfeita.

\begin{enumerate}
\def\labelenumi{\arabic{enumi}.}
\setcounter{enumi}{4}
\item
  \textbf{a carneirada}
\end{enumerate}

O escritor luso"-angolano Valter Hugo Mãe introduz um velho \emph{silva}
português salazarista internado pela filha em um sanatório para a
``feliz idade'' após a viuvez. Adepto do salazarismo por ignorância,
medo, esperança, omissão e conformismo, almejava filhos ``com nossos
nomes portugueses e orgulhosos'', e devotos da igreja. Ainda que
constatasse que ``ser religioso é desenvolver uma mariquice no
espírito'' (Hugo Mãe, 2011: 83), ele ainda era a parte que ia ``pela
vida abaixo como carneirada, tão bem enganados'' (Ibidem: 86).

Como a massa amorfa, analfabeta e ``criada deliberadamente pelo poder'',
como situava Vila"-Matas/Riba, essa carneirada em V. H. Mãe é também cada
assujeitado, amante da obediência, que como situa o poeta Francisco
Alvim, no poema"-verso ``Obrigação'': ``a gente tem é que se acostumar''
(Alvim, 2004: 170). O ``lar da feliz idade'' leva o velho a gostar da
morte: ``é como se fossemos cortejando a confiança dessa desconhecida,
para nos encantarmos, quem sabe'' (Hugo Mãe, 2011: 102). Mas hoje é
preciso um pouco mais que asilo. Ocupa"-se o povo da \emph{feliz idade},
em um outro jeito de educar pela continuidade da vida por meio de
diversos programas institucionais para aposentados, a fim de apoderar"-se
deles depois do desgaste no tempo de uso social de seu trabalho, com
equipamentos sociais que possam dar"-lhes novas oportunidades para serem
\emph{inteligentes} em novas ocupações. São programas de efeitos
destinados a uma população de vivos (não há mais o \emph{deixar morrer}
generalizado do direito de soberania acoplado à biopolítica) formada e
utilizada na sociedade disciplinar; agora, estão redimensionados na
ocupação de energias inteligentes pela sociedade de controle com sua
materialidade computo"-informacional, conectando, nessas práticas, a
produtividade de velhos ocupados como nicho de mercado e
produtores"-consumidores.

Entretanto, há outros cuidados distintos aos jovens considerados
marginalizados, semi"-marginalizados, delinquentes, \emph{ralé} ou
\emph{novos batalhadores} (Souza, 2012) não mais tratados segundo o
princípio da carneirada, e, portanto, não mais como \emph{massa}.
Programas e organizações da sociedade civil organizada procuram dar
conta de ocupá-los, levá-los a participar, mesmo que isto requeira
informar que se deve participar de algo definido anteriormente em alguma
instância superior que foi capaz de compreendê-los. Não é mais a massa
carneirada seguindo o líder. Agora cada um deve estar consciente de que
participa do que é imediato à sua comunidade, à sua vida onde habita, às
suas disposições para o trabalho empreendedor, ao afastamento dos
desvios, ou, no melhor dos casos, aprender a conviver com os riscos
esperados. Não é mais alguém que depende do conjunto para dispor ao
líder suas vontades, posto que cabe a si mesmo realizá-las em companhia
de semelhantes. Agora, cumpre a cada umdescobrir suas vontades, zelar
pelos pares, ser pastor de cada um e de todos irmanados em torno de uma
\emph{melhoria} social em função do \emph{empreendedorismo} de cada um,
valorizar sua cultura popular e imbuir"-se de que as religiões passam a
ter papel fundamental na construção das identidades agrupadas sob o
signo do \emph{ecumenismo}. Para cada diferença normativa um direito, e
isso assemelha todos como divíduos livres que compartilham desafios,
trabalhos e participação política. Eles não caminharão para o campo de
concentração, mas pedem punições para as irmãzinhas ovelhas negras. Seu
espaço de existência elastifica"-se, passa a ser apreciado, assim como
toda a configuração \emph{resiliente} da vida.

O \emph{silva} pensava que o maior rebelde para Salazar era quem
abdicava da Igreja, como ele, depois do filho natimorto. Mas na ditadura
ou na democracia burguesa, neste vaivém do século \versal{XX}, a família esteve
no centro das coisas: \emph{eu deixava que a sociedade fosse apodrecendo
sob aquele tecido de famílias de bem, um mar imenso de família de
aparências, todas numa lavagem cerebral social que lhes punha o mundo
diante dos olhos sublinhados a lápis azul, para melhor vermos o que
melhor queriam que apreciássemos. ai as glórias de glórias de salazar,
eram tão grandes as pontes e longas as estradas, eram tão bonitas as
criancinhas a fazerem desporto e a cantarem letrinhas patrióticas.
parecíamos um grande cenário de legos, pobrezinhos mas tão lavadinhos
por dentro e por fora, a obedecer. divirtam"-se, gentes da minha terra,
não é desgraça ser pobre, punha"-se a amália a dizer, e que numa casa
portuguesa pão e vinho e um conforto pobrezinho e fartura de carinho, e
ela que ia à frança comprar vestidos onde se vestiam as estrelas de
cinema americano e se embonecava de jóias e até tinha visto o brasil e a
espanha, servia para que amássemos e fôssemos pensando que estávamos
todos tão bem ali metidos, éramos todos tão boas pessoas, tão bons
homens, realmente. e eu, de facto, ainda adoro a amália (...). a maior
voz da desgraça e do engano dos portugueses. pena não haver paraíso, já
não haver amália e ter havido e sobrar para aí tanta desgraça e engano}
(Hugo Mãe, 2011: 133-134).

Aqui se encontram passado e presente vinculando governo e as gentes, com
suas misérias, culturas, constrangimentos. A ditadura de Salazar
pretendia que cada português se sentisse seguro, amado pelo seu chefe,
orgulhoso da simplicidade lusitana e das obras erguidas pelo Estado. A
democracia contemporânea aprendeu com a ditadura e redefiniu a relação
verticalizada pela conexão Estado"-sociedade civil organizada, pela qual
seus mediadores entrecruzam ações policiais que levam adiante novas
configurações da subjetividade capazes de conviver com \emph{protestos},
\emph{marchas}, ser solidárias com movimentos antitirânicos, acolher
\emph{pacificações}, estimular acertos entre movimentos e Estado por
meio de direitos, conter situações limites de falência (Grécia) como
novo preço a ser pago para encontrar \emph{melhorias}.

Enfim, é preciso \emph{melhorar}, e isto implica governo da vida para o
futuro melhor\emph{.} Não se trata mais de uma biopolítica como governo
da vida, mas da vida para o futuro; não mais vida no \emph{presente}
somente, o presente deve estar repleto de práticas voltadas ao futuro
melhor de si e do planeta. Não se trata de um círculo vicioso, mas de um
círculo \emph{entreaberto}, inacabado em possíveis variadas conexões,
compondo correntes, fluxos, nuvens de controles, nuvens que funcionam
como novo território de um presente projetado em arquivos virtuais que
se guardam para o futuro, ou são deletados a seguir porque somente
contam as vantagens do presente.

Não se trata de um governo da população como na biopolítica, mas de
governo com cada população agrupada, móvel, \emph{resiliente},
participativa, em função de cada um, de seu agrupamento e do planeta.
Indivíduo, redimensionado em \emph{divíduo} por direitos e identidades
que produzem conexões temporárias, paradoxalmente tênues e consolidadas;
grupos abertos ou fechados funcionando internamente com alguma conexão,
mesmo que mínima, com o exterior \emph{universalizante}. É preciso viver
para fora e por dentro, do lado de fora e conectado com vários
\emph{ambientes} \emph{resilientes}, o Estado e organizações
transterritoriais: é preciso fazer parte de tecnologias sociais, ser
reconhecido e premiado, mas também saber fazer \emph{negócios sociais}
sustentáveis. Uma subjetividade \emph{resiliente} em \emph{ambientes
resilientes} deve reduzir vulnerabilidades, ampliar a qualidade de vida,
produzir riqueza sustentável para o planeta: empresariado (capital) e o
capital humano (empreendedores) produzem nova cooperação liberal,
nomeada de produção compartilhada. E todos amam ou devem amar sua
condição no manejo da \emph{erradicação da pobreza}.

Depois de Salazar, tudo em letras minúsculas como se o maiúsculo fosse
apenas do soberano. Mas há também o inverso desta escrita que abole a
maiúscula para dissolver a autoridade da disposição das frases e
períodos no papel, lugar regrado da escrita literária e científica, como
dimensionou o poeta e.e. cumings. A escrita em minúscula está para a
subordinação da maiúscula convencional, assim como o aforismo está para
o discurso filosófico. Trata"-se menos de estilo e mais de estética como
ética igualitária, como nos contos de Prévert:

\emph{Noutros tempos, os burros eram totalmente selvagens, quer dizer,
eles comiam quando tinham fome, bebiam quando estavam com sede, e
corriam no mato quando lhes dava na telha.}

\emph{Às vezes, um leão vinha e devorava um burro, então todos os outros
burros fugiam zurrando feito burros, mas no dia seguinte eles já não
pensavam mais nisso e recomeçavam a zurrar, a beber, a comer, a correr,
a dormir... Em resumo, tirando os dias em que o leão vinha, tudo corria
bastante bem.}

\emph{Um dia, os reis da criação (que é como os homens gostam de se
chamar entre si) chegaram ao país dos burros, e os burros, muito
contentes de ver gente nova, foram galopando ao encontro dos homens.}

\emph{Os burros (falam galopando): ``Que gozados, esses bichos
branquelos, eles andam em duas patas, as orelhas são pequenininhas, eles
não são bonitos, mas devemos de qualquer forma fazer uma pequena
recepção para eles... é o mínimo...''}

\emph{E os burros fazem graça, rolam no capim agitando as patas, eles
cantam a canção dos burros, e depois, só para dar risada, empurram os
homens, fazendo"-os cair de leve no chão; mas o homem não gosta muito de
graça, quando não é ele quem faz a graça, e os reis da criação estão no
país dos burros há menos de cinco minutos e todos os burros já estão
amarrados que nem linguiça.}

\emph{Todos, menos o mais jovem, o mais tenro, que foi morto e assado no
espeto, rodeado por homens de faca na mão. Quando o burro está no ponto,
os homens começam a comer e fazem uma careta de mau humor e atiram as
facas no chão.}

\emph{Um dos homens (falando sozinho): ``Isso não dá um bom bife, não dá
um bom bife!''}

\emph{Um outro: ``Não está bom, eu gosto mais de carneiro!''}

\emph{Um outro: ``Ai que ruim (chorando).''}

\emph{E os burros presos, vendo o homem se lamentar, pensam que a causa
da lágrima é o remorso. }

\emph{Vão nos deixar ir embora, pensam os burros, mas os homens se
levantam e falam todos juntos, gesticulando muito.}

\emph{Coro de homens:}

\emph{``Esses animais não são bons para comer, têm um grito
desagradável, umas orelhonas ridículas, com certeza são estúpidos e não
sabem nem ler nem contar, vamos chamá-los de burros, segundo o nosso
bel"-prazer, e eles carregarão os pacotes para a gente.}

\emph{Os reis somos nós, avante!''}

\emph{E assim os homens levaram os burros} (Prévert, (2007: 44-45)
2008)\emph{.}

Depois de Salazar, o \emph{silva} encontra"-se no asilo para a ``feliz
idade''. Nas palavras do outro silva, o da Europa, ``a paz está toda
metida na ignorância, pronta para levar as pessoas à felicidade'' (Hugo
Mãe, 2011: 154). Emana o ponto crucial de Kant --- o efeito da
menoridade reproduzido pela tirania --- e assumem o primeiro plano as
práticas de assujeitamentos derivadas deste regime político, instaurando
o que era de desejo de cada português, ou de cada fascista entregando a
um condutor suas fraquezas e buscas desesperadas para conservação de sua
condição sócio"-econômica. ``Vestígios do fascismo cotidiano, em cada um
e para todos, tendem a se atualizar em modulações diversas, e que não
exigem, pelo menos por enquanto, uma totalização personalista; tampouco
as condutas fascistas, tal como as conhecemos no passado, tendem a se
perpetuar, pois elas sucumbem aos mecanismos democráticos para se
acomodarem e renovarem em fluxos ininterruptos. Os direitos de minorias
imobilizam ou impulsionam forças, enquanto o fascismo exige que se abra
mão de direitos em favor do condutor apoiado em mobilização de massa.
Fascismo e direito se excluem, mesmo depois da argumentação
constitucional de Carl Schmitt para o exercício do ditador. Mas direitos
e não"-direitos se completam nas democracias e nos campos de concentração
da primeira metade do século \versal{XX}: lá houve uma ação de direito que dizia
quem podia viver e quem devia ser recolhido, protegido, separado, e
talvez morto ou até mesmo servir de experimentações científicas. Eram os
loucos, os anarquistas, os deficientes, certas etnias, os judeus, os
criminosos, os subversivos, enfim, os identificados como perigosos ou a
serem protegidos, devido a situações de guerra, como, por exemplo, os
japoneses nos Estados Unidos durante a \versal{II} Guerra Mundial ou nos campos
de refugiados a qualquer momento. Na mesma medida, hoje o
multiculturalismo orienta e identifica quem é assimilável pelos
direitos, quem comporá as elites secundárias e quem dele escapa no
governo da verdade. E quem são eles? Pobres e miseráveis? Não, mas o
exército político de reserva composto pelos perdedores radicais. Estes
pertencem a todas as classes sociais: são pais, alunos, egressos,
drogados, contingente do tráfico, mascates eleitorais, terroristas,
mercenários... matam no anonimato, tornam"-se visíveis por instantes com
ou sem direitos. São terroristas que não aceitam a diferença e combatem
pessoas, ideias, associações e Estados democráticos que se afirmam como
tal com base na diferença pluralista. Todavia, o perdedor radical e o
Estado operam por identidades e como tal buscam transformar o outro em o
mesmo de si, senão matá-lo. É impossível para eles pensar pelo outro com
o outro, dissolver o indivíduo no grupo. São como os intelectuais
acomodados que só pensam pelas evidências --- no interior do discurso
empírico, estatístico e quantitativo --- ao elaborarem a crítica
dirigida e reformuladora para a continuidade das propostas de
governamentalização e refugiarem"-se em velhas teorias anunciadoras de
profecias'' (Passetti, 2010: 279; 291-292).

Assim, resta a interpretação de Kant redimensionada no final da vida,
também, por Michel Foucault (1999) que nos leva a recusar e praticar os
efeitos de resistência à \emph{maioridade}, talvez pelo viés de Deleuze
(2010) acerca das minorias potentes em oposição a qualquer maioria
numérica, porque estas tendem a acomodar posicionamentos
\emph{moderados} mesmo diante dos efeitos do \emph{exército de reserva
de poder} (Foucault, 1977) redimensionado e também composto de
\emph{perdedores radicais} (Enzensberger, 2007): \emph{e isto era a
receita do regime}, continua o silva da Europa, \emph{igualzinho. hoje
podemos ver mas não há quem queira ver. temos um povo que compra o
jornal para ler as futilidades, e compra mais ainda as revistas de
alcoviteirice, e nem sequer entenderia notícias diferentes. isso não
mudou tanto assim, caros amigos, apenas a falta de vergonha, que
antigamente havia vergonha, e agora devem estar a tirá-la dos
dicionários. (...) anda tudo assim um bocado defasado do que é e parece
ser. olhe, hoje é possível reviver o fascismo, quer saber. é possível na
perfeição. basta ser"-se trabalhador dependente. é suficiente para
perceber o que é comer e calar, e por vezes nem comer, só calar} (Hugo
Mãe, 2011: 154-155).

``Na sociedade de controle de comunicação contínua e modificável por
convenções, combina"-se a herança disciplinar das estruturas hierárquicas
com a descentralização por meio de variadas agendas. Ao mesmo tempo, o
mundo governando por elites e vanguardas, mostra"-se permeável a novas
formas da governamentalização em que aparecem derivados da proliferação
de direitos, normativas, diretrizes e principalmente programas
confiáveis, tolerantes e seguros, compondo fluxos elitistas sustentados
pelos agrupamentos diferenciados por direitos de terceira geração, que
dão a cada minoria numérica, acesso a trânsitos pelas hierarquias, por
intermédio de gerenciamentos compartilhados descentralizados.
Configuram"-se, assim, as \emph{elites secundárias}, compostas por
mulheres, gays, pretos, deficientes físicos, jovens, lideranças
indígenas, enfim, a população organizada em elites minoritárias
dirigidas por comandos capturados, e muitas vezes cooptados pelos
dispositivos de descentralidades na produção do produto e da cultura,
judicialização, escolarização, habitação... São minorias numéricas que
atravessam as maiorias parlamentares, empresariais, sindicais,
configurando um novo desenho das burocracias, acomodando a continuidade
de uma \emph{maioria sequenciada} e governada por dentro, em função da
centralidade do comando, no exercício de controles avaliativos. Na
sociedade de controles em fluxos, acontece, simultaneamente, a ampliação
da participação e do acesso à informação e à comunicação, ao mesmo tempo
em que, cada um, deve transitar pelas suas minorias de acolhimento,
defender seus direitos, praticar ações comunitárias em função da redução
de anomias, irregularidades e disfunções no interior de suas
comunidades, para elevá-las à condição de espaço desejado e amado, e não
mais de periferias abandonadas ou reduzidas a um ou outro programa
social de Estado'' (Passetti, 2011: 48).

Os portugueses, como também tantos latino"-americanos, são da fábrica de
espanhóis, espanhóis que fugiram de Franco e caíram em Salazar,
latino"-americanos que fugiram de ditadores e morreram ou escaparam para
serem presos em outras tiranias em nome da democracia; umas ditaduras
que foram nomeadas militares, mas que nunca deixaram de ser civis, e
foram apoiadas e sustentadas por massas amorfas, mas vibrantes.

No meio dessas massas é possível um salazarista tornar"-se amigo de um
esquerdista por longos anos, mas quando acuado pela polícia, ele é capaz
de entregar esse ``amigo'', em favor do medo e da esperança, como
\emph{silva} o fez: como um pequeno Salazar que não foge de sua
menoridade. Novamente a literatura é incisiva no diferencial que situa a
distinção entre a transcendental amizade e a ética dos amigos, naquele
instante preciso em que é possível distinguir o bajulador do amigo,
segundo Plutarco, os efeitos de semelhança, apontados por Montaigne, as
longas reflexões de Nietzsche e as de Foucault, atravessadas por Sêneca,
por fora da herança filosófica traçada desde Aristóteles em \emph{Ética
Nicômaco} (Passetti, 2003; Stern"-Gillet, 1995).

Ali onde habita a amizade transcendental, a vitória da força sustenta o
direito, institui a guerra justa. A física da análise serial
proudhoniana estabelece que todo fundamento do direito é divino
(Proudhon, 2011; Resende e Passetti, 1986; Rodrigues, 2010). Situa o
direito por meio desta proveniência fundadora da soberania
\emph{aristói} e que por si só define quem são os demais sob os quais
recaem os efeitos de soberania, como obediência e como multiplicidade de
dominações praticadas pelos súditos e/ou cidadãos. É a força que
institui o direito. Força física explicitada pela guerra, força da
astúcia que define o domínio, força de direito que governa. Desta
maneira, a paz é sempre metafísica. Da comunidade ao regime da
propriedade, o direito transita entre fracos (exploração do forte devido
às intempéries da natureza) e fortes (exploração e dominação do fraco
pela força e astúcia).

O divino e a Ideia se equivalem como categorias do entendimento na
relação serial entre guerra (reivindicação de paz) e paz (demonstração e
confirmação da guerra). A contradição entre as séries não encontra
síntese, pois à paz não corresponde uma ação própria, é preparação para
nova guerra. Enquanto perdura a paz acontecem as \emph{pequenas
guerras}, que se em Foucault expressam os efeitos da dominação entre os
assujeitados na modernidade da sociedade disciplinar, em Proudhon
preparam a nova guerra (ou revolução) e se perpetuam como discordâncias
e enfrentamentos mesmo na Anarquia, uma possível situação em que a paz é
possível como ação própria, em estágio adiantado, mas que ainda não
encontrará descanso. A agonística do poder não encontra solução
(Foucault, 1995), mas pode produzir redutores de possibilidades de
guerra (Rodrigues, 2010), outros embates de violência após a
ultrapassagem da guerra tradicional (Gros, 2009) e outras tensões que
afastem o perigo da guerra pela possibilidade da ação própria da paz na
Anarquia (Proudhon in Resende \& Passetti, 1986).

A possibilidade para Proudhon decorre da superação do direito livre de
transcendentalidade, segundo o objeto da troca, que ele chama de
contrato sinalagmático e comutativo, estabelecido entre dois ou mais de
dois e relacionado ao mutualismo econômico e ao federalismo político. O
direito relacionado ao objeto encontra ressonâncias em Max Stirner,
ainda que este evite aderir à \emph{nova sociedade}, a da Anarquia, por
considerá-la, também outro efeito da utopia. Em Stirner (2004) a
sociedade é algo em \emph{morte anunciada}, acontecendo. Então, as novas
relações não devem pautar"-se em termos de sociedade e formas de
organização, mas derivam da revolta: demolir a sociedade. Todavia, em
Proudhon como em Stirner, a crítica ao direito pelas categorias do
entendimento (o divino e a Ideia) ultrapassa o universal e situa outro
campo, ou melhor, fluxo, no qual o direito deixa de ser força e astúcia,
renovadas e restauradoras da soberania. Trata"-se da formação de uma
ética voltada ao indivíduo como outro início da política (Foucault,
2004), distanciada da revolução, como novo apogeu da \emph{maioridade},
de um novo direito antissoberania, como sublinhou Foucault, que não se
institui \emph{depois}, mas \emph{neste} instante. Neste instante em que
há um \emph{ingovernável} que não cessa ou que não se apressa em novas
relações de soberania, como supõe Agamben (2009).

Deixa"-se de estar no embate entre \emph{amigos} e \emph{inimigos},
expressão da divergência com rompimentos que explicitam os que se
aproximam e os que se afastam, segundo os acontecimentos atravessados
pela diplomacia e exercícios dos exércitos realizados pelos Estados
(Foucault, 2008), seja destruindo, imobilizando ou subordinando os
inimigos (Estados e povos) pela ameaça da força. Até mesmo no
desdobramento de Westfalia, na Convenção de Viena de 1815, em que estava
em jogo o \emph{equilíbrio} (contraforças) \emph{de poder}
\emph{internacional}, reiterava"-se o direito como eficácia, com a
diplomacia funcionando como razão da força e da guerra. Então, a
\emph{amizade} entre Estados se definia e define pela identidade
circunscrita, pela submissão do outro abdicando da guerra, pela ameaça
da guerra iminente, pelo \emph{equilíbrio internacional} ou pelo simples
reconhecimento da superioridade do outro; o \emph{inimigo} é o que
questiona, inquire e desafia (pode estar dentro e provocar insurreições
ou revoluções; pode vir de fora rompendo o arranjo pelo
\emph{equilíbrio}).

\emph{Amizade} e \emph{inimizade}, portanto, dependem de um \emph{justo}
colocado de antemão na relação de forças relacionado ao \emph{amor} aos
homens, à nação, à pátria, ao Estado, mas também ao condutor do Estado.
Se a força é a condição para a existência do direito, fundado no divino
ou na Ideia, a guerra explicita o desacato à força e à fé (incluindo a
racionalidade moderna formatada juridicamente) que \emph{legitima}
qualquer guerra \emph{santa} e \emph{justa}. Entre o final do século \versal{XX}
e o início do século \versal{XXI} procura"-se, por meio de agendamentos
internacionais, superar a relação amizade/inimizade pela diplomacia
multilateral. E isto repercute internamente por meio das
\emph{pacificações}, do amor à ocupação como capital humano, dos efeitos
de protestos \emph{efêmeros}.

A vitória na guerra produz direito, porque nela há dignidade: a guerra é
sempre \emph{julgamento} em nome da vontade da força. Enfim, as relações
entre Estados, sob os intervalos de paz e guerra, fundam"-se no
\emph{amor} (Stirner, 2002) e o amor antecede e dimensiona a amizade.
Assim, seja pelo Édito de Caracala --- pelo qual Roma levava a
\emph{isonomia} a todos os povos que acatassem sua soberania (benefício
que exigia o suicídio dos demais povos diante da ameaça) ---, seja pela
\emph{Declaração Universal dos Direitos Humanos} --- pela qual se
ajustam excessos e ameaças à humanidade e se espera a partir daí uma
possível \emph{paz perpétua} (ainda que o estágio de guerra tradicional
tenha cedido lugar a \emph{estados de violência}) ---, importa que o
direito permaneça \emph{santo}, e é por isso que o direito se diferencia
das paixões humanas. Desde Aristóteles, a guerra mais natural é aquela
contra animais e homens que a estes se assemelham.

A guerra permaneceu na história da Europa ampliada aos demais
continentes como julgamento diferenciado dos tribunais. Agora, pelos
desdobramentos dos tribunais internacionais, ajustam"-se os excessos
próprios de ditadores. As pontas extremas dos fascismos pretendem"-se
aparadas por condutas \emph{resilientes}. Sua proveniência principal
está nos tribunais de Nuremberg e Tóquio (Passetti, 2011), que
contemporizaram com os fascismos de espanhóis, portugueses e demais
Estados europeus que \emph{fabricavam espanhóis} depois da \versal{II} Guerra
Mundial. Mas, após Berlim"--1989, abriu"-se uma nova era para a irmandade
universal (a amizade transcendental traduzida em fraternidade) depois de
encerrado o ciclo das guerras tradicionais, das ameaças socialistas, da
subordinação da ditadura do proletariado, ainda que taticamente, à
democracia, da ameaça de guerra pacificada em \emph{missões de paz},
organização do comércio internacional, controle da energia atômica,
proximidades com o desenvolvimento sustentável: as novas amizades são
protocolares por meio de programas transterritoriais no cotidiano das
redes sociais digitais.

Onde há amor e amizade transcendental não há \emph{amigos}, as
possibilidades de direito sinalagmático e comutativo, dos \emph{meus
direitos}, como assinalou Max Stirner, ou do direito antissoberania,
ainda que a guerra tenha se metamorfoseado em \emph{estados de
violência}. Ali prevalece a trapaça e a traição, enquanto a ética dos
amigos supõe o reconhecimento das diferenças na igualdade. Sua base é a
\emph{indisciplina}, a recusa à obediência ou aos desdobramentos da
\emph{servidão voluntária} em \emph{servidão maquímica}, afirmação de
uma ética que começa, como a política, em cada um, como um \emph{abrigo
precário} (Passetti, 2003).

A situação da diferença relacionada à semelhança aproxima o salazarista
e o comunista (a qual o livro de Mãe situa com clareza) e, ao mesmo
tempo, os opõe diante de um \emph{interesse maior}, a preservação da
vida sob a ameaça da permanência do direito fascista do soberano em
causar \emph{quem deve viver} e que é maior do que os desfrutes
diferenciados na amizade cotidiana diante da \emph{pequena guerra},
também travada diariamente. A amizade se estende enquanto permanecem
intocáveis as diferenças políticas (não há vida como direito cotidiano
diante do direito de soberania como sustentou Carl Schmitt; não é
possível estabilizar a máxima \emph{política e religião não se
discutem}; não há cotidiano anestesiado pelas relações afetivas ou
sensações, mas sobredeterminado).

Política e religião (a Ideia e o divino, o Estado como ideia reguladora
do entendimento e do funcionamento das instituições, como ideia"-fixa)
podem não ser discutidas ao supor contenção da polêmica (situação em que
os oponentes procuram manter suas doutrinas), em benefício de uma
suposta dimensão autônoma da afetividade. Todavia, há um limite situado
pela força e o amor ao exigirem obediência e fidelidade (de silva com o
salazarismo) e a infidelidade (de silva em omitir a delação do ``amigo''
até mesmo de sua fiel esposa --- de quem dizia não esconder nada ---, ou
dos frequentadores da barbearia). Os pequenos fascismos pelas suas
práticas fazem com que cada um se sinta chamado a permanecer como uma
vara que compõe o lictor comandado pelo condutor (seja ele homem ou o
procedimento na participação).

Na democracia que convoca à participação, o \emph{sentimento} entre os
seus muitos adeptos, principalmente entre os componentes dos programas,
é o de esquecer, em função da proximidade e da afetividade, as surpresas
dos \emph{inimigos}. Estes efeitos foram também vividos por diversos
integrantes de \versal{ONG}s durante os primeiros investimentos nas favelas
brasileiras diante do controle do tráfico e não raramente custaram"-lhes
uma morte \emph{digna}. A combinação de práticas compartilhadas
(diplomáticas e policiais) proporcionam o assentimento em direção a uma
política de \emph{frente}. De modo análogo às ultrapassagens das
ditaduras latinoamericanas que levaram à reposição de forças liberais
diante das demais, como no Brasil da \emph{abertura política}, nos
recentes efeitos da Praça Tahir, quando a tirania foi substituída por
uma junta militar composta dos mesmos homens que sustentavam o regime de
exceção, ou nas \emph{comunidades} dos morros cariocas, todos apostam em
democracia e em novas \emph{amizades}. Entretanto, também na oposição
entre anarquistas coletivistas e de \emph{estilo de vida}, como a
elaborada por Murray Boockchin e assimilada majoritariamente entre
anarquistas (Augusto, 2011-2012), transparecem os mesmos equívocos
éticos fundados na concepção transcendental de amizade.

Por diversos ângulos, a massa amorfa/carneirada, em seus fins claros ou
escuros, submissos ou revolucionários, torna"-se a \emph{carneirada filha
da puta}, licença literária que explicita o que o conceito por vezes
camufla com a linguagem polida e objetiva. Disputa"-se o poder soberano e
se negligenciam os pequenos fascismos, sob a maioridade e mesmo quando
os trabalhadores se metamorfoseiam em \emph{empreendedores}. Novos
desafios se colocam na construção de um discurso sustentável,
democrático, estável e compartilhado que se redesenha em variados
encontros internacionais e recolocaram suas bases na Rio +20, em 2012.

\begin{enumerate}
\def\labelenumi{\arabic{enumi}.}
\setcounter{enumi}{5}
\item
  \textbf{ideias e práticas }
\end{enumerate}

Na carneirada e entre os seguidores há sempre o traidor, o \emph{silva}:
``\emph{o traidor ocupa a posição clássica do herói utópico: homem de
lugar nenhum, o traidor vive entre duas lealdades; vive no duplo
sentido, no disfarce} (Piglia, 2010: 67). A busca da utopia permite
todos os crimes, pensa o personagem de Piglia no exílio, em busca do
ouro da Califórnia, no século \versal{XIX}: \emph{só poderão chegar ao reino
suave e feliz da pura utopia aqueles que (como eu) sonham se arrastar
pela maior degradação. Só na mente de traidores e infames, dos homens
como eu, podem surgir os belos sonhos que chamamos de utopia} (Ibidem).
A conquista do ouro, da riqueza, nos extremos da pirâmide social onde se
instala a utopia do ``para sempre'', onde o capitalismo melhor se
realiza como utopia, e na qual alojam"-se traidores também transitando
pela estratificação social.

A vida capitalista também oferece com o liberalismo outras escalas
possíveis na ascensão e a principal entre as inferiores é para cada
pobre vir a ser um classe média, deixar de ser \emph{ralé} e se tornar
um \emph{batalhador}. É deste estrato que o capitalismo mais precisa
para se impulsionar, favorecer e fortalecer por meio do liberalismo
econômico, cujo complemento é a democracia, para que em momentos de
crise a revolta se configure com reivindicação por empregos, fim da
corrupção, \emph{novas} possibilidades de cooperação, ainda que
estratégicas no interior de uma luta \emph{mais ampla} (Harvey, Telles
et al., 2012).

O liberalismo proporciona o acesso à propriedade e aos benefícios e
instaura o dilema do próprio capitalismo: ele precisa de impostos para
gerir benefícios aos pobres, visando reduzir as convulsões entre os
estratos. Ele é obrigado a viver com a reação burguesa quanto à taxação
de impostos e esta se transforma em plataforma de defesa da classe
média, que também se expressa avessa a pagar impostos, seja por sonhar
(\emph{sonho de paz}) em um dia ser burguesia e usufruir dos privilégios
de sua própria riqueza, seja pela ameaça do \emph{sonho da guerra} dos
inferiores, seja somente para se manter (sem ter de realizar seu
\emph{sonho fascista}). Pressionada, em tempos de crise, essa classe
média ajuda a impulsionar os distúrbios de matizes diversos como forma
de contenção de outras radicalidades, possíveis e viáveis, expressas
pela organização ou pela associação de pobres e trabalhadores. Porém,
como nos tempos atuais de neoliberalismo --- quando a desigualdade é
vista como positiva porque estimula a reprodução do capital humano ---,
estas organizações e associações são minadas, aninhadas, imobilizadas ou
simplesmente se encontram estagnadas por incapacidade de seus membros em
atualizar seus discursos; esperam, novamente, a chamada intervenção do
Estado, para além das práticas já consagradas e recicladas
institucionalmente de controle da educação e da saúde com o retorno à
intervenção econômica, ou mais precisamente, nos termos atuais, de mais
\emph{regulamentações econômicas}.

A regulação da população não se restringe mais à biopolítica, com ou sem
intervenção de cunho social"-democrata, mas precisa se conectar a uma
nova produção da verdade sobre o capitalismo e o \emph{ambiente}, de
trabalhador como empreendedor, da democracia com gestão do planeta, com
sentimentos e afeições. Emergem o desenvolvimento sustentável, as
\emph{tecnologias sociai}s, os \emph{negócios sociais}, as cidades
sustentáveis e criativas, os controles sobre emissão de carbono, a
política e a ciência sobre o clima, a vida dos refugiados (políticos e
climáticos), todas as possíveis seguranças, enfim, novos direitos com
base na força, na Ideia, no divino, na cultura democrática,
participativa e normalizadora. Não há mais carneirada, mas indivíduos
divizíveis transitando pelos fluxos em busca de sustentações.

Uma primeira conclusão, talvez mais fácil, restrita aos ``comentários''
poderia, de um lado, reiterar a conformação da ``carneirada'', e de
outro, estimular novas formas populares de integração ao capitalismo.
Diante do ``comentário'' repaginado sobre a carneirada, próprio do
discurso de esquerda radical, encontra"-se a possibilidade de
estabelecer, cientificamente, novas modulações da biopolítica, e
principalmente situar, mesmo tenuemente, o perfil ecopolítico. Neste
sentido, a massa está novamente em questão no interior do discurso
liberal e consequentemente no da esquerda partidária recoberta pelas
práticas de cidadania. Estamos diante da institucionalização do
capitalismo sustentável que se volta para o gerenciamento do planeta e
que exige práticas de \emph{resiliência}.

Tomemos a discussão que Piglia trata em seu livro sobre frenologia e
teoria da relatividade. A frenologia (freno: controle; logia: sociedade
secreta, lógica"-conhecimento) estabelece a distinção segundo o tamanho
do crânio: ``\emph{a maldade sempre obedece a uma estrutura
geométrica}''\emph{,} compondo o ``\emph{enlace secreto entre geometria
(círculo) e a moral (vicioso)}'' (Piglia, 2010: 102-103), o que explica
a noção de círculo vicioso, onde estão aninhados criminosos e depravados
(a frenologia, também, aparece hoje reciclada nas neurociências,
principalmente pela via da chamada \emph{psiquiatria do desenvolvimento}
voltada para crianças e jovens). A teoria da relatividade depende da
presença do observador, portanto ``\emph{teoria da ação relativa.
Relativa, de {relatar}: narrar. O que narra, o narrador. {Narrador} quer
dizer: aquele que fala}'' (Ibidem, grifo do autor). O relativismo
carrega a variedade de narradores, ampliando as diversas maneiras de
falar a mesma coisa, de disputar a verdade entre verdades científicas;
permite interpor ações governamentais e não"-governamentais diante de
circunstâncias de arranjos institucionais. O que se relata em bases
científicas humanistas põe sempre a ignorância de quem não se
``percebe'' como diferente capaz de ser capturado; doravante, o
relativismo propicia renovações de assujeitamentos pela inclusão e, ao
mesmo tempo, renova o conhecimento científico com foco no indivíduo e/ou
no planeta. A frenologia no sentido literário de seu aparecimento pode
estar operando em termos de comprovação empírica, entretanto, os
sujeitos de seu discurso recolocam freios e controles ampliados: não há
mais um âmbito de ação estatal, mas democratização ampliada como
controle de indivíduos entre si por monitoramento de ambientes; e isso
ultrapassa os sujeitos de controle social descritos pela escola de
Chicago no século passado.

Configura"-se, assim, o protesto como nova faceta derivada dos
``comentários'', reconfigurados pelo exercício de cidadania, efeitos de
custos econômicos, desarticulação de organizações e associações,
sinalizando para o elenco de problemas a que a programação procedimental
de Estado e sociedade civil deverá voltar os olhos: desemprego, clima,
áreas de preservação, efeitos de poluentes, secas e enchentes,
contingentes de refugiados, populações tradicionais e ribeirinhas,
reflorestamentos, enfim, demarcações que levam ao principal elemento
articulador do discurso democrático e sustentável: a segurança.

Segurança desses muitos bens, da sua vida, comunidade, emprego, saúde,
habitação, educação dos filhos, maneiras de governar a família com
segurança: policiar, gerenciar, formar, acumular, beneficiar. Enfim, os
novos empreendimentos ecológicos exigem segurança no \emph{ambiente}.
Refluxo das guerras, novas maneiras de governar (Campbell, 2011;
Oliveira, 2011; Rodrigues, 2011; Optiz, 2012).

Estamos diante de uma nova configuração em que o modelo de intelectual
europeu do século \versal{XIX} supostamente se transfere para o estadunidense.
Todavia, tanto o europeu como o estadunidense são modelos de
antepassados, ou do que a filosofia maturou ao longo do tempo desde
Platão e Aristóteles, enfim, narradores voltados à relativização cujos
limites foram demarcados por Kant e Hegel e suas atualizações na
tipologia de Max Weber e na transformação histórica por Marx. Ainda
assim, premente era a condição de existência, direito, força e guerra
pelo Estado, e neste interstício foi o liberalismo que se fortaleceu de
Locke a Stuart Mill e as variadas conformações da democracia política
combinadas com participação, como nas propostas conservadoras de
Tocqueville, nas atenções com cidadania de T. S. Marshall, nos conflitos
entre elites por Wirght Mills, pelo apogeu do kantismo com Max Weber e a
crítica ao Estado e às massas.

O dilema dos discursos políticos entre o individual e o social não
encontrou o meio"-termo pelas exigências dos modelos. A questão do
meio"-termo é o que os ``comentários'' proporcionam na construção de uma
teoria democrática do capitalismo, ainda que isso seja um paradoxo. A
analítica genealógica insere"-se aqui. Não exige o balanço a respeito da
produção científica em humanidades, segundo suas repercussões. Ao
contrário, toma essa exigência como evidência da proliferação de
narradores. Opõe perspectivismo a relativismo, situando o espaço do
ponto de vista no qual se dilui o diálogo (supressão do ser consciente e
conhecedor) pela conversação (ponto de vista como resistência,
aproximando sua simples sinonímia com linha de fuga, no sentido
deleuziano, por expressar também a possibilidade de uma linha de fuga
fascista).

A articulação dos fluxos meio ambiente, direitos, segurança e
penalização a céu aberto traça os movimentos do discurso e das práticas
internacionais e comunitárias, enquanto efeito de recomendações que
encontram nos Estados legislação próprias.

``\emph{Em vez de ser respeitoso, fui me arrastando cada vez mais para a
franqueza, delito imperdoável entre acadêmicos}'' (Piglia, 2010: 156).
``Foucault revirou \emph{as luzes} para afirmar a urgência de nos
voltarmos contra o que somos. Reparou para além da isonomia (direitos de
todos os cidadãos perante a lei) e da isegoria (o direito legal de cada
um pronunciar sua opinião) gregas, a atitude do parresiasta, aquele que
pronuncia uma verdade e para quem não há proteção institucional para a
vingança sobre quem a proferiu. Se não era nada fácil pronunciar uma
verdade, sabendo dos riscos diante de um superior, também não foi
difícil à democracia facilitar a acomodação do parresiasta em demagogo.
Procede daí o fale por mim do rebanho e ao mesmo tempo o mundo das
opiniões conduzidas por um pastor'' (Passetti, 2011: 129).

Tardewiski, filósofo polonês, quando eclodiu a guerra não se encontrava
na Inglaterra, mas em sua terra natal, e redigia sua tese com
Wittgenstein sobre Heidegger e os pré-socráticos. Fugiu para o
desconhecido num navio que imaginava seguir para os Estados Unidos e foi
dar na Argentina. Frequentou brevemente o círculo de intelectuais
portenhos e, inviabilizado por este, transformou"-se num professor
secundário longe de Buenos Aires. Quando redigia sua tese,
acidentalmente leu \emph{Mein Kampf}, de Adolph Hitler (havia solicitado
ao bibliotecários o volume de Hippias) e por meio de citações, análises
históricas localizadas, e considerando o suposto encontro de Hitler, ou
do jovem Adolph, com Kafka, em Praga, em 1910, elaborou uma analítica
que traduziu em breve artigo para o jornal (\emph{La Prensa}), graças à
influência da embaixada da Polônia, intitulado ``A encruzilhada
Hiltler"-Kafka; uma hipótese de investigação'', publicado como o nome de
Wladimir Tard\emph{o}wiski, na seção cultural. Ele o redigiu em inglês e
posteriormente encontrou uma tradutora para o espanhol. Restou"-lhe do
artigo, enfim, uma versão em espanhol, pois naquela madrugada, quando
saíra para comprar o jornal, seu quarto de hotel foi furtado, nada lhe
restando mais que a roupa do corpo e o jornal.

O livro de Piglia encerra com a conversa de Tardewiski com Emílio Remi,
sobrinho do professor de história que está sendo aguardado, Marcelo,
cuja grande tarefa tinha sido a de se dedicar a escrever uma biografia
completa de um exilado que vai em busca da utopia e passa pela
Califórnia, sem jamais retornar à Argentina.

\begin{enumerate}
\def\labelenumi{\arabic{enumi}.}
\setcounter{enumi}{6}
\item
  \textbf{O rebanho}
\end{enumerate}

Se a carneirada é uma expressão portuguesa salazarista, se Ortega y
Gasset escreveu sobre o homem médio na Espanha de Franco, e se Portugal
é a máquina de fabricar espanhóis (e latino"-americanos), o rebanho é a
noção própria da Alemanha: designação de Nietzsche para o que se
avizinhava com o homem da modernidade, os obedientes educados segundo
Stirner, ou até mesmo a sombra de Bismark sublinhada por Max Weber.

Tardewiski vivia sob a insatisfação de Wittgenstein para quem sua
própria filosofia, ``\emph{tal como Husserl dissera que a psicanálise
devia ser vista como uma enfermidade que se empreende com a própria
cura}'' (Piglia, 2010: 148), envereda para \emph{Investigações
filosóficas}, livro inacabado cuja frase final, muitas vezes citada é:
\emph{sobre aquilo que não se pode falar, é preciso calar}. Ou, como
pronuncia Agamben, o que se situa entre o \emph{não mais} e o
\emph{ainda não}?

Não. A resposta é kafkiana. \emph{Ele sabe ouvir, ele é aquele que sabe
ouvir}. É assim que Tardewiski compreende Kafka. Ler \emph{Mein Kampf} a
partir da leitura anotada de Joachim Klinge\footnote{Um hakikimori, como
  Riba, recorre ao Google --- ou neste caso Riba não correria? ---e só
  encontra um homônimo no \emph{Myspace}.}, amigo de Walter Benjamin.
Compreende \emph{Mein Kampf} como o complemento do \emph{Discurso sobre
o método} de Descartes. Em ambos a ``\emph{dúvida não existe, não pode
existir, e que a dúvida não passa de sinal de fraqueza de um pensamento,
que não é a condição necessária de seu rigor}'' (Ibidem: 172). Estes
monólogos de ``\emph{um indivíduo mais ou menos alucinado}'' (Ibidem:
173) supõem haver um lugar de onde se edifica um sistema coerente
filosoficamente imbatível, como sublinha Tardewiski, monólogo no qual é
narrada a história de uma ideia, maneira pela qual Valéry considera o
\emph{Discurso} de Descartes como o primeiro romance moderno: ``\emph{o
sonho {dessa} razão produz monstros}'' (Ibidem: 176, grifos do autor).
Heidegger leu \emph{Mein Kampf} e aí começou a pensar, encerra
Tardewiski. Compõe"-se, então, um triângulo equilátero monstruoso:
Hitler, Descartes e Heidegger.

Entra Kafka, em seus momentos em 1910, no café Arcos, com a veracidade
do encontro com o jovem Adolph {[}Hitler{]}, sua capacidade de ouvir e
de pressentir a \emph{Colônia penal} como uma coincidência: a palavra
\emph{ungeziefer} (animal nocivo, não doméstico, que não presta para o
sacrifício\footnote{``O adjetivo \textbf{ungeheuer} (que significa
  monstruoso e, como substantivo --- das \textbf{Ungeheuer} ---
  significa `monstro'), quer dizer, etimologicamente, "aquilo que não é
  mais familiar, aquilo que está fora da família, \textbf{infamiliaris"}
  e se opõe a \textbf{geheuer,} isto é, aquilo que é manso, amistoso,
  conhecido, familiar. Por sua vez, o substantivo \textbf{Ungeziefer}
  (inseto), ao qual \textbf{ungeheuer} se liga, tem o sentido original
  pagão de `animal inadequado ou que não se presta ao sacrifício', mas o
  conceito foi se estreitando e passou a designar animais nocivos,
  principalmente \textbf{insetos,} em oposição a animais domésticos como
  cabras, carneiros, etc \textbf{(Geziefer)''. }

  \emph{http://www.revistasusp.sibi.usp.br/scielo.php?pid$=$S1678-51771992000100013\&script$=$sci\_arttext}})
designa para os nazistas os presos no campo de concentração; é a mesma
palavra que Kafka usa para designar no que se transformou Gregor Samsa,
em \emph{Metamorfose}. Tardewiski associa, enfim, a utopia a uma grande
colônia penal. Tanto a conhecida por meio do nazismo --- assim como o
gulag na \versal{URSS}, até mesmo Clevelândia, no Amapá-Brasil, na década de
1920, antes do quase tudo das utopias totalitárias ―, como a designação
insuportável à sua utopia, efeito de confronto de utopias que o Estado
sempre foi capaz de criar para se manter. Até mesmo no que hoje em dia
chama"-se de campo de refugiados ou na emergência de \emph{países"-campos
de concentração}, como o Haiti ocupado pela força articulada de
militares multinacionais, organizações internacionais e \versal{ONG}s
transterritoriais atuando em nome da ``construção de um
Estado''\footnote{Sobre \emph{state-building} ver \versal{ONU}-Agenda da Paz,
  \emph{http://www.oecd.org/dataoecd/62/9/41212290.pdf}

  e sistematizações em
  \emph{http://www.sumarios.org/sites/default/files/pdfs/58503\_6774.PDF}

  \emph{http://cabodostrabalhos.ces.uc.pt/n3/documentos/2\_Ramon\_Freitas.pdf}}.

Assim como prossegue enquanto confirmação da utopia democrática dos dias
de hoje em que as periferias das cidades configuram-se como \emph{campos
de concentração a céu aberto}, sem muros, mas monitorados pela
eletrônica e pelas práticas policiais e democrático-cidadãs,
desconhecidas por Kafka. Periferia equipada socialmente para educar
crianças e gente da \emph{feliz idade} ocupada, espaço de lazer,
cultura, religião, organização e gestão de \versal{\versal{ONG}}s, fundações e institutos,
por vezes policiados ostensivamente por \versal{UPP}s --- reverso do combate
anterior pelo tráfico ―, propiciando entre os beneficiários o amor a ali
permanecer, enquanto o tráfico, como as mercadorias, gira, busca outras
circulações pelas cidades, pela \emph{urbe}, cada trabalhador, ou não,
busca encontrar seu lugar como empreendedor, na medida em que a reforma
do espaço da favela-comunidade responda como \emph{ambiente}
compartilhado de cultura e seguro comércio.

O campo de concentração agora como espaço de segurança e de atendimento
de saúde a uma população escolarizada com família renovada (a família
burguesa monogâmica em variadas composições de gênero, ligações
afetivas, casamentos e religiões), quando possível como \emph{insetos
nocivos domesticados}. Uma população também capaz de manejar o seu
próprio contingente penalizado, cumprindo sentenças ao ar livre, a céu
aberto, prestando serviços à comunidade ou sob o seu monitoramento.
``\emph{Sobre aquilo que não se pode falar, o melhor é calar, dizia
Wittgenstein. Como falar do indizível? Essa é a pergunta que a obra de
Kafka, tenta repetidamente, responder. É uma obra que fala do que não se
pode nomear}'' (Piglia, 2010: 194).

É neste instante que ― regressando a Vila-Matas ― emerge Beckett e o
inominável, o que ultrapassa os exercícios literários de Joyce; é onde
Kafka ganha outro impulso, livre do arame farpado do campo de
concentração.

Configurações possíveis hoje com o final da guerra tradicional nos
mostram como são redimensionados estes espaços segundo o controle por
satélites e, por conseguinte, monitoramentos que criam condições para
que se delimitem áreas móveis vulneráveis, composta de gente vulnerável
exposta a ser a intempérie de novidades ou os efeitos aguardados da
\emph{resiliência}. Configuram-se assim os traçados \emph{da cultura de
paz}, abarcando desde \emph{economia verde} até os protestos contra o
desemprego, fiscalizando tiranias, e principalmente regulando e
regulamentando condutas, compondo novas governamentalidades.

O rebanho não precisa mais de um pastor no poder, mas cada um deve saber
ser pastor de outro e estar no rebanho, ainda que pelos ideais; qual
\emph{ungeziefer} se é neste espaço? Boockchin tinha sua razão ao
distinguir \emph{urbe} de \emph{polis}, ainda que sua utopia também não
admita diferentes entre os próprios anarquistas e institua o similar ao
pluralismo democrático que é a diluição das diferenças em direitos. Por
isso a pletora de direitos inexequíveis! Não é mais necessária uma
biopolítica como tradução racional do pastorado cristão. O pastor se
democratizou.

Seguimos: sem guerras tradicionais e sob os registros constantes do
liberalismo (neo ou pós) e suas formas democráticas de fazer política e
produzir com participação, minando as argumentações tradicionais do
operariado, as idas e vindas do socialismo na América Latina, as
empresas burguesas chinesas, as redes sociais digitais, as wikis, e
estabelecendo um ensurdecedor silêncio.

\begin{enumerate}
\def\labelenumi{\arabic{enumi}.}
\setcounter{enumi}{7}
\item
  \textbf{sustentabilidade e ecopolítica}
\end{enumerate}

A sustentabilidade se firma como o meio para o capitalismo realizar de
maneira adequada, adaptável e consensual sua utopia de \emph{um futuro
melhor}, a partir dos três pilares: o social, o econômico e o ambiental.
As intervenções na natureza por meio de regulamentações
\emph{internacionais} repercutem em regulações nacionais, as empresas
aderem à \emph{responsabilidade social}, cresce o investimento em
redutores de \emph{vulnerabilidades}, aplica-se com rigor o \versal{IDH} (Índice
de Desenvolvimento Humano) e suas variações para monitoramento de
vulneráveis, e por ele, também, medem-se liberdades, convoca-se à
participação por programas de pacificação e missões de paz, amplia-se o
leque de seguranças, incluindo alimentação, clima, securitizações e
leva-se adiante as \emph{Metas do Milênio}, para as quais a Rio +20
apresentou-se como fórum de tendências e espaço para implementações da
\emph{economia verde} e de institucionalização da \emph{cultura de paz}.
Acresce-se a necessidade de monitorar a corrupção do Estado e
criminalizar novas condutas, exigindo-se zerar as impunidades.

A relação entre empresários, Estado, organizações internacionais e as
diversas \emph{comunidades} busca solidificar, plasticamente, um
\emph{mundo melhor}, que funciona como redutor de resistências.

A \emph{ecopolítica}, muito menos que disciplina de conhecimento e
política governamental específica relacionada ao meio ambiente, mostra
como a governamentalização dos \emph{ambientes} também está em função de
um planejamento para a institucionalização de cidades
\emph{resilientes}, conceito muito mais adequado e que ultrapassa o de
sociedade civil global.

Os termos genéricos deixam de sê-lo para assumir contornos definidores
mais claros, segundo a análise da história do presente. Trata-se da
racionalidade neoliberal, por suas práticas e não como ideologia, que
atrai e dissolve em grande medida as distinções entre \emph{direita} e
\emph{esquerda} tão próprias do século \versal{XX}. É imperativo que os
\emph{esforços}, agora, sejam de todos em função de \emph{melhorias}
(até mesmo das tradicionais estratégias revolucionárias). Neste sentido,
há um uso tático das chamadas esquerdas nas práticas de ecopolítica, na
medida em que buscam ideologicamente encontrar soluções para a sua
transformação, ainda que não mais sob o princípio da revolução comandada
pelo partido único, recompondo desafortunadamente, pela defesa da
democracia como valor e método, o repudiado reformismo de outrora.

A sustentabilidade exige inovação constante na produção de protocolos
gerando interfaces, com ênfase na diplomacia e práticas democráticas
para além da política institucional: a vida deve tornar-se democrática
com pletora de direitos.

Constata-se a captura de práticas políticas radicais dos anarquistas na
sucessão de protestos desde o movimento antiglobalização até os efeitos
no Norte da África e repercussões planetárias subsequentes a 2011. A
morte do líder da Al Qaeda mostra a força de polícia do Estado em âmbito
transterritorial, assim como leva adiante práticas de exceção na
democracia e proclama a presença forte do argumento favorável à
\emph{resiliência}. Da mesma maneira a cidadania digital passa a ser
regulamentada, acomodando relações com Estados totalitários em função de
interesses do próprio controle monitorado, e coloca em xeque a tese da
democratização da informação e da comunicação eletrônica.

A \emph{sustentabilidade} é uma prática que vai do indivíduo
multifacetado por direitos inacabados e inexequíveis à \emph{economia}
\emph{verde}, ao cálculo político a respeito da democratização dos
Estados e das relações com a natureza, e confirma a prevalência das
forças que defendem as práticas de \emph{conservação} diante das de
\emph{preservação}.

A \emph{ecologia} se transformou em tema que atravessa a \emph{direita}
e a \emph{esquerda}, em função do \emph{mundo melhor}. Apesar das
diferentes abordagens, o ambiente político-intelectual institui um
trajeto no qual a ecologia é o centro da vida, da \emph{vida
sustentável}, na qual as práticas de \emph{resiliência} dão forma às
modulações na sociedade de controle.

As crises localizadas parecem indicar oscilações entre o rebaixamento e
o ascendente consumo, respectivamente nos países ricos e pobres, e
sinalizam um efeito de nivelamento no âmbito planetário como redutor
equalizado das situações de miséria (equacionamento da erradicação da
pobreza com base em oscilações entre ascensão e rebaixamento de classes
médias, posto não haver ampliação dos segmentos com maior acesso ao topo
da pirâmide de rendimentos). Entre os protestos da Europa e a ascensão
para um patamar de classe média baixa no Brasil, temos os efeitos
consecutivos da sustentabilidade: multiplicação de direitos, seguranças
e medidas em relação ao meio ambiente. Os índices alcançados pelos
Objetivos do Milênio (\versal{ODM}) para o período 2000-2015 orientaram os
Objetivos de Desenvolvimento Sustentável (\versal{ODS}) a serem alcançados até
2030. Os \emph{ambientes} seguros, por conseguinte, necessitam de
penalizações ampliadas a céu aberto e de criminalização de novas
condutas como efetivação de direitos para sedimentar a cultura de paz e
a formação do cidadão \emph{resiliente}.

Os fluxos se cruzam, misturam, metamorfoseiam constantemente em função
de \emph{um mundo melhor}. As viagens siderais repercutem na indústria,
no conhecimento dos potenciais do corpo e da inteligência, das proteínas
inteligentes às neuromodulações, assim como na busca por outros espaços
para além do planeta, que também entram no cálculo \emph{inteligente} de
uma sociedade que deve se aperfeiçoar à imagem da democracia. Então, se
esta é a imperfeição em movimento, o capitalismo sustentável é a
desigualdade democratizada (nivelada) em movimento.

Exige-se que todos colaborem principalmente em seus locais de moradia,
trabalho e \emph{ocupações} (todos precisam estar ocupados para se
sentirem \emph{vivos}: da criança ao idoso, do divíduo são ao
deficiente, do saudável ao doente, do louco ao transtornado, do
abandonado ao possível incluído). A \emph{inclusão} se torna o principal
efeito da sustentabilidade pela proliferação de \emph{cares} que atraem
quem está em zona de escape ou marginal. Tudo deve estar seguro, das
práticas computo-informacional ao sexo.

A \emph{energia inteligente} necessita incluir. É a inteligência que
prepondera sobre o corpo, como se o \emph{espírito} da soberania
hobbesiano finalmente encontrasse um novo trânsito contratual, no qual o
assujeitamento revestisse as mais variadas contestações.

É como se a cada nova anomia anunciada a anterior já estivesse
normalizada, dando \emph{sentidos} às velozes práticas que também
\emph{normalizam os normais}. Diante desta situação, explicita-se o
esforço da racionalidade neoliberal em administrar o que chama de
\emph{crime}, em reequacionar a reforma da prisão pelo que ela tem de
mais seguro e jamais alcançado, ou seja, sua capacidade de conter
rebeliões. A penalização a céu aberto cada vez mais reitera a
metamorfose da polícia de exército de reserva do poder (composta de
infiltrados, delatores e delinquentes) alargado para um grande
contingente de cidadãos-polícia, sob o regime da denúncia, que atualizam
as \emph{lettres-de-cachet} em monitoramento de cada um por cada um, um
novo pastorado a serviço da soberania.

Ecopolítica, governo do planeta para um futuro melhor.

\section{adendo}

\begin{quote}
A dessacralização do espaço ocorre na sociedade de controle de maneira
veloz, segundo os fluxos, levando o trabalhador intelectual a atuar
desprendido dos lugares fixos. Navega-se no espaço sideral por meio de
fluxos computacionais. Não é mais um barco que nos leva a surpreendentes
e até exóticos pontos. As aventuras voltam a ocorrer dando fim à
espionagem (ultrapassagem da guerra-fria o paradigma da espionagem
contemporânea). Os novos corsários, como sabotadores nas redes e fluxos,
emparedam a polícia e provocam os múltiplos dispositivos de segurança,
nomeados segundo os sonhos de proteção divina como os programas
\emph{anjos da guarda} ou localizadores de invasores, e podem num
segundo se transformar em agentes de segurança. Se o anarquismo foi para
a sociedade disciplinar uma heterotopia, o que será para a sociedade de
controle?

Os anarquismos foram inventores de heterotopias intensas, o reverso da
sociedade disciplinar e inspiradores nas revoltas de 1968. Dali se
anunciou um deslocamento dos posicionamentos para os percursos. O que
estava esboçado na sociedade disciplinar por artistas e socialistas
libertários ganhará agora outra dimensão, a da intensidade diante da
velocidade.

A sociedade de controle gera velocidade, atravessa territórios e
fronteiras e faz seus fluxos se perderem no espaço sideral. Na história
do espaço, dizia Foucault, passamos pelos conjuntos hierarquizados de
lugares (as localizações que nos foram legadas da Idade Média), a
extensão infinitamente aberta (do Renascimento) e os posicionamentos
dispostos segundo séries, organogramas e grades (da sociedade
disciplinar). Agora, os fluxos se fazem e refazem segundo velocidades,
programas, interfaces, protocolos, acrescentados a hierarquias,
extensões, posicionamentos. A velocidade nos leva por transportes
materiais (barco, automóvel, avião, foguetes) e imateriais (os
programas) a espaços, culturas, lazeres, famílias, sociedades; leva-nos
à exclusiva sociedade da comunicação, da participação constante: todos
pela sociedade democrática que nos convoca a atuar na política
aperfeiçoando a democracia, o mais precioso valor universal, um
investimento em programas que vão da contenção à anulação das
resistências. Mais do que um risco para a democracia, como sublinhou
Alexis de Tocqueville, a opinião sobre todas as coisas e a participação
ativa por meio de atuação na economia e na política, fazem a vida do
rebanho contemporâneo, como alertaram Stirner e Nietzsche, chamando
atenção para as religiões da razão.

Os anarquismos entram para as redes e seus fluxos eletrônicos como
sabotadores de programas e inventores de vida. Os anarquismos vivem na
sociedade de controle não mais pelos lugares em que criavam
heterotopias, mas por percursos em que inventam experimentos. Eles,
enfim, não possuem lugares fixos, constantes e imutáveis, como
constataram Proudhon e Bakunin a respeito da existência anarquista
(Passetti, 2003b: 49-50).
\end{quote}

\emph{Como viver sem o desconhecido diante de si? (...) Nascido do apelo
do futuro e da angústia da retenção, o poema, elevando-se de seu poço de
lama e estrelas, será testemunha em quase total silêncio, que não há
nada nele que não exista, verdadeiramente noutra parte, nesse rebelde e
solitário mundo de contradições} (Char, 1995: 159).

\begin{enumerate}
\def\labelenumi{\arabic{enumi}.}
\setcounter{enumi}{8}
\item
  \textbf{sustentável captura}
\end{enumerate}

A sustentabilidade emerge como reação conservadora, e neste sentido, faz
funcionar a captura capitalista sobre os efeitos das lutas esboçadas
pelo acontecimento \emph{1968}. Diante das experiências nucleares em
lugares distintos, isolados e em muitos casos considerados paradisíacos
--- não só por serem lugares distantes, mas também por se desconhecer
propositalmente as populações ali residentes, suas cercanias e futuras
construções de usinas atômicas, proveniência da futura emergência dos
\emph{verdes}---; dos acúmulos de poluições urbanas e das movimentações
em defesa de santuários ecológicos --- equidistantes da lei do
nacional-socialismo de proteção à natureza de 1933---, para os quais
colaborou, inclusive, o movimento hippie ---; da proliferação da miséria
nos centros urbanos e rurais com degradação das cidades e modos de vida;
do desemprego crescente e das condições de trabalho exaustivas sob as
ditaduras, denegrindo os espaços com vista ao desenvolvimento de forças
produtivas; das constantes guerras e a utilização de armas bioquímicas;
enfim, as movimentações ecológicas não só tematizaram críticas diretas
ao capitalismo como também ao socialismo.

A situação de desestabilização destas formas de existência da produção
provocou não só uma reação política conservadora contra o keynesianismo,
o \emph{welfare state} e o socialismo estatal, como revelou a condição
estratégica da dominação em capturar esta latente tematização resistente
capaz de articular diversas tendências de combate ao desenvolvimento
industrial pela ampliação das práticas democráticas. A racionalidade
neoliberal se instituirá, desde as décadas de 1970 e 1980, e encontrará
na \versal{ONU} o espaço de configuração de uma nova situação de ordenamento
capitalista, não só em torno das reduções de poluentes, mas
principalmente enquanto recomendações que levaram a uma nova
configuração viável ao capitalismo, e que encontrou na \emph{Rio 92} o
ponto de confluência para o debate e o traçado de uma elaboração de
verdade acerca da \emph{nova ordem mundial}, sustentável e democrática.

O processo gradual se institucionaliza por meio de protocolos
administrados pela \versal{ONU} e gerenciados pelos Estados-nacionais e Europa,
em um momento de reconfiguração capitalista que combina a produção de
mercadorias a baixo custo (China, Índia e outros Estados asiáticos) com
redutores de benefícios de direitos sociais em polos avançados do
capitalismo, ajustando a transformação do trabalhador em capital humano
com políticas ambientais, parcimoniosamente incentivadas, em escala
planetária (e aqui é de pouca relevância a disputa pela verdadeira
sustentabilidade).

Os efeitos mais recentes situam a configuração da população japonesa
como expressão da \emph{resiliência}, após os efeitos devastadores do
terremoto de 2011 que atingiu a costa do país: é preciso colaborar,
adaptar-se, seguir pacífica diante das intempéries da natureza tendo em
vista a preservação da produção da energia atômica como forma de
acelerar ou manter o desenvolvimento, ou seja, estabilizar níveis
ascensionais de consumo com estabilidade política. As demais poluições
que têm levado a debates e tentativas de medidas de controle de
poluentes, principalmente o carbono, independentemente de outras
disputas sobre o \emph{crédito verde}, institucionalizam a luta política
com base na ciência ― colocada pelo \versal{IPCC} (Intergovernmental Panel on
Climate Change) e por pressões relativas ao cumprimento do Protocolo de
Kyoto, que redundou no Acordo de Paris assinado em 22 de abril de 2016
―, bem como produzem seus dissidentes, também apoiados em laudos
científicos capazes de orientar os \emph{policy makers} e fazê-los
crescer. Todavia, neste caso, o contraste de resultados situa embates
entre financiadores de pesquisas e coloca em discussão a produção da
verdade científica na orientação da política, ao mesmo tempo em que
revela os objetivos políticos na produção de uma verdade científica.
Enfim, por ambos os lados, a produção da verdade institui formas
econômicas e políticas de existência.

As guerras, redimensionadas em estados de violência, funcionam para
exercitar novos armamentos e táticas de combate por meio de
gerenciamentos eletrônicos, elevando a capacidade de segurança de
controle dos Estados, para além dos monitoramentos por satélites, do
combate sem prejuízo da vida humana dos pilotos ou daqueles que acionam
bombas localizadas: a população civil passa a ser o alvo tanto das novas
formas da guerra impessoal, \emph{cirúrgica}, com \emph{seus efeitos
colaterais}, e sempre \emph{justa}, quanto de ataques terroristas e suas
tentativas de combinar a fusão religião-Estado, como dos demais
perdedores radicais pela racionalidade neoliberal, realizando homicídios
localizados.

No interior dos Estados, projetos de conservação da natureza e
populações locais (indígenas ou ribeirinhas) começam a tratá-las por
meio de \emph{gestão territorial} combinada com \emph{economia verde},
ainda voltada a bens de consumo de preço alto destinados a setores da
sociedade com alto padrão aquisitivo e \emph{portadores} de consciência
ambiental; nas áreas urbanas, as restaurações de espaços degradados, a
inclusão de populações periféricas com incentivos à \emph{participação}
nas decisões locais, ampliam as \emph{tecnologias sociais} em
\emph{negócios sociais} e dão novos contornos que ultrapassam a
requerida \emph{responsabilidade social} proposta por empresários
ambientalistas.

As taxas de desemprego (em certos \emph{ambientes} do planeta) começam a
apresentar reduções por combinarem diversas formas de trabalhos
colaborativos que funcionam como \emph{ocupações} (Lazzarato, 2001;
2011) cada vez mais constantes no interior de desdobramentos econômicos,
situando-se como complementares à economia computo-informacional,
ampliando as ações em conformidade com a gestão do local. Trabalha-se
mais, ganha-se menos, sob contratos precarizados, mas se participa
muito.

Os espaços rurais e urbanos, por meio da sustentabilidade, configuram-se
em \emph{campos de concentração a céu aberto}, onde a gestão da miséria
com incentivos a acesso a bens de consumo, pelo circular movimento do
capital que provoca crises e avanços simultâneos e localizados,
combinados a programas sociais sob a conexão
empresários-governos-populações periféricas, articulam as novas fusões
entre sociedade civil e Estado em condições de transformar um fluxo que
funcionaria em progressão aritmética em fluxo que deve seguir em
progressão geométrica, a partir dos \emph{novos} negócios sociais.

A sustentabilidade propõe \emph{um futuro melhor para as próximas
gerações} e redimensiona a luta pelo presente levada a cabo pelo
\emph{1968}. Contudo, é preciso sublinhar que não se trata tão somente
de uma reação conservadora provocada pela racionalidade neoliberal que
soube conjugar interesses econômicos, práticas democráticas de gestão e
institucionalidade de regimes como contenção de resistências ou captura
de movimentos, como se esta estratégica fosse capaz de prever as demais
táticas de enfrentamentos, ou suficientemente preparada para cobrir
todos os monitoramentos. As \emph{novas} lutas, ainda sob a forma de
\emph{protestos}, nas primeiras décadas do século \versal{XXI}, mostram as
capacidades de incorporação de práticas anarquistas radicais em suas
produções. Mas quais os efeitos visíveis? Isso por si só não é garantia
de vitória à vista, na medida em que os poucos efeitos têm produzido,
até o momento, certas desestabilidades no controle e nas disciplinas
(ainda em vigência e das quais a sociedade de controle não prescinde ou
está mostrando, mais uma vez, sua capacidade de combinação ao fazê-las
funcionar por meio de \emph{interfaces}).

Determinar um \emph{juízo} a respeito destas novas lutas situando seus
aspectos de protestos e capturas seria próprio a uma conduta precipitada
e niilista, no sentido reativo indicado por Nietzsche. Também
incensá-las como anúncio de uma nova era não deixaria de sê-lo, ainda
que destas práticas radicais possam ser extraídas condutas de niilismo
ativo. Contudo, não se trata de niilismo por si só, um apreço amoroso
pela humanidade expresso em momentos de crise. Vivemos novas
configurações de reforma do capitalismo que prepara nova maneira de
produzir, em que a \emph{sustentabilidade} se firma como a verdade mais
conectada. Nesse sentido, as pontas soltas desses protestos que não
buscam sustentação, mas afirmam a necessidade da discrição,
apresentam-se como meios disruptivos nos quais o sujeito pode se apartar
dessa verdade sustentável.

Se tomarmos a formação de uma cultura sustentável pela educação de
crianças (não apenas escolar, como expressam as reformas curriculares
que incorporam o problema do \emph{meio ambiente} e/ou da
\emph{ecologia} chegando até às universidades com os cursos sobre
\emph{gestões} e \emph{engenharias}, entre eles as ambientais),
configura-se a construção de uma sociabilidade a partir do sensível
(impacto de efeitos de degradação ao meio ambiente que transita de
televisões à internet), das sensibilidades produzidas voltadas para
correções de rotas racionalmente traçadas, segundo o acesso condizente a
bens de consumo e programas sociais. Contudo, novamente, a sociabilidade
se refaz com base na \emph{escassez}, com uma educação voltada para a
gestão compartilhada produzida no momento em que se contornam os efeitos
capitalistas e socialistas em função do que se passa no instante: a
\emph{erradicação da pobreza}. Trata-se de um ``sigamos juntos'' dentro
do possível a todos.

A educação de crianças está atravessada pelas novas formas de controlar
o uso da água para banho e escovação dentária, combinada com os impactos
subjetivos produzidos por desmatamentos e intempéries da natureza.
Educação para \emph{melhorar hoje} como forma para obter maior
\emph{segurança} no futuro: aprender a gerir a escassez é também
\emph{melhorar} as condições de vida nas periferias, incentivar a
participar, exercitar-se em discussões democráticas com tomadas de
decisões nas escolas por meio de encenações de situações, jogos ou
enfrentamento de uma controvérsia circunstancial, combiná-las com
internet, fazer da vida um jogo a partir de simulações.

Nada a surpreender se a \emph{dissimulação} for complementar e conectada
ao pragmatismo. Não se trata de desvio da consciência, mas de maneira
eficaz e eficiente de produzir verdades que sustentam uma \emph{amizade}
universal, regida não mais por negócios entre \emph{homens de bens} como
no cânone aristotélico, mas como paz entre todos os homens (o que
combina cultura judaico-cristã, fraternidade revolucionária e kantismo,
outro triângulo equilátero perfectível).

A educação de crianças e de jovens, em especial, em função da
\emph{sustentabilidade} faz-se, agora, com o grande poder da
racionalidade neoliberal, a partir da ênfase na continuidade da
desigualdade como forma de implementar o capital humano, pelo
\emph{empreendedorismo}. Desta maneira, torna ou retorna a naturalização
da desigualdade sob a forma de investimento de cada um em trabalho e
ocupação, em trabalho e práticas sociais, em trabalho e certa atenção
necessária do Estado para com a saúde e a educação como intervenções
corretivas.

A convivência com a desigualdade sustentável é capaz de provocar
resistências que se articulem com práticas radicais como as
experimentadas na última década? Destas resistências podem emergir
associações mutualistas que ultrapassem limites da gestão territorial,
da economia solidária, dos efeitos das sensações e amor ao planeta? Até
que ponto o amor ao planeta é apenas a reescrita da verdade que sustenta
o amor à humanidade e ao Estado? Cabe e caberá aos jovens produzir lutas
que expressem as atuais condições de existência, encontrando seus modos
de contestar e ultrapassar, liberados das implicações difíceis das
formas de emancipação que governaram as resistências nos últimos dois
séculos.

A emancipação, o tema tão caro a socialistas, comunistas e anarquistas,
não pode ser negligenciada nem colocada de lado neste instante de
expansão da sociedade de controle diante dos efeitos ainda constantes da
sociedade disciplinar, em que vigilância e monitoramentos se combinam.
E, no último caso, governando condutas sustentáveis.

O \emph{arquivo ecopolítica} se abre em vários acessos a serem
complementados, mas jamais deletados. Trazer a produção da verdade,
ainda que desassossegue cânones, pode ser negligenciado, mas não admite
o ostracismo. Impõe-se como um arquivo parresiasta no combate pela
verdade. O \emph{arquivo ecopolítica} enfrenta os baixos começos da
comunicação contínua e computacional, buscando suas referências e
relações, desassujeitando a busca pela palavra-chave em favor da
emergência do acontecimento pelos combates entre as forças. O
\emph{arquivo ecopolítica} revira a história política para dissolver
oposições e dicotomias captando seus fluxos nas instantaneidades e,
portanto, repletos de intensidades que situam semelhanças estranhas e
diferenças uniformes e radicais.

O \emph{arquivo ecopolítica} absorve a biopolítica para indicar
metamorfoses e desdobramentos de uma sociedade disciplinar fundada na
extração de energias economicamente úteis e politicamente dóceis em uma
sociedade de controles de extração de energias inteligentes,
participativas e moderadas. Daí o embate atual entre conduta resiliente
e atitude de resistências, entre uma nova política e a antipolítica,
entre o cidadão-polícia e o revoltado. O \emph{arquivo ecopolítica} tem
um interesse sim, o de tornar possível cartografar a interceptação da
racionalidade neoliberal. A crítica não se destina aos outros, mas ao
modo como nos governamos monitorados pela crença no Estado como
categoria do entendimento.

A pesquisa na internet nos leva a documentos oficiais, situações prévias
e posteriores, comentários, artigos, os vários \versal{PDF}s disponibilizados,
sites que remetem a outras constatações mas, principalmente, colocam-nos
diante de derivas, e de volta aos livros, às artes, à literatura. Somos
navegadores que desconhecem o ponto em que pretendem ancorar, mas que
nele chegam sem se assustar, como nômades, que não almejam sedentarizar.
Outras navegações se anunciam depois da aportagem. Estão tranquilos pois
naquele ponto também está o refúgio de piratas. Talvez por isso, não se
trata de encontrar o paraíso, mas de aportar com os olhos no céu,
localizar o monitoramento celestial e compreender como transitar.

Meninos e meninas, hoje, olham para o céu e veem além das estrelas
cadentes, os satélites. Sabem onde está Vênus, olham as constelações com
telescópios, recorrem ao computador para verem o ``nascimento da
Terra'', como ela transita no universo e simulações de exoplanetas,
ouvem rumores na Terra e ainda podem se livrar da escuta. A escola não é
mais a mesma, também se democratizou, está governada por didáticas
artísticas, é da comunidade, e suas famílias heterodoxas são
orquestradas por direitos. Dentre eles, estarão outros indisciplinados,
revoltados a chocalhar seus colegas medicalizados.

O povo virou população pela economia política e pela estaística; pela
biopolítica, conformou-se em uma massa. Os divíduos de hoje caminharão
resilientes, até quando? Outros vivem nessa sociedade que está morrendo
e que almeja rejuvenescer experimentando o devir criança.

\chapter{O meio e o ambiente}

Paris, 12 de dezembro 2015. Salão da Conferência das Nações Unidas sobre
o Clima, a 21ª Conferências das Partes --- \versal{COP} 21. O presidente do
evento bateu na mesa do palco do grande auditório um pequeno martelo de
cabeça de borracha verde em formato de uma folha. Encerraram-se os
questionamentos. Ao som abafado da martelada, seguiram-se aplausos
esfuziantes, lágrimas de alegria e brados diversos.

Os 193 países-membros da \versal{ONU}, somados à Nova Zelândia e às ilhas Cook e
Niue, associados da Convenção Quadro das Nações Unidas sobre Mudanças
Climáticas (United Nations Framework Convention on Climate Change ---
\versal{UNFCC})\emph{,} comprometeram-se a adotar medidas para a contenção do
aquecimento do planeta. Mantiveram a meta de que a temperatura média da
Terra ficaria abaixo da elevação de 2º C ao longo do século \versal{XXI}, a
partir dos níveis pré-industriais estabelecidos em torno das condições
meteorológicas de 1750\footnote{``Os termos \emph{pré-industrial} e
  \emph{industrial} se referem, de uma maneira um tanto arbitrária, a
  períodos antes e depois de 1750, respectivamente\emph{''} (\versal{IPCC}, 2013:
  1456).}. Concordaram que os esforços do governo de cada Estado seriam
para limitar o aumento de temperatura a 1,5º C, tido como um índice
seguro para se controlar ``os piores efeitos das mudanças
climáticas''\footnote{\emph{Historic Paris Agreement on Climate Change.}
  \emph{http://newsroom.unfccc.int/unfccc-newsroom/finale-cop21/}}.

Os países desenvolvidos se comprometeram a conservar a posição de
redução em todos os setores da economia e os países em desenvolvimento,
a pelo menos mitigar os efeitos de emissões, caso seja impossível
coibi-las. Para essa adaptação à meta acordada, os países em
desenvolvimento terão um aporte financeiro de cerca de 100 bilhões de
dólares anuais vindo dos países considerados desenvolvidos, item cuja
difícil discussão quase impediu o consenso. O atual acordo entrará em
vigor em 2020, data do encerramento da validade do Protocolo de Kyoto, o
primeiro tratado internacional para reduzir a emissão de gases do efeito
estufa, lançado em 1997, na 3ª Conferência das Partes --- \versal{COP}-3, no
Japão. Nesse acordo anterior sobre o clima não se previam metas de
redução de emissão para os países em desenvolvimento, mas, a partir de
agora, todos os países membros da \versal{ONU} estão comprometidos na consecução
do objetivo, estabelecendo-se revisões a cada cinco anos, mediante
relatórios atualizados e planos de ação dos signatários.

As metas de Paris parecem modestas, além disso, não foram estabelecidos
prazos para sua efetivação. No entanto, entusiasmado, o presidente do
\versal{EUA}, Barack Obama postou no Twitter horas depois do evento: ``isto é
enorme: quase todos os países no mundo acabam de assinar o acordo de
Paris sobre mudanças climáticas --- graças à liderança dos Estados
Unidos''\footnote{``Obama diz que acordo não é perfeito, mas é
  ambicioso''.

  \emph{http://g1.globo.com/natureza/noticia/2015/12/acordo-climatico-de-paris-e-ambicioso-diz-obama.html}}.
Liderança tardia no tema, pois os \versal{EUA} demoraram para assinar o Protocolo
de Kyoto, ainda em vigor, e ainda não o ratificaram. Contudo, é
inquestionável que a adesão do governo estadunidense ao acordo
contribuiu para a celebração do consenso planetário a respeito da
atmosfera do planeta como alvo de ações que enfrentem efeitos deletérios
de atividades humanas. O Secretário Geral da \versal{ONU}, Ban Ki-Moon, comentou
na ocasião: ``entramos em uma nova era de cooperação global para a
humanidade enfrentar uma das questões das mais complexas. Pela primeira
vez, todos os países do mundo se comprometeram a reduzir as emissões,
reforçar a resiliência e unir-se na causa comum de tomar medidas
climáticas conjuntas''\footnote{\emph{Historic Paris Agreement on
  Climate Change}, op. cit\emph{. }}.

Nesse momento, importa menos a definição das ações que decorrerão do
acordo do que a união e o compromisso dos Estados em cooperar na
prevenção das alterações na atmosfera da Terra, recomendada por
cientistas de diversas nacionalidades dos grupos de trabalho do Painel
Internacional sobre Mudança Climática (Intergovernmental Panel on
Climate Change ― \versal{IPCC})\footnote{``Working Groups''.
  \emph{http://www.ipcc.ch/working\_groups/working\_groups.shtml}} e em
artigos científicos divulgados on-line\footnote{``For most vulnerable,
  1.5°C warming limit is critical: above it, climate impacts rise
  rapidly''.

  \emph{http://climateanalytics.org/hot-topics/for-most-vulnerable-1-5c-warming-limit-is-critical-above-it-climate-impacts-rise-rapidly}}.
Considera-se que as alterações de temperatura impactam o regime dos
ventos, os níveis de umidade do solo, a circulação das correntes
marítimas dos oceanos, as espécies vivas e, consequentemente, a vida
humana no planeta. Desde a fundação do \versal{IPCC} em 1988, a questão da
mudança climática ampliou seu escopo: de um problema ambiental restrito
às ações do órgão da \versal{ONU} voltado ao meio ambiente, o Programa das Nações
Unidas para o Meio Ambiente-\versal{PNUMA} (United Nations Environmental Program
--- \versal{UNEP}), o tema clima consolidou a presença no debate sobre a
segurança planetária. Na Conferência das Nações Unidas sobre o Meio
Ambiente e Desenvolvimento --- \versal{CNUMAD} (United Nations Conference on
Environment and Development --- \versal{UNCED}), realizada no Rio de Janeiro em
1992, conhecida também como \versal{ECO}-92, com a assinatura dos Estados
participantes, estabeleceu-se a base institucional para acordos
climáticos de alcance planetário: a Convenção--Quadro sobre Mudança
Climática\footnote{Convenção Quadro sobre a Mudança do Clima\emph{.}
  \emph{http://www.mct.gov.br/upd\_blob/0005/5390.pdf}}. Antes disso, em
relação à proteção da atmosfera do planeta, havia o Protocolo de
Montreal de 1987\footnote{Protocolo de Montreal\emph{.}
  \emph{http://ozone.unep.org/pdfs/Montreal-Protocol2000.pdf}}, ainda em
vigor, destinado a regular e coibir o uso de substâncias que destroem a
camada de ozônio que envolve o planeta e o protege dos raios
ultravioletas do Sol. O protocolo contou com ratificação de todos os
Estados membros da \versal{ONU}, tornando-se o primeiro acordo da instituição com
adesão total. Além disso, durante esses anos as medidas de substituição
de carbofluorcarboneto-\versal{CFC} --- composto gasoso presente em aerossóis e
refrigeradores responsável em parte pela destruição da camada de ozônio
da Terra --- fizeram com que ela se recuperasse parcialmente\footnote{``Camada
  de Ozônio estará recuperada até 2050, diz a \versal{ONU}''\emph{.}

  \emph{http://sustentabilidade.estadao.com.br/noticias/geral,camada-de-ozonio-estara-recuperada-ate-2050-diz-a-onu,1557931}}.
Essa vitória de um acordo internacional sobre um tema de abrangência
planetária, a camada de ozônio da atmosfera da Terra, passa a ser uma
referência para a eficácia das ações da \versal{ONU} nas questões ambientais.

\emph{2015, tempo da ação global para as pessoas e o planeta.} Sob esse
slogan, em 26 de setembro de 2015, em Nova Iorque, durante a Assembleia
Geral, e ao mesmo tempo em que se comemorava os seus 70 anos, as Nações
Unidas lançaram e aprovaram uma nova agenda com dezessete objetivos a
serem cumpridas no período de 2016 até 2030, são os Objetivos de
Desenvolvimento Sustentável (\versal{ODS}). O Papa Francisco abriu essa
Assembleia da \versal{ONU} com um discurso que reiterou a posição ética da Santa
Sé\footnote{A Santa Sé, desde 1957, integra a \versal{ONU} como observadora. A
  partir de 1964, tem o status de Estado não membro e observador
  permanente das Nações Unidas e de outras instituições do sistema como
  \versal{FAO}, \versal{UNESCO}, \versal{PNUMA}.} acerca do cuidado com o meio ambiente associado à
garantia de uma vida digna ao pobre\footnote{``Discurso do Santo Padre
  em visita às Nações Unidas''\emph{,} Nova Iorque, 25 de setembro de
  2015.

  \emph{http://w2.vatican.va/content/francesco/pt/speeches/2015/september/documents/papa-francesco\_20150925\_onu-visita.html}}.

Cada um dos 17 \versal{ODS} conta com um detalhamento de metas secundárias,
somando 169, que passam a ser referências para a elaboração de planos e
programas de governo em cada Estado, com ou sem auxílio financeiro de
fundos transnacionais e bancos de fomento, e serão acompanhados pelos
órgãos do sistema \versal{ONU} por meio de prestação de contas, relatórios e
produção de dados ao longo de quinze anos. São eles:

``\emph{Objetivo 1. Acabar com a pobreza em todas as suas formas, em
todos os lugares;}

\emph{Objetivo 2. Acabar com a fome, alcançar a segurança alimentar e
melhoria da nutrição e promover a agricultura sustentável;}

\emph{Objetivo 3. Assegurar uma vida saudável e promover o bem-estar
para todos, em todas as idades;}

\emph{Objetivo 4. Assegurar a educação inclusiva e equitativa de
qualidade, e promover oportunidades de aprendizagem ao longo da vida
para todos;}

\emph{Objetivo 5. Alcançar a igualdade de gênero e empoderar todas as
mulheres e meninas;}

\emph{Objetivo 6. Assegurar a disponibilidade e gestão sustentável da
água e o saneamento para todos;}

\emph{Objetivo 7. Assegurar a todos o acesso confiável, sustentável,
moderno e a preço acessível à energia;}

\emph{Objetivo 8. Promover o crescimento econômico sustentado, inclusivo
e sustentável, emprego pleno e produtivo e trabalho decente para todos;}

\emph{Objetivo 9. Construir infraestruturas resilientes, promover a
industrialização inclusiva e sustentável e fomentar a inovação;}

\emph{Objetivo 10. Reduzir a desigualdade dentro dos países e entre
eles;}

\emph{Objetivo 11. Tornar as cidades e os assentamentos humanos
inclusivos, seguros, resilientes e sustentáveis;}

\emph{Objetivo 12. Assegurar padrões de produção e de consumo
sustentáveis;}

\emph{Objetivo 13. Tomar medidas urgentes para combater a mudança do
clima e os seus impactos (reconhecendo que a Convenção Quadro das Nações
Unidas sobre Mudança do Clima é o fórum internacional intergovernamental
para negociar a resposta global à mudança do clima);}

\emph{Objetivo 14. Conservar e usar sustentavelmente os oceanos, os
mares e os recursos marinhos para o desenvolvimento sustentável;}

\emph{Objetivo 15. Proteger, recuperar e promover o uso sustentável dos
ecossistemas terrestres, gerir de forma sustentável as florestas,
combater a desertificação, deter e reverter a degradação da terra e
deter a perda de biodiversidade;}

\emph{Objetivo 16. Promover sociedades pacíficas e inclusivas para o
desenvolvimento sustentável, proporcionar o acesso à justiça para todos
e construir instituições eficazes, responsáveis e inclusivas em todos os
níveis;}

\emph{Objetivo 17. Fortalecer os meios de implementação e revitalizar a
parceria global para o desenvolvimento sustentável}''\footnote{Transformando
  Nosso Mundo: a Agenda 2030 para o Desenvolvimento Sustentável\emph{.}
  \emph{http://www.pnud.org.br/Docs/Agenda2030completo\_PtBR.pdf}

  Agenda 2030\emph{.} \emph{https://nacoesunidas.org/pos2015/agenda2030/}}.

A noção de desenvolvimento sustentável preconizada para o planeta por
meio dessas metas afirma a conservação do ambiente planetário para a
manutenção da vida humana, na qual o crescimento econômico deverá
ocorrer conjuntamente com o desenvolvimento social, abrangendo a
promoção de direitos, da segurança e da proteção do meio ambiente, aqui
entendido como suporte natural e também artificial em interação com a
população humana. Os objetivos diretamente ligados às questões
ecológicas (interação população humana e meio natural) são os de número
6 (água doce), 7 (energia e suas fontes), 13 (clima), 14 (oceanos e
recursos marinhos), 15 (ecossistemas terrestres: solo, biodiversidade) e
parte do Objetivo 2 referente à agricultura, dentro do contexto de uma
gestão dos recursos naturais em escala global.

O lançamento dessa nova agenda, a Agenda 2030, seguiu-se após o anúncio
do cumprimento dos Objetivos do Desenvolvimento do Milênio (\versal{ODM}) no
período entre 2000 e 2015. Os \versal{ODM} compuseram o projeto que orientou a
reestruturação dos propósitos da \versal{ONU}, suas agências e órgãos no ano
2000. Foram oito grandes objetivos, ``\emph{8 jeitos de mudar o
mundo}''\footnote{``8 jeitos de mudar o mundo''.
  \emph{http://www.objetivosdomilenio.org.br/}}, cada qual com algumas
metas detalhadas a serem alcançadas pelos países em desenvolvimento até
2015: 1) redução da pobreza; 2) ensino básico universal; 3) igualdade de
gênero e autonomia das mulheres; 4) redução da mortalidade infantil; 5)
a saúde materna e da gestante; 6) combate a malária, \versal{AIDS} e outras
doenças; 7) sustentabilidade ambiental; 8) estabelecimento de parcerias
mundiais para o desenvolvimento. Apenas o Objetivo 7 tratava
especificamente do meio ambiente, contando com quatro metas específicas:
desenvolvimento sustentável e reversão da perda de recursos; proteção da
biodiversidade; acesso à água potável e ao esgotamento sanitário; e
melhoria de bairros degradados\footnote{O detalhamento das metas do
  Objetivo 7 com os resultados das ações estão descritos em:

  \emph{http://www.unric.org/pt/objectivos-de-desenvolvimento-do-milenio-actualidade/27671}

  No Brasil: \emph{http://www.pnud.org.br/ODM7.aspx}}. Foram avaliados,
periodicamente, o grau de obtenção de cada uma das metas de cada
objetivo e a adequação dos resultados.

\begin{quote}
Em setembro de 2010, a Assembleia Anual das Nações Unidas para avaliar e
fomentar a consecução das Metas do Milênio solicitou ao Secretário Geral
um projeto para as etapas de aplicação da Agenda de Desenvolvimento da
\versal{ONU} para depois de 2015, denominado Keeping the Promise: United to
Achieve the Milleniun Goals\footnote{Keeping the promise: united to
  achieve the Millennium Development Goals.
  \emph{http://www.un.org/en/mdg/summit2010/pdf/outcome\_documentN1051260.pdf}}.
Em 2011, foi criada a Força Tarefa do Sistema \versal{ONU} (United Nations System
Task Team) \footnote{\versal{UN} System Task Team.
  \emph{http://www.un.org/en/development/desa/policy/untaskteam\_undf/}}
para a agenda de desenvolvimento após 2015\emph{,} sob a liderança do
Conselho Econômico e Social da \versal{ONU} (Economic and Social Council ---
\versal{ECOSOC}). Elaborou-se o rascunho do documento \emph{O Futuro que
queremos,} cuja versão final foi aprovada durante Conferência das Nações
Unidas sobre Desenvolvimento Sustentável (United Nations Conference on
Sustainable Development --- \versal{UNCSD}), em 2012. A Conferência, também
sediada no Rio de Janeiro, ficou conhecida como Rio+20, por ter sido
realizada vinte anos depois da \versal{ECO}-92, durante a qual se estabeleceu a
sustentabilidade como a conexão entre o desenvolvimento econômico e o
meio ambiente por meio de uma agenda socioambiental e econômica
(\emph{Agenda 21}), visando o novo século. Por sua vez, o documento de
2012 consolidou definitivamente o desenvolvimento sustentável como o
conceito a nortear os programas de desenvolvimento do sistema \versal{ONU} para o
futuro.
\end{quote}

A renovação das metas planetárias mantém e distende os \versal{ODM} colocando-os
em um programa amplo em termos sustentáveis, no qual a erradicação da
pobreza permanece em primeiro lugar. Os novos objetivos se interligam
pelo fortalecimento da chamada governança internacional, o conjunto de
instituições e normas para gerir o planeta, que os conecta como
interdependentes e responsáveis para a promoção do desenvolvimento
(Veiga, 2013). A resiliência, a inclusão, a educação, regulamentações e
presença do Estado, assim como a harmonização de conflitos, a
conservação e o chamado ``uso eficiente'' dos recursos naturais são
considerados meios fundamentais para a realização da igualdade social e
do equilíbrio entre economia capitalista e conservação do planeta,
cernes do ideal da sustentabilidade.

Os \versal{ODS} são metas aplicáveis a todos a países\emph{,} diferentemente dos
\versal{ODM}, que foram destinados especificamente aos países em desenvolvimento.
A proposta atual situou a erradicação da pobreza como condição
indispensável para a promoção do desenvolvimento sustentável e indicou
que os países em desenvolvimento necessitam de recursos extras para
atingir as metas. O processo de implantação dos objetivos conta também
com uma institucionalização mais eficaz de mecanismos de participação
dos grupos de interesse denominados \emph{Major Groups} e dos
\emph{stakeholders,} denominados pela sigla \versal{MG}o\versal{S} (Major Groups e outros
Stakeholders) que inclui consultas on-line\footnote{Quadro de
  Monitoramento e Indicadores para os Objetivos Sustentáveis. O
  relatório esteve em consulta pública on-line em 2014 e no início de
  2015.
  \emph{http://unsdsn.org/what-we-do/monitoring-the-sdgs/indicators/consultation/}}.
Reiterou-se o compromisso de uma parceria global para a mobilização de
recursos necessários, com a participação da sociedade civil, governos,
setor privado e \versal{ONU}.

Março de 2015, Sendai, Japão, Terceira Conferência Mundial das Nações
Unidas sobre a Redução dos Riscos de Desastres (Third United Nations
World Conference on Disaster Risk Reduction). Depois de três anos de
estudos e consultas, concluiu-se o Marco de Sendai para a Redução de
Risco de Desastres para o período de 2015-2030\footnote{Marco de Sendai
  para a Redução de Riscos de Desastres.

  \emph{http://www.preventionweb.net/files/resolutions/N1516720.pdf}},
aprovado em junho de 2015 pela \versal{ONU}. Este documento renova e amplia as
ações do Marco de Hyogo sobre o mesmo tema, cuja abrangência se
encerrara em 2015\footnote{Marco de Ação de Hyogo.
  \emph{http://www.eird.org/cdmah/}}. O marco atual faz uma avaliação dos
resultados de Hyogo e estabelece novas metas. A prioridade permanece em
torno de medidas para mitigação dos efeitos destrutivos nas populações
humanas de fenômenos naturais ou daqueles resultantes de ações
antrópicas conectadas com eventos naturais, envolvendo não apenas os
Estados, mas a ``comunidade internacional'', além da população local. O
novo marco diferencia-se do anterior ao propor ações para uma prevenção
mais efetiva e redução dos efeitos de desastres, agora incluindo os de
pequena escala, aliada a um compromisso mais estreito para uma
cooperação entre os Estados nacionais e um foco maior na promoção da
resiliência nas comunidades envolvidas. Estabelecem-se quatro
prioridades: compreensão dos riscos de desastres; fortalecimento da
governança dos riscos de desastres para uma gestão melhor; investimento
na redução dos riscos de desastres para incrementar a
resiliência\footnote{Na terminologia da \versal{ONU}, \emph{resiliência} seria a
  ``capacidade de um sistema, comunidade ou sociedade expostos a uma
  ameaça para resistir, absorver, adaptar-se e recuperar-se de seus
  efeitos de maneira oportuna e eficaz, que inclui a preservação e a
  restauração de suas estruturas e funções básicas''. Cf.
  \emph{http://www.unisdr.org/files/7817\_UNISDRTerminologySpanish.pdf}};
ampliação da preparação para casos de desastres para dar uma resposta
eficaz e melhor na recuperação, reabilitação e reconstrução\footnote{Artigo
  20 do Marco de Sendai.}.

No início do terceiro milênio, o ano de 2015 caracterizou-se pela
renovação de programas da \versal{ONU} em torno da promoção da sustentabilidade
como pedra de toque do desenvolvimento socioeconômico em todos os
continentes, da gestão da atmosfera e do clima da Terra e dos riscos dos
desastres no globo, e do investimento na resiliência das populações e do
planeta. Estes programas recentes procedem de acordos, debates,
conferências, encontros e relatórios agregados pela noção de meio
ambiente, construída, gradualmente, a partir da segunda metade do século
passado. O meio ambiente desdobrou-se em diversos ramos de uma
governamentalidade planetária à qual ajudou a conformar reunindo e
unindo elementos heterogêneos e, por vezes, disparatados. Enquanto
estratégia de respostas e equacionamento de conflitos e disputas que
surgiram no século passado, referentes à segurança e à manutenção da
vida no planeta, o funcionamento do termo meio ambiente demarca umas das
ultrapassagens da biopolítica para a ecopolítica.

No Brasil, seguindo as definições canônicas fixadas internacionalmente,
o termo meio ambiente aparece definido pela Lei 6.938 que instituiu a
Política Nacional de Meio Ambiente em 1981, como ``o conjunto de
condições, leis, influências e interações de ordem física, química e
biológica, que permite, abriga e rege a vida em todas as
formas''\footnote{Artigo 3º, item I; Lei Federal nº 6.938 de 31 de
  agosto de 1981.

  \emph{http://www.planalto.gov.br/ccivil\_03/leis/L6938.htm}}\emph{.} As
expressões \emph{meio ambiente}, em português, \emph{medio ambiente}, em
espanhol, e \emph{medi ambient}, em catalão, traduzem
\emph{environnement}, palavra em francês existente pelo menos desde o
século \versal{XIII} com o sentido inicial de \emph{contorno,
circunvizinhança}\footnote{Environnement
  \emph{http://www.cnrtl.fr/etymologie/environnement}

  Environ \emph{http://www.cnrtl.fr/etymologie/environ}}, e que no século
\versal{XIX} passou para o inglês gerando \emph{environment}, com o mesmo
sentido. A língua francesa conta também com a palavra \emph{milieu} (em
português, \emph{meio}) e a locução \emph{milieu ambiant}, pouco
utilizada no francês atual. A locução \emph{meio ambiente} não consiste
em pleonasmo, pois há uma distinção de sentido entre as palavras
\emph{meio} e \emph{ambiente}, apesar delas se apresentarem como
sinônimos em dicionários comuns. \emph{Meio} veio do latim
\emph{medium}, empregado na física newtoniana do século \versal{XVII}, como o
agente que intermedeia e transmite influências de uma coisa a outra. Com
uma conotação mecânica, \emph{medium} assinalava a distância da ação de
um corpo sobre o outro e relacionava‐se com a circulação: os movimentos
dos corpos e mútuas interferências (Spitzer, 1942: 37-38).
\emph{Ambiente} derivou do latim \emph{ambiens} (o que circunda), e foi
ganhando o sentido de um espaço amplo que envolve algo, uma atmosfera,
entorno ou um clima psicológico e moral (Ibidem: 01). O linguista
austríaco e professor de literatura nos \versal{EUA}, Leo Spitzer (1887-1960),
publicou em 1942 um estudo sobre o uso das expressões \emph{ambiente} e
\emph{meio} e as transformações de seus significados e usos desde o
latim, especialmente na literatura, mas sem deixar de comentar as
devidas aplicações na ciência biológica e social durante a passagem para
o século \versal{XX}. Reconhecia que as duas palavras atravessaram os séculos e
contextos diversos sem se separarem demais, mas mantendo distinções
entre os respectivos sentidos (Ibidem: 02). \emph{Meio} dizia respeito a
um lugar mais definido; \emph{ambiente} tinha uma conotação de
atmosfera, às vezes emanando de um meio (Ibidem: 188). A História
Natural empregou o termo meio (\emph{milieu)} nos séculos \versal{XVIII} e \versal{XIX},
como o ``conjunto de circunstâncias externas das quais dependem um
organismo'' (Ibidem: 177), acepção que apareceu também na sociologia com
Augusto Comte. O naturalista Saint Hilaire chegou a usar
\emph{milieu-ambiant} (Ibidem: 174). Havia os adeptos de um determinismo
do \emph{meio} geográfico, predominantemente do clima e da paisagem, no
caráter das populações. Na arte, escritores como Balzac e Emile Zola,
inspirados pelo cientificismo da época, colocavam personagens em
\emph{meios} que impulsionavam suas ações. Muitas vezes, na literatura,
a referência ao \emph{ambiente} fornecia ``um escape dentro de uma
poética mais vaga e imponderável: a antítese do meio determinante''
(Ibidem: 191)\emph{.}

A noção de meio (\emph{milieu}) como um espaço no qual pode se inscrever
uma série de eventos aleatórios, mas também permitindo um planejamento
de seus elementos e um cálculo de prováveis acontecimentos, foi se
constituindo ao longo do século \versal{XVIII}, na Europa ocidental. A noção
ultrapassou seu sentido de uma distância entre elementos físicos e
recebeu injunções de sentido político como um espaço passível de
intervenção programada, relacionado com a circulação de mercadorias,
pessoas e efeitos capitalistas. Antes mesmo da palavra \emph{meio} ser
utilizada para designar parte do alvo dessas técnicas de governo que
incluem intervenções em edifícios, ruas, localização de atividades,
vegetação, regime hidrográfico das cidades, afirma Foucault, o ``esquema
técnico dessa noção, (...) a estrutura pragmática que a desenha
antecipadamente estava presente na maneira com que os urbanistas
tentavam pensar e modificar o espaço urbano. Os dispositivos de
segurança trabalham, criam, organizam, planejam um meio antes mesmo da
noção ter sido formada e isolada'' (Foucault, 2008: 28).

Para se chegar à emergência histórico-política do conceito de população,
o novo alvo de um exercício específico de poder, o \emph{meio} faria
parte das técnicas de governo, marcadas pelo controle das condutas e
pela prevenção de ameaças à vida social: a partir do final do século
\versal{XVIII}, como mostrou Foucault, entre as ``coisas'' que se tornaram alvo
da governamentalidade estavam o uso dos recursos na economia e o
\emph{meio}: a materialidade no interior da qual e \emph{profundamente}
ligados a ela existiam os múltiplos indivíduos que compunham a população
humana. Mediante um meio construído, no qual se intervinha, atingia-se a
``naturalidade da espécie humana'' (Ibidem: 28) para governá-la por
novas técnicas no âmbito da biopolítica (Ibidem: 29-30).

Nas definições de meio e de ambiente até metade do século passado não
havia a conotação direta e necessária de \emph{natureza}, a não ser que
se acompanhassem do adjetivo ``natural'', ou aparecessem como noção de
estudos biológicos. O \emph{meio} referia-se com maior frequência ao
\emph{habitat} artificial da espécie humana, apesar do uso crescente
pela História Natural e, posteriormente, pela Biologia e pela Ecologia,
a ciência que trata da interação das espécies vivas entre si e com o
meio físico, inicialmente da natureza. Na passagem de uma biopolítica
para uma ecopolítica nos últimos 25 anos do século \versal{XX}, busca-se o
governo do planeta, por meio de, entre outras técnicas, intervenções no
\emph{meio ambiente}, cujo sentido paulatinamente se cristalizou em
definições institucionais.

A partir de determinados eventos com efeitos que ultrapassavam as
fronteiras nacionais, regimes de verdades e práticas foram se conjugando
e se voltaram para a atmosfera e o espaço circundante, redefinindo as
palavras \emph{ambiente} e \emph{meio}, reunindo-as numa mesma locução:
\emph{meio ambiente}, conforme a língua, ou fechando seu significado em
uma única palavra, como \emph{environment} (inglês),
\emph{environnement} (francês), ou acompanhada de um adjetivo, como no
alemão em que \emph{Umwelt} (meio), ao se referir à natureza e traduzir
meio ambiente nas definições institucionais, recebe o adjetivo
\emph{natürlich.} O sentido foi inicialmente assimilado à natureza, em
especial quando as ciências naturais e a ecologia transformaram
\emph{meio} e \emph{ambiente} em conceitos e os utilizaram em pesquisas
e experiências.

Na segunda metade do século \versal{XX}, houve um momento para o qual confluíram
vertentes discursivas de proveniências diversas que, ao se reunirem sob
uma mesma noção, \emph{meio ambiente}, buscavam enfrentar conflitos e
efeitos de acontecimentos díspares que ameaçavam a segurança do planeta,
apesar de diretamente não implicarem guerras e tampouco corrida
armamentista.

A Conferência das Nações Unidas para o Meio Ambiente Humano (United
Nations Conference on the Human Environment), realizada em Estocolmo em
1972, pode ser considerada como o evento espetacular que marca o
direcionamento da governamentalidade planetária pela noção de meio
ambiente. Desde a fundação da \versal{ONU} em 1945, foi esta Conferência um dos
primeiros eventos a reunir grande número de participantes: 113 Estados,
incluindo a Santa Sé, de um total de 132 membros. A \versal{URSS} e alguns países
europeus do Leste, como a Tchecoslováquia, a Bulgária, a Hungria, a
Polônia e a Albânia, boicotaram o evento, apesar da \versal{URSS} participar dos
encontros preparatórios. Nessas reuniões de 1970 e 1971, ocorreram
efetivos debates, não apenas sobre a proteção do meio ambiente, mas
sobre desenvolvimento econômico, demografia, gestão de recursos
planetários e maneiras de institucionalizar a cooperação entre as nações
em assuntos que atravessavam fronteiras nacionais.

A recomendação para que a \versal{ONU} realizasse uma conferência referente aos
``problemas do meio ambiente (\emph{milieu}) humano''\footnote{Resolução
  1346, 45° Sessão \versal{ECOSOC}.

  \emph{https://documents-dds-ny.un.org/doc/UNDOC/GEN/NR0/762/47/IMG/NR076247.pdf}}
decorreu de uma iniciativa em 1968 do \versal{ECOSOC}, órgão da \versal{ONU} voltado para
o desenvolvimento dos países membros. A motivação inicial do Conselho
procedeu de relatórios resultantes de estudos e encontros promovidos por
outros órgãos da \versal{ONU} sobre saúde humana, desenvolvimento econômico,
fertilidade dos solos, degradação dos ecossistemas, além de um pedido
direto do governo sueco ao Secretário da \versal{ONU} para convocar um debate
internacional sobre os problemas ambientais\footnote{Histórico da
  Declaração de Estocolmo.
  \emph{http://legal.un.org/avl/pdf/ha/dunche/dunche\_ph\_f.pdf}}. A
discussão sobre poluição industrial entrara na pauta sueca devido às
constantes chuvas ácidas no país, resultantes da poluição das indústrias
da Europa ocidental, especialmente da Alemanha. A sugestão foi
indiretamente reforçada pelas nascentes discussões de especialistas fora
do sistema da \versal{ONU} sobre a necessidade de limites ao crescimento
econômico global em um período em que as Nações Unidas elaboravam
programas e conferências para incrementar o desenvolvimento de países
pobres e estimular o comércio internacional. Em dezembro, a recomendação
do Conselho foi acatada pela \versal{ONU} e a Conferência marcada para 1972,
contando com a Suécia como sede do evento.

Os temas que se uniram neste primeiro grande encontro em torno da noção
de meio ambiente podem ser agrupados em quatro vertentes básicas (as
três primeiras aqui mostradas basicamente por relatórios e encontros
dentro do sistema da \versal{ONU}).

A primeira vertente envolvia a saúde pública e o saneamento, composta de
questões ligadas aos efeitos da poluição resultantes de atividades
econômicas na saúde humana. Um relatório da Organização Mundial da Saúde
― \versal{OMS} (World Health Organization -- \versal{WHO})\footnote{World Health
  Organization. \emph{Technical Report Series n° 406. Research into
  Environmental Pollution.} Genebra: \versal{WHO}, 1968.
  \emph{http://apps.who.int/iris/bitstream/10665/40709/1/WHO\_TRS\_406.pdf}}\textsuperscript{,}
órgão da \versal{ONU} então presidido pelo médico brasileiro Marcelo Candau, foi
publicado em 1968. Detalhava os efeitos de substâncias poluentes nos
seres humanos e causou grande impacto em outras agências da \versal{ONU}. Novas
doenças eram arroladas como resultantes de um meio poluído, não apenas
por esgotos, resíduos sólidos ou efluentes industriais, mas por gases
particulados e outros elementos pouco conhecidos que atingiam o corpo
humano ou impregnavam a cadeia alimentar dos ecossistemas.

Nos anos 1960, livros e estudos sobre o tema meio ambiente serviram de
referência para informações divulgadas na mídia, a qual na época
consistia em jornais, rádio, televisão, cinejornal e revistas
ilustradas. Alguns livros sobre a contaminação ambiental ganharam
destaque e foram muito divulgados em todos os continentes, como \emph{A
Primavera Silenciosa}, de 1962, de Rachel Carson, sobre os efeitos de
organoclorados e pesticidas não apenas em humanos, mas também nos
ambientes ecológicos. O silêncio a que se refere o livro de Carson seria
principalmente resultado da morte de pássaros nas áreas rurais dos \versal{EUA}
sob a ação de pesticidas, deixando de anunciar a primavera, fato
registrado com frequência desde a década de 1950 (Carson, 1969: 113).
Segundo a biógrafa da autora, Linda Lear: ``{[}Rachel Carson{]} escreveu
um livro revolucionário nos termos do que era aceitável para uma classe
média emergindo da letargia da afluência do pós-guerra e a despertou
para suas esquecidas responsabilidades'' (Lear, 1997: 45). Além disso,
os efeitos da radioatividade, desastres ambientais como o naufrágio e
vazamento de óleo do navio Torrey Canyon, na Grã-Bretanha, e casos
extremos de poluição, como a contaminação por mercúrio na baía de
Minamata, no Japão, eram amplamente divulgados pelos meios de
comunicação, mobilizando pessoas e despertando uma sensibilidade mais
aguda para o meio ambiente.

A segunda vertente seria a da natureza na perspectiva ecológica, como
interação de espécies vivas entre si e com o meio físico, e os modos
específicos e eficientes de protegê-la e gerenciá-la. A conservação da
biosfera entrava na agenda de intervenções de organismos internacionais
do pós-guerra porque a perda e a degradação de ecossistemas se
aceleraram em função de ações antrópicas. Uma das soluções encontradas,
a preservação de parte dos ecossistemas em santuários intocados se
mostrou insuficiente, e muitas vezes, inviável, por razões econômicas e
mesmo jurídicas, pois envolvia a propriedade. Além disso, muitas doenças
causadas pela contaminação e poluição atingiam indiretamente os seres
humanos, em ambientes deteriorados. Em 1968, a Organização das Nações
Unidas para a Educação, a Ciência e a Cultura (United Nations
Educational, Scientific and Cultural Organization ― \versal{UNESCO}) organizou a
Conferência da Biosfera, em Paris\footnote{Conferência da Biosfera.
  \emph{http://annals.org/content/73/4/675.4.extract}}, com o objetivo de
montar uma base científica de alto nível para o uso racional e a
conservação dos recursos naturais, marcando a entrada na política
internacional do termo biosfera \emph{---} a porção do planeta onde
ocorre a vida, envolvendo a crosta terrestre, a atmosfera e as águas.
``A Conferência foi palco de um alerta da comunidade científica
internacional aos governos do mundo. Chamou atenção para o nível de
criticidade ambiental que o planeta experimentava''\footnote{Reserva da
  Biosfera da Mata Atlântica\emph{.} Caderno nº 2, p. 28.

  \emph{http://www.rbma.org.br/rbma/pdf/caderno\_02.pdf}}\emph{.} Entre os
resultados da reunião, destaca-se a criação pela \versal{UNESCO}, em 1971, de uma
estrutura institucional para estudos e atuação para o estabelecimento de
reservas da biosfera em vários locais do planeta: o Programa Homem e a
Biosfera (Man and Biosphere ―\versal{MAB}), assegurando a preservação de algumas
áreas e a conservação racional de biomas identificados como
importantes\footnote{Preservação se refere à proteção integral de uma
  área, proibindo ocupação e uso humanos e evitando a predação por
  espécies invasoras. Conservação implica uso racional, ou seja,
  balizado pelo conhecimento técnico e científico de modo a manter os
  recursos dentro de sua capacidade de suportar as atividades humanas
  sem se degradarem, ou seja, sem se alterarem de modo irreversível.}.
Na Conferência de Paris, ``a preservação da natureza sob forma de
santuários é abandonada em prol da conservação pela qual se visa a
melhoria das qualidades humanas'' (Acot, 1990: 167).

O termo biosfera foi divulgado em 1926, pelo geoquímico russo Vladimir
Vernadsky (1863-1945). Foi uma das primeiras descrições da Terra como um
planeta esférico, autorregulado no espaço sideral, interagindo com
forças cósmica: o sol, partículas, a energia gravitacional dos elementos
do chamado universo. A biosfera também seria ``tanto criação do Sol
quanto produto dos processos da Terra'' (Alker Jr \& Haas, 1993: 141).
Para Vernadsky, ``o homem se esquece, em um sentido prático, que ele e
toda humanidade, da qual não pode ser separado, está inevitavelmente
ligado à biosfera (...). Na realidade nenhum organismo vivo encontra-se
em uma circunstância livre sobre a Terra. (...) Todos esses organismos
são constante e inextrincavelmente ligados ― antes de tudo por sua
alimentação e respiração ― com o meio ambiente material e energético ao
redor deles. Fora disso, não podem existir em condições naturais''
(Vernadsky apud Young, 2012: 160). O uso do conceito em um encontro do
sistema \versal{ONU} demonstra a disseminação crescente da imagem Terra como uma
unidade.

Antes da Conferência de Paris, em 1949, em seus primeiros anos, a \versal{ONU} já
se ocupara da questão da conservação e uso dos recursos naturais e da
ecologia, mediante discussão entre técnicos e cientistas. Duas
conferências de cunho científico aconteceram em Lake Sucess, Nova
Iorque. A primeira, Conferência Científica da \versal{ONU} sobre Conservação e
Uso dos Recursos (United Nations Scientific Conference on Conservation
and Utilizations of Resources ― \versal{UNSCCUR})\footnote{Proceedings of the
  United Nations Scientific Conference, Vol. I (1949).

  \emph{http://www.archive.org/stream/proceedingsofthe029855mbp/proceedingsofthe029855mbp\_djvu.txt}},
resultou de iniciativa conjunta da \versal{ONU}, \versal{UNESCO}, \versal{OMS} e Organização
Internacional do Trabalho ― \versal{OIT} (International Labour Organization ―
\versal{ILO}). Teve um caráter de fórum de discussões e trocas de experiências e
pesquisas envolvendo representantes de 49 países. A segunda, a
Conferência Técnica Internacional para a proteção da Natureza\footnote{International
  Technical Conference on the Protection of Nature (1949).

  \emph{http://unesdoc.unesco.org/images/0013/001335/133578mo.pdf}}, foi
organizada apenas pela \versal{UNESCO}, com a participação de representantes de
39 países, e girou em torno da educação e ecologia humana (McCormick,
1992: 53). Ambas não tinham força institucional para sugerir acordos
internacionais nem mesmo recomendações, porém, sua realização, quase
simultaneamente, destaca a especificidade dos recursos naturais como
alvo possível de políticas transnacionais e a necessidade de estudos
científicos e avaliações financeiras para melhor administrá-los. Suas
agendas foram retomadas apenas vinte anos depois, na Conferência da
Biosfera, em Paris, e a mobilização em torno desse evento e de seus
resultados institucionais impressionaram o \versal{ECOSOC} sobre a relevância do
tema.

A terceira vertente trata da economia e da gestão dos recursos naturais
para o chamado desenvolvimento econômico. Em dezembro de 1961, as Nações
Unidas declararam que a década de 1960 seria a década do
desenvolvimento\footnote{Resolution 1710 (\versal{XVI}), 19 de dezembro de 1961.

  \emph{http://www.un.org/en/ga/search/view\_doc.asp?symbol=A/RES/1710\%20(XVI)}}.
O programa Conferências das Nações Unidas sobre Comercio e
Desenvolvimento (United Nations Conferences on Trade and Development ―
\versal{UNCTDA}), estabelecido em 1964, buscava responder às demandas de Estados
em desenvolvimento por acordos comerciais e por auxílios financeiros. Os
Estados do bloco socialista participavam dos encontros e se incluíam na
economia global, principalmente mediante o comércio. Nas discussões para
o estabelecimento de uma Segunda Década do Desenvolvimento, avaliou-se
que as expectativas não foram respondidas e as metas necessitavam ser
reformuladas e ampliadas\footnote{The Yearbook of the United Nations
  1968, Chapter \versal{II}, The \versal{UN} Development Decade.

  \emph{http://www.unmultimedia.org/searchers/yearbook/page.jsp?volume=1968\&page=337\&searchType=advanced}}.
Havia um empenho das Nações Unidas na promoção do crescimento econômico
global, considerado como um desdobramento da Carta das Nações Unidas de
1945 e o ``caminho essencial para a paz e a justiça''\footnote{Resolução
  da 25ª Assembleia da \versal{ONU} nº 2626 (\versal{XXV}) International Development
  Strategy for the Second United Nations Development Decade (1970).
  \emph{http://www.un-documents.net/a25r2626.htm}}. O propósito do
desenvolvimento seria o de ``prover a todas as pessoas oportunidades
crescentes para uma vida melhor''\emph{,} incluindo acesso a saúde,
educação, nutrição, bem-estar social e moradia, atenção especial à
criança e proteção ao meio ambiente. Entre outros itens, as metas
propunham a participação dos jovens e a integração das mulheres nos
``esforços para o desenvolvimento''\footnote{Ver Artigo 18 da Resolução
  nº 2626 (\versal{XXV}) de 1970.}.

Fora do âmbito do sistema das Nações Unidas, porém, a discussão sobre a
economia mundial contava com alguns questionamentos sobre a pertinência
de crescimento econômico. Em 1968, na Itália, o alto executivo e
consultor das empresas italianas Fiat e Olivetti, Aurelio Peccei, reuniu
um grupo de industriais, cientistas, políticos e outros interessados
para discutir os rumos econômicos da humanidade e buscar soluções
inovadoras para os problemas a serem diagnosticados no encontro.
Fundou-se o Clube de Roma, um grupo de empresários, diplomatas e outras
personalidades dispostos a interferir nas decisões planetárias e
existente até hoje\footnote{Clube de Roma
  \emph{http://www.clubofrome.org/}}. Nas reuniões iniciais elaborou-se o
Projeto sobre a Condição da Humanidade, cujo objetivo era examinar a
`problemática mundial', na qual se incluíam pobreza, desemprego,
poluição, crescimento urbano desordenado, desemprego, inflação, ``perda
da confiança nas instituições, alienação da juventude, rejeição de
valores tradicionais'' (Watts in Meadows et al., 1972: 12).

O final da década de 1960, principalmente \emph{1968}, foi de amplas
manifestações de protesto nas ruas de grandes cidades do planeta,
especialmente levadas adiante por jovens, inclusive no chamado bloco
socialista. Questionava-se com veemência as guerras, em especial a do
Vietnã, que envolvia os \versal{EUA}; as instituições e seus valores
conservadores na família, escola, manicômios, exércitos entre outros;
nas relações hierárquicas rígidas; as ditaduras; as relações desiguais
de trabalho sob a propriedade; a tecnologia; o consumismo e/ou escassez;
a destruição avassaladora da natureza; e expunha-se uma reviravolta nos
valores, nos estilos de vida e no regime da propriedade. Encontros como
o do Clube de Roma buscavam entender e responder, imediatamente, à
convulsão anunciada pelos protestos e, no caso desse grupo de
empresários capitalistas, avaliar a crise no aspecto econômico e nos
seus reflexos morais.

Para a compreensão do conjunto e da interação dos fatores dentro do
Projeto sobre a Condição da Humanidade, o Clube de Roma propôs o
desenvolvimento de um modelo matemático para o diagnóstico e o
prognóstico de cenários futuros da produção econômica planetária. Contou
com o apoio de profissionais do Massachussetts Institute of Technology ―
\versal{MIT} na elaboração desse modelo, cujo resultado foi o livro \emph{Os
limites ao crescimento}, de Denis Meadows e equipe, publicado em 1972.
Concluiu-se que o planeta não suportaria o crescimento populacional,
caso fosse mantido o ritmo corrente da economia. O aumento da população
pressionava a utilização dos recursos naturais e energéticos e levava ao
aumento da poluição e da degradação do ambiente natural ou artificial
para condições que dificultariam as atividades humanas. A poluição do
meio ambiente entrou nas discussões econômicas como uma
``externalidade'' do chamado sistema produtivo que acabava por produzir
efeitos deletérios. Diagnosticado o problema, concluiu-se que ``é
possível alterar essas tendências de crescimento e chegar a uma condição
de estabilidade ecológica e econômica que fosse sustentável por muito
tempo no futuro'' (Meadows et al., 1972: 24). A data de lançamento do
livro coincidiu com a realização da primeira Conferência Mundial, em
Estocolmo e isso pode ter contribuido para o tom mais otimista da
conclusão. No entanto, a proposta para se atingir a chamada
``estabilidade ecológica e econômica sustentável'' consistia em uma
desaceleração do crescimento econômico associada ao controle
demográfico.

Outro tema em debate, corroborado pelas conclusões do Clube de Roma, foi
especificamente o limite dos recursos do planeta para suportar a vida
humana em um futuro próximo. Muitos estudos focavam o problema do
crescimento da população numa chave malthusiana, apontando o controle da
natalidade como a solução para os problemas do globo, inclusive da
poluição, como apresentado no livro \emph{A bomba populacional,} de Paul
Ehrlich, publicado nos Estados Unidos, em 1968. No mesmo período, em um
breve artigo, ``A tragédia dos Comuns'' (Hardin, 1968), o biólogo Garret
Hardin introduziu na questão do crescimento da população os chamados
`bens comuns' da humanidade: oceanos, estoques de peixes em alto mar,
florestas, ar, os quais, sendo de acesso comum e não sujeitos a se
tornarem propriedade privada, eram utilizados sem limites e regras
compulsórias. A questão seria a de encontrar maneiras para se moderar
esse uso, o que era dificultado, novamente, pela alta taxa de natalidade
no planeta. \emph{Como legislar a temperança?} ― perguntou o autor.
Segundo Hardin, para refrear o uso exaustivo dos bens comuns causado
pela superpopulação, seria necessária uma coerção não simplesmente
proibicionista, mas moral, com ou sem respaldo em normas
administrativas, propiciando condutas com um forte apelo à
responsabilidade social. ``'Liberdade é o reconhecimento da necessidade.
O papel da educação seria de revelar a todos a necessidade de abandonar
a liberdade para se reproduzir. Somente assim poderemos colocar um fim a
este aspecto da tragédia dos comuns'' (Ibidem: 1248).

A pobreza e a fome apareceram também na discussão como efeitos de uma
deterioração do meio ambiente. Constatou-se que o predomínio de solos
poucos férteis piora com a falta de ações para coibir usos predatórios,
muitas vezes causados pela própria população miserável. Um mapa dos
solos do planeta estava em confecção, sob a responsabilidade da
Organização das Nações Unidas para Alimentação e Agricultura (Food and
Agriculture Organization of the United Nations -- \versal{FAO}) desde os anos
1950, e tinha a América do Sul como ponto de partida. Em 1961, esse
mapeamento tornou-se um projeto conjunto com a \versal{UNESCO}\footnote{\versal{FAO}/\versal{UNESCO}
  Soil Map of the World. South America.

  \emph{http://www.fao.org/docrep/019/as361e/as361e.pdf}}, buscando-se
uma avaliação detalhada da fertilidade das terras dos continentes
associada às condições efetivas de seu uso na agricultura. Esboços dos
resultados preliminares foram divulgados e apontavam para uma escassez
de áreas com solos férteis e suas fragilidades, gerando preocupações nos
orgãos da \versal{ONU} voltados para o desenvolvimento e corroborando a
necessidade de um encontro entre os países membros para discutir a
questão relativa ao meio ambiente humano.

Uma quarta vertente, que também alertou o sistema \versal{ONU} sobre a
importância crescente do tema, foi a mobilização social e política em
torno do meio ambiente, do uso dos recursos naturais e dos efeitos da
poluição. Nos anos 1960, configurou-se uma grande desconfiança na
tecnologia e na ciência de ponta como poderes que afastaram a humanidade
de modos de vida harmônicos e equilibrados. ``Materialismo, tecnologia,
poder, lucro e crescimento eram caracterizados como símbolos do que
havia de pior na sociedade ocidental e como ameaças ao meio ambiente
{[}...{]} O movimento hippie do final dos anos 60 encarnou a escola
moral e anti-establishment do ambientalismo nos \versal{EUA}, em que o retorno às
áreas virgens e à natureza era o único caminho para manter os valores da
terra num mundo materialista'' (McCormick, 1992: 77). As experiências do
poeta e naturalista Henry David Thoreau ao encontrar a liberdade na
natureza selvagem (\emph{wildernes}s) e de outros preservacionistas
estadunidenses do século \versal{XIX} foram redescobertas e engrossaram a
discussão sobre a verdade da defesa da natureza. As questões referentes
à destruição do meio ambiente, junto à revalorização da natureza,
entraram na agenda das mobilizações do período.

A percepção real de habitar não mais um local, uma região, uma cidade, e
sim um planeta aumentou com a divulgação de fotos da Terra tiradas do
espaço e das declarações, até singelas, de astronautas. A questão da
sensibilização ao meio adquire um peso grande na mobilização em torno de
questões consideradas ambientais. Muitos efeitos da contaminação
ambiental não facilmente identificados, muitas vezes surgem em longo
prazo, e dependem da divulgação científica que, ao mostrar novas
evidências, adquire um papel disseminador nas mobilizações ambientais
que escapavam de mero sentimento. Ainda que as mobilizações lançassem
mão de um apelo emocional, o conhecimento científico não apenas
facilitava o foco para uma reivindicação fundamentada como permitia que
se discutisse um rol de soluções. Havia uma ênfase grande em
``conscientizar'' sobre a questão ambiental, situar aspectos de difícil
percepção ao cidadão comum, fazer do perigo genérico de um planeta que
se degrada um problema a ser equacionado no quintal de cada residência.
A ecologia tornou-se o setor do conhecimento científico que mais se
aproximou dos ativistas que defendiam a natureza\footnote{Há uma
  distinção entre ambientalismo e ecologismo, na qual os ecologistas
  aparecem como mais contundentes e os ambientalistas mais aptos a se
  integrarem ao sistema. Entretanto, aqui não será sugerida tal
  distinção, pois além de divergências na nomenclatura conforme a língua
  e o teor da reivindicação, isso pouco auxilia a análise do ativismo
  referente ao meio ambiente.}.

A palavra ecologia pareceu pela primeira vez em uma nota de rodapé de
uma obra monumental, publicada em 1866, \emph{Morfologia Geral dos
Organismos}, do biólogo alemão Ernst Haeckel (1834-1919), definida como
``totalidade da ciência das relações do organismo com o meio,
compreendendo, no sentido lato, todas as condições de existência''
(Acot, 1990: 27). O biólogo alemão cunhou este e outros conhecidos e
utilizados neologismos no decorrer de seus trabalhos, entre eles
\emph{phylum}, \emph{ontogenesis} e \emph{phylogenia}. Para formar a
expressão ecologia, \emph{Œcologie}, foram unidas duas palavras gregas:
\emph{oikos}, casa, habitat, e \emph{logos,} estudo, razão. Contudo,
Haeckel nunca pesquisou o que denominou ecologia; seus temas de
investigação eram morfologia e metamorfoses dos organismos, sem
enfatizar o meio em que estes nasciam e cresciam. Ele cunhou a palavra,
mas a prática desse conhecimento envolvendo seres vivos e o meio
percorria vários outros autores e outras expressões. Um termo que
concorria com ecologia era \emph{biocenose}, mas seu sentido pendia
apenas para a relação dos seres vivos entre si, enquanto que a ecologia
tratava da relação dos seres entre si e com o meio. Outras noções
procuravam dar conta da especificidade desse tipo de estudo e estavam em
circulação no ambiente das ciências da época: etologia (termo utilizado
por Saint-Hilaire), hexicologia (noção de Mivart), mesologia (noção de
Bertillon) (Drouin, 1993: 88).

A palavra \emph{ecologia} serviu para renomear a chamada geografia das
plantas (noção do naturalista Humbolt) ou geobotânica. Em 1895, o termo
\emph{ecologia} apareceu destacado no título de um estudo de Eugen
Warming, geobotânico de Copenhagen, que lhe deu um sentido conceitual
mais preciso. Ele dividiu a geobotânica em florística e ecológica. A
primeira compila flores, divide a terra em zonas florísticas, estudando
a extensão das espécies; a geobotânica ecológica tem outro propósito,
nas palavras de Warming, em \emph{Lehrbuch der Œkologishen
Pflanzengeographie}\footnote{Traduzido para o inglês em 1909, com o
  título \emph{Œcology of Plants}. Cf. Drouin, 1993: 21.}: ``ela nos
ensina como as plantas e as comunidades vegetais ajustam suas formas e
seu comportamento aos fatores do seu meio efetivamente atuante, tais
como a quantidade de calor, de luz, de alimentação e de água que se
acham disponíveis'' (Warming apud Acot, 1990: 32)\emph{.} Nos \versal{EUA}, um
pouco mais tarde, surgiu a \emph{ecologia dinâmica,} interessada nas
sucessões vegetais, introduzindo o fator tempo na análise dos
\emph{habitats}. Os animais foram incluídos levando as comunidades vivas
a serem consideradas um sistema, inclusive com a possibilidade de
previsões. A ecologia designou um ramo de estudos sobre a vida em que,
de modo geral, as espécies isoladas deixaram de ser o alvo central das
pesquisas. Os minerais e seus elementos químicos --- os fatores
abióticos desse fluxo de trocas --- foram considerados parte
constitutiva dos diversos \emph{ecossistemas,} termo cunhado em 1935
pelo botânico e pioneiro da ecologia inglesa Arthur Tansley. Com a
ecologia, investiga-se o que ocorre entre as espécies através das trocas
energéticas da cadeia alimentar.

``A `era da ecologia' começou no deserto dos arredores de Alamogordo,
Novo México, em 16 de julho de 1945, com uma ofuscante bola de fogo e um
intumescente cogumelo formado por uma nuvem de gases radioativos''
(Worster, 1994: 342), afirmou o historiador do ambientalismo D. Worster.
Menos de um mês depois, dois cogumelos semelhantes destroem Hiroshima e
Nagasaki, no Japão. Lançadas por estadunidenses, as \emph{bombas da paz}
encerraram definitivamente a \versal{II} Guerra Mundial, que já terminara na
Europa. Os efeitos radioativos no ambiente ainda eram praticamente
desconhecidos. Experiências e testes posteriores com explosões atômicas
demonstraram a impossibilidade de um controle desses efeitos que se
espalhavam pela atmosfera do planeta e impregnavam o ambiente. Segundo
Worster, esse momento marcou uma problematização, envolvendo inclusive a
opinião pública, sobre os limites da ciência. O otimismo em relação à
tecnologia arrefeceu, seria como se o ``lado escuro do
Iluminismo\emph{''} tivesse enfim se manifestado, não só com as
horríveis mortes, mas com a liberação mortal de forças naturais
impossíveis de serem detidas (Ibidem: 343). Essa questão também serviu
de referência para a revolta de jovens dos anos 1960 contra a tecnologia
e aspectos artificiais da vida humana e os fez valorizarem o saber
ecológico. Sendo uma ciência com rigor e prestígio para produzir
verdades, a ecologia nesse momento foi evocada como força capaz de
limitar a racionalidade do Estado e a própria ciência moderna, que, com
a justificativa legítima de soberania em destruir um inimigo, ameaçou um
bem comum e vital para a humanidade, o planeta. Sob esta ameaça à vida,
``uma nova consciência moral denominada ambientalismo começou a tomar
forma com o propósito de usar as descobertas da ecologia para restringir
o uso da ciência moderna baseada no poder sobre a natureza'' (Ibidem:
343-344).

Contudo, a era atômica fomentou os estudos ecológicos de ponta,
especialmente nos \versal{EUA}. As agências nucleares que realizavam testes
financiavam pesquisas para que se avaliassem os efeitos da
radioatividade nos seres vivos (Hagen, 1992: 100-121). Investigações
detalhadas sobre o alcance das contaminações também serviram para se
compreender os movimentos da energia e de materiais através dos
elementos vivos em interação com o meio. Eugene Odum foi um dos
pioneiros do uso de marcadores radioativos para estudar a cadeia
alimentar, o fluxo de nutrientes e os animais (Mongillo \& Booth, 2001:
205).

O conceito de ecossistema tornou-se central em uma nova ecologia,
apresentada no livro \emph{Fundamentos da Ecologia,} dos irmãos Eugene e
Howard Odum, publicado pela primeira vez em 1953, com edições revistas e
ampliadas em 1959 e 1971. Para os autores, ``o princípio do ecossistema
é o mais fundamental e importante princípio subjacente à conservação''
(Odum apud Hagen, 1992: 139). Essa nova ecologia trouxe um otimismo em
relação à tecnologia derivada desses saberes que apontavam para uma
efetiva engenharia ecológica: ``os seres humanos poderiam aprender
valiosas lições da natureza, mas esse conhecimento também permitiria aos
humanos intervir, manipular os ecossistemas naturais e criar
artificiais'' (Hagen, 1992: 139). O discurso ecológico contém e alimenta
um ``ethos gerencial'' (\emph{managerial ethos}) (Worster, 1994: 139).
As verdades ecológicas produzidas pelo saber explicam os processos da
natureza, e também trazem subsídios às intervenções, ao manejo e ao
gerenciamento de seus elementos, possibilitando equacionamentos de
efeitos de catástrofes e ações de prevenção, monitoramento da dinâmica
dos ecossistemas e mitigação de impactos de atividades humanas.

Presente nas ciências biológicas, o conceito de ecologia foi incorporado
à sociologia para dar conta de questões urbanas, alvo das pesquisas da
Escola de Chicago, desde os estudos pioneiros de Robert Park no início
do século \versal{XX}. A cidade foi vista como uma natureza, \emph{habitat} da
espécie humana, para a qual a visão sistêmica trazida pela abordagem
ecológica contribuiu para a compreensão do meio urbano, que incluía
desde os elementos materiais até os ``temperamentos morais'' de seus
habitantes. Os fluxos de transporte e comunicação, ``linhas de bonde e
telefones, jornais e publicidade, construções de aço e elevadores ― na
verdade todas as coisas que tendem a ocasionar a um mesmo tempo maior
mobilidade e maior concentração de populações urbanas ― são fatores
primários da organização ecológica das cidades'' (Park, 1967: 30). A
abordagem serviu para descrições mais sistêmicas da vida nas cidades e
para projetos de intervenção visando corrigir problemas e equacionar
conflitos, mas não se afastou das analogias procedentes do século \versal{XIX}
entre sociedade e organismo vivo. Entretanto, a ecologia urbana
instrumentalizou, mais especificamente, a imbricação entre população e
meio em \emph{ambientes}, que podem ser analisados por si e,
consequentemente, podem ser alvo de intervenções com resultados no
comportamento das espécies, no caso da ecologia humana, individual ou
socialmente.

Há uma relação estreita entre os variados ativismos voltados para a
defesa de aspectos pontuais da natureza e a criação e fortalecimento de
instituições e regulamentos e os diversos ativismos. Para as
reivindicações serem implantadas da maneira como são exigidas, faz-se
necessária a atuação de uma autoridade e, em geral, cabe ao Estado tal
função. As respostas institucionais ocorrem na forma normativa e
punitiva. Os ativistas lutavam e lutam para serem atendidos pelos
governos de algum modo e, por vezes, exigem a mediação do Estado em
alguma disputa. Além da força dos movimentos ambientais, que aumentavam
em número, causas e visibilidade global a partir dos anos 1960, as
instituições jurídico-políticas voltadas a atender as reivindicações por
uma gestão mais ecológica passaram a ser criadas e aprimoradas com maior
frequência. Nos \versal{EUA}, ``a partir do final dos anos 1960, tinham sido
redefinidos em termos `ambientais' os embates contra as condições
inadequadas do saneamento, de contaminação química de locais de moradia
e trabalho e de disposição indevida de lixo tóxico e perigoso''
(Acserald et al., 2009:17).

A resposta institucional do governo estadunidense aos problemas surgidos
pelas atividades humanas e à pressão das manifestações foi a aprovação,
em 1969, da Lei de Política Ambiental Nacional (National Environmental
Policy Act), e a criação da Agência Federal de Proteção Ambiental
(Environmental Protection Agency -- \versal{EPA}), ligada à Presidência da
República, em 1970 (McCormick, 1992: 136-137), tendo como atribuições:
controle da poluição mediante o licenciamento de atividades
potencialmente danosas às condições do meio ambiente; monitoramento,
pelo qual se certifica a qualidade do meio ambiente pelo registro de
indicadores pertinentes; cumprimento das medidas mitigadoras e do grau
de controle de impactos exigidas pelo licenciamento\footnote{\versal{EPA}.
  \emph{https://www3.epa.gov/}}.

Ao mesmo tempo, estudos mostraram que a distribuição espacial de riscos
e acidentes ambientais era socialmente desigual, e que os territórios de
concentração de minorias raciais ou de baixa renda eram os locais de
maior incidência de ações de impacto e de insuficiência de medidas de
controle e proteção. Surgiu nos \versal{EUA} o movimento denominado Justiça
Ambiental, pressionando o governo para que elaborasse políticas mais
distributivas e igualitárias, unindo em sua prática reivindicatória as
lutas ambientais e os direitos civis. Esse modelo espalhou-se pelo
planeta\footnote{No Brasil, ver:
  \emph{https://redejusticaambiental.wordpress.com/}}, e pressionou para
que as reivindicações ambientais fizessem parte de programas de defesa
de direitos e qualidade de vida, além de realizar estudos sobre
conflitos e as populações envolvidas\footnote{Mapa de Conflitos
  envolvendo Injustiça Ambiental e Saúde no Brasil.

  \emph{http://www.conflitoambiental.icict.fiocruz.br}}.

Porém, essa produção de verdades gerou também o seu reverso
complementar: o ``ecologismo dos pobres, ativismo de mulheres e homens
pobres ameaçados pela perda dos recursos naturais e dos serviços
ambientais de que necessitam para sobreviver'' (Martínez-Alier, 2012:
170). Dois exemplos são emblemáticos de resistências diretas às forças
que obstruíam seus modos de vida. Na Índia, nas nascentes do Ganges, em
1973, liderado por mulheres, um grupo de camponeses abraçou
(\emph{chipko}) as árvores de sua aldeia que haviam sido leiloadas pelo
governo para uma empresa de raquetes de tênis. Com o enfrentamento corpo
a corpo, conseguiram manter a floresta, da qual dependiam, e serviram de
referência para outras lutas semelhantes, formando o Movimento Chipko.
No Brasil, os seringueiros do Acre, ao terem suas terras passadas para
fazendeiros que derrubavam e queimavam a floresta, defenderam-nas por
meio do que ficou conhecido como \emph{empate}. Homens, mulheres e
crianças andavam pela mata de mãos dadas, desafiando os madeireiros e
fazendeiros. O primeiro episódio ocorreu em 1976, e o movimento logo
ficou internacionalmente conhecido devido à figura emblemática de Chico
Mendes. Ambos os movimentos não eram propriamente ecológicos, mas foram
assimilados às lutas ambientais tradicionais devido à defesa do meio
onde realizavam suas atividades.

A partir dos anos 1970, quando ``o meio ambiente se torna objeto de
políticas públicas, enquanto medidas jurídicas e instituições se
multiplicam nos níveis nacional e internacional'' (Le Preste, 2005:
166), surgiu o neologismo \emph{ecopolítica,} unindo dois substantivos:
\emph{ecologia} e \emph{política}. Sob um mesmo conceito reuniu-se as
políticas que envolviam assuntos referentes esse novo objeto de
intervenção: o \emph{meio ambiente}, no plano internacional. Segundo
Philippe Le Prestre, o uso pioneiro do termo \emph{ecopolítica} coubera
a Dennis Pirages, professor de política da Universidade de Maryland, nos
\versal{EUA}, ``para designar relações políticas no âmbito da proteção do meio
ambiente e dos seus recursos'', em 1978 (Ibidem: 19). O termo fora
utilizado anteriormente por Karl Deutsch, em uma publicação da \versal{UNESCO},
para nomear um campo emergente tanto de estudos científicos quanto de
ação: a \emph{ecopolítica}, o conjunto de intervenções, decisões
políticas relativas ao \emph{sistema ecossocial}, que consiste no
resultado da interação entre o sistema social e o ecossistema (Deutsch,
1977: 13). Como tal interação ultrapassa fronteiras dos Estados-nação,
Deutsch apontava para o aumento da importância de atores e instituições
internacionais e transnacionais em processos de mudanças, sem deixar de
lado possíveis novas atribuições dos governos dos Estados nacionais
(Ibidem: 16).

Alguns autores fazem uma distinção entre \emph{ecologia política} e
\emph{ecopolítica}. Philippe Le Preste, por exemplo, define ecologia
política como a ``doutrina que se articula sobre a crítica da sociedade
industrial com a pretensão de oferecer, sobre essa base, um projeto
global de sociedade, comparável e suscetível de oposição às duas grandes
ideologias da era industrial: o liberalismo e o socialismo'' (Le Preste,
2005: 19). Refere-se ``a uma ideologia ou programa político empenhados
em influenciar as políticas públicas, ou seja, a conquistar o poder
(Ibidem: 129). A ecologia política seria a propulsora do ativismo
ecológico e da crítica ao capitalismo, colocado em questão pela
percepção relativa à finitude dos recursos.

Por sua vez, a \emph{ecopolítica} se manifesta na mudança nas relações
internacionais, a qual, segundo Dennis Pirages, decorre da consideração
de ``preocupações ecopolíticas e novos assuntos correlatos como acesso a
recursos naturais, novos tipos de alavancagem econômica em temas
internacionais, adequação de suprimento alimentar para o mundo futuro,
questões de justiça social e direitos humanos, novos problemas de
desenvolvimento econômico para os países menos desenvolvidos e uma
equânime e eficiente distribuição de recursos mundiais'' (Pirages, 1978:
ix). O sentido de ecopolítica reforça a chamada governança ambiental
global, em que que todo um conjunto de normas e regulamentos decididos
multilateralmente e disseminados em níveis locais tem como alvo o
planeta. Para Le Prestre, ``o meio ambiente se tornou rapidamente a
matriz das novas relações internacionais. (...) A dinâmica da
ecopolítica indica a ordem internacional para a qual se encaminham as
sociedades humanas (Le Preste, 2005: 471). (...) Mesmo os indiferentes
às questões ambientais não podem ignorar as implicações {[}da
ecopolítica{]} para a segurança interna e externa dos Estados e suas
populações'' (Ibidem: 481).

Os ativismos em torno de temas ligados à natureza e ao que
posteriormente, nas últimas décadas do século passado, denominou-se
\emph{meio ambiente} se iniciaram como pressões para respostas a
problemas específicos e segmentados. Algumas das primeiras e efetivas
ações de preservação jurídico-institucional da natureza surgiram em
países que se industrializavam com rapidez. Paisagens e modos de vida se
alteravam, devido à chamada ``Revolução Industrial e Agrária'', iniciada
na transição do século \versal{XVIII} para o \versal{XIX}\footnote{\emph{Revolução
  Industrial} foi termo inicialmente divulgado pelo historiador
  britânico Arnold Toynbeepara se referir às mudanças sociais advindas
  do capitalismo, em \emph{Lectures on The Industrial Revolution in
  England} (1884).
  \emph{http://socserv2.socsci.mcmaster.ca/~econ/ugcm/3ll3/toynbee/indrev}.},
especialmente na Inglaterra, na França e nos \versal{EUA}. A indústria crescia
simultaneamente à concentração e à redefinição da propriedade de terras,
de novas técnicas agrícolas com a decorrente expulsão de parte das
gentes que tinha seu sustento e modo de vida em torno da agricultura.
Nos \versal{EUA}, especialmente durante o século \versal{XIX}, também havia um processo de
ocupação por parte de migrantes agricultores ocupando vastas áreas a
oeste, terras indígenas e as subtraídas do México, compostas de grandes
regiões naturais ainda preservadas da ação predatória humana.

Os rumos do chamado ``progresso'' mereceram desconfiança entre alguns
cientistas já no século \versal{XIX} devido aos problemas gerados pela rápida
urbanização de locais onde se instalaram as fábricas. O ambiente
artificial das cidades passou a manifestar aspectos insalubres que não
dependiam mais de miasmas provenientes de pântanos ou má circulação de
ar, mas de chaminés de fábricas, de efluentes de processos industriais
lançados em corpos d'água e do crescimento populacional, pela natalidade
e migração, sem uma correspondente infraestrutura de saneamento e
moradia. Em 1852, o químico escocês Robert Angus avaliou o pH de chuvas
que afetavam negativamente a vegetação na cidade de Manchester, na
Inglaterra, e cunhou o termo ``chuva ácida'', tema investigado a fundo e
alvo de ações internacionais 120 anos depois. O chamado \emph{efeito
estufa} causado pelo dióxido de carbono, por sua vez, foi estudado com
mais detalhe pelo sueco Arrhenius em 1896. A primeira legislação de
controle de gases na atmosfera foi a lei de Alcalis (\emph{Alkali Act}),
promulgada na Inglaterra em 1863 para coibir a emissão de ácido
clorídrico das indústrias químicas de ponta. Reclamações de moradores
dos arredores dessas fábricas se aliavam a laudos dos médicos que
tratavam dessa população e de estudos científicos dos químicos que
conheciam bem esses gases. Os efeitos da industrialização eram tratados
de modo indelével no âmbito das políticas de saúde pública e de
manutenção da salubridade do meio.

Para mitigar as condições das cidades britânicas surgiram entidades de
apoio à construção de parques urbanos e jardins públicos, e uma das
primeiras foi a londrina Common, Foot Paths, Open Spaces Preservation
Society\emph{,} em 1865. Começavam a aparecer as organizações não
estatais reunidas em torno de um tema hoje denominado ambiental (Mc
Cormick, 1992: 34). A história natural e outras ciências da natureza
progrediram nesse período, em função do crescimento industrial que
exigia novas técnicas produtivas desenvolvidas pela aplicação dos
conhecimentos da física e química. As primeiras propostas de gestão de
áreas naturais e recursos da natureza, portanto, resultaram das ações de
associações de estudiosos e cientistas.

No início do século \versal{XIX}, a percepção de artistas levou também a
questionamentos a respeito da destruição de paisagens e modos de vida
locais para a ampliação da economia capitalista industrial. Atribuiu-se
aos movimentos estéticos da época, reunidos sob a rubrica de Romantismo,
uma valorização da natureza como um ``sistema simétrico e
bem-composto'', capaz de restaurar a harmonia dos seres humanos.
``Embora houvesse diversas visões sobre a natureza (...) --- visões
mecanicistas, visões biológicas, visões orgânicas, visões físicas
(usando todo tipo de metáforas) ---, o refrão era sempre o mesmo:
Senhora Natureza, Mãe Natureza, os cordões da natureza que nos seguram e
dos quais não devemos nos separar'' (Berlin, 2015: 121). Na Inglaterra,
em 1810, o poeta W. Wordsworth escreveu um guia de visitação à região
dos lagos na Cumbria, no norte do país, destinado a ``pessoas de gosto e
sentimentos em relação à paisagem'', no qual descreve o local como ``uma
espécie de propriedade nacional na qual há o direito e o interesse de
cada homem que tem olhos para perceber e um coração para desfrutar''
(apud McCormick, 1992: 17). Na Escócia, em 1828, o poeta Thomas Carlyle
(1795-1881) cunhou o termo \emph{environment} a partir da palavra
\emph{environ} do francês arcaico, para traduzir a palavra alemã
\emph{umgebung} em um texto de W. Goethe (Jessop, 2012). Passou a
utilizar o termo em diversos ensaios sobre filosofia e literatura,
buscando novas relações entre o ser humano, a paisagem natural e as
atmosferas morais que suscitavam, contrapondo-se a uma visão mecanicista
da vida humana e à guerra contra a Natureza (Adam Rome \emph{apud}
Jessop: 718).

Nos Estados Unidos, no século \versal{XIX}, a preocupação com o meio recaia nos
ambientes das áreas naturais intocadas pelos civilizados, não tanto
sobre poluição ou exigências de saneamento. A concepção de natureza
intocada orientou as políticas de preservação de áreas especiais que se
efetivavam mediante a criação de parques e reservas naturais pelos
governos. Em 1832, o pintor estadunidense George Catlin (1796-1872)
sugeriu que os indígenas americanos do seu país fossem preservados ``por
alguma grande política governamental, em sua primitiva beleza e
\emph{wilderness} dentro de um parque magnífico. (...) Que bela e
emocionante espécie para a América preservar e exibir para ser vista por
cidadãos refinados e pelo mundo nas eras futuras! Um parque da nação,
contendo homens e feras, todos no frescor selvagem de sua beleza
natural'' (Catlin, 1841: 261-262). Com objetivo expresso de preservação
de uma paisagem natural de ``beleza cênica'', surgiu em 1872 o primeiro
parque nacional, Yellowstone, em Wyoming. Anos depois, graças à
iniciativa do preservacionista estadunidense John Muir (1838-1914), foi
criado o Parque Nacional de Yosemite, a primeira reserva conscientemente
designada à proteção de áreas virgens, que abrangia não apenas o vale de
grande beleza, mas áreas do entorno. Em 1892, em defesa da Sierra
Nevada, na Califórnia, John Muir e seus discípulos (Fox, 1981) criaram a
associação Sierra Club, uma pioneira organização de defesa da natureza.
Ao longo do século \versal{XX}, ampliou seu escopo para questões ambientais como
os recursos energéticos e permanece atuante até hoje\footnote{\emph{http://www.sierraclub.org/index.asp}}.

No Brasil, em 1876, inspirado pela criação de parques nacionais como
Yellowstone, André Rebouças, engenheiro preto do Império brasileiro,
propôs a criação de dois parques nacionais, um em Sete Quedas e outro na
Ilha no Bananal, consolidando o efeito da divulgação internacional da
política de criação de áreas preservadas pelos Estados nacionais.
Entretanto, seus projetos não foram implantados. O primeiro parque
nacional de preservação integral só viria a ser institucionalizado em
Itatiaia, no Rio de Janeiro, pelo Decreto nº 1.713 de 1937, no governo
provisório de Getúlio Vargas.

A relação conflituosa das atividades humanas com a natureza foi tratada
pelo estaduninense George Perkins Marsh (1801-1882) em \emph{Man and
Nature}, publicado 1864 e reeditado com alterações em 1873, com o título
\emph{The Earth as Modified by Human Action}. Ao longo de extensos
capítulos, o autor procura demonstrar a destruição da fauna, das
paisagens, dos solos e a formação de desertos; a devastação das
florestas; a alterações dos regimes hidrológicos, fluviais e estuarinos
pelos humanos desde o início da história a ponto de ameaçar tornar a
Terra inabitável em um futuro breve. ``O homem é um agente perturbador
em qualquer lugar. Onde ele fixa o pé, a harmonia da natureza se
transforma em discórdia'' (Marsh, 1907: 46). Seus argumentos
corroboraram iniciativas de proteção de elementos naturais e também de
pesquisas sobre o uso desses recursos. Ao mesmo tempo, Marsh procurou
encontrar ações que mostrassem possibilidades de restauração de danos,
como a construção de diques para impedir assoreamento de cursos d'agua
ou ações de reflorestamento (Ibidem: 55-56). Ao publicar a edição
revisada de seu livro, considerou o trabalho do geógrafo anarquista
francês Elisée Reclus (1830-1905), \emph{A Terra}, então recém-publicado
em inglês, como um estudo complementar ao seu. ``{[}Reclus{]} se ocupou
com os efeitos conservacionistas e restauradores da atividade humana, e
não as destrutivas; elaborou um quadro atraente e encorajador das
influências aperfeiçoadoras da ação do homem e das compensações pelas
quais ele, consciente ou inconscientemente, faz as pazes com a
degradação que ele produziu no meio {[}medium{]} em que habita''
(Ibidem: 3).

Élisée Reclus publicou dezenas de livros e artigos, tanto sobre a
geografia quanto resultantes de sua prática anarquista. Em um de seus
textos, ``O sentimento da natureza nas sociedades modernas'', investigou
a atração e os sentimentos das pessoas em relação à natureza, e concluiu
que estes decorreriam da percepção humana ativada pelos estímulos
sensoriais do contato com os elementos naturais, aquém dos conceitos e
do intelecto (Reclus, 1866). Estudava a natureza aliada à vida humana,
pois a humanidade era um dos fenômenos da Terra. ``A geografia
reclusiana caracteriza-se pela descrição da produção social do espaço e
pela análise das relações entre as sociedades e o quadro físico e
biológico em dimensões, simultaneamente, espaciais e temporais. A vida e
a natureza coincidem, para além do orgânico, pelo movimento constante''
(Carneiro, 2011: 105). A relação de Reclus com a natureza ultrapassava o
utilitarismo que nela busca recursos para atividades econômicas ou uma
contemplação visando transcendê-la. Consiste em uma relação física,
sensorial, entre o seu corpo e os corpos dos elementos de seu meio. Em
relação a uma experiência dentro de um riacho, Reclus comentou:
``Parece-me que me tornei de fato parte do meio que me envolve, eu me
sinto um com as ervas flutuantes, com o saibro movente sobre o fundo,
com a correnteza que faz oscilar meu corpo... Todo esse mundo exterior é
real?'' (Reclus apud Carneiro, 2011: 106).

Com atitudes em relação à natureza semelhantes às de Reclus, o escritor
e naturalista libertário estadunidense David Henry Thoreau (1817-1862)
buscava também uma relação de intimidade física com os elementos
naturais. ``Suas excursões científicas (...) mantinham essa busca por um
contato sensual, por um visceral senso de pertencimento à terra e a seu
círculo de organismos'' (Worster, 1994: 78). Enquanto pesquisador
naturalista, desejava ser ``natureza examinando natureza'' (Ibidem: 78).
Todavia, ao contrário de Reclus que não recusava interferir na natureza
para facilitar vida humana, Thoreau voltava-se para a natureza selvagem
(wilderness), e nela encontrava uma experiência de liberdade.

Nos \versal{EUA}, em relação à proteção da natureza, ao lado do preservacionismo,
cujo expoente foi John Muir, inspirado na filosofia de Ralph Waldo
Emerson e nas ações de Thoreau, havia os conservacionistas, como Gilford
Pinchot, que propunham a \emph{conservação} dos elementos naturais. Esta
tendência se define pelas propostas de um uso planejado de recursos,
especialmente dos florestais, inspirado na escola alemã de manejo
florestal. Pinchot tinha bom trânsito no governo de T. Roosevelt
(presidente dos \versal{EUA} de 1901 ao início de 1909) e entre suas propostas
estava a convocação de um encontro entre países para discutir a
conservação de recursos naturais a ser realizada em 1909, em Haia,
cidade holandesa que no início do século \versal{XX}, que sediava a primeira
corte permanente para equacionar conflitos internacionais. Houve adesão
de dezenas de Estados para a realização o encontro, mas o projeto acabou
cancelado com a posse de W.H.Taft, o novo presidente dos \versal{EUA} (McCormick,
1992: 34).

Parques nacionais de um lado e, de outro lado, mitigação de condições
insalubres fizeram a gestão de recursos naturais, água, ar, solo e
territórios entrar na pauta das responsabilidades dos governos e do
Estado nacional desde os séculos \versal{XVIII} e \versal{XIX}. Essas intervenções no meio
faziam parte das técnicas de governo da população descritas por
Foucault, que, em geral, compunham-se de leis e regulamentos envolvendo
a natureza e os \emph{habitats} humanos em territórios nacionais. Porém,
já se notava que os efeitos de atividades econômicas ultrapassavam as
fronteiras dos Estados-nações, reforçando o papel dos Estados nas
negociações com populações de áreas fronteiriças. As organizações de
proteção à fauna e à flora, capitaneadas ou não por naturalistas ou
``amantes da natureza'', logravam influenciar governos e propor
normativas e acordos internacionais multilaterais quando o assunto
ultrapassava as fronteiras nacionais. Estas primeiras Convenções e
Tratados envolviam animais, em especial aves, além de mar, rios de
fronteiras e o transporte de vegetais, como mudas e sementes.

Em 1868, em Viena, a questão de proteção de pássaros selvagens úteis à
agricultura fora tema de um encontro de naturalistas alemães que
propuseram elaborar um grande acordo internacional para a proteção de
animais e que fosse também útil ao manejo florestal e agricultura. Em
1902, assinou-se a Convenção Europeia para a Proteção de Pássaros Úteis
à Agricultura\footnote{Em espanhol:
  \emph{http://www.ecolex.org/server2.php/libcat/docs/TRE/Full/Sp/TRE-000067.txt}.}.
A profusão de sociedades de ornitólogos e de amantes de pássaros
reunidos em um Comitê Internacional de Ornitologia colaborou para que
finalmente essa convenção fosse assinada, e suas técnicas de manejo dos
pássaros são usadas até hoje (Sands, 2003: 27). No entanto, apenas as
aves insetívoras foram consideradas `boas' para serem protegidas, posto
que úteis ao combate de pragas nos campos cultivados; as aves de rapina
não obtiveram menção à proteção. Porém, em 1950, uma nova Convenção
Internacional para a Proteção das Aves veio a ser assinada em Paris,
substituindo a de 1902, antecedida pela bem-sucedida Conferência sobre a
Proteção a Aves Aquáticas Migratórias, em 1927, em Genebra\emph{.}
Enfim\emph{,} coube a esses cientistas as mais efetivas políticas
internacionais de defesa da fauna no começo do século passado:
\textbf{``}todo progresso que possa ter se realizado na cooperação
internacional durante os anos de entreguerras coube aos ornitólogos''
(McCormick, 1992: 40).

Os Estados europeus ampliavam sua soberania em colônias na Ásia e na
África, e as questões envolvendo a fauna e flora locais começaram a
repercutir nas metrópoles. Em 1903, foi fundada a organização
internacional Sociedade para a Preservação da Fauna Selvagem do Império,
para estimular a proteção da fauna nas colônias inglesas, alvo de caça
indiscriminada. Na passagem do século, pode-se citar a Convenção para a
Preservação de Animais, Pássaros e Peixes da África, ou Convenção de
Londres\footnote{\emph{http://iea.uoregon.edu/pages/view\_treaty.php?t=1900PreservationWildAnimalsBirdsFishAfrica.EN.txt\&par=view\_treaty\_html}},
assinada em Londres pela Grã-Bretanha, França, Portugal, Itália, Bélgica
e Alemanha, e cujo objetivo era controlar a exploração de marfim, peles
e a caça esportiva nas colônias. O propósito do documento nunca foi
levado adiante, mas marcou um dos primeiros acordos internacionais sobre
a chamada \emph{natureza}. Apenas mais tarde, em 1933, houve um acordo
mais efetivo, com a Convenção de Londres para a Preservação da Fauna e
Flora em seu Estado Natural\footnote{.\emph{http://www.webcitation.org/69hsbtkwR}}\emph{.}
Este tratado reuniu preservacionistas, cientistas e governos e foi
ratificado, aos poucos, pela maioria dos Estados coloniais, visando uma
proteção mais efetiva do que o estabelecido no tratado de 1900, pois se
programou a criação de áreas protegidas nas colônias e ``estabeleceu-se
o precedente de organizações não governamentais desempenhando um papel
técnico consultivo em iniciativas desse tipo'' (McCormick, 1992: 37).

A questão dos recursos hídricos de fronteiras também foi alvo de
convenções multilaterais. Em 1906, foi assinado um dos primeiros acordos
bilaterais na América do Norte acerca de questões do uso dos recursos
hídricos em comum, a Convenção entre México e \versal{EUA} sobre a distribuição
das águas do Rio Grande para irrigação\footnote{\emph{http://www.theodore-roosevelt.com/images/research/trtreaties/treaty60.pdf},}.
Três anos depois, surgiu outro pioneiro acordo bilateral, o Tratado das
Águas Fronteiriças entre a Grã-Bretanha (no interesse do Canadá) e os
Estados Unidos da América.

Em 1909, Paris sediou o Congresso Internacional para a Proteção da
Paisagem\footnote{Congrès International pour la Protetion des
  Paysages\emph{.}
  \emph{http://gallica.bnf.fr/ark:/12148/bpt6k63535711.r}=}. Um dos
objetivos era o de ``encontrar as melhores maneiras de proteger as
fontes mundiais de energia natural, as florestas, as minas d'água, os
lagos, os rios e de fazer um inventário de todos os recursos naturais
indispensáveis à vida econômica de cada país''\footnote{Ibidem: 72.}.
Nessa reunião propôs-se a criação de um organismo internacional, a
Federação Internacional de todas as Sociedades para a Preservação das
Riquezas naturais e regionais\emph{.} Em 1913, em Berna, 17 países
assinaram o Ato de Fundação da Comissão Consultiva para a Proteção
Internacional da Natureza (Estados Unidos, Alemanha, Argentina, Áustria,
Hungria, Bélgica, Dinamarca, Espanha, França, Grã-Bretanha, Itália,
Noruega, Países Baixos, Portugal, Rússia, Suécia e Suíça \footnote{Andreas
  Kley. ``Die Weltnaturschutzkonferenz 1913 in Bern''. Em alemão
  \emph{http://www.rwi.uzh.ch/lehreforschung/alphabetisch/kley/ka/person/publikationen/weltnaturschutz.pdf}}),
resultado dos esforços de Paul Sarasin, naturalista suíço, que
incorporou a proposta do Congresso Internacional de Paris para se formar
uma federação. Entre as primeiras ações estava programada a realização
de uma grande conferência sobre a caça à baleia, o comércio
internacional de peles e de plumagens, e a proteção aos pássaros
migratórios (Ibidem: 39). A eclosão da I Guerra Mundial, entre 1914 e
1918, cancelou os encontros previstos para a consolidação desse órgão
que acabou dissolvido.

Entretanto, em 1923, Paris sediou o 1° Congresso Internacional sobre a
Proteção da Flora, da Fauna e dos Panoramas e Monumentos Naturais,
considerando-se a ação da pesca e da caça em pequena escala, e se
elaborou uma lista de animais em extinção acompanhada de propostas de
reserva para protegê-los. O segundo Congresso Internacional sobre
Proteção da Natureza aconteceu em 1931, em Paris, onde foram
apresentadas iniciativas voltadas à defesa da natureza, destacando-se a
preservação de espaços de beleza cênica e de proteção à fauna e flora
(Franco, 2002).

Quando os \versal{EUA} se recuperavam da crise mundial de 1929 por meio do
\emph{New Deal} (Novo Acordo), que contava com políticas sociais ao lado
de planos de reestruturação da economia, o presidente F. Roosevelt,
sobrinho de T. Roosevelt, aproximou-se, em 1933, dos conservacionistas
ligados a Pinchot e providenciou várias medidas econômicas com uso
planejado de recursos naturais, inclusive a de encaminhamento de
desempregados para o manejo florestal e para ações de combate à erosão
do solo.

Entre 1934 e 1937, ocorreu uma grande catástrofe ambiental nas Planícies
estadunidenses: o \emph{Dust Bowl} (caldeirão de pó), acentuado pela
exploração agrícola que desconsiderou as características ambientais do
local. Tempestades de areia e tornados tornaram o solo ainda mais seco e
erodido, levaram poeira até grandes cidades, e o país chegou a
necessitar importar trigo. Nessa época, a corrente conservacionista
entrou em evidência e se voltou para uma ``perspectiva ecológica mais
abrangente e coordenada'' (McCormick, 1992: 39). A desertificação dessas
grandes áreas agrícolas alertou para um processo similar que se iniciara
na África Oriental com a exploração colonial e favoreceu algumas medidas
de prevenção (Ibidem: 39).

Devido a seu aspecto de alcance transterritorial, mesmo que incipiente,
estes primeiros regulamentos contribuíram para a construção da noção de
meio ambiente planetário. Os pássaros, as bacias hidrográficas, a fauna
marinha e o próprio mar não se circunscrevem a fronteiras. As
proposições para protegê-los respondem a problemas que também atravessam
limites nacionais: na passagem do século \versal{XIX} para o \versal{XX}, havia a moda de
uso de plumas e peles exóticas nos trajes elegantes da burguesia
europeia e seus imitadores pelos cinco continentes; a contaminação do
mar por efluentes de navios de qualquer bandeira; a pesca predatória
diminuindo os estoques da fauna marinha, com destaque para o bacalhau e
as baleias. Associações científicas ou mesmo de ``amantes da natureza''
de países diversos mantinham contatos mediante intercâmbio de
publicações e viagens, e passaram a tomar iniciativas para a formulação
das primeiras proposições transterritoriais de defesa da fauna, da flora
e das paisagens naturais. As propostas de ações de proteção podiam
resultar de encontros nacionais ou internacionais de especialistas ou de
leigos envolvidos com a temática, mas dependiam da força coercitiva do
Estado para se efetivarem internamente ou da negociação internacional
mediante a prática da diplomacia.

No Brasil, o primeiro grande encontro sobre o tema da proteção à
natureza ocorreu em abril de 1934. Foi a Primeira Conferência Brasileira
da Proteção à Natureza, por iniciativa da Sociedade dos Amigos das
Árvores e contando com amplo apoio do Museu Nacional do Rio de Janeiro ―
\versal{MNRJ}. Seus organizadores, como Alberto Sampaio, botânico do \versal{MNRJ},
acompanhavam os encontros internacionais presencialmente e mediante o
acesso a publicações especializadas (Franco \& Drummond, 2009: 108). A
Conferência tinha duas concepções: a preservacionista se propunha a
criar áreas protegidas, e a conservacionista enfatizava as práticas de
manejo das florestas desenvolvidas na Alemanha desde o século \versal{XIX}. Ambas
as posições coexistiam no Brasil e convergiram para um projeto comum: a
preservação das florestas se relacionava à produção agrícola e à
proteção dos solos agriculturáveis.

A proteção à natureza era justificada dentro de um projeto mais amplo de
sociedade em que as riquezas naturais e a paisagem nativa eram fontes da
própria nacionalidade. A formação de uma nação na clave do romantismo do
século \versal{XIX} dependia, também, do reconhecimento do ambiente natural na
história de um povo. A partir da Independência em 1822, a formação
institucional da nação brasileira incorporou esse ideário valorizando
uma ``paisagem nacional'', ao mesmo tempo em que as atividades
econômicas baseadas na agricultura de exportação destruíam a Mata
Atlântica, nativa do sul e sudeste do Brasil. A Conferência de 1934
tinha uma intenção de ``articular os ideais românticos de natureza e
nacionalidade (...) com uma abordagem racionalista, segundo a qual tanto
o mundo natural, como o próprio povo poderiam ser `melhorados' por meio
do aprendizado de uma ciência universal e de comparação com outros
povos'' (Ibidem: 59)\emph{.} Essa `melhora' viria pela promulgação de
leis a serem cumpridas pela autoridade de um Estado intervencionista.

Os protetores da natureza brasileiros tomaram as ideias do pensador
político autoritário Alberto Torres (1865-1917) em \emph{A organização
nacional} (1914) e \emph{O problema nacional brasileiro} (2002) como um
programa de ação, que equacionava ``as preocupações com o mundo natural
e a questão da construção da nação e da identidade nacional'' (Ibidem:
39), mediante o Estado considerado como a instituição capaz de organizar
a sociedade em nome dos interesses coletivos e cuja defesa justificava o
necessário uso da força. Entretanto, o cumprimento da lei deveria ser
também resultado da educação e do esclarecimento à população,
mobilizando a opinião pública para a defesa do patrimônio natural da
nacionalidade brasileira. Segundo seu presidente, Leôncio Corrêa, a
Sociedade dos Amigos das Árvores fora ``fundada por patriotas, e queria
ser a sentinela vigilante do nosso ameaçado patrimônio florestal''
(Ibidem:44)\emph{.} No ano da Conferência fora promulgado pelo Governo
Provisório de Getúlio Vargas o Decreto nº 23.793\footnote{Decreto nº
  23.793, de 23 de janeiro de 1934.
  \emph{https://www.planalto.gov.br/ccivil\_03/decreto/1930-1949/D23793.htm}},
do qual surgiu o primeiro Código Florestal brasileiro, cuja elaboração
contou com estudos e propostas desses cientistas afinados com as
correntes conservacionistas da Europa e América do Norte. Em um de seus
livros dos anos 1930, Alberto Sampaio, o ``amigo da árvore'', elogiou a
Milícia Florestal, um dos braços armados do dirigente fascista italiano
Benito Mussolini, criada em 1926: ``o senhor Mussolini (...) tomou a si
impulsionar a proteção à natureza em seu país, e, reconhecendo a
necessidade de rígida disciplina e continuidade, militarizou o serviço
florestal italiano e deu-lhe um chefe general'' (Sampaio apud Franco \&
Drummond, 2009: 77).

O regime fascista italiano não foi o único a defender a natureza de seu
país com disciplina e constância. Na Alemanha, o nacional-socilaismo
promulgou a Lei do Reich de Proteção à Natureza
(\emph{Reichsnaturschutzgesetz} -- \versal{RNG}), em 1935, resultante de alguns
anos de elaboração dos conservacionistas alemães ainda na República de
Weimar. Os defensores da natureza germânica encontraram sob o nazismo a
força para proteger os animais e as florestas, além de apoio à
agricultura de pequenos proprietários (Radkau, 2008: 261-265), e não se
preocuparam em avaliar os desdobramentos dessa aliança: ``o
conservacionismo alemão agiu baseado em uma filosofia política
extremamente simples: `não há problemas em qualquer disposição legal e
qualquer aliança com o regime nazista, desde que ajude nossa causa!'''
(Uekoetter, 2006: 16).

A natureza defendida pela legislação nazista consistia na natureza da
paisagem \emph{germânica}, símbolo de uma raça ancestral, a raça ariana,
que o regime purificaria ainda mais eliminando outros povos,
especialmente os judeus, do território alemão que se alargava pelas
ações bélicas da guerra declarada em 1939 (Ferry, 2009). Apenas no final
do século \versal{XX} é que se investigaria mais sistematicamente a intersecção
entre uma política nacional de proteção à natureza e um regime
assombroso como o nazismo, responsável pela morte de milhões de pessoas
em campos de concentração e de extermínio, em batalhas e massacres
(Brüggemeier et al., 2005). Estas pesquisas se motivaram também pela
frequente referência ao autoritarismo nazista por parte de opositores ao
ativismo ecológico nas últimas décadas, especialmente o que preconizava
a retirada de pessoas de áreas protegidas e penalidades crescentes para
os recém-criados ``crimes ambientais''\footnote{No início dos anos 1990,
  surgira uma expressão relativamente popular na França:``\emph{écolos
  fachos}'' que ilustra a tendência. Cf. ``Les écolos
  fachos.''.\emph{Actuel}. Octobre 1991, nº 10, pp.8-20.}, exigindo
investigações detalhadas e precisas desse conservacionismo sob o
nazismo, em geral, empreendidas por pesquisadores alemães de um recente
campo de estudos: a história ambiental.

O nazismo, para Foucault, ``é o desenvolvimento até o paroxismo dos
mecanismos de poder novos que haviam sido introduzidos desde o século
\versal{XVIII}. Não há Estado mais disciplinar do que o regime nazista; tampouco
há Estado onde as regulamentações biológicas sejam adotadas de uma
maneira mais densa e mais insistente. Poder disciplinar, biopoder: tudo
isso percorreu, sustentou a muque a sociedade nazista (assunção do
biológico, da procriação, da hereditariedade; assunção também da doença,
dos acidentes). Não há sociedade mais disciplinar e mais previdenciária
do que a que foi implantada, ou em todo caso projetada, pelos nazistas.
(...) ao mesmo tempo em que se tinha essa sociedade universalmente
previdenciária (...) {[}se tinha{]} o desencadeamento mais completo do
poder assassino, do velho poder soberano de matar'' (Foucault, 1999:
301-310). A legislação de defesa da natureza da Alemanha nazista
demonstra ao extremo uma política voltada para dentro de um projeto de
nação, o qual desencadeou um conflito global que só se encerrou com a
explosão de bombas H., massacrando a população de duas cidades no Japão.

Durante os anos de guerra, os mais importantes acordos referentes à
proteção da natureza selvagem, não circunscrita às fronteiras, foram
celebrados no continente americano: o Tratado sobre Aves Aquáticas
Migratórias, de 1937, assinado entre \versal{EUA}, Canadá e México, e
principalmente a Convenção da Proteção da Natureza e Preservação da Vida
Selvagem no Hemisfério Ocidental\footnote{Em português:
  \emph{http://www.pickupau.org.br/mundo/convencao\_fauna\_flora\_america/convencao\_fauna\_flora.htm}.},
aberta para assinatura na União Pan-Americana em Washington, em 12 de
outubro de 1940, e assinada por Brasil, Colômbia, Equador, Bolívia,
Cuba, El Salvador, Nicarágua, Peru, Estados Unidos, Haiti, Venezuela,
Trinidad Tobago, Suriname, Panamá, Guatemala, Paraguai, República
Dominicana, Costa Rica, Argentina, Uruguai, México e Chile. Os objetivos
dessa Convenção eram: preservar da extinção todas as espécies e gêneros
da flora e fauna nativas da América; e preservar áreas de beleza
extraordinária, de formações geológicas raras ou de valor histórico ou
científico. Os países contratantes se comprometiam a criar áreas
protegidas, além de leis e regulamentos internos de proteção da fauna e
flora, e as informações e estudos científicos resultantes dessas ações
de proteção deveriam ser compartilhados entre os signatários.

Em 1941, depois de um litígio de 17 anos, saiu a sentença no caso da
Fundição de Trail. Instalada no Canadá desde 1896, a fábrica lançava
efluentes nos cursos d'água atingindo estadunidenses, na maioria
agricultores, que se organizaram e fizeram com que o governo submetesse
o caso a um Tribunal Arbitral, em 1925. Esse caso registra a primeira
decisão da jurisdição internacional relativa ao meio ambiente, ``que
declara um Estado não ter o direito de usar o seu território, ou
permitir o seu uso de modo a que emissões gasosas causem prejuízo no
território alheio ou nas propriedades das pessoas que nele se
encontrem'' (Antonio Neto, 2009), e se tornou uma referência para o
Direito Internacional Ambiental.

Até o final dos anos 1960, a noção de meio ambiente, conforme hoje
utilizada, praticamente não existia. Os sentidos de \emph{meio} e de
\emph{ambiente} e da locução \emph{meio ambiente} dispersavam-se em
alusões a atmosferas morais, habitats de espécies vivas, à geografia de
ecossistemas, locais específicos. O minucioso artigo de Spitzer
publicado em 1942 não permite uma previsão inequívoca de que, em menos
de trinta anos, os termos \emph{meio} e \emph{ambiente} se conjugariam
para abarcar a economia, a biosfera e as demais questões pontuais que se
agruparam sob a égide de Conferências internacionais, gerando novas
instituições e recomendando novas condutas das pessoas e de grupos
sociais em um âmbito planetário.

Foram dispersos os trajetos até se chegar ao Acordo de Paris sobre o
clima. As práticas e os juízos acerca da natureza, seus animais,
vegetação, rios e mares e suas situações decorrentes da industrialização
e colonizações vieram de organizações da sociedade civil, cientistas,
inovadores estudos sobre ecologia, artistas e escritores marcados pelo
romantismo, atenção com comércio de sementes, relações em zonas
fronteiriças, defesa de paisagens imperturbáveis, argumentações sobre a
necessidade de parques nacionais, que procedem do século \versal{XIX} e chegam ao
\versal{XX}.

A questão se torna cada vez mais internacional, as defesas da natureza
chegam a ser militarizadas e a funcionar para o racismo de Estado do
nazismo, fazendo com que as conferências internacionais ganhem
importância. Com os testes com energia atômica, constata-se que não se
estava somente no campo da paz pelo fim das guerras. Estudos, pesquisas,
reuniões, organizações da sociedade civil, empresários, ativistas e
políticos confluem para questionamentos sobre os usos da natureza e sua
equivocada dissociação ou subordinação ao humano.

A \versal{ONU} emerge como centro catalizador e irradiador de discussões visando
um consenso ambiental, e neste trajeto, aos poucos, chega-se à
sustentabilidade como meta comum dos Estados, assim como o tema do clima
foi responsável por conectar as variadas sugestões de preservacionistas
e conservacionistas e a sociologia absorve a noção de ambiente para
designar espaços diferenciados no meio urbano.

O meio ambiente se configura como dispositivo; a ecopolítica começa a
ser uma designação para onde se destinam as sociedades humanas com
segurança, envolvendo transterritorialidades, populações e uma concepção
de vida relacionada ao universo.

\chapter{O dispositivo meio ambiente}

O ano de 1945 marca a destruição de Hiroshima e Nagasaki por duas bombas
de Hidrogênio lançadas pelos \versal{EUA}, em agosto, e o encerramento da \versal{II}
Guerra Mundial. Iniciou-se, desde então, uma nova configuração das
forças internacionais, e o tema da segurança global se impôs. Surgiu a
Organização das Nações Unidas ― \versal{ONU}, pela qual foram programadas as
ações de reconstrução de guerra na Europa pelo Plano Marshall, ao mesmo
tempo em que se formava o bloco comunista nos países do Leste europeu.

Além da \versal{ONU}, outros órgãos internacionais ligados a ela são criados,
entre eles a Organização Mundial de Saúde -- \versal{OMS} e a \versal{UNESCO}. Nas Nações
Unidas predominavam, especialmente no Conselho Econômico Social --
\versal{ECOSOC}, os temas do desenvolvimento econômico e da aplicação da
tecnologia de ponta na utilização desses recursos. A prioridade das
agências internacionais era reabilitar o mundo no pós-guerra e combater
a fome mediante o crescimento da produção e fornecimento de alimentos,
assunto já pautado pela extinta Liga das Nações. A agência criada para
tratar da fome dentro de um amplo planejamento foi a \versal{FAO} (Food and
Agriculture Organization of the United Nations). Os assuntos relativos
ao que hoje se define como meio ambiente ― mar, fauna marinha, fauna,
vegetação ―, que estavam sob a égide de tratados entre países,
parcialmente multilaterais, mediados ou não pela Liga das Nações, foram
assimilados pelo sistema da \versal{ONU}.

Durante a década que se seguiu ao fim da guerra, o mundo político se
dividiu em dois blocos, um liderado pelo capitalismo estadunidense,
outro pelo socialismo soviético, em conflito não diretamente bélico. A
\versal{ONU} abarcava membros dos dois grupos no Conselho de Segurança e muitos
dos seus esforços se concentraram em equacionar a tensão entre eles para
a manutenção da segurança do planeta, evitando-se uma possível e
devastadora nova guerra mundial.

No interior do sistema da \versal{ONU}, nos anos 1960, dois conjuntos de tratados
envolveram temas inéditos com probabilidade de afetar a continuidade da
vida humana. Um destes conjuntos se refere, obviamente, ao uso bélico da
energia nuclear. Apesar da desconfiança de parte da opinião pública
mundial, os testes nucleares a céu aberto continuaram até a década de
1960, trazendo contaminação radioativa, espalhada pelo vento a locais
distantes da explosão. A ameaça da radioatividade tornou-se real por
meio da comprovação científica de seus efeitos, para além das evidências
inscritas nos corpos dos afetados pelas explosões, ambas amplamente
divulgadas. Nesse sentido, o Tratado de Proibição Parcial de Testes
Nucleares, de 1963, por iniciativa de União Soviética, Grã-Bretanha e
Estados Unidos, é considerado o acordo internacional pioneiro em que a
segurança mundial associada ao meio ambiente tornou-se objetivo
prioritário (McCormick, 1992: 19). O destaque recaiu sobre um
desarmamento genérico, quando o objetivo era procurar encerrar a
contaminação do ambiente humano pela radioatividade. Em 1968, com o
Tratado da Não Proliferação de Armas Nucleares\footnote{\emph{http://www.cedin.com.br/wp-content/uploads/2014/05/Tratado-de-N\%C3\%A3oProlifera\%C3\%A7\%C3\%A3o-de-Armas-Nucleares.pdf}},
assinado pelo Brasil em 1992, os signatários que não tinham ainda
produzido bombas atômicas se comprometiam a não fabricá-las, além de
permitirem inspeções periódicas da Agência Internacional de Energia
Atômica\emph{.}

O assunto permaneceu gerando discussões sobre a desigualdade entre os
países. Com o pretexto de desenvolver fontes de energia, muitos Estados
deixavam transparecer a suspeita de estarem criando a bomba, como
ocorre, por exemplo, em relação à Coréia do Norte, que, de fato,
recentemente realizou testes nucleares. No caso do Brasil, a questão
está revestida de decisões que envolvem sigilo militar, como o Acordo
Brasil-Alemanha, firmado durante a ditadura civil-militar para a
construção das Usinas de Angra e o projeto Aramar em São Paulo (\versal{CEDI},
s/d), culminando na relutância em assinar um Protocolo Adicional do
Tratado. A energia nuclear não saiu da pauta da defesa nacional. No
documento Estratégia Nacional de Defesa, de 2008, aparece um motivo da
demora brasileira em assinar um protocolo adicional mais restritivo do
Tratado de Não Proliferação de Armas Nucleares: ``O Brasil zelará por
manter abertas as vias de acesso ao desenvolvimento de suas tecnologias
de energia nuclear. Não aderirá a acréscimos ao Tratado de
Não-Proliferação de Armas Nucleares destinados a ampliar as restrições
do Tratado sem que as potências nucleares tenham avançado na premissa
central do Tratado: seu próprio desarmamento nuclear'' (Azevedo, 2010).
A questão específica dos testes nucleares é, enfim, regida pelo Tratado
de Interdição Completa de Testes Nucleares\emph{,} de 1996, que proíbe
explosões nucleares sob qualquer finalidade ou lugar. Contudo, por não
ter sido ratificado por todos os Estados, este Tratado não entrou em
funcionamento\footnote{Comprehensive Nuclear Test-Ban Treaty.
  \emph{http://www.ctbto.org/member-states/country-profiles/?country=24\&cHash=25626fb675}}.

O outro conjunto se refere ao espaço exterior. Em 1959, a União
Soviética acelerou seu programa espacial e procurou criar uma era de
otimismo ao lançar o primeiro satélite em órbita, o Sputnik, e, dois
anos depois, em 1961, o primeiro homem no espaço: Yuri Gágarin. Os
comentários desse primeiro astronauta sobre a beleza do planeta Terra
foram divulgados como uma mensagem de paz e humanismo, enquanto o uso do
espaço exterior passou a ser realidade. A preocupação em garantir o uso
do espaço ``para o benefício da humanidade'' já era anterior ao
lançamento do Sputnik. Em 1958, a \versal{ONU}\footnote{Texto da resolução:

  \emph{http://www.unoosa.org/oosa/SpaceLaw/gares/html/gares\_13\_1348.html}.}
criou o Comitê do Uso Pacífico do Espaço Exterior\emph{,} para
estabelecer a cooperação entre os Estados no assunto \footnote{Committee
  on the Peaceful Uses of Outer Space:

  \emph{http://www.unoosa.org/oosa/COPUOS/copuos.html}.}. Como resultado
das atividades do Comitê, apareceu o Tratado do Espaço Exterior ou
Tratado sobre os Princípios que Governam as Atividades dos Estados na
Exploração e Uso do Espaço Exterior, Incluindo a Lua e Outros Corpos
Celestes\footnote{Em espanhol:
  \emph{http://www.unoosa.org/pdf/publications/STSPACE11S.pdf}.}, em
1967, considerado o marco na legislação internacional.

O tratado proíbe que qualquer Estado reivindique a posse de corpos
celestes, pois estes pertencem a toda humanidade, e proíbe armas
nucleares e de destruição em massa na órbita da terra. A ameaça do uso
de armas de destruição em massa, inclusive a partir do espaço sideral,
contrastava com a imagem pacífica, cooperativa e humanista que os
astronautas tentavam transmitir. Em final de 1968, a espaçonave
estadunidense Apollo 8 saiu da órbita da Terra e deu a volta na Lua. As
primeiras fotos da Terra flutuando no espaço foram divulgadas em
revistas e televisão obtendo grande repercussão. Os astronautas da
Apollo 11 pisaram na lua em 1969, deixando no solo lunar pegadas, uma
bandeira dos \versal{EUA} e a frase em uma placa de metal: ``aqui os homens do
planeta Terra puseram pela primeira vez os pés na Lua. Julho de 1969 \versal{DC}.
Viemos em paz em nome de toda a humanidade''\footnote{Biografia de Neil
  Armstrong: \emph{http://www.imdb.com/name/nm0035842/bio}}.

Radioatividade e espaço sideral alimentaram a noção de meio ambiente ao
mostrarem a abrangência do humano e ao fazê-lo componente de um ambiente
planetário. As questões de segurança se mesclaram às questões ambientais
planetárias nessa fase inicial de construção da noção de meio ambiente.
Devido à força das verdades ecológicas em torno da conservação da vida,
consolidou-se uma associação mais forte entre meio ambiente e a noção de
natureza, reforçando-se a necessidade de evitar ou, ao menos, minimizar
ameaças ao planeta por meio de ações de manutenção ou restauração de
seus aspectos naturais. Portanto, conservar o meio ambiente mediante
instituições cada vez mais abrangentes surgiu, inicialmente, como
resposta a crescentes sentimentos de medo e de insegurança produzidos
por catástrofes e desastres, pela ameaça de guerras nucleares, pelos
efeitos deletérios da poluição, fatos amplamente divulgados por uma
mídia que atingia todos os continentes em meados do século \versal{XX}.

A proposta do \versal{ECOSOC} de organizar um encontro para comentar os problemas
relativos ao meio ambiente, caracterizados como decorrentes de
``catástrofe'' ecológica, procede não apenas de denúncias acerca de uma
``crise ambiental'', mas de insatisfações e pressões de autoridades de
países pobres em relação ao protecionismo ou ausência de incentivos a
eles em relação à produção e ao comércio global. A incorporação de novos
países-membros no sistema, incluindo muitas ex-colônias europeias,
acarrretou conflitos no interior do sistema \versal{ONU}, envolvendo ``o acirrado
antagonismo entre protecionistas e desenvolvimentistas'' em questões
relativas às regras comerciais e econômicas (Amorim, 2015: 117). Havia
uma crítica contundente às propostas de colocar limites malthusianos ao
crescimento e ao ambientalismo que predominavam entre os especialistas
das nações industrializadas. Discutidas apenas em seus ambientes, as
propostas de \emph{crescimento zero} ensejavam medidas drásticas de
intervenção dos governos nos rumos do desenvolvimento econômico e
causavam desconfiança nos dirigentes dos países chamados
subdesenvolvidos ou em desenvolvimento. O tema do limite de crescimento
não entrara na \versal{ONU}, que se mantinha preocupada em elaborar programas
para o desenvolvimento das trocas comerciais e a cooperação entre os
países. Anunciavam-se revisões.

O tema do desenvolvimento econômico era central na \versal{ONU} nos anos 1950, e
a década de 1960 foi por ela declarada a Década do Desenvolvimento. Os
esforços para equacionar a desigualdade entre economias envolveram novas
instituições, destacando-se, em 1964, a Primeira Conferência das Nações
Unidas para o Comércio e Desenvolvimento (United Nations Conference on
Trade and Development ― \versal{UNTACD}), programada para reuniões quadrienais.
Mesmo assim, a perspectiva de uma discussão sobre o uso dos recursos
naturais e a poluição provocou desagrados na maioria dos Estados-membros
da \versal{ONU}. Quando a Conferência Ambiental foi sugerida, imediatamente foi
vista como uma iniciativa de Estados desenvolvidos, no sentido de coibir
o crescimento dos demais, revestida do pretexto de se encontrar medidas
contra a chamada deterioração do ambiente e a poluição. Os Estados
membros buscavam evitar que, a pretexto de controle da poluição e defesa
da natureza, empresas poluidoras deixassem de se ampliar em seus
territórios, ou que investimentos fossem bloqueados. Tratava-se de
percorrer o labirinto.

Cogitou-se que fossem estabelecidos programas de ação para equacionar
tais problemas, mediante a cooperação internacional. Para organizar o
encontro em 1972, em Estocolmo, o Secretário das Nações Unidas, U-
Thant, nomeou Maurice Strong, biólogo de formação, empresário canadense
do setor de petróleo e colaborador assíduo da instituição com eficácia,
até ser convidado a assumir, vinte anos depois, a Conferência Rio 92. No
primeiro momento, Strong trabalhou para obter a adesão das economias
industrializadas para participarem da Conferência, ao mesmo tempo em que
visitou dezenas de países considerados não desenvolvidos buscando reunir
autoridades locais para assegurar debates em reuniões preparatórias e na
Conferência, sublinhando não haver discussão fechada e decisões
acordadas previamente.

Na primavera de 1971, em Founex, nos arredores de Genebra, ocorreu um
seminário para debater as questões do desenvolvimento econômico com a
presença de 27 especialistas de várias nações\footnote{Relatórios de
  Founex. \emph{http://www.mauricestrong.net/}}. O representante do
Brasil foi o diplomata Miguel Ozório de Almeida, que posteriormente veio
a ser o chefe da delegação brasileira em Estocolmo. A posição brasileira
nesse encontro inicialmente foi de confronto ao defender do programa de
crecimento econômico traçado pelo governo civil-militar do país.
Evocando a soberania dos Estados nacionais sobre seus recursos naturais,
estabelecida inclusive por uma Resolução da \versal{ONU} de 1962 para atender os
novos Estados das ex-colônias\footnote{Resolução 1.8003 da \versal{XVII}
  Assembleia Geral da \versal{ONU}, de 14 de dezembro de 1962.}, a representação
brasileira se manifestou contra qualquer medida de controle das
atividades econômicas por injunções internacionais recorrendo ao
princípio da soberania nacional. Porém, não deixou de aludir à criação
de medidas financeiras de apoio por parte de países ricos às ações de
proteção ambiental em países pobres.

O Encontro de Founex foi uma prévia do que ocorreria em Estocolmo,
quando transpareceu o conflito Norte-Sul relativo ao chamado
desenvolvimento e com o Brasil se mantendo crítico às tentativas de
imposições de limites ao crescimento. Os seminários preparatórios da
Conferência, enfim, marcaram a inclusão no tema `meio ambiente', e não
apenas de questões ecológicas ou efeitos de poluição na saúde humana.
Trouxeram problemas como fome, miséria, más condições de habitações,
saneamento, doenças. ``A noção de meio ambiente se complementou com
questões sociais. A noção de meio incluiu algumas questões que, um
século antes, estariam restritas ao que se concebia como população''
(Carneiro, 2012: 9).

Outra medida para organizar o encontro em Estocolmo foi solicitar um
estudo preparatório a ser discutido na reunião. Dessa iniciativa
resultou o livro \emph{Uma terra somente}, de Bárbara Ward e René Dubos
(1972), preparado com o auxílio de 152 consultores de todos os
continentes. O objetivo principal do livro era apresentar a Terra como
um patrimônio comum da Humanidade, com a finalidade de sensibilizar cada
um para uma consciência e possíveis soluções. Barbara Ward cunhou, em
1965, a expressão ``espaçonave terra'', amplamente utilizada na época
para assinalar a percepção do \emph{habitat} comum da humanidade: o
planeta. Na Introdução, ela comenta: ``a Humanidade agora está espalhada
sobre toda superfície do globo (...) aprender a manejá-lo
inteligentemente é um imperativo urgente. O Homem deve aceitar a
responsabilidade de administração da Terra. A palavra administração
implica, naturalmente, governo para o bem comum. (...) na prática, a
responsabilidade da \versal{ONU} na Conferência era claramente a de definir o que
deveria ser feito para manter a Terra como um lugar adequado para a vida
humana, não somente agora, mas para as gerações futuras'' (Ward \&
Dubos, 1972: 20-21).

Entre 5 e 12 de junho de 1972, em Estocolmo, ocorreu a Primeira
Conferência das Nações Unidas sobre o Meio Ambiente Humano, com
representantes de 113 países, incluindo a China, mais 250 organizações
não-governamentais e representantes de outros organismos da \versal{ONU}. As
discussões foram acirradas, apesar das reuniões prévias em Founex e Nova
Iorque, e giraram em torno da relação entre desenvolvimento econômico e
deterioração ambiental, tema pela primeira vez destacado na agenda
internacional, como um explícito conflito entre os chamados Primeiro e
Terceiro Mundos. Estados e sociedades civis organizadas compuseram sob a
gestão da \versal{ONU}, com base em estudos preparatórios de consulta e
sugestões, o primeiro grande encontro voltado a equacionar uma
governamentalidade planetária. A ecopolítica começava a ser visível.

Entre os Estados-membros não desenvolvidos da \versal{ONU} e fora do bloco
socialista, o temor de um incentivo ao estancamento do crescimento da
economia, como sugerido por estudos sobre a capacidade de suporte do
planeta, contribuiu para que o tema específico da pobreza entrasse com
destaque nessa primeira Conferência Ambiental e se tornasse parte da
questão do meio ambiente humano como problema a ser equacionado. Tensões
sobre a questão do desenvolvimento que não foram resolvidas em outras
reuniões da \versal{ONU} retornaram em Estocolmo agregadas à questão do meio
ambiente. Na plenária do Encontro, o discurso de Indira Gandhi, único
chefe de Estado presente além do Primeiro Ministro sueco, foi
emblemático e posteriormente citado com regularidade: ``por um lado, os
ricos olham com desconfiança nossa contínua pobreza, por outro, eles nos
advertem contra os seus próprios métodos. Não queremos empobrecer o
ambiente ainda mais, mas não podemos esquecer por um momento a pobreza
sombria de um grande número de pessoas. Não são a pobreza e a
necessidade os maiores poluidores?''\footnote{Discurso de Indira Gandhi
  na Conferência de Estocolmo em 1972.

  \emph{http://lasulawsenvironmental.blogspot.com.br/2012/07/indira-gandhis-speech-at-stockholm.html}}.

Houve consenso em torno da afirmação de que a ``pobreza é a pior
poluição'', alertando para os perigos de se limitar o crescimento de
países pobres. O Papa Paulo \versal{VI} enviou uma mensagem para a abertura da
Conferência, elogiando este ``primeiro gesto de cooperação mundial''.
Dentre os pontos principais da sua mensagem, destacavam-se a ênfase no
meio ambiente como patrimônio da humanidade, que não deve dele se
apropriar de modo egoísta; a importância do tema do desenvolvimento, que
serviu de orientação central à Conferência, na busca não só de um
``equilíbrio ecológico'', mas também de um ``equilíbrio entre países
industrializados e a periferia''; e a reiteração de que ``a miséria é a
pior poluição''\footnote{Não há versão oficial em português, apenas em
  espanhol, inglês e francês. Em espanhol: ``Mensaje de su Santidad D.
  Pablo \versal{VI} a la Conferencia de las Naciones Unidas sobre el Medio
  Ambiente''\emph{.}
  \emph{http://w2.vatican.va/content/paul-vi/es/messages/pont-messages/documents/hf\_p-vi\_mess\_19720605\_conferenza-ambiente.html}},
com base na atuação da Igreja da época junto aos pobres.

A Conferência produziu a Declaração sobre o Meio Ambiente
Humano\footnote{Declaração de Estocolmo sobre o meio ambiente humano.

  \emph{http://www.direitoshumanos.usp.br/index.php/Meio-Ambiente/declaracao-de-estocolmo-sobre-o-ambiente-humano.html}},
uma asseveração de princípios de comportamentos e responsabilidades
sobre as questões ambientais. A declaração contempla os 26 princípios
agrupados em cinco blocos: 1) os recursos naturais deveriam ser
conservados, a capacidade da terra de produzir recursos renováveis
deveria ser mantida e os recursos não renováveis deveriam ser
compartilhados; 2) o desenvolvimento e a preocupação ambiental deveriam
andar juntos e os países desenvolvidos deveriam ajudar os não
desenvolvidos a implantarem políticas de proteção ambiental; 3) cada
país deveria estabelecer seus padrões de gestão ambiental, explorar seus
recursos sem prejudicar ou colocar em perigos outros países, e contar
com a cooperação internacional na questão do meio ambiente; 4) a
poluição não excederia a capacidade do meio ambiente de se recuperar, e
a poluição dos mares deveria ser evitada; 5) ciência, educação,
tecnologia e pesquisa deveriam ser utilizadas para a proteção ambiental
(McCormick, 1992: 110).

A Declaração de Estocolmo veio a ser equivalente ou mesmo uma ampliação
da declaração dos Direitos Humanos da \versal{ONU} (Soares, 1995: 82) ao incluir
entre os direitos fundamentais de cada um o acesso a um meio ambiente de
qualidade. O primeiro princípio afirma\emph{:} ``o homem tem o direito
fundamental à liberdade, à igualdade e ao desfrute de condições de vida
adequadas em um meio ambiente de qualidade tal que lhe permita levar uma
vida digna e gozar de bem-estar, tendo a solene obrigação de proteger e
melhorar o meio ambiente para as gerações presentes e
futuras''\footnote{Ibidem.}\emph{.} Nesta Declaração, o meio ambiente
humano é definido tanto como o natural quanto artificial: ``o homem é ao
mesmo tempo obra e construtor do meio ambiente que o cerca (...). Os
dois aspectos do meio ambiente humano, o natural e o artificial, são
essenciais para o bem-estar do homem e para o gozo dos direitos humanos
fundamentais, inclusive o direito à vida''\footnote{Ibidem.}.

Outro resultado formal de Estocolmo 72 foi um Plano de Ação que
convocava os governos, as organizações internacionais e, principalmente,
os diversos órgãos das Nações Unidas, como \versal{FAO}, \versal{UNESCO} e \versal{OMS}, a
cooperarem na busca de soluções para uma série de problemas agrupados
como ambientais. Houve o compromisso de que o tema \emph{meio ambiente}
entraria oficialmente na agenda de cada país integrante do encontro.

O plano compunha-se de três grandes grupos de ação: um programa global
de avaliação ambiental, denominado Earthwatch (Vigilância da Terra) que
previa a avaliação e a revisão das condições ambientais, a pesquisa, o
monitoramento e o intercâmbio de informações; um programa de gestão
ambiental que incluía a definição de metas e o planejamento, consultas
internacionais e acordos; e por fim, medidas para apoiar as ações
nacionais e internacionais de gestão e avaliação mediante troca de
informações, financiamento, cooperação internacional, educação e
treinamento\footnote{Report of the United Nations Conference on Human
  Environment. June, 1972.

  \emph{http://www.un-documents.net/aconf48-14r1.pdf}}. O Plano de Ação
contava com 109 recomendações distintas, cada uma detalhada em itens.
Metade tratava de conservação de recursos naturais, e o restante cobria
temas como assentamentos humanos, rurais e urbanos; poluição,
desenvolvimento e meio ambiente, educação e informação (McCormick, 1992:
110-111). Tais recomendações, direta ou indiretamente, dialogavam com
convenções, tratados e programas da \versal{ONU} e de órgãos do sistema,
referentes à biosfera, saúde humana, pobreza, recursos da natureza e
assentamentos humanos. Essas práticas se agruparam, mas nem todos os
assuntos que hoje possuem interfaces com o meio ambiente estiveram
contemplados ou registrados nos documentos da Conferência.

O Plano de Ação ainda previa a criação de um órgão da \versal{ONU} voltado para o
meio ambiente no seu aspecto mais ecológico, os recursos naturais,
estimulando a coordenação do Plano de Ação de Estocolmo articulado a
outros programas que poderiam surgir. Em 1972, foi criado o Programa das
Nações Unidas para o Meio Ambiente ― \versal{PNUMA}, com sede em Nairobi, capital
do Quênia, cabendo-lhe implementar o Plano de Ação e o programa de
vigilância Earthwatch\emph{,} rede planejada para pesquisar, monitorar e
avaliar as tendências e processos do meio ambiente, identificando riscos
ambientais e situação de recursos naturais. No entanto, como o \versal{PNUMA}
fora concebido sem um orçamento condizente com as tarefas a ele
destinadas e sem poderes executivos para implantá-las em seus primeiros
anos, teve sua consolidação dificultada.

Ao longo da sua história política, marcada pelo ponto de inflexão do
acontecimento Conferência de Estocolmo, o meio ambiente incorporou em
sua definição e às ações levadas adiante em seu nome instituições,
filosofias, planejamentos territoriais, conhecimentos científicos,
comportamentos, paisagens, emoções e afetos; abrigou em sua definição
elementos heterogêneos como saúde humana e fauna; reorientou e forneceu
outras dimensões aos elementos que agregou sob a promessa de salvação da
vida e de segurança do planeta.

A noção de meio ambiente passou a funcionar a partir de \emph{1968} como
um \emph{dispositivo}, no sentido inaugurado por Michel Foucault. Um
dispositivo promove conexões entre elementos heterogêneos, oferece um
novo campo de racionalidades, um novo campo de verdades e qualificação
de saberes, e principalmente se conforma como uma estratégia que
responde a uma urgência (Foucault, 2001: 299). Um dispositivo programa o
que deve ser percebido e o que deve ser qualificado como verdade,
atualiza-se a qualquer momento, \emph{aperfeiçoando-se} para cumprir
outras funções e adaptando-se a novas e velozes exigências.

Uma vez consolidado o \emph{dispositivo meio ambiente}, tornou-se
possível a análise sobre o redimensionamento de questões segmentadas ou
dispersas, não apenas referentes aos temas trabalhados pelos órgãos da
\versal{ONU} ou por setores dos governos dos Estados, mas também aqueles
produzidos por organizações não governamentais, algumas de alcance
internacional, como a União Internacional para a Conservação da Natureza
― \versal{UICN} (International Union for Conservation of Nature and Natural
Resources ― \versal{IUCN}), Friends of the Earth e World Wildlife Fund
―\versal{WWF}. O encontro de Estocolmo proporcionou às \versal{ONG}s conectarem-se
aos temas comuns e a ganharem força na produção da verdade. Apesar de
não terem tido voz nas reuniões plenárias, as \versal{ONG}s realizaram reuniões
paralelas durante o evento, uma prática inovadora que desembocou na
institucionalização da participação desses grupos pelo sistema \versal{ONU},
favorecendo a criação de outras organizações transnacionais, como o
Greenpeace, e inúmeras repercussões em organizações locais\emph{.}
Voltadas para a defesa do meio ambiente, passaram a participar mais
ativamente nas políticas de governos de Estados que, por sua vez,
passaram, em graus diversos, a se adequar ao acordado na Conferência,
criando legislações e normas internas para a questão ambiental.

A gestão de temas antes deixados a critério dos empresários passou a ser
alvo de ações conjuntas, como por exemplo, o trânsito de lixo tóxico e
produtos perigosos pelo planeta, regulado em 1989 pela Convenção da
Basileia\footnote{Convenção da Basiléia\emph{.}
  \emph{http://www.mma.gov.br/cidades-sustentaveis/residuos-perigosos/convencao-de-basileia}}.
Além disso, o \emph{dispositivo meio ambiente} comporá com o Direito
Internacional, posto que a Conferência de Estocolmo passou a ser
considerada o marco do início do direito ambiental internacional
(Amorim, 2015: 118). O \emph{dispositivo meio ambiente} abarca
organizações, Estados, legislações, normas, conduções de condutas
esperadas, variações em regulamentações, regulações orquestradas,
situação de espaços, etnias e povos, programáticas científicas e
educacionais, um domínio, uma palavra, inúmeros programas que atravessam
populações e divíduos, formas de educar e cuidar do planeta degradado,
visando melhorias e o futuro das gerações, a conduta de cada um. O
\emph{dispositivo meio ambiente} expressa a governança sob as relações
compartilhadas a respeito do humano conectado à natureza.

À parte da crise econômica mundial reconhecida em 1973 com o aumento dos
preços do petróleo em quase 300\%, agravada por uma mudança cambial e
desaceleração da produção, o assunto ambiental reconfigurou sua
capacidade de mobilizar e outras propostas se aglutinaram ao
dispositivo. A Declaração de Cocoyoc, de 1974, resultado de um evento no
México organizado pelo \versal{PNUMA} e pela \versal{UNTACD}, mostra uma avaliação das
ações da \versal{ONU} e a busca por soluções: ``trinta anos se passaram desde que
a assinatura da Carta das Nações Unidas lançou o esforço para
estabelecer uma nova ordem internacional. Hoje, esta ordem atingiu um
ponto crítico. As esperanças de criar uma vida melhor para toda a
família humana têm sido amplamente frustradas. (...) Pelo contrário,
mais pessoas estão com fome, doentes, sem abrigo e analfabetos hoje, do
que quando a \versal{ONU} se configurou. Ao mesmo tempo, novas e inesperadas
preocupações começaram a escurecer a perspectivas internacionais. A
degradação ambiental e a crescente pressão sobre os recursos levantam a
questão de saber se os `limites externos' da integridade física do
planeta não poderiam estar em risco. Partindo do pressuposto de uma
ordem econômica internacional mais justa, alguns dos problemas da má
distribuição de recursos e uso do espaço poderiam ser tratados por uma
alteração na geografia industrial do mundo. (...) O ideal que buscamos é
um mundo cooperativo harmonizado em que cada parte é um centro, vivendo
às próprias custas, em parceria com a natureza e em solidariedade com as
gerações futuras'' \footnote{Symposium on Patterns of Resource Use,
  Environment and Development Strategies. Declaração de Cocoyoc.
  \emph{http://helsinki.at/projekte/cocoyoc/COCOYOC\_DECLARATION\_1974.pdf}}.

A crise econômica, principalmente financeira, forneceu mudanças nas
políticas e na gestão dos recursos naturais no planeta, ao mesmo tempo
em que diminuiu a oferta de recursos para a proteção ambiental. O
desafio à conservação do planeta passou a ser o de equacionar os
desdobramentos da recessão econômica agregada ao aumento dos desastres e
catástrofes com efeitos mortíferos de difícil solução: o acidente
nuclear da usina estadunidense Three Miles Island, em 1979; o vazamento
de gás mortal em Bhopal, na Índia, e o incêndio na Vila Socó, em
Cubatão, Brasil, ambos em 1984; a explosão do reator nuclear de
Tchernóbil, na Ucrânia, em 1986; a contaminação por Césio 137
proveniente do lixo radiotivo do Instituto Goiano de Radioterapia,
Brasil, em 1987. Os conflitos em torno dos temas que estavam sendo
reunidos pelo \emph{dispositivo meio ambiente} aumentaram e a
perspectiva do esgotamento dos recursos continuava na pauta dos
governos, em pequena e em grande escala. Apesar disso, o encontro de
Estocolmo estimulou em muitos locais a criação de normas ou leis, mesmo
que incipientes, de defesa de aspectos do meio ambiente, acirrando
disputas.

Foi neste momento de reconfiguração do sistema econômico mundial que o
reflexo no Brasil, em linhas gerais, repercutiu na desaceleração do
processo de crescimento econômico, também conhecido como ``milagre
brasileiro''. Entretanto, algumas ações para evitar ou ao menos
controlar os efeitos da poluição que já afetavam cidades
industrializadas passaram a ser colocadas em funcionamento, ao menos nos
locais de maior concentração industrial. Em 28 de fevereiro de 1967, o
Decreto-Lei nº 303 definia o conceito de poluição e criava o Conselho
Nacional de Controle da Poluição; pela primeira vez, o termo
``meio-ambiente'' aparece em um texto legal brasileiro de abrangência
nacional\footnote{Decreto-Lei nº 303, de 28 de fevereiro de 1967.
  \emph{http://www.planalto.gov.br/ccivil\_03/decreto-lei/1965-1988/Del0303.htm}}.
Na mesma data, criava-se a Política Nacional de Saneamento pelo
Decreto-Lei nº 248, legislação revogada meses depois pela nova Política
Nacional de Saneamento\footnote{Lei nº 1538, de 26 de setembro de 1967.
  \emph{http://www.planalto.gov.br/ccivil\_03/decreto-lei/1965-1988/Del0303.htm}}.
Nesta são incluídos ações e o controle da poluição industrial (Bursztyn
\& Persegona, 2008: 193), ainda circunscritos no âmbito da engenharia
sanitária e da saúde, sem ``campo ambiental'' institucionalizado.

Nos Estado de São Paulo, fundada em 1968, a Companhia de Tecnologia de
Saneamento ― \versal{CETESB} incorporou a Superintendência de Saneamento
Ambiental ― \versal{SUSAM}, vinculada à Secretaria da Saúde, e estava encarregada
de controlar e monitorar a poluição e seus efeitos. Anteriormente, a
\versal{SUSAM} absorvera a Comissão Intermunicipal de Controle da Poluição das
Águas e do Ar ― \versal{CICPAA} que atuava, desde 1960, nos municípios de Santo
André, São Bernardo do Campo, São Caetano do Sul e Mauá, cidades de
concentração industrial da Grande São Paulo\footnote{\emph{http://www.cetesb.sp.gov.br/2015/07/24/no-aniversario-de-47-anos-conheca-10-fatos-importantes-sobre-a-cetesb-2/}.
  Atualmente a \versal{CETESB} denomina-se Companhia Ambiental do Estado de São
  Paulo e está vinculada à Secretaria de Estado de Meio Ambiente.}.

O Brasil teve uma atuação importante nas discussões da Conferência de
Estocolmo, levando uma posição de defesa do desenvolvimentismo
confrontando uma proposta mais preservacionista do meio natural. No
entanto, cumpriu as recomendações do encontro. Em 1973, criou-se a
Secretaria Especial do Meio Ambiente, ligada à Presidência da República,
tendo como secretário o ambientalista Paulo Nogueira Neto, criador, em
1958, da Fundação Brasileira de Conservação da Natureza ― \versal{FBCN}. Em sua
gestão, o Brasil aderiu ao Programa da \versal{UNESCO}: Homem e Biosfera (\versal{MAB}) e
comprometeu-se em criar áreas de reserva para cada bioma (Bursztyn \&
Persegona, 2008: 163-164). Em 1975, promulgaram-se leis de controle da
poluição em nível federal, acompanhando a experiência de unidades da
federação industrializadas, como São Paulo, Minas Gerais, Rio de
Janeiro, e se previam planos de ação e exigências de controle e
monitoramento de efluentes, além de definição de áreas críticas de
poluição do país\footnote{Decreto- Lei nº 1413, de
  1975.\emph{http://www.planalto.gov.br/ccivil\_03/Decreto-Lei/1965-1988/Del1413.htm}

  Decreto nº 76389, de 1975.
  \emph{http://www2.camara.leg.br/legin/fed/decret/1970-1979/decreto-76389-3-outubro-1975-424990-publicacaooriginal-1-pe.html}}.

No ano de 1981, ainda durante a ditadura civil-militar, promulgou-se a
Lei nº 6.938, vigente até hoje com alterações, que criou o Sistema
Nacional de Meio Ambiente ― \versal{SISNAMA} e a Política Nacional de Meio
Ambiente ― \versal{PNMA}\footnote{Lei Federal nº 6.938, de 31 de agosto de 1981.

  Disponível em
  \emph{http://www.planalto.gov.br/ccivil\_03/leis/L6938.htm}}. Entre
outras medidas, exigiu-se estudos de impacto ambiental para o
licenciamento de empreendimentos que acarretam alterações no meio
ambiente. Previu-se a participação de representantes da sociedade civil,
em decisões e consultas sobre a gestão e licenciamento no Conselho
Nacional de Meio Ambiente ― \versal{CONAMA} e, por isso, foi considerada ``uma
iniciativa profundamente transformadora no que diz respeito ao papel do
Estado e da organização do Poder Executivo para a sua aplicação, à
medida que com ela se introduzem, pela primeira vez no Brasil,
mecanismos de gestão colegiada e participativa'' (Carvalho, 2008: 261).
A mesma estrutura de licenciamento e participação da sociedade civil
começaram a ser reproduzidas nos níveis estaduais e municipais em órgãos
integrantes do \versal{SISNAMA} (Bursztyn \& Pessegona, 2008: 187-189). O
estabelecimento da Política Nacional de Meio Ambiente, afinada com a
nascente ``governança ambiental global'', consolida o funcionamento do
\emph{dispositivo meio ambiente} no Brasil, reunindo temas e
instituições dispersas, como ecossistemas, desenvolvimento
socioeconômico, uso racional do solo, subsolo, água e ar. A efetivação
dessa política ocorreu, gradativamente, mas com a Constituição Federal
de 1988 acelerou-se a criação de projetos e instituições que favoreceram
sua implantação.

O tema ambiental entrou também na vida político-partidária em vários
locais do planeta, focando os aspectos referentes à natureza e recursos,
pois outros temas que compunham a noção de meio ambiente das últimas
décadas, como saneamento, pobreza, saúde consistiam já em objetos de
muitas propostas parlamentares e de gestão governamental. Na França, as
lutas ecológicas engrossaram os protestos em 1968, introduzindo na pauta
de discussões de grupos radicais da esquerda francesa os temas
ambientais como o dos efeitos da poluição industrial e a destruição de
paisagens naturais. No final da década de 1970, os ecologistas se
mobilizaram para discutir a fonte nuclear da matriz energética da França
e buscar meios de impedir novas usinas. O assunto se disseminou para
outros setores sociais e a luta contra a energia nuclear se tornou uma
questão relativa à organização institucional dos movimentos ambientais,
não apenas na França. Os movimentos ecológicos se organizaram e chegaram
a apoiar um candidato com uma pauta ecologista, René Dumont, à
presidência da república da França nas eleições de 1974, fato inédito
até então.

A ecologia não seria apenas um ramo das ciências naturais, em sentido
amplo funcionaria como a ``crítica global e radical do modo de produção
industrial'' (Dupuy, 1980: 27-36). Tal crítica pode ser conservadora e
retrógrada, mas também pode fazer acionar práticas de contestação ao
capitalismo, ao modelo industrial e ao Estado. Conforme declara um
adepto da \emph{ecologia política}: ``queremos um capitalismo ecológico
ou aproveitaremos a crise ecológica para instaurar outra lógica social
onde `o livre desenvolvimento de todos seria ao mesmo tempo o fim e a
condição do livre desenvolvimento de cada um?''' (Ibidem).

O filósofo francês Michel Serres mostra dois sentidos da ecologia: ``1.
o sentido de uma disciplina científica dedicada ao estudo dos conjuntos
mais ou menos numerosos e seres vivos interagindo com o meio. {[}...{]}
2. o sentido ideológico e político de uma doutrina, variável segundo os
diversos autores e grupos, visando à proteção do meio ambiente''
(Serres, 2006)\emph{.} Assim como a ecologia, o ativista ecológico teria
também uma dupla face: a do cientista e a do militante. O dilema a ser
enfrentado seria o de encontrar meios para lutar pelo equilíbrio
natural, regido por leis independentes do homem, e ao mesmo tempo o de
combater pela liberdade humana (Dupuy, 1980: 15-16). Em todo caso, ``por
meio da crítica ecológica, pode-se atacar o modo de produção capitalista
e socialista, enquanto formas agressivas e predatórias de viver,
redimensionar as formas de participação política que afastam dos
processos decisórios; questionar as maneiras de consumir e a própria
ativação perpétua de um consumo desenfreado, e, por último, pensar as
relações com as novas tecnologias e suas implicações para o indivíduo e
meio ambiente. As formas que essa crítica toma para um ou outro
formulador é que se diferencia: ela pode ir desde uma visão holística de
mundo, (...), indicar possibilidades e necessidades de reformas
políticas, até apontar para a derrubada definitiva do Estado e do
sistema capitalista'' (Augusto, 2012: 80-81).

Na Alemanha ocidental, grupos alternativos ao estilo de vida e ao
sistema político e econômico criaram o Partido Verde, em janeiro de
1980, em busca de uma nova sociedade pela via parlamentar. Meses antes,
não formalizados ainda como partido, mas como Outra Associação Política
os Verdes (Sonstige Politische Vereinigung die Grunen)\footnote{Na
  Alemanha era possível se candidatar em eleições sem vínculos a
  partidos institucionalizados.}, convidaram o artista plástico Joseph
Beuys a se candidatar ao Parlamento Europeu (Stachelhaus, 1994: 13) em
sua primeira eleição direta. O partido surgiu em um momento de protestos
contra a política alemã e contra a presença de militares e instalações
bélicas estadunidenses. O tema ecológico, para seus defensores,
ultrapassava a luta de classes e divisões sociais, associava-se ao
pacifismo para se voltar à salvação do planeta, \emph{habitat} da
humanidade, ou, no caso, pelo menos do ambiente circundante dos
ativistas. Beuys foi um dos artistas mais atuantes da Alemanha
pós-guerra, sua arte não se apartava de ações de cunho político, e ele
se tornou um expoente dos movimentos alternativos, questionando os
partidos e a democracia representativa em busca de um ``terceiro
caminho'' (Beuys, 2010: 49-55). Em 1971, em \emph{Ação no Pântano}
(\emph{Aktion im Moor}), mergulhou em um pântano em Zuider Zee, na orla
holandesa, ameaçado de ser drenado (Stachelhaus, 1994: 134). Em 1972,
uma ação denominada \emph{Vencer finalmente a ditadura dos Partidos}
(\emph{Überwindet endlich die Parteienditaktur}) consistia em varrer,
junto com simpatizantes da causa, uma floresta em Grafenberg, na
Baviera, ameaçada de corte para dar lugar a quadras de tênis. ``Todos
falam em proteger o meio-ambiente, mas ninguém age'', declarou Beuys na
ocasião (Ibidem: 102).

Em 1983, com propostas pacifistas e ecológicas, elegeram-se muitos
representantes do Partido Verde para o Parlamento alemão. O nome de
Beuys fora cortado da lista dos candidatos pelos próprios fundadores do
Partido Verde já em 1980 (Ibidem: 104). Anos depois, o partido compôs
com outros em coalizão de governo, tomou decisões convencionais e perdeu
sua força inovadora\footnote{``Há 30 anos, o Partido Verde chegava ao
  Parlamento da Alemanha''.

  \emph{http://www.dw.com/pt/h\%C3\%A1-30-anos-partido-verde-chegava-ao-parlamento-da-alemanha/a-16647547}}.
As utopias partidárias verdes se espalharam pelo planeta, mas o sucesso
no legislativo passou a depender de coalizões e, às vezes, de um fugaz
bom momento circunstancial.

No Brasil, em 1986, em um inicial momento da redemocratização após a
ditadura civil-militar e preparação para se elaborar uma nova
Constituição, fundou-se o Partido Verde, com Fernando Gabeira Além de
abordar a questão ecológica, o \versal{PV} se propunha a lutar por formas
alternativas de vida, pela autonomia e liberdade e por uma sociedade
cada vez mais descentralizada. Durante os seus anos iniciais, o \versal{PV}
participou da constituinte como colaborador do grupo ambientalista.
Contribuiu para a realização do Fórum Global 92, reunindo \versal{ONG}s de todos
continentes durante a \versal{ECO}-92, ao conseguir a aprovação de uma verba
governamental da prefeitura para o evento mediante uma lei de incentivos
fiscais, conhecida por Lei Sirkis, proposta por Alfredo Sirkis,
ex-guerrilheiro urbano e vereador do \versal{PV} do Rio de Janeiro. Em 2010, o \versal{PV}
lançou a candidatura para a Presidência da República de Marina Silva,
que viera dos seringais extrativistas do Acre com Chico Mendes e fora
Ministra do Meio Ambiente no governo Lula da Silva. Marina Silva saiu do
\versal{PV} em 2011 e tentou fundar um movimento que denominou
``transpartidário''. No entanto, para entrar na disputa pelo governo foi
necessário adaptar-se à estrutura partidária, de onde surgiu a Rede de
Sustentabilidade, ou simplesmente Rede, que se propõe afinada com a
questão do desenvolvimento sustentável\footnote{Em 2014, a Rede não
  conseguiu credenciar-se a tempo como um partido e Marina Silva se
  filiou ao \versal{PSB}, tornando-se vice do candidato presidencial pelo \versal{PSB}
  Eduardo Campos. Com a morte deste em um acidente, assumiu a
  candidatura à Presidência e obteve 21,32\% dos votos válidos, mas não
  logrou passar para o segundo turno.}.

As pautas e os programas de quase todos os partidos, inclusive os
denominados \emph{conservadores}, contam agora com questões ambientais e
ecológicas, tendência que se espalhou por todos os continentes. Não se
governa nem se propõe um estilo de governo sem que se tenha ao menos
algumas propostas para gestão de aspectos ambientais. Um partido
especializado em meio ambiente, portanto, tende a ter sua força
dissolvida porquanto o próprio dispositivo se expandiu e agregou outras
injunções, como desenvolvimento, direitos e participação da sociedade
civil. Os partidos mais afeitos às causas ecológicas buscam, então,
inovar nas formas de arregimentação de seguidores, garantindo-lhes que
poderão efetivamente decidir nas escolhas partidárias visando o governo.

Nos primeiros anos de sua criação, o Programa das Nações Unidas para o
Meio Ambiente -- \versal{PNUMA} realizou uma série de estudos ecológicos e
econômicos, destacando-se, em 1980, a Estratégia Mundial para a
Conservação (World Conservation Strategy: living resources conservation
for a sustainable development)\footnote{World Conservation Strategy:
  living resources conservation for a sustainable development { }
  (1980). (grifo nosso)
  \emph{https://portals.iucn.org/library/efiles/edocs/WCS-004.pdf}},
junto com a \versal{UICN} e a \versal{WWF}, com sugestões para a Terceira Década do
Desenvolvimento da \versal{ONU}. Nesse documento aparece o uso pioneiro da
expressão ``desenvolvimento sustentável'' como meta a ser alcançada pela
conservação da natureza. O desenvolvimento econômico deveria considerar
a ``capacidade de suporte'' dos ecossistemas utilizados. ``A noção de
sustentabilidade ou de durabilidade se origina de teorizações e práticas
ecológicas que tentam analisar a evolução temporal de recursos naturais,
tomando por base a sua persistência, manutenção ou capacidade de retorno
a um presumido estado de equilíbrio, após algum tipo de perturbação''
(Raynaut et al., 2000: 74).

Dez anos depois de Estocolmo, na sede do \versal{PNUMA}, em Nairobi, elaborou-se
a Declaração de Nairobi, avaliando a pouca repercussão do Plano de Ação
de Estocolmo 72 e conclamando esforços para seu fortalecimento, ``para
que o legado do nosso pequeno planeta às gerações futuras seja feito em
condições que garantam uma vida digna a todos os seres
humanos''\footnote{Nairobi Declaration.
  \emph{http://www.un-documents.net/nair-dec.htm}. Em português: Tamanes
  (1983).}. Em outubro do mesmo ano, a \versal{ONU} adotou em Assembleia Geral a
Carta Mundial da Natureza, composta de 24 artigos que abordam três
temas: os princípios gerais para a defesa dos ecossistemas; as ações
para integrar desenvolvimento econômico à natureza; e, por fim, o apelo
à incorporação desses princípios e ações por cada Estado-membro da \versal{ONU}.
Dentre os tópicos do arrazoado da Carta, destaca-se: ``cada forma de
vida é única, merecendo respeito independente de seu valor para o homem
e, para conceder a outro organismo tal reconhecimento, o homem deve
estar guiado por um código de ação moral''. O último artigo afirma que
``cada pessoa tem o dever de agir de acordo com as disposições desta
presente Carta, agindo individualmente, associada a outros ou através da
participação no processo político. Cada pessoa deve se esforçar em
garantir que os objetivos e exigências desta carta sejam
cumpridos''\footnote{World Chart of Nature.
  \emph{http://www.un.org/documents/ga/res/37/a37r007.htm}.}. Dois pontos
sobressaem nesse documento: o reforço ao aspecto ecológico do meio
ambiente e a ênfase na conduta moral que deve balizar a relação de cada
um com a natureza, convocando todos à participação e à adequação de
condutas.

Nos anos 1980, ao lado das manifestações políticas da sociedade civil e
do sucesso do meio ambiente como alvo de ações de gerenciamento dos
Estados, o impasse entre a proteção ao meio-ambiente e o desenvolvimento
econômico permanecia oscilante nas Nações Unidas e não se vislumbrava
uma resolução consensual, nem mesmo entre os órgãos do sistema. Em 1983,
em um documento da Assembleia Geral intitulado Process of preparation of
the Environmental Perspective to the Year 2000 and Beyond\footnote{Process
  of preparation of the Environmental Perspective to the Year 2000 and
  Beyond.

  \emph{http://www.un.org/documents/ga/res/38/a38r161.htm}}, a \versal{ONU}
proporcionou as bases para a formação de um grupo fora dos órgãos
instituídos de seu sistema, com tarefa e prazos definidos. A Comissão
Mundial de Meio-Ambiente e Desenvolvimento (World Comission on
Environment and Development ― \versal{WCED}), presidida pela ex-Primeira Ministra
da Noruega, Gro Brundtland, contava entre os membros efetivos da
Comissão com Paulo Nogueira Neto, ambientalista, ex-secretário do Meio
Ambiente no Brasil de 1974 a 1986, durante os governos de Ernesto Geisel
e João Figueiredo\footnote{Paulo Nogueira Neto: biografia resumida.
  \emph{http://eco.ib.usp.br/nogueirapis/autor.htm}.}. A elaboração do
relatório teve por base consultas, reuniões deliberativas e audiências
públicas em várias cidades e continentes; no Brasil ocorreram duas: uma
em Brasília e outra em São Paulo, ambas em 1985 (Brundtland et al.,
1988). Foram quatro anos de consultas e seminários com representantes,
técnicos e diplomatas de todos os países membros da \versal{ONU} e das principais
\versal{ONG}s internacionais. A desigualdade social planetária e a consequente
pobreza da maior parte da população foram diagnosticadas e associadas ao
esgotamento dos recursos naturais.

Como resposta específica às questões envolvendo meio ambiente e
economia, diagnosticadas durante o processo de organização da
Conferência de Estocolmo, Maurice Strong e o economista Ignacy Sachs
cunharam o termo \emph{ecodesenvolvimento}. Definiu-se uma trajetória de
crescimento econômico adequado ao Terceiro Mundo em que as atividades
produtivas se realizariam com mínima alteração ambiental. O
\emph{ecodesenvolvimento} comportaria uma forma de crescimento que
investiria nas soluções específicas de problemas particulares,
considerando dados ecológicos e culturais locais assim como as
necessidades das pessoas envolvidas, tanto as imediatas quanto as de
longo prazo (Sachs, 1986: 18). No entanto, ao discutir soluções de longo
alcance para o impasse proteção ambiental e desenvolvimento econômico, o
\emph{ecodesenvolvimento} se revelou insuficiente por depender das
especificidades locais sem possibilidades de se tornar um modelo
aplicável a todos, além de não apresentar consistência necessária para
se tornar uma solução genérica para o impasse enfrentado pelo sistema
\versal{ONU}. O conceito foi abandonado pelos seus autores. Maurice Strong foi
membro efetivo da \versal{WCED} e abriu caminho para um discurso que se
coadunasse com a esperada abrangência global.

O termo ``sustentável'', definindo a ``capacidade de suporte do
ecossistema'' e a ``harmonia possível entre elementos inter-relacionados
no meio ambiente humano'', fora citado de passagem no relatório do Clube
de Roma, como resultado de uma ``condição de estabilidade ecológica e
que fosse sustentável por muito tempo no futuro'' (Meadows, 1972: 24) e
empregado no relatório World Conservation Strategy, do \versal{PNUMA}\footnote{World
  Conservation Strategy: living resources conservation for a sustainable
  development\emph{.}

  \emph{https://portals.iucn.org/library/efiles/edocs/WCS-004.pdf}.}. Em
1977, Dennis Pirages utilizou o conceito de \emph{crescimento
sustentável} (\emph{sustainable growth}) como meta de um novo `desenho
social' e novas instituições para uma era pós-industrial na qual a
escassez de petróleo como fonte de energia traria uma desaceleração do
crescimento. Para ele, ``uma sociedade sustentável seria aquela cujo
crescimento futuro poderia ser indefinidamente sustentado pelos recursos
energéticos disponíveis'' (Pirages, 1977:18). No entanto, prossegue o
autor, como cada fonte alternativa de energia tem custos socioeconômicos
específicos, o desenho de uma nova sociedade, uma sociedade sustentável,
dependeria das escolhas das fontes de energia a serem feitas
considerando seus impactos e efeitos a longo prazo.

Pirages foi um dos colaboradores do relatório encomendado pelo
presidente estadunidense Jimmy Carter ao Conselho de Qualidade
Ambiental, órgão ligado à presidência, e criado pela Lei de Política
Ambiental Nacional de 1970, publicado em 1980 como \emph{The Global 2000
Report to the President: entering the twenty-first century}\footnote{\emph{Global
  2000 Study Statement on the Report to the President} (1980).
  \emph{http://www.geraldbarney.com/G2000Page.html}.}. O relatório visava
orientar as políticas interna e externa dos \versal{EUA}. Contava com a
metodologia e técnicos graduados que elaboraram o relatório do Clube de
Roma em 1972, inclusive D. Meadows. Para promover ``o desenvolvimento
econômico sustentável, junto com proteção ambiental, gerenciamento de
recursos e planejamento familiar'', os quais seriam
``essenciais''\footnote{\emph{The Global 2000 Report to the President:
  entering the twenty-first century}, v. I (volume síntese), p. \versal{IV}.

  \emph{http://www.geraldbarney.com/Global\_2000\_Report/G2000-Eng-GPO/G2000\_Vol1\_GPO.pdf}.},
o governo estadunidense se propunha a ampliar a colaboração com outros
países, desenvolvidos ou não\footnote{Entretanto, o presidente que
  suscedeu Carter, Ronald Reagan cortou boa parte do orçamento dos
  programas ambientais e deixou o tema em segundo plano (McComick, 1992:
  139).}. As conclusões reafirmaram o lugar-comum das pressões
populacionais como fator decisivo nas previsões catastróficas para o
futuro: ``se as tendências atuais se mantiverem, o mundo em 2000 será
mais populoso, mais poluído, e mais vulnerável a rupturas do que
hoje''\footnote{\emph{The Global 2000 Report to the President: entering
  the twenty-first century}, v. I (volume síntese), p. 1.}.

O esquema para a utilização dos recursos naturais sem prejuízos para as
gerações futuras já estava delineado desde as primeiras constatações dos
limites da natureza, discutidas ainda que não exaustivamente desde o
século \versal{XIX}. Os preservacionistas levavam em conta o futuro ao proteger a
natureza em \emph{santuários}; os conservacionistas defendiam um uso
prudente, e até baseado na ciência, dos recursos naturais pensando
também no futuro. Lamentava-se a destruição de uma paisagem em nome da
impossibilidade de usufruir de seus benefícios em momentos posteriores.
As mobilizações de naturalistas e os ativistas ecológicos do século \versal{XX}
procuraram olhar para o meio ambiente visando um bem comum que não
desaparecesse pouco tempo adiante. O desenho estava pronto, só não
recebera um nome válido e consensual para se disseminar pelo planeta.

A Comissão Mundial de Meio-Ambiente e Desenvolvimento procurou evitar a
``reiteração de problemas e tendências que já tinham sido bem
contemplados pelo Global 2000 e outros relatórios'' (McCormick, 1992:
189). Foram selecionadas as questões decisivas do impasse, que se
conectaram pelo funcionamento do \emph{dispositivo meio ambiente}.
Energia, assentamentos urbanos, indústria, ecossistemas, população,
segurança alimentar, relações internacionais, gestão dos bens comuns
(oceanos, espaço sideral, clima) foram debatidos em audiências públicas
e conferências locais em várias cidades e continentes envolvendo uma
extensa lista de organizações e indivíduos\footnote{Ver o anexo com os
  agradecimentos a mais de 900 pessoas e instituições em \versal{WCED}, 1987:
  366-387.}. No Brasil ocorreram duas: em Brasília e em São Paulo, ambas
em 1985 (\versal{WCDE}, 1987; Brundtland, 1988). Foram quatro anos de consultas e
seminários com representantes, técnicos e diplomatas de todos os países
membros da \versal{ONU} e das principais \versal{ONG}s internacionais.

Ao final das discussões, consultas e relatórios, bateu-se o martelo na
locução \emph{desenvolvimento sustentável} como resposta ao impasse,
tornando-o meta estratégica para atuação em várias frentes: ambiental,
social, econômica e, especialmente, política. Em 1987, publicou-se o
documento \emph{Nosso futuro comum}, ou \emph{Relatório Brundtland,}
harmonizando desenvolvimentistas econômicos e conservacionistas
ambientais em um mesmo diapasão: o estímulo institucional e planetário a
um crescimento econômico que não esgotasse os recursos naturais para as
gerações futuras: o \emph{desenvolvimento sustentável}. A desigualdade
social planetária e a consequente pobreza da maior parte da população
foram diagnosticadas e relacionadas ao esgotamento dos recursos naturais
e, principalmente, aos seus incipientes gerenciamentos que impediam a
união harmônica entre meio ambiente e desenvolvimento. O impasse entre
crescer e conservar o ambiente foi equacionado com o consenso em torno
do \emph{desenvolvimento sustentável}, fundado em três pilares:
desenvolvimento econômico, desenvolvimento social e proteção ambiental.
A partir daí, não haveria mais o perigo de incentivos a políticas de
limite ao crescimento, tampouco a aposta na desaceleração econômica; os
recursos poderiam ser usados e o lucro, crescer, desde que se mantivesse
a \emph{sustentabilidade}, mediante a qual ``o desenvolvimento satisfaz
as necessidades presentes, sem comprometer a capacidade das gerações
futuras de suprir suas próprias necessidades'' (\versal{WCED}, 1987: 8).

A meta de um desenvolvimento socioeconômico sustentável por conservar os
recursos para as futuras gerações foi a resposta imediata e conveniente
aos problemas planetários reunidos por meio dos desdobramentos
produzidos pelo \emph{dispositivo meio ambiente}. A noção de
sustentabilidade redefiniu as ações da \versal{ONU}, não apenas as relativas às
questões ambientais, mas suas principais propostas sociais econômicas,
pautadas pelo aperfeiçoamento de uma ``governança planetária'', por meio
do conjunto de instituições e normas voltadas para a gestão do planeta.
Redimensionou a noção corrente da atividade econômica e do
relacionamento do humano com o seu meio, tanto natural quanto o
resultante de suas atividades. A relação homem-meio passou a ser
considerada a grande questão a ser encaminhada, inclusive no que tange à
pobreza e à fome, deixando para o plano secundário, ou de revestimento,
o equacionamento como restauração das relações hierárquicas e desiguais
entre as pessoas.

O socialismo presente em alguns países da Europa começou a desaparecer
com o marco emblemático da queda do muro de Berlim em 1989, que arruinou
utopias de transformação social e econômica mediante a concretização
desse regime político-econômico. Com o alargamento do crescimento pelas
modulações capitalistas, a sustentabilidade tornou-se o mais abrangente
e consensual programa global para alcançar a ``salvação da vida humana
no planeta'', utilizando-se aqui termos difundidos pela divulgação do
programa, transitando por várias composições econômicas sob a governança
capitaneada pela racionalidade neoliberal.

\begin{quote}
Uma das resoluções do \emph{Relatório Brundtland} fora programar uma
grande Conferência sobre meio ambiente e desenvolvimento, aprovada na
Assembleia Geral da \versal{ONU} de 1989. Por iniciativa do Itamaraty, no governo
Sarney, o Brasil se ofereceu para sediar o evento convocado para 1992. O
intuito era o de melhorar a imagem internacional do país que vinha se
deteriorando nas questões ambientais e sociais (Lago, 2007: 151).
Todavia, o líder dos extrativistas da borracha do Acre, Chico Mendes,
conhecido internacionalmente como respeitável expoente ambiental, seria
assassinado a mando de fazendeiros locais semanas depois do anúncio do
Brasil como candidato a sediar o encontro da \versal{ONU}. Divulgavam-se em todos
os continentes as notícias e imagens das queimadas e do desmatamento na
Amazônia, assim como a perseguição aos indígenas e extrativistas.

O \emph{Relatório Brundtland} norteou novas preocupações ambientais, e a
poluição e outros problemas mais localizados deixaram de ser o alvo
principal das propostas e das intervenções dando lugar à mudança
climática e à biodiversidade, temas mais abrangentes (Ibidem: 146-147).
O anúncio da Conferência da \versal{ONU} para 1992 mobilizou a comunidade
internacional em torno de questão de uma ``governança global para
preservar a vida na Terra'', enfatizando a necessidade de uma imediata e
efetiva ``mudança de comportamento'' dos Estados, das organizações
internacionais, das empresas e dos indivíduos para uma chamada
``sociedade civil planetária''.

A conferência ocorreu em junho, no Rio de Janeiro, com a presença de 172
países, representados por aproximadamente 10.000 participantes,
incluindo 116 chefes de Estado. Ao mesmo tempo, vários outros encontros
sobre o assunto foram realizados na mesma semana, como o Fórum Global de
\versal{ONG}s de todos os continentes e um encontro de empresários. Além de
relatórios e manifestos produzidos nesses diversos encontros, cinco
documentos importantes foram assinados pelos chefes de Estado: 1)
\emph{Declaração do Rio sobre Meio Ambiente e Desenvolvimento}\footnote{\emph{Declaração
  do Rio sobre Meio Ambiente e Desenvolvimento}.

  \emph{http://www.onu.org.br/rio20/img/2012/01/rio92.pdf}}; 2)
\emph{Agenda 21}, conjunto de recomendações para se atingir o
desenvolvimento sustentável, visando o próximo século\footnote{\emph{Agenda
  21} (United Nations).

  \emph{https://sustainabledevelopment.un.org/content/documents/Agenda21.pdf}},
cabendo a cada país elaborar seus próprios planos a partir das
recomendações e implantar a sua própria Agenda; 3) \emph{Princípios das
Florestas}\footnote{Declaração oficial de princípios sem força jurídica
  obrigatória para um consenso mundial sobre a gestão, conservação e
  desenvolvimento sustentável de todos os tipos de florestas.

  \emph{http://www.un.org/documents/ga/conf151/spanish/aconf15126-3annex3s.htm}},
uma primeira tentativa de obter um consenso planetário sobre o uso
sustentável das florestas, com a garantia aos Estados do direito
soberano de usá-las de acordo com suas necessidades de desenvolvimento;
4) \emph{Convenção-Quadro sobre Mudança do Clima}\footnote{\emph{Convenção-Quadro
  sobre Mudança do Clima}.
  \emph{http://www.mct.gov.br/upd\_blob/0005/5390.pdf}} (\emph{United
Nations Framework Convention on Climate Change} ― \versal{UNFCC}), com o suporte
científico do Intergovernamental Panel of Climate Change ― \versal{IPCC}\emph{,}
fundado em 1988 por iniciativa do \versal{PNUMA}, e da qual resultou, em 1997, o
Protocolo de Kyoto, com estratégias de combate ao efeito estufa e
mudanças climáticas em geral; 5) \emph{Convenção-Quadro sobre
Diversidade Biológica}\footnote{Convenção sobre Diversidade Biológica.

  \emph{http://www.mma.gov.br/estruturas/sbf\_dpg/\_arquivos/cdbport.pdf}},
com o objetivo de preservação da diversidade biológica e uso sustentável
do patrimônio genético. Entretanto, apenas em 2010, em uma Conferência
das Partes (\versal{COP}), em Nagoya, foi estabelecido um Protocolo sobre o tema,
dando início, com apoio administrativo do \versal{PNUMA}, à criação de uma
plataforma nos moldes do \versal{IPCC}: a Plataforma Intergovernamental para a
Biodiversidade e Serviços Ecossistêmicos (Intergovernamental Plataform
on Biodiversity and Ecosystem Services ― \versal{IPBES}).\footnote{Encontra-se em
  curso a consolidação da \versal{IPBES}, com a meta inicial de organizar o
  conhecimento, inclusive os saberes de populações tradicionais e
  indígenas, sobre biodiversidade para subsidiar decisões políticas em
  âmbito mundial. \emph{http://www.ipbes.net/}}

A \emph{Agenda 21} consiste em um extenso Plano de Ação com centenas de
propostas, a ser replicado pelo planeta. No Preâmbulo do documento base,
conclama-se a união consensual de todos para o desenvolvimento
sustentável: ``a integração das preocupações ambientais e de
desenvolvimento e uma maior atenção a isso levará à satisfação das
necessidades básicas, a melhores padrões de vida para todos, a
ecossistemas mais protegidos e gerenciados e a um futuro mais seguro e
próspero. Nenhuma nação pode conseguir isso por conta própria, mas
juntos podemos, em uma parceria global para o desenvolvimento
sustentável. (...) A Agenda 21 aborda os problemas prementes de hoje e
também visa preparar o mundo para os desafios do próximo século. Ela
reflete um consenso global e compromisso político do mais alto nível
sobre a cooperação para o desenvolvimento e meio ambiente''\footnote{\emph{Agenda
  21}. Preâmbulo (documento-base em inglês).

  \emph{https://sustainabledevelopment.un.org/content/documents/Agenda21.pdf}}.
\end{quote}

No Brasil, desde o final dos anos 1990 e início do novo milênio, a
sustentabilidade organizada pela \emph{Agenda 21} consta da pauta do
governo brasileiro. ``A Agenda 21 não é apenas um documento. Nem é um
receituário mágico, com fórmulas para resolver todos os problemas
ambientais e sociais. É um processo de participação em que a sociedade,
os governos, os setores econômicos e sociais sentam-se à mesa para
diagnosticar os problemas, entender os conflitos envolvidos e pactuar
formas de resolvê-los, de modo a construir o que tem sido chamado de
sustentabilidade ampliada e progressiva'' (Novaes, s/d: 5).

A partir de 1996, ocorreram sucessivas consultas coordenadas pela
Comissão de Políticas de Desenvolvimento Sustentável e da Agenda 21
Nacional (\versal{CPDS}), chegando-se a uma agenda própria que orientou diversos
programas de governo nos níveis municipal, estadual e federal, pela qual
a sustentabilidade aparece em quatro dimensões: ambiental, social,
econômica e político institucional, cada uma com suas metas,
contempladas nos programas. As metas básicas da Agenda brasileira, de
acordo com as consultas junto à população e afinadas com as resoluções
internacionais, consistiram em:

Dimensão Ambiental

\begin{quote}
\emph{1. Uso sustentável, conservação e proteção dos recursos naturais.}

\emph{2. Ordenamento territorial. }

\emph{3. Manejo adequado dos resíduos, efluentes, das substâncias
tóxicas e radioativas. }

\emph{4. Manejo sustentável da biotecnologia.}
\end{quote}

Dimensão Social

\begin{quote}
\emph{5. Medidas de redução das desigualdades e de combate à pobreza.}

\emph{6. Proteção e promoção das condições de saúde humana e seguridade
social. }

\emph{7. Promoção da educação e cultura, para a sustentabilidade. }

\emph{8. Proteção e promoção dos grupos estratégicos da
sociedade.}\footnote{Nesse item estariam as questões de gênero e de
  grupos étnicos e tradicionais.}
\end{quote}

Dimensão Econômica

\begin{quote}
\emph{9. Transformação produtiva e mudança dos padrões de consumo. }

\emph{10. Inserção econômica competitiva.}

\emph{11. Geração de emprego e renda, reforma agrária e urbana. }

\emph{12. Dinâmica demográfica e sustentabilidade.}
\end{quote}

Dimensão Político-Institucional

\begin{quote}
\emph{13. Integração entre desenvolvimento e meio ambiente na tomada de
decisões. }

\emph{14. Descentralização para o desenvolvimento sustentável. }

\emph{15.Democratização das decisões e fortalecimento do papel dos
parceiros do desenvolvimento sustentável. }

\emph{16. Cooperação, coordenação e fortalecimento da ação
institucional. }

\emph{17. Instrumentos de regulação. }
\end{quote}

Informação e conhecimento

\begin{quote}
\emph{18. Desenvolvimento tecnológico e cooperação, difusão e
transferência de tecnologia}

\emph{19. Geração, absorção, adaptação e inovação do conhecimento.}

\emph{20. Informação para a tomada de decisão.}

\emph{21. Promoção da capacitação e conscientização para a
sustentabilidade}.\footnote{\emph{Agenda 21 brasileira}. Resultado da
  Consulta Nacional. 2ª edição.

  \emph{http://www.mma.gov.br/responsabilidade-socioambiental/agenda-21/agenda-21-brasileira}}
\end{quote}

Além disso, no Brasil, seria possível criar Agendas 21 locais, para
territórios e setores definidos: municípios, bairros, comunidades,
escolas, associações. Para isso, divulgou-se também o ``Passo-a-Passo da
Agenda 21 Local, que propõe um roteiro organizado em seis etapas:
mobilizar para sensibilizar governo e sociedade; criar um Fórum de
Agenda 21 Local; elaborar um diagnóstico participativo; e elaborar,
implementar, monitorar e avaliar um plano local de desenvolvimento
sustentável''\footnote{\emph{Agenda 21}.
  \emph{http://www.mma.gov.br/responsabilidade-socioambiental/agenda-21}}.
Depois do ano 2000, os programas dessas agendas locais se integraram ou
buscaram se integrar à consecução dos \emph{Objetivos do
Milênio}\footnote{``Agenda 21 e os Objetivos do Milênio''.

  \emph{http://www.odmbrasil.gov.br/legislacao/agenda-21-e-os-odm}}. As
metas gerais foram incorporadas ao Plano Plurianual 2004-2007 e a outras
iniciativas dos governos brasileiro subsequentes.

No limiar do terceiro milênio, o ano 2000 marcou novas
institucionalizações e uma reorganização dos encaminhamentos de
políticas planetárias. A Cúpula do Milênio da \versal{ONU}, por meio do Programa
das Nações Unidas para o Desenvolvimento ― \versal{PNUD}, reuniu as propostas
apresentadas por seus membros e, com a presença de 189 Estados, lançou a
\emph{Declaração do Milênio}. Oito Objetivos de Desenvolvimento do
Milênio foram definidos\textbf{,} resultantes de debates e estudos
realizados em uma serie de conferências\footnote{As conferências
  referiam-se à questão do desenvolvimento econômico, com referência à
  educação, crianças, juventude, mulheres, destacando-se a Conferência
  do Cairo (\versal{IPCD}) de 1994, sobre população e desenvolvimento; a Cúpula
  Mundial de Desenvolvimento Social, de 1996, em Copenhagen; além da
  \versal{ECO}-92.} para ``estabelecer o que deve fazer parte da agenda de
desenvolvimento. Isso foi necessário, pois se chegou à conclusão de que
a pobreza estava aumentando e que a falta de recursos não era a única
explicação para tanto''\footnote{``Introdução aos Objetivos do
  Milênio''\emph{.}
  \emph{http://www.pnud.org.br/popup/download.php?id\_arquivo=497}}. O
trabalho técnico de unificação das propostas foi realizado pelo
Secretariado das Nações Unidas, em conjunto com o Fundo Monetário
Internacional ― \versal{FMI}, a Organização para a Cooperação e o Desenvolvimento
Econômico ― \versal{OCDE} e o Banco Mundial. A presença de entidades
financiadoras garantia os meios concretos para a consecução das metas,
divulgadas como uma ``verdade mundial'' mobilizando esforços de todos.

Esse ano ficou marcado também por iniciativas para se concretizar a
\emph{Agenda 21} e associá-la às metas preconizadas pela
\emph{Declaração do Milênio} e, com isso, organizar a Conferência da
\versal{ONU}, em 2002, sobre Desenvolvimento Sustentável (World Summit on
Sustainable Development ― \versal{WSSD}), em Johanesburgo. Nesta conferência, a
tônica recaiu nas medidas de combate à pobreza, e um dos instrumentos
jurídico políticos internacionais para garantir isso mediante cooperação
entre nações seria cumprir as Metas do Milênio e implantar a
\emph{Agenda 21} em todo mundo.

Ainda em 2000, a \versal{UNESCO} divulgou a \emph{Carta da Terra}, propondo ``uma
visão compartilhada de valores básicos para proporcionar um fundamento
ético à comunidade mundial emergente''\footnote{\emph{Carta da Terra}.
  Preâmbulo.
  \emph{http://www.mma.gov.br/estruturas/agenda21/\_arquivos/carta\_terra.pdf}}.
O documento começara a ser debatido durante a \versal{ECO}-92 para se transformar
em código de uma ética universal de conduta dos povos em direção a um
mundo sustentável, mas o seu esboço não obteve consenso na ocasião,
dando lugar à \emph{Declaração do Rio}. Duas organizações não
governamentais internacionais assumiram a continuidade das discussões da
\emph{Carta}: a Cruz Verde Internacional, presidida por Mikhail
Gorbachev, e o Conselho da Terra, apoiado pelo governo holandês. Durante
o Fórum Rio+5 criou-se a Comissão Carta da Terra, que envolveu 46 países
e 100 mil pessoas pelo planeta para a elaboração do texto final (Boff,
2004: 59). A \emph{Carta da Terra} conta com 16 princípios, pelos quais
se distribuem 77 recomendações, visando a sustentabilidade, ``através
dos quais a conduta de todos os indivíduos, organizações, empresas,
governos, e instituições transnacionais será guiada e
avaliada''\footnote{\emph{Carta da Terra}. Responsabilidade Universal.},
reunidas em quatro temas: 1) respeitar e cuidar da comunidade de vida;
2) integridade ecológica, 3) justiça social e econômica e 4) democracia,
não violência e paz. As 77 recomendações preconizam ações contempladas
em outros documentos da \versal{ONU}, como por exemplo, erradicar a pobreza,
garantir os direitos das mulheres e das meninas, impedir a poluição,
eliminar armas nucleares. Para chegar a uma sociedade sustentável
mediante a viabilização dessas metas, ``é imperativo que nós, os povos
da Terra, declaremos nossa responsabilidade uns para com os outros, com
a grande comunidade da vida, e com as futuras gerações''\footnote{\emph{Carta
  da Terra}. Preâmbulo.}.

O planeta aparece na \emph{Carta da Terra} como uma entidade viva, que
hoje está ameaçada, propositalmente ressoando a \emph{hipótese de Gaia}
do químico e ambientalista James Lovelock (para quem a Terra seria um
ser vivo autônomo), e também a ``crença ancestral dos povos pela qual a
Terra é a Grande Mãe'' (Ibidem: 60). Nesse sentido, há uma conclamação a
todos os habitantes da ``Terra Nosso Lar'' para uma mudança de
comportamento, de hábitos, de modos de vida. Imediatamente após sua
divulgação, enquanto expressão de uma tomada de ``consciência
ecológica'', houve diversos programas em escolas e outras instituições
de todos os continentes para discuti-la e fazer com que o cuidado ---
``a relação amorosa com a realidade para além dos interesses de uso''
--- fosse o ``valor principal de uma ética ecológico-social-espiritual''
(Ibidem: 63-64)\emph{.} A ética consolidada pela \emph{Carta}
apresenta-se como ecumênica e valeria para todos os habitantes do
planeta, de todas as religiões e culturas, contemplando uma
espiritualidade laica voltada para o cuidado amoroso. Ao mesmo tempo,
aspira se tornar uma lei dura (\emph{hard law}) com poder de coerção
punitiva, pois seu objetivo, a salvação da humanidade, a justifica.

A \emph{Carta da Terra} ainda espera o reconhecimento pela Assembleia da
\versal{ONU} para ganhar o mesmo valor da \emph{Declaração dos Direitos Humanos},
``inicialmente como lei branda (soft law) e depois como lei de
referência mundial, em nome da qual os violadores da dignidade da Terra
poderão ser levados aos tribunais'' (Ibidem: 59). Apesar de programas de
educação ambiental e outras iniciativas que exaustivamente disseminam o
conteúdo da \emph{Carta}, avalia-se que seus princípios não estão sendo
colocados em prática e que as transformações pessoal e social
necessárias para efetivá-los não ocorrem conforme esperado\footnote{\versal{RSE}
  e Carta da Terra: dois movimentos para a transformação\emph{.}
  \emph{http://www.parceirosvoluntarios.org.br/rse-e-carta-da-terra-dois-movimentos-para-a-transformacao/}}.

Sobre a questão ética em relação ao meio ambiente, a Igreja Católica se
manifestou explicitamente, em 2015, com a Encíclica \emph{Laudato Si},
elaborada pelo Papa Francisco (2015). A Santa Sé, o alto comando da
Igreja que exerce sua soberania sobre o Vaticano e personifica no âmbito
jurídico-político um sujeito de direito internacional, sempre esteve
próxima do sistema \versal{ONU} e participou de todas as conferências ambientais,
contribuindo principalmente para que a questão da pobreza fizesse parte
das ações para proteção do ambiente do planeta, e insistindo na
necessidade de mudança de hábitos e na transformação de um estilo de
vida marcado pelo desperdício e interesses imediatos. O Papa João Paulo
\versal{II} já explicitara essa transformação como sendo uma \emph{conversão
ecológica}, o que foi retomado com mais clareza pelo papa atual. Ao
reconhecer haver uma maior conscientização e uma sincera preocupação com
o planeta, Francisco recoloca a \emph{conversão ecológica} como um
assunto do cristianismo hoje: ``para o cristão, a crise ecológica traz a
necessidade de `uma profunda conversão interior', uma reorientação
efetiva da conduta consigo mesmo, com a natureza, com os outros enquanto
consequência do encontro com Cristo'' (Carneiro, 2015: 63). Essa
transformação decorreria de uma experiência interior poderosa como a fé:
``com efeito, não é possível empenhar-se em coisas grandes apenas com
doutrinas, sem uma mística que nos anima'' (Papa Francisco, 2015: §216).

O \emph{dispositivo meio ambiente} também incorporou tanto o combate à
pobreza quanto o cuidado com a criação divina e possibilitou uma
desejável \emph{conversão} em direção a uma ecologia integral.
Redimensionam-se questões como as mudanças climáticas, a perda da
biodiversidade e propostas de desenvolvimento sustentável, ``levando-as
para dentro do coração de cada cristão que as recebe como tarefa para
cumprir a vontade de Deus'' (Carneiro, 2015: 65).

O \emph{dispositivo meio ambiente} não tem um caráter estático, não
consiste em um conceito, mas funciona como um foco de conexões de forças
e produção de verdades que programam as sensações e orientam
comportamentos alinhados com as instituições que gera. Produzem-se novos
objetos e se redimensionam ou redirecionam o curso dos elementos que
aglutina. O movimento ambientalista ampliou seu escopo com o dispositivo
para o qual confluíram temas como direitos civis, desenvolvimento
econômico, questões de gênero. Mobilizações e associações esparsas em
torno de temas disparatados --- como ações de defesa de árvores de um
bairro, protesto contra energia nuclear, contra lixões, abaixo-assinado
pela implantação de ciclovias ---, ao longo dos anos, agruparam-se como
lutas e reivindicações \emph{ambientais}, entram na pauta de reuniões
\emph{ambientais}. Os ativistas de todos esses temas se tornaram
\emph{ambientalista}s ou \emph{ecologistas}, estes quando os assuntos
envolvem ecossistemas, e os movimentos pelos direitos civis incorporaram
reivindicações \emph{ambientais} como parte da cidadania que defendem.

Desde a preparação da Conferência de Estocolmo, exercita-se uma dinâmica
nova de participação da chamada sociedade civil no sistema \versal{ONU}, com
associações ambientalistas e empresariais de outros setores sociais
organizando encontros paralelos para influir direta ou indiretamente nas
decisões. Na \emph{Carta das Nações Unidas} de 1945, no capítulo
referente às atribuições e ao funcionamento do \versal{ECOSOC} (artigo 71),
previa-se a possibilidade deste órgão consultar organizações não
governamentais em assuntos de suas respectivas competências\footnote{Carta
  de las Naciones Unidas de 1945.
  \emph{http://www.un.org/es/sections/un-charter/chapter-x/index.html}}.
Em 1968, o \versal{ECOSOC} especificou com detalhes as relações consultivas entre
o sistema \versal{ONU} e as organizações sociais, abrindo maior espaço para essa
interação\footnote{Resolução 1296 (\versal{XLIV}) 1968.

  \emph{https://www.globalpolicy.org/component/content/article/177/31832.html}}.

Durante a Conferência Rio-92, o envolvimento desses grupos sistematizou
e ampliou as aproximações. As questões reunidas sob a égide da temática
ambiental ganharam mais visibilidade e apoio de vários setores sociais,
acarretando aumento de grupos de pressão e uma maior diversidade de
demandas por novas instituições. As diversas associações presentes nas
atividades que compunham a \versal{ECO}-92, em especial a Conferência dos Povos,
em paralelo ao evento diplomático oficial entre os Estados,
organizaram-se por seus interesses. Passaram a ser reconhecidas
institucionalmente como \emph{Major Groups}, definidos em nove setores
pela \emph{Agenda 21} ``como os principais canais pelos quais uma ampla
participação seria facilitada em atividades das Nações Unidas
relacionadas ao desenvolvimento sustentável''\footnote{Major Groups and
  other stakeholders.
  \emph{https://sustainabledevelopment.un.org/majorgroups}}. Os noves
grupos são: mulheres; crianças e jovens; povos indígenas; organizações
não governamentais; autoridades locais; trabalhadores e sindicalistas;
industriais e empresários de negócios; comunidade científica; e
agricultores. Os grupos majoritários comportariam os grupos de interesse
(\emph{stakeholders}) principais e essa divisão, na ocasião,
correspondeu às associações presentes, respondendo a uma demanda
imediata da movimentação ambientalista e reivindicações de grupos
sociais durante a Eco-92.

Os encontros e conferências atuais das Nações Unidas, e não apenas as
estritamente ambientais, incluem a presença de \emph{Major Groups} por
meio de representantes e organizações que se cadastram no sistema \versal{ONU}
para participarem dos grupos de discussão e dos próprios encontros. Esta
prática tende a se estender com maior formalidade para os governos
locais para que estes gerenciem as questões ambientais e as ações de
incremento da sustentabilidade, levando em conta a população por meio de
representantes. No entanto, ao longo das discussões no âmbito do sistema
\versal{ONU}, outros grupos de interesse (\emph{stakeholders}), não diretamente
ligados aos nove grupos oficializados, passaram a atuar e participar,
como os portadores de deficiência física que ganharam o estatuto de
``outro importante grupo de interesse'' na Conferência Mundial da \versal{ONU}
sobre Redução dos Riscos de Desastres, em Sendai, Japão, em
2015\footnote{\emph{http://www.wcdrr.org/majorgroups/other}}. No conjunto
todos esses \emph{stakeholders} que orbitam em torno do sistema \versal{ONU},
passaram a ser denominados \versal{MG}o\versal{S}s (\emph{Major Groups and other
stakeholders}). Não estão necessariamente limitados por fronteiras
nacionais, mas se vinculam transterritorialmente por seus interesses
específicos.

Ao contrário do que muitas vezes se argumenta contra a atuação dos
\versal{MG}o\versal{S}s, Felix Dodds\footnote{Felix Dodds foi diretor do Forum dos
  \emph{Stakeholders} para um Futuro Sustentável, com sede em Londres no
  período de 1992-2012. Esta instituição foi formada como um comitê
  nacional do \versal{PNUMA} no Reino Unido e desde o ano 2000 tem atuado na
  interface da ``sociedade civil organizada'' e seus múltiplos grupos de
  interesse (\emph{stakeholders}) e as Nações Unidas
  (\emph{http://www.stakeholderforum.org}). Atualmente, Dodds é consultor
  independente (\emph{http://www.felixdodds.net/aboutfelix}).}, um dos
organizadores da Conferência das Nações Unidas sobre Meio Ambiente e
Desenvolvimento, no Rio de Janeiro, em 1992, e articulador do Fórum dos
Stakeholders para o encontro Rio+20, declarou que eles não estão
basicamente ``contra isso ou aquilo''\emph{.} Há uma aposta no
crescimento de espaços pelos quais os \versal{MG}o\versal{S}s podem cooperar com os
governos e exercer um ``papel mais proativo'' nas decisões e, ao mesmo
tempo, fazer com que os governos funcionem como facilitadores na
implementação destas escolhas (Dodds \& Strauss, 2012: 245). Dodds
declarou que nos vinte anos que separam os encontros no Rio ocorreu a
``transição de uma democracia madisoniana (ou representativa) para uma
democracia jeffersoniana (ou participativa), e agora estamos entrando em
um período de democracia por, de e com stakeholders. A democracia
stakeholder é a ideia de que envolver todos os participantes em cada
nível resultará na tomada de decisões mais fundamentadas. Significa que
estes stakeholders se sentirão mais donos dos resultados e serão mais
ativos na implementação das políticas. (...) Em um momento de crescente
desilusão com os governos, a \emph{democracia stakeholder} oferece uma
plataforma que pode ajudar fornecer aos governos a força moral e
política para enfrentar as difíceis decisões que precisam ser tomadas
agora e no futuro próximo para assegurar a viabilidade, a equitabilidade
e a sustentabilidade da vida nesse planeta'' (Ibidem: 232). A proposta
de Dodds para efetivar essa nova democracia seria a de uma maior
abertura do sistema \versal{ONU} à participação dos \versal{MG}o\versal{S}s, incentivando a mais
ampla participação da chamada sociedade civil planetária nas decisões
dos órgãos internacionais, trazendo novas práticas de governo. As
reuniões para debater os meios para efetivar a implantação dos Objetivos
de Desenvolvimento Sustentável incluíram essa participação que tem
ocorrido mediantes reuniões de representantes dos \emph{stakeholders}
cadastrados ou diretamente nos sites dos projetos\footnote{Cf. Relatório
  dos representantes dos \versal{MG}o\versal{S}s na preparação do Forum Politico de Alto
  Nivel de 2016.

  \emph{https://sustainabledevelopment.un.org/content/documents/9798MGOS\%20report\%20on\%20HLPF\%20Retreat\%20Feb\%2023\%202016\_FINAL.pdf}}.

A quais ``urgências'' e demandas o \emph{dispositivo meio ambiente}
construiu-se como resposta nas últimas décadas? Em seus começos, em
torno das prévias à Conferência de Estocolmo e como efeito do
acontecimento \emph{1968}, o dispositivo respondeu a um cenário
catastrófico no planeta, derivado do crescimento econômico planejado
pelo capitalismo com as medidas de reconstrução do pós-\versal{II} Guerra
Mundial. Novos Estados se formaram com a descolonização, abrindo
oportunidades de novos negócios, levando à ameaça de uma maior crise
ecológica. No contexto de crise capitalista do início dos anos 1970, o
\emph{dispositivo meio ambiente} se compôs, reuniu os elementos
heterogêneos que se amalgamaram com a ecologia e a natureza,
constituindo-se assim como um novo objeto abrangendo o planeta. Nesse
contexto de crescimento econômico e de alertas de uma crise ambiental
planetária, os primeiros princípios da \emph{Declaração de Estocolmo} de
1972 reafirmaram a melhoria do ambiente e a preservação dos recursos
naturais como necessárias para as gerações atuais e futuras\emph{.} As
intervenções planejadas se dariam diretamente no meio ambiente ou em
atividades humanas, incluindo-se as econômicas, para que o meio fosse
conservado.

O sistema \versal{ONU} procurava retomar seus ideais de mundo harmônico e seguro,
porém, tinha que aprender a equacionar as disputas internas entre os
vários setores. O impasse interno da \versal{ONU} começou a ser ultrapassado
quando, em 1987, a Comissão Mundial sobre o Meio Ambiente e o
Desenvolvimento direcionou as ações para o desenvolvimento sustentável.
O meio ambiente retornou ao centro da cena da governança global com uma
meta para o planeta: a sustentabilidade do planeta. O desenvolvimento
voltou a ser o principal objetivo para o planeta com o auxílio do
sistema \versal{ONU} conectado à conservação do meio ambiente, visando
consolidar-se como sustentável.

Um novo itinerário se traça pelo qual os Estados, apartados da chamada
Guerra Fria e sintonizados com a racionalidade neoliberal, com ênfase na
participação inteligente e inovadora, exercitam práticas democráticas,
ainda que regimes políticos não democráticos permaneçam. Entretanto, os
Estados são concordes com o sistema \versal{ONU} em função de uma governança
global sustentável, o que supõe produção de conexões principalmente com
mobilizações alternativas, como a Cúpula dos Povos na Rio+20. Importa
trazer todos para os variados fluxos de modo resiliente, praticando-se
sustentáveis e dinamizando o \emph{dispositivo meio ambiente}.

Há um consenso em torno da salvação do planeta e a promoção genérica do
bem-estar humano, sem o qual um planeta ``salvo'' não teria sentido. O
\emph{dispositivo meio ambiente} tem facilitado a convergência de
interesses. O \emph{Acordo de Paris} foi assinado por todos os países
membros nas Nações Unidas, da Coreia do Norte aos \versal{EUA}. O mesmo sucesso
não acontece, por exemplo, em tentativas de tratados sobre desarmamento
ou em questões dos direitos humanos, tampouco em discussões sobre
relações comerciais, como demonstra o constante impasse em que se
encontra a Rodada Doha, processo de discussão encabeçado pela
Organização Mundial do Comércio sobre o fim dos protecionismos
comerciais entre as nações\footnote{Rodada Doha.
  \emph{http://www.itamaraty.gov.br/pt-BR/politica-externa/diplomacia-economica-comercial-e-financeira/694-a-rodada-de-doha-da-omc}}.

Quando a Comissão Mundial sobre o Meio Ambiente e o Desenvolvimento
sugeriu as ações para o desenvolvimento sustentável, o meio ambiente
entrou no centro da cena da governança global com uma meta para o
planeta: a sustentabilidade.

Não pode haver proteção ambiental sem que se considere o
desenvolvimento, esse é o outro lado da \emph{Declaração do Rio} de
1992. O princípio declara que: ``para alcançar o desenvolvimento
sustentável, a proteção ambiental constituirá parte integrante do
processo de desenvolvimento e não pode ser considerada isoladamente
deste''\footnote{\emph{Declaração do Rio sobre Meio Ambiente e
  Desenvolvimento}.

  \emph{http://www.onu.org.br/rio20/img/2012/01/rio92.pdf}}. Para as
próximas décadas do milênio se prevê uma fusão entre proteção e
desenvolvimento (desenvolvimento sustentável), temperada pela construção
da resiliência, que se anuncia permanente com detalhados projetos de
melhorias, crescentes bancos de dados e minuciosos monitoramentos das
etapas de aplicação dos programas e resultados destes. Espera-se a
propagação caudalosa de fluxos consensuais em torno do desenvolvimento
sustentável distribuído por setores e temas, detalhados nas 169 metas
dos \versal{ODS}.

Comparando-se os \emph{Objetivos de Desenvolvimento Sustentável} com as
metas da \emph{Agenda 21}, assinala-se a confluência de temas: combater
a pobreza, diminuir as desigualdades, assegurar a educação, promover a
paz e a inclusão, entre outras metas. Uma diferença de base é a
proveniência dos dois documentos. A \emph{Agenda 21} foi elaborada no
âmbito de uma conferência ambiental, a \versal{ECO}-92, enquanto que os \versal{ODS}s
formam o grande programa das Nações Unidas envolvendo todo sistema e
todos os Estados-membros. Ambos procuram, porém, efetivar o
desenvolvimento econômico capitalista com proteção ao meio ambiente,
marca da sustentabilidade; ambos têm o planeta como alvo e ambos
expressam as interfaces diplomáticas como prática recomendável e
preponderante.

No documento \versal{ODS} introduziu-se em alguns objetivos uma noção citada na
\emph{Agenda 21} básica aprovada na Conferência \versal{ECO}-92:
\emph{resiliência}\footnote{Na versão brasileira da \emph{Agenda 21},
  aprovada em 2004, não há referências à \emph{resiliência.}}\emph{.}
Segundo a \versal{ONU}, ``resiliência significa a capacidade de resistir ou de
ressurgir de um choque. A resiliência de uma comunidade a possíveis
eventos que resultam de uma ameaça se determina pelo grau dos recursos
necessários que possui e da capacidade de se organizar tanto antes
quanto durante os momentos de urgência''\footnote{Verbete
  \emph{Resiliência}.
  \emph{http://www.unisdr.org/files/7817\_UNISDRTerminologySpanish.pdf}}.
Resiliência é a capacidade de se adaptar e se reorganizar frente a
impactos e forças destrutivas, inicialmente descrita pela física. O uso
pioneiro do conceito em ecologia foi um estudo sobre estabilidade dos
ecossistemas, \emph{Resilience and stability of ecological systems},
publicado em 1973, pelo ecólogo canadense Stanley Holling, reconhecido
atualmente como a maior autoridade dessa teoria\footnote{Stanley Holling
  e Lance Gunderson estão diretamente vinculados ao Stockholm Resilience
  Centre advances research on the governance of social-ecological
  systems with a special emphasis on resilience ― the ability to deal
  with change and continue to develop
  (\emph{http://www.stockholmresilience.org/2.aeea46911a3127427980003200.html}).
  Este centro mostra-se como referência para o planeta e a marca de suas
  pesquisas estabelece relação direta entre resiliência,
  sustentabilidade e empreendedorismo.}. A partir de 1997, data da
fundação da revista \emph{Conservation Ecology}, por Holling, renomeada
em 2003, como \emph{Ecology and Society: a jornal of integrative science
for resilience and sustainability}\footnote{\emph{http://www.ecologyandsociety.org/index.php}},
o uso do conceito e seus efeitos começaram a ser disseminados.

Na versão base da \emph{Agenda 21}, elaborada em 1992, a resiliência
apareceu como uma qualidade dos ecossistemas do planeta a ser
preservada. Caberia ao conhecimento científico ``fornecer uma maior
compreensão dos solos, da terra, dos oceanos, da atmosfera e sua conexão
com a água; nutrientes, os ciclos biogeoquímicos e energia, todos os
fluxos que fazem parte do sistema Terra. Isso é essencial para uma
estimativa mais acurada da capacidade de suporte da Terra e de sua
resiliência às tensões produzidas pelas atividades humanas''\footnote{United
  Nations Conference on Environment \& Development. Rio de Janeiro,
  1992. \emph{\versal{AGENDA} 21} (inglês), § 35.2 (Não há numeração de páginas
  no documento citado).

  \emph{https://sustainabledevelopment.un.org/content/documents/Agenda21.pdf}}.
Há duas premissas que marcam a entrada da resiliência na problemática
ambiental: a primeira é que não se pode separar sistema natural do
humano, e a outra, que as respostas dos ecossistemas ao uso humano e os
efeitos dessa interação, embora não previsíveis, podem ser gerenciados.

Trinta anos após o impacto da publicação de \emph{Limites do
Crescimento}, resultado dos estudos promovidos pelo Clube de Roma,
Donella Meadows e equipe reavaliam as conclusões e introduzem a
necessidade de se considerar a resiliência do sistema que suporta a
população humana (Meadows, 2004: 49). Para medir a pressão sobre a
resiliência do planeta e estabelecer ações de mitigação para se atingir
a sustentabilidade, os autores apontam para o monitoramento da pegada
ecológica (\emph{ecological footprint}), termo que alude aos rastros que
a humanidade deixa na Terra com indicadores referentes ao consumo de
recursos, como ar, água e solo. A \versal{ONG} \emph{World Wildlife Fund ―}
\versal{WWF}\footnote{\emph{http://wwf.panda.org/}} realiza e publica estudos
sobre as condições dos recursos planetários com essa metodologia,
reunidos no relatório bianual \emph{Living Planet Report.} Não apenas
governos e organizações, mas também cada pessoa pode avaliar o grau de
sua ``pegada ecológica'' e orientar \emph{pari passu} o comportamento
para colaborar com o que se estabeleceu para um planeta sustentável.
\footnote{``Calcule sua pegada ecológica''.

  \emph{http://www.wwf.org.br/natureza\_brasileira/especiais/pegada\_ecologica/}}

No entanto, foi durante a Conferência Mundial sobre o Desenvolvimento
Sustentável de Johannesburgo, em 2002, que a resiliência ganhou destaque
nas discussões ambientais da \versal{ONU}. O governo sueco contribuiu com um
relatório elaborado da teoria com pesquisas recentes e estudos de casos
sobre o gerenciamento ambiental em um mundo em transformação, cujos
resultados apontaram a conexão estreita entre resiliência, diversidade e
sustentabilidade dos sistemas socioambientais (Folke et al., 2002). A
resiliência apresenta-se como uma qualidade de objetos, sistemas e
pessoas que lhes garante a sustentabilidade e que deve ser gerenciada e
fortalecida, ou mesmo construída. Notou-se que a sustentabilidade não é
alcançada pela promoção de um equilíbrio estático, mas pela capacitação
em se adaptar a situações em constante transformação. O funcionamento do
sistema se mantém, mas sem deixar de ter suas capacidades e limites
testados e as inovações promovidas. Um efetivo desenvolvimento
sustentável seria possibilitado com o fortalecimento ou construção da
resiliência das pessoas e do ambiente capacitando-os a se adaptarem ao
inesperado.

O sentido geral de resiliência, a capacidade de adaptação de um
ecossistema às pressões sem se degradar, serve de base à
sustentabilidade e possibilita avaliações dos efeitos atribuídos às
chamadas ações humanas. A teoria da resiliência no gerenciamento do meio
ambiente é colocada em prática pela gestão e governança adaptativas que
permitem tanto evitar configurações indesejáveis que destruam o sistema
quanto conhecer os pontos de resiliência dos sistemas para
fortalecê-los. A noção de resiliência traz o deslocamento de uma visão
da natureza e da sociedade baseada em um equilíbrio estrutural para uma
posição relativa baseada no movimento e no dinamismo, a qual enfatiza
relações não lineares entre elementos em constante mudança dentro do
sistema, enfrentando descontinuidades e incertezas trazidas por choques
e impactos, repercutindo conflitos, mas tendo por alvo encontrar uma
situação de estável harmonia.

Para as discussões da Rio+20, em 2012, o Secretário Geral da \versal{ONU}
organizou um documento intitulado \emph{Povos resilientes, Planeta
Resiliente: um futuro digno de escolha}, com 65 recomendações dirigidas
aos governos dos Estados-membro. ``O desenvolvimento sustentável não é
uma meta, mas um processo dinâmico de adaptação, aprendizagem e ação.
Trata-se de reconhecer, compreender e atuar nas interconexões ---
especialmente aquelas entre a economia, a sociedade e o meio ambiente
natural. (...) Já houve progresso, mas ele não tem sido nem rápido nem
profundo o suficiente e a necessidade de uma ação de maior alcance está
se tornando cada vez mais urgente'' \footnote{\emph{Povos Resilientes,
  Planeta Resiliente: um Futuro Digno de Escolha}, p. 9.
  http://www.onu.org.br/docs/gsp-integra.pdf}. As recomendações se
concentram em três grupos de ações: 1) qualificar as pessoas para
escolhas sustentáveis assegurando-se ``os direitos humanos, a satisfação
de necessidades básicas e a resiliência humana'', 2) trabalhar para uma
economia sustentável, e principalmente, 3) fortalecer a governança
institucional para o desenvolvimento sustentável\footnote{Ibidem: 47.}.
A resiliência seria construída pelos Estados por duas vias: mecanismos
de proteção social (recomendação 24) e redução dos riscos de desastres
(recomendação 26)\footnote{Ibidem: 72-76.}.

O termo resiliência apareceu também no documento final da Conferência
das Nações Unidas para o Desenvolvimento Sustentável de 2012\emph{, O
futuro que queremos}\footnote{\emph{O futuro que queremos}.
  \emph{http://www.mma.gov.br/port/conama/processos/61AA3835/O-Futuro-que-queremos1.pdf}},
como uma capacidade dos ecossistemas e da população, mediante medidas de
proteção social, dos ambientes urbanos, e de enfrentamento dos
desastres. Ambos documentos forneceram subsídios para a elaboração de
uma nova agenda para o planeta: a \emph{Agenda 2030} e os
\emph{Objetivos de Desenvolvimento Sustentável}.

No preâmbulo da \emph{Agenda} \emph{2030} estabeleceu-se um compromisso:
``Todos os países e todos os grupos interessados {[}stakeholders{]},
atuando em parceria colaborativa, implementarão este plano. Estamos
decididos a libertar a raça humana da tirania da pobreza e da privação e
a sanar e proteger o nosso planeta. Estamos determinados a tomar medidas
ousadas e transformadoras que se necessitam urgentemente para pôr o
mundo em um caminho sustentável e resiliente. Ao embarcarmos nessa
jornada coletiva, comprometemo-nos a não deixar ninguém para
trás''\footnote{\emph{Transformando Nosso Mundo: a Agenda 2030 para o
  Desenvolvimento Sustentável}, p. 1.

  \emph{http://www.pnud.org.br/Docs/Agenda2030completo\_PtBR.pdf}}. A
imagem de um planeta como um ambiente a ser \emph{sanado} da degradação
e da miséria visando sua melhoria mediante um gerenciamento
\emph{coletivo e colaborativo} foi possibilitada pela paulatina
construção e funcionamento do \emph{dispositivo meio ambiente} enquanto
ponto de conexão de elementos dispersos e de pacificação de conflitos
entre eles.

A resiliência apresenta duplo aspecto, como qualidade de objetos e
pessoas que promovem a sustentabilidade dos sistemas socioambientais e
também como qualidade a ser gerenciada e construída em objetos e
pessoas. Como a sustentabilidade não se alcança pela promoção de um
equilíbrio estático, a capacitação em se adaptar a um mundo em constante
transformação testa suas capacidades e traz inovações. Um efetivo
desenvolvimento sustentável é possibilitado, portanto, com a construção
de condutas resilientes de pessoas e, por conseguinte, do ambiente por
meio de capacitação para estes se adaptarem ao inesperado com risco
calculável para conter rupturas. Resiliência não se confunde com a
sustentabilidade, esta é uma meta a ser atingida, mas surge como um
atributo que precisa ser estudado, protegido e estimulado, como uma
qualidade \emph{adquirida} das coisas e gentes que as faz se adaptarem
às mudanças climáticas, inclusive às catástrofes naturais e a outras
denominadas situações de risco. A capacidade de suporte de um
ecossistema pode ser trabalhada visando fortalecer a resiliência desses
locais e, com isso, possibilitar um uso mais eficiente, um
desenvolvimento \emph{melhor}. Protege-se o ambiente por meio da
promoção da resiliência para que catástrofes e pressões, climáticas ou
não, sejam enfrentadas e equacionadas sem a perda de sua qualidade,
destinada a futuras gerações.

O \emph{dispositivo meio ambiente} colocou a humanidade em um mesmo
patamar de risco, de algum modo unificando-a como habitante de
``\emph{uma só Terra}''; e, simultaneamente, promoveu soluções para a
segurança do planeta, considerado um ambiente em que ocorre a vida.

\chapter{Segurança planetária}

Europa centro-ocidental, período que se convencionou designar como final
da Idade Média e início da Era Moderna. Entre os séculos \versal{XIV} e \versal{XVII}, foi
de intensa transformação nos campos político, militar, cultural e
econômico. A dinamização das atividades produtivas e comerciais foi, ao
mesmo tempo, instigadora e provocada pelo fortalecimento de cidades de
vocação mercantil e de importantes associações entre poderes políticos e
econômicos voltadas ao controle territorial e à busca de novos mercados
e fontes de produtos cobiçados. O crescimento da circulação pelos mares
e rios interiores foi potencializado pelas navegações pela costa
africana que prepararam, no final do século \versal{XV}, a chegada à Índia e,
imediatamente depois, a travessia do Atlântico.

Os modos medievais de pensar, de ver o mundo e o cosmo, de classificar e
de se apropriar do território e suas riquezas, assim como das populações
e suas forças, foram fortemente abalados, pressionando para a
modificação de antigas modalidades de governo de coisas, espaços e
pessoas. Interessado em analisar genealogicamente como emergem e se
redimensionam as práticas de governo, Michel Foucault voltou-se àquele
momento histórico, perscrutando proveniências de uma nova razão
governamental que se cristalizou nos séculos seguintes até o despontar
da biopolítica das populações, já na passagem do século \versal{XVIII} para o
século \versal{XIX}. Assim, no curso \emph{Segurança, Território, População},
apresentando no Collège de France entre janeiro e abril de 1978,
Foucault dedicou-se a identificar e analisar aquilo que reputou como uma
\emph{nova arte de governar}, formulada nos finais da Idade Média, que
``tinha sua própria razão, sua própria racionalidade, sua própria
\emph{ratio}'' (2008: 383).

Essa nova \emph{racionalidade de governo} afirmou-se diante de outras
modalidades concorrentes, principalmente a fragmentação político-militar
feudal e os projetos de unificação do espaço europeu que remetiam à
memória do Império Romano, depois herdada pelo universalismo da Igreja
Católica, e que permaneceria no imaginário político europeu do
Iluminismo em diante. Essa nova racionalidade ― consagrada no discurso
jurídico-político como ``Razão de Estado'' ― despontou como um conjunto
tático que combinava elementos do poder político centralizado
proveniente da experiência romana sem, no entanto, reconfigurar um novo
\emph{imperium} sobre toda a Europa.

Foucault (2008) recuperou, então, os Tratados de Westfália, assinados
nas cidades de Münster e Osnabrück no ano de 1648 e que colocaram fim à
Guerra dos Trinta Anos, compreendendo-os como um acontecimento em meio à
emergência dessa nova razão governamental. Tais acordos instituíram uma
inédita e precária forma de equilíbrio demográfico, econômico, militar e
territorial entre unidades políticas que simultaneamente manifestavam o
projeto de não se submeter a qualquer poder político exterior ou
universal, combinado com a meta de se afirmar como fontes únicas de
autoridade política dentro de um território delimitado.

A negação dos príncipes modernos de se sujeitar a qualquer outro poder
se explicava pelo fato de cada novo Estado considerar-se ``seu próprio
fim'', subordinado apenas ``a si mesmo'' (Foucault, 2008: 389). Para
Foucault, como todas essas novas unidades soberanas buscavam o mesmo fim
― subsistir e crescer ― num espaço geográfico limitado, a competição e
as fricções entre elas seriam inevitáveis. Desse modo, a Europa moderna
emergiu como um ``espaço de concorrência'' (Ibidem: 389) e de tensa
relação entre unidades assemelhadas com objetivos semelhantes, num
estado de inter-relações supostamente estático, sem mudanças nas
disputas das relações entre Estados, mas também, sem escatologia ou fim
diante do despontar de um possível ``novo Império'' a unificar todos os
povos europeus.

O problema central para cada Estado seria, então, como permitir que cada
unidade soberana não apenas sobrevivesse, mas também crescesse, sem que
essa busca por aumento de riqueza e poder militar levasse ao fim do novo
sistema de Estados pelo retorno do \emph{princípio} \emph{imperial}. Daí
a importância dos Tratados de Westfália, pois foram eles que encontraram
na mencionada fórmula do equilíbrio ou balança de poder, um modo de
produzir um \emph{efeito de segurança} gerado, precisamente, pela
semelhança do poderio material de uma aristocracia de Estados, ao redor
dos quais orbitariam pequenos Estados satélites. Essa situação de
equilíbrio instituiria uma \emph{previsibilidade} na ação dos grandes
Estados e, ao menos temporariamente, serviria como estímulo dissuasório,
ou seja, não incentivaria que um Estado atacasse outro rival em
condições de resistir ou, até mesmo, de vencer uma guerra.

A atenção de Michel Foucault aos Tratados de 1648 não o fez tomar esses
documentos como o fazem as teorias tradicionais das Relações
Internacionais. As tradições realista e liberal nas \versal{RI} entendem os
acordos de Westfália como o momento de instauração ou de \emph{origem}
das relações internacionais. De modo distinto, Foucault compreendeu
Westfália como um momento de \emph{emergência} de uma nova Razão de
Estado e de um \emph{sistema geral de segurança entre os Estados}
vinculado à afirmação dessa própria razão governamental.

Esse sistema não seria natural, tampouco estático, mas um tenso,
precário e historicamente constituído equilíbrio de Estados em um
``espaço internacional'' emergente. Ademais, Foucault indicou que não se
poderia analisar a formação dos Estados Modernos europeus focando apenas
nos processos internos de concentração de poder ou, inversamente,
reduzindo-se a análise apenas ao plano externo representado pelas
celebrações de tratados e de equilíbrios de poder entre os príncipes.
Para Foucault, a segurança de cada nova unidade soberana dependeu da
acoplagem entre dois ambientes de segurança, um interno e outro externo,
em co-constituição.

Essa articulação foi necessária para enfrentar o desafio de governar a
tensão interna entre crescimento das forças econômicas e a maximização
das riquezas humanas e materiais dos reinos sem que isso provocasse
desequilíbrios entre grupos sociais e regiões que abalassem a própria
existência do Estado. No entanto, esse equacionamento interno não seria
possível sem garantir alguma proteção com relação a ataques e agressões
de outros Estados. Segundo Foucault, para gerir esses dois ambientes de
segurança, os Estados modernos e a emergente Razão de Estado elaboraram
dois dispositivos irmanados: o \emph{diplomático-militar}, voltado para
a o exterior, e o \emph{de polícia}, visando o interior de cada Estado.

O componente \emph{diplomático} do dispositivo diplomático-militar foi
constituído pela produção de um sofisticado aparato de representação
oficial dos príncipes gerido por funcionários especializados ― os
diplomatas ― instruídos para falar pelo soberano em negociações junto a
outros representantes estatais e, ao mesmo tempo, capazes de informar os
seus príncipes sobre as forças e capacidades econômicas e militares dos
demais Estados. De forma distinta das antigas missões diplomáticas
medievais ― efêmeras e destinadas a solucionar problemas pontuais ou a
selar alianças circunstanciais, a diplomacia moderna possibilitou que os
príncipes se comunicassem constantemente e conseguissem informações a
respeito da força e vigor dos seus concorrentes. Esses dois elementos
foram importantes para que os grandes Estados pudessem manter certa
equiparação das suas capacidades militares (pois cientes do aumento da
força militar de seus rivais), além de planejar, no plano interno, ações
e incentivos para setores produtivos visando a autossuficiência e a
possibilidade de exportar de modo a acumular ouro (seguindo um dos
princípios do mercantilismo).

O componente \emph{diplomático} foi complementado pelo \emph{militar},
baseado numa ``nova economia das armas'' que, para Foucault (2002) foi
fundamental para garantir a autoridade dos novos monarcas sobre seu
próprio reino e para confrontar seus novos competidores no espaço
europeu. Os novos exércitos régios, de dispendiosa manutenção, exigiram
uma reformulação do aparato tributário dos emergentes reinos modernos,
levando a uma capacidade ampliada de controlar e extorquir riqueza da
população. Ao mesmo tempo, o desenvolvimento de novas tecnologias
bélicas ― como as armas de fogo e artilharia ― provocaram modificações
no treinamento e habilidades de soldados, novos problemas para o
deslocamento de tropas e seus armamentos e uma redimensionada
arquitetura militar capaz de resistir aos tiros de canhão e às táticas
de cerco (Keegan, 2002).

Em suma, quando voltado para as relações com outros Estados, o
dispositivo militar foi, antes de tudo, destinado a dissuadir Estados
concorrentes a atacar para, somente então, ser um elemento agressivo
diante da possibilidade de conquistar vantagens pela violência. Assim, a
relação entre \emph{diplomacia} e \emph{força militar} foi estabelecida
como um ``complexo político-militar absolutamente necessário à
constituição desse equilíbrio europeu como mecanismo de segurança''
(Foucault, 2008: 409) no qual a \emph{guerra} passou a ser uma política
de Estado aplicada em relação aos ``interesses'' da Razão de Estado.

O complemento interno do dispositivo diplomático-militar foi o
\emph{dispositivo de polícia}. Segundo Foucault, o conceito de
``polícia'' designou, a partir do século \versal{XVII}, ``o conjunto de meios
pelos quais seria possível fazer as forças do Estado crescerem, mantendo
ao mesmo tempo a boa ordem desse Estado'' (2002: 421). Assim,
``polícia'' seria o conjunto de técnicas e de cálculos que permitiriam o
difícil equacionamento entre o crescimento interno das forças do Estado
(a dinâmica interna das forças) e a ordem geral desse Estado.

O objetivo das técnicas de ``polícia'' visava a ``assegurar o esplendor
do Estado'' e a ``felicidade de todos os cidadãos'' (Ibidem: 422),
fazendo ``da felicidade dos homens a utilidade {[}e a própria força{]}
do Estado'' (Ibidem: 439). Para tanto, foram desenvolvidos novas
técnicas e saberes, como a \emph{estatística}, a ``ciência do Estado'',
voltados para calcular e medir as ``forças constitutivas do Estado''
(Ibidem: 424). Conhecer para regulamentar; regulamentar como prática de
governo. Por isso, um dos alvos da ``polícia'' foi conhecer as
capacidades e habilidades dos súditos de modo a produzir regulamentos
sobre como e onde empregá-las e melhor organizá-las para seu benefício
individual e do reino.

Evitar a escassez de alimentos, por exemplo, foi preocupação das
técnicas do dispositivo de polícia, pois significava, ao mesmo tempo,
promover a saúde física para o trabalho e inibir revoltas e sublevações.
Desse modo, o \emph{dispositivo de polícia} foi ``o conjunto de
intervenções e meios que garantem que viver, melhor que viver,
coexistir, será efetivamente útil à constituição, ao aumento das forças
do Estado'' (Foucault, 2008: 438). O conceito de ``polícia'' começou a
ser superado apenas com a crítica dos economistas fisiocratas que, a
partir do século \versal{XVIII}, procuraram demonstrar que a ênfase na busca de
produção a baixos preços para um lucro no comércio exterior não
aumentava o retorno financeiro ao camponês, devendo haver uma regulação
de preços pelo mercado e não pela intervenção estatal. Os fisiocratas
acreditavam que uma \emph{regulação} \emph{natural} regeria tanto a
produção e oferta de grãos quanto o aumento ou decréscimo das
populações, levando ao equilíbrio entre ambos.

A crítica fisiocrata coincide historicamente com a emergência do que
Foucault caracterizou como ``problema político das populações'',
concentradas em número cada vez maior nas cidades europeias do início da
Revolução Industrial, em finais do século \versal{XVIII}. A identificação desse
problema levou à emergência de uma nova governamentalidade que ele
denominou como a biopolítica das populações (Foucault, 2003). A partir
de então, ``polícia'' deixou de ser esse dispositivo de regulamentação
geral da vida do Estado, passando a ser apenas o braço armado dos
Estados voltados para dentro das fronteiras políticas: modalidade de
exército menos poderoso em suas capacidades bélicas e destinado a
controlar populações desarmadas ou quase, enquanto os exércitos e
marinhas de guerra continuavam com recursos mais expressivos para
possíveis guerras interestatais. Portanto, com a entrada do século \versal{XIX},
o ``dispositivo de polícia'' teria se redimensionado em biopolítica das
populações, enquanto o dispositivo diplomático-militar, voltado para o
plano exterior, continuava com suas práticas e objetivos.

Na aula de 22 de março de 1978, ao comentar como se consolidou o sistema
de Estados do início da Era Moderna, Foucault afirmou que ``a Europa
como região geográfica de Estados múltiplos, sem unidade, mas com
desnível entre pequenos e grandes, tendo com o resto do mundo uma
relação de utilização, de colonização, de dominação (...) é a realidade
histórica de que ainda não saímos. É isso que é a Europa'' (2008: 400).
Como desdobramento dessa reflexão, Foucault, afirmará duas semanas
depois, na aula de 29 de março, que apesar das diferenças importantes no
modo como cada Estado organizou seu dispositivo de polícia, as formas de
articulação dos dispositivos diplomático-militares foram
``compartilhadas'' entre os Estados europeus e, depois deles,
universalizadas no sistema internacional a partir da adoção global do
modelo europeu de Estado, seguindo os processos de independência nas
Américas, nos séculos \versal{XVIII} e \versal{XIX}, e na África e Ásia na segunda metade
do século \versal{XX}.

No plano doméstico, segundo Foucault, as táticas governamentais sofreram
transformações ainda mais significativas com a emergência da biopolítica
sobrepondo-se à Razão de Estado das sociedades de soberania e, já a
partir dos anos 1970, com o despontar de uma outra governamentalidade ―
a neoliberal ― que Foucault começava a observar e problematizar (em
especial, a partir do seu curso de 1979, \emph{O nascimento da
biopolítica}). No entanto, ainda em 1978, Foucault nota as modificações
nas práticas de governamentalização do Estado, mas afirma que a lógica
do equilíbrio de poder num sistema de competição, apesar de não ser mais
exclusivamente europeu, tendo abarcado todo o globo, continuaria a ser
``a realidade histórica de que ainda não saímos'' (Foucault, 2008: 476).

Se é possível aceitar que, em linhas gerais, o dispositivo
diplomático-militar tenha passado mais ou menos inalterado pelos
redimensionamentos da razão governamental no plano interno aos Estados
entre os séculos \versal{XVII} e \versal{XX}, talvez não o seja mais a partir de todas as
alterações na dinâmica das relações políticas e econômicas mundiais,
principalmente aquelas ocorridas ou aceleradas a partir da \versal{II} Guerra
Mundial.

A análise dos novos problemas de segurança, dos redimensionamentos
jurídico-políticos nacionais e internacionais e da emergência de novas
formas de violência para além das guerras entre Estados evidenciam uma
transformação no sistema de segurança no plano global. Os modos pelos
quais os Estados em suas coalizões novas ou atualizadas, em parceria com
dimensões locais e globais de empresas e ativismos na chamada sociedade
civil, passam a gerenciar a segurança do planeta não prescindem das
instituições inauguradas pelos Estados Modernos e operadas pelo
dispositivo diplomático-militar. No entanto, há um importante
redimensionamento de práticas de segurança e táticas de governo das
``inseguranças'' que não respeitam os limites das fronteiras nacionais e
as distinções entre ``dentro'' e ``fora'' tão caras às teorias das
Relações Internacionais.

Na ecopolítica do planeta não está mais em jogo o governo dos Estados de
modo isolado, preocupados com a gestão de suas populações e territórios.
Do mesmo modo, alteram-se as práticas para a segurança do Estado, do
conjunto de Estado no globo, do próprio planeta e do entorno sideral
como ``espaço comum'' da Humanidade. Enfim, a preocupação em governar
não desaparece, mas se redimensiona em programas para a gestão do
planeta com seus recursos, novas e antigas ameaças à paz e à ordem
política e econômica, com suas inteligências produtivas, sua circulação
de produtos e seus espaços com populações aproveitáveis e não
aproveitáveis pelo capitalismo global.

Desse modo, as táticas para o governo do planeta acionam um novo
dispositivo de segurança voltado para o globo e seu entorno, utilizando
as tecnologias computo-informacionais e as tecnologias de rastreamento e
monitoramento por satélites para mapear, demarcar e regular fluxos
transterritoriais. Garantir a segurança do planeta significa, então,
manejar ou gerir ``ameaças'' de modo a contê-las, direcioná-las,
manejá-las. Essa \emph{gestão das ameaças} faz parte, então, de um
emergente governo transterritorial das violências e das práticas
individuais e coletivas, vivas e circulantes nos fluxos eletrônicos e
nos espaços terrestres.

\emph{A guerra é a saúde do Estado}.

Randolph Bourne.

Noite de 16 de janeiro de 1991. Em pronunciamento na televisão, o então
presidente dos Estados Unidos, George H. W. Bush, anunciou o início de
uma grande operação militar, a \emph{Desert Storm} (Tempestade no
Deserto), liderada por seu país em coalizão com outros sete Estados,
dentre os quais se destacavam aliados da Organização do Tratado do
Atlântico Norte (\versal{OTAN}), como o Reino Unido, França e Canadá\footnote{Os
  outros Estados que enviaram tropas e/ou equipamentos militares foram
  Itália, Austrália, Nova Zelândia e Argentina.}. O objetivo da ação era
expulsar as tropas iraquianas que ocupavam o Kuwait desde agosto de
1990.

Na \versal{TV}, Bush afirmou que o que estava ``em jogo era muito mais que um
país pequeno {[}o Kuwait{]}; mas uma grande ideia: uma nova ordem
mundial, na qual diversas nações se unem em torno de uma causa comum
visando às aspirações universais da humanidade ― paz e segurança,
liberdade e o Estado de direito {[}\emph{rule of law}{]}'' (Bush, 1991:
01).

A nova ordem mundial seria aquela que despontava com o enfraquecimento
das tensões próprias da Guerra Fria, com a debilidade e o iminente
desmonte do projeto socialista de Estado defendido pela \versal{URSS}. Nessa
``nova ordem'', a invasão de um país por outro não seria mais tolerável
e os Estados, coligados na \versal{ONU}, ficavam, a um só tempo, obrigados e
habilitados a agir em nome da paz e da segurança internacional.

Esses dois elementos, ``paz'' e ``segurança'' internacionais, são as
bases sobre as quais foi alicerçado o próprio direito internacional que
emergiu da \versal{II} Guerra Mundial. O preâmbulo da \emph{Carta de São
Francisco}, assinada em junho de 1945 e que marca a fundação da \versal{ONU},
traz precisamente essas duas expressões como o duplo fundamental sobre o
qual seria constituída uma \emph{ordem} planetária articulada para
evitar o ``flagelo da guerra''\footnote{\emph{A} c\emph{arta das Nações
  Unidas}. Preâmbulo. \emph{https://nacoesunidas.org/carta/}}. Bush, ao
evocá-los, fazia mais do que simplesmente justificar uma mobilização
militar. Estava em jogo afirmar uma nova lógica para gerir o planeta, na
qual o antagonismo entre capitalismo e socialismo estaria superado
diante da vitória política, ideológica, militar e econômica do primeiro
sobre o segundo. A ``libertação do Kuwait'', portanto, marcaria o
surgimento de novas correlações de força no planeta, nas quais teriam
triunfado a democracia liberal, como regime político, e capitalismo
globalizado como sistema econômico.

A Operação Tempestade no Deserto foi, de fato, inédita. Pela primeira
vez na história, o sistema de ``segurança coletiva'' foi colocado em
prática. Previsto no direito internacional desde o Pacto da Liga das
Nações, firmado em 1919, e incorporado à \emph{Carta de São Francisco},
esse princípio fez parte do redimensionamento da noção de \emph{guerra
justa} (\emph{jus ad bellum}, o direito de recorrer à guerra) presente
num momento histórico marcado pelas duas guerras mundiais e pela
superação de uma ordem europeia diante da emergência dos \versal{EUA} como
potência mundial. Com proveniências nas recomendações para a paz
perpétua, elaboradas pelo filósofo prussiano Immanuel Kant, ainda no
século \versal{XVIII}, o princípio da ``segurança coletiva'' compreende que num
sistema regido por compromissos universais de autonomia
jurídico-política, a violação da soberania de um Estado mobilizaria a
ação concertada de todos os demais para reparar a agressão, punir o
agressor e manter a ordem mundial pactuada (Dinstein, 2004; Herz,
Hoffmann e Tabak 2015).

A ativação desse recurso foi bloqueada desde que o Conselho de Segurança
da \versal{ONU} foi efetivamente paralisado após a cristalização do enfrentamento
entre a \versal{URSS} e os \versal{EUA}. Esse embate, que marca a chamada Guerra Fria, não
se seguiu imediatamente ao final da \versal{II} Guerra e à celebração de pactos
entre potências aliadas diante da derrota dos Estados do Eixo (Alemanha,
Japão e Itália). A divisão do planeta em grandes esferas de influência,
tendo a Europa como espaço inicial de disputa, ganhou impulso com a
explosão da primeira bomba atômica pela \versal{URSS}, em 1948, e com a
assinatura do \emph{Tratado do Atlântico Norte}, em 1949, criando a \versal{OTAN}
como aliança militar liderada pelos \versal{EUA} contra a potencial ameaça
soviética.

O Conselho de Segurança foi instituído na \versal{ONU} como o centro de decisões
sobre os temas de ``paz e segurança'' regido pela regra do consenso
entre os cinco membros permanentes ― \versal{EUA}, \versal{URSS}, França, Reino Unido e
China ― representando os Estados vencedores do segundo conflito mundial.
O envolvimento maior da \versal{ONU} e de seus Estados-membros em questões de
segurança internacional, como, por exemplo, a decisão sobre uma
intervenção coletiva em um conflito específico, precisaria ao menos de
discordância explícita (o ``veto'') de um desses membros. Por isso, a
rivalidade entre estadunidenses e soviéticos fez do Conselho de
Segurança um órgão praticamente inoperante diante das maiores ``crises
de segurança'' e guerras, como o conflito no Vietnã, de finais dos anos
1950 até princípios dos anos 1970, ou do Afeganistão, entre 1979 e 1989,
respectivamente, datas da invasão e da retirada militar soviética.

No entanto, no momento em que George H. W. Bush anunciou a intervenção
militar no Kuwait, a \versal{URSS} não vetou a decisão, permitindo que uma
resolução fosse aprovada e conferindo, diante do direito internacional,
legalidade e legitimidade à coalizão. Eram tempos de acirramento da
crise política, econômica e social pela qual passava a \versal{URSS}, indicando o
esgotamento do seu modelo político relacionado à economia planificada
dependente do mercado internacional capitalista. Nesse contexto, após
meses de sanções econômicas e reprimendas diplomáticas contra o Iraque,
o Conselho de Segurança, sem a oposição de chineses e soviéticos,
aprovou a Resolução 678 de 29 de novembro de 1990, que estabeleceu o dia
15 de janeiro de 1991 como prazo final para que o Iraque desocupasse o
Kuwait. Se isso não acontecesse, estava autorizado o ``uso de todos os
meios necessários'' para implementar as decisões do Conselho, jargão
diplomático utilizado para designar a permissão para utilizar força
militar. O prazo chegou, o Estado comandado por Saddam Hussein não
acatou o que a \versal{ONU} lhe impôs e, no dia seguinte, Bush fez seu famoso
pronunciamento televisivo.

O sucesso militar da Tempestade no Deserto foi estrondoso. Em pouco mais
de um mês, mísseis teleguiados lançados de navios de guerra e submarinos
nucleares, caças supersônicos e invisíveis, blindados computadorizados e
soldados com treinamento e equipamentos de última geração fizeram as
precárias tropas iraquianas retroceder a Bagdá. A guerra foi transmitida
ao vivo pelo canal de televisão a cabo estadunidense \versal{CNN} e por demais
agências internacionais, enquanto outras imagens de vídeo, filmadas
desde os aviões militares ou das ogivas dos mísseis, mostravam alvos na
mira e explosões supostamente precisas.

A chamada Guerra do Golfo foi celebrada pela coalizão vitoriosa como a
prova de como seriam as guerras da ``nova ordem mundial'': precisas,
pontuais, com alvos bem delimitados atingidos por bombardeios
``cirúrgicos'', legalmente autorizadas pela \versal{ONU} e moralmente legitimadas
em nome da defesa da ``liberdade'', da ``paz'' e da ``segurança''
\emph{para todos}. A mudança das formas de governar as relações entre
Estados anunciada pelo presidente estadunidense Thomas Woodrow Wilson,
ainda em 1918, quando da sua defesa de uma Liga da Nações como foro para
superar a lógica da guerra como instrumento da política exterior dos
Estados, parecia, então, ter finalmente se realizado entre os escombros
iraquianos, os corpos no deserto e as imagens apocalípticas de poços de
petróleo incendiados com suas imensas colunas negras de fumaça.

Naquele momento, a euforia da ``nova ordem mundial'' se espraiou nos
discursos acadêmicos. Ganharam destaque obras como \emph{O fim da
História e o último Homem}, livro do cientista político estadunidense
Francis Fukuyama, publicado em 1992, em que se defende o argumento de
que a vitória ideológica e material da democracia liberal capitalista
anunciava a ``paz perpétua'' para o mundo. Para Fukuyama, à medida em
que se universalizasse esse modelo, as novas democracias capitalistas
colaborariam em harmonia. No entanto, guerras, como a do Golfo, seriam
necessárias para enfrentar os Estados ou elites políticas que
insistissem no autoritarismo político e no protecionismo econômico. Essa
tese, portanto, não era nova, pois traduzia o projeto kantiano à luz do
universalismo estadunidense (Doyle, 1983).

O novo ``governo do mundo'', enfim, seria \emph{partilhado} por
\emph{todos} os povos ― ``we the peoples of the United Nations'' (``Nós,
os povos das Nações Unidas''), como registra o Preâmbulo da \emph{Carta
da \versal{ONU}}. Uma projeção política e econômica de um mundo em que os
``povos'' estariam conectados pelas tecnologias de informação e de
transporte produzidos e potencializados por um capitalismo globalizado,
enquanto irmanados pela democracia liberal e pela observância do regime
universal dos direitos humanos. Em termos de segurança, a ``nova ordem
mundial'' parecia, então, superar a racionalidade da guerra entre
Estados consolidada desde a formação dessas unidades políticas, na
Europa entre os séculos \versal{XIV} e \versal{XVII}: a busca do ``interesse nacional'',
traduzido em termos de sobrevivência do Estado e de ampliação das suas
capacidades de poder, justificou historicamente o recurso à guerra desde
que ele obedecesse a um cálculo utilitário de vantagens ou custos de uma
campanha militar (Keegan, 2002). Essa lógica foi resumida, no início do
século \versal{XIX}, pela máxima do general prussiano e teórico da guerra Carl
von Clausewitz, para quem a guerra era ``o verdadeiro instrumento
político, a continuação da interação política {[}\emph{political
intercourse}{]} por outros meios'' (Clausewitz, 1976: 87).

Nesse sentido, a guerra de agressão foi entendida, praticada e
legitimada como uma opção à qual os Estados poderiam recorrer se a
possibilidade de vitória militar fosse crível. Assim foram a prática da
guerra e o ``\emph{jus ad bellum}'' no ``sistema de segurança''
governado pelo dispositivo diplomático-militar emergido na Europa,
conforme a análise de Foucault (2008). Não é fortutito que a Prússia, de
Kant e Clausewitz, e não a Inglaterra ou a França, tenha sido sublinhada
por Foucault ─ quando ele estudava o nascimento da medicina social,
neste caso em especial, a medicina de Estado onde ele se constitui,
simultaneamente, como objeto e instrumento de conhecimento ─ tenha sido
o primeiro modelo de ``Estado Moderno'' na Europa e para a Europa, e não
por razões negativas como pretendem os teóricos da soberania de liberais
a marxistas que depreendem ``o poder'' como uma dedução sinonímia do
Estado ou este como originário da criação contratual que demarca sua
origem gloriosa assim como a do próprio Direito. No entanto, a ``nova
ordem'' anunciada por Woodrow Wilson logo após a I Guerra Mundial foi
adensada nos anos subsequentes, como indica a assinatura do \emph{Pacto
Briand-Kellogg}, em 1928, que exortou os Estados a nunca mais decidirem
pela guerra de agressão, considerando apenas a possibilidade de ação
militar de autodefesa (em caso de ataques estrangeiros) ou a coletiva,
quando o princípio da segurança coletiva assim o exigisse. Essa premissa
foi reafirmada na institucionalização da \versal{ONU} e, para a literatura
liberal, nos inícios dos anos 1990, teria, finalmente, se libertado para
organizar o mundo segundo os princípios da ``paz'' e da ``segurança''.

As guerras entre Estados pareciam, então, com os dias contados frente à
vitória moral com a ideia da ``paz perpétua'' ao modo estadunidense, ou
seja, liberal, democrática e capitalista. As guerras pareciam, assim,
condenadas às brumas violentas da História e uma nova ``segurança
mundial'' foi vislumbrada como uma modalidade de gerenciamento
diplomático e técnico ― multilateral e partilhado ―, militar quando
necessário (mas mesmo assim, legalizado e coligado), de um destino comum
e inevitável para a humanidade.

\emph{De um Trácio é agora o meu tão belo escudo.}

\emph{Que havia eu de fazer? Perdi-o na floresta.}

\emph{Mas salvei minha pele, no aceso da luta.}

\emph{Sei bem onde comprar um escudo novo.}

Arquíloco de Paros, Grécia, \versal{IV} a.C.

Julho de 1995, Bósnia. Em meio às múltiplas violências que tomaram a
Iugoslávia a partir de 1991, quando Eslovênia e Croácia declararam-se
independentes do poder central em Belgrado, os acontecimentos na região
bósnia de Sbrenica eclodiram. Em duas semanas, cerca de oito mil bósnios
muçulmanos foram mortos por tropas militares bósnias cristãs ortodoxas e
de origem sérvia, apoiadas por paramilitares sérvios. No início daquele
mês, a invasão sérvia de Sbrenica havia levado milhares de bósnios
muçulmanos a pedir proteção numa base de militares holandeses
pertencentes à Missão das Nações Unidas na Bósnia-Herzegovina (\versal{UNMIBH}).
Após dias de cerco, isolados e sem conseguir autorização da \versal{ONU} para que
pudessem usar a força contra os sitiadores, os militares holandeses
abriram os portões de sua base.

Ato contínuo, homens de todas as idades foram fuzilados e enterrados em
covas coletivas, enquanto mulheres e crianças eram brutalizadas,
violentadas sexualmente e assassinadas naquilo que foi designado pelo
Tribunal Penal Internacional das Nações Unidas para a ex-Iugoslávia como
o primeiro genocídio e o maior crime de guerra cometido na Europa desde
a \versal{II} Guerra Mundial (Gibbs, 2015). Essa corte foi instituída em Haia,
Holanda, em 1993, especialmente para julgar os casos de violações de
direitos humanos ocorridos nas guerras que eclodiram na Iugoslávia em
fragmentação política, com base nos princípios e procedimentos previstos
em documentos como a \emph{Declaração Universal dos Direitos Humanos} e
a \emph{Convenção para a Prevenção e Repressão ao Crime de
Genocídio}\footnote{\emph{http://www.direitoshumanos.usp.br/index.php/Sistema-Global.-Declara\%C3\%A7\%C3\%B5es-e-Tratados-Internacionais-de-Prote\%C3\%A7\%C3\%A3o/convencao-para-a-prevencao-e-a-repressao-do-crime-de-genocidio-1948.html},},
ambos de 1948. Os eventos de Sbrenica acrescentaram elementos de horror
a continuadas violências e conflitos que não cessaram apesar do anúncio
triunfante da ``nova ordem mundial''.

Além dos Balcãs, guerras civis, genocídios e assassinatos em massa
continuaram a acontecer na América Latina, na Ásia e na África, com
destaque para os quase um milhão de mortos nos meses de enfrentamento,
em 1994, entre tutsis e hutus em Ruanda. As guerras não acabaram, porém,
tampouco pareciam coincidir com as definições consagradas no direito
internacional e nas práticas diplomático-militares estabelecidas pelos
Estados nacionais.

A Guerra do Golfo não foi uma mera ``guerra entre Estados'', como poucos
anos antes tinham sido as entre Irã e Iraque (de 1980 a 1988) ou entre o
Reino Unido e a Argentina (em 1982). Na guerra para a ``libertação'' do
Kuwait, em 1991, a \emph{justificativa moral e formal} não foi mais a do
``interesse nacional'', mas a da luta por valores universais (como a
``liberdade política'' e os ``direitos dos povos à autodeterminação'').
Ao mesmo tempo, formas de violência que emergiram durante a Guerra Fria,
e principalmente desde os anos 1970, como modalidades de ilegalismos
(com destaque para o narcotráfico) e novas práticas terroristas que não
mais se limitavam à identificação nacional e que realizavam seus
propósitos atravessando fronteiras políticas, começaram a despontar como
``problemas de segurança'', com variados matizes, para muitos Estados de
todo o globo.

Se o parâmetro para decretar a ``nova ordem mundial'' era o fim das
guerras interestatais, os motivos para celebração liberal se
justificavam. Porém, um olhar mais atento para as dinâmicas das
violências pôde captar que as formas de embate e os modos de matar e
morrer eram redimensionados para \emph{além} e para \emph{aquém} das
fronteiras nacionais. Ficou mais evidente, a partir dos anos 1990, o
descolamento entre a prática da violência e a pretensão de monopólio da
coerção física pelos Estados com suas forças militares regulares.
Todavia, esse deslocar-se das violências para além do Estado não surgiu
de repente, pois desde o final da \versal{II} Guerra Mundial, guerrilhas, grupos
insurgentes, levantes populares, guerras de libertação nacional e
movimentos revolucionários radicalizaram a afronta ao pretendido
monopólio estatal, transtornando as tentativas de regulamentar a prática
da guerra como jogo político entre os Estados e voltados apenas para os
interesses estatais (Haydte, 1990; Rodrigues, 2010).

Não tardou para que os especialistas no campo das Relações
Internacionais e dos Estudos Estratégicos e da Segurança Internacional
passassem a criticar o prognóstico da ``paz perpétua''. Uma ampla
literatura surgiu nesse período tendo como denominador comum, apesar dos
conceitos, metodologias e princípios teóricos distintos, a identificação
de ``atores não-estatais'' como ``agentes de violência'' que passavam a
rivalizar ou, ao menos, serem localizados como ``ameaças'' pelos Estados
nacionais (Buzan e Hansen, 2012). Os Estados e suas forças coercitivas
não deixavam de existir e combater nesses novos e fluídos cenários, mas
também se reconfiguravam diante de outras modalidades de enfrentamento.

No entanto, uma marca do que Mary Kaldor (2008) denominou ``novas
guerras'', em contraposição às ``antigas guerras'' de tipo interestatal,
seria, precisamente, a capacidade de deslocamento de grupos, guerrilhas
e ``células terroristas'' conectados por laços e motivações que remetiam
a renovados nacionalismos e ``identidades étnicas'' que articulavam
levantes em Estados que se decompunham como unidades políticas
centralizadas, como a Iugoslávia e a \versal{URSS}. Kaldor (2008) e Hansen e
Buzan (2012), alinhavando o argumento central dessa literatura,
destacaram os chamados ``atores não-estatais'', com capacidade de ação e
de trânsito \emph{transnacionais}, como novos agentes de violência a
gerar problemas de segurança para os Estados. No entanto, a emergência
de modalidades de conflito para além e para aquém do Estado explicitou
os limites das teorias de Relações Internacionais e dos Estudos
Estratégicos para compreender o que aconteceu com a ``guerra'' na
passagem do século \versal{XX} para o \versal{XXI}. Uma análise das proveniências dessas
novas conflitividades nos leva ao mundo que emergiu das duas guerras
mundiais.

Ambas as guerras, como indica sua designação, tiveram impacto global. Os
campos de combate, principalmente na \versal{II} Guerra Mundial, se estenderam
dos mares do Pacífico Sul, à Europa, passando pelo Oceano Atlântico, o
norte da África, o Oriente Médio e a Ásia continental e insular. No
entanto, essas guerras ainda foram grandes conflitos mobilizados por
Estados nacionais que colocaram em marcha uma articulação sem
precedentes das atividades econômico-industriais com as mais variadas
forças sociais no chamado ``esforço de guerra''. Tamanha mobilização fez
com que as guerras mundiais fossem denominadas como ``guerras totais''
(Keegan, 2002). Nelas, apesar dos esforços dos Estados centrais a partir
da segunda metade do século \versal{XIX} em separar combates (os militares
estatais) de não-combatentes (os civis), reservando direitos e cuidados
humanitaristas para ambas as categorias, soldados e não soldados foram,
na prática, reunidos pelo ``esforço de guerra'' como partes de uma mesma
maquinaria voltada à guerra de proporções mundiais.

A emergência de uma combinação entre as figuras do combatente e do
não-combatente anunciou, na prática das violências, os limites da
definição da ``guerra'' como um assunto restrito do Estado e voltado
apenas para o choque violento com outros Estados, com forças armadas
regulares, normas humanitaristas e protocolos diplomático-militares. No
plano tático, dos combates em si, tal separação também começou a ceder
com a atuação de grupos guerrilheiros ou \emph{partisans}, como a
Resistência Francesa ou as milícias comunistas chefiadas por Tito na
Iugoslávia, apoiando as forças militares Aliadas. No entanto, as guerras
mundiais foram ainda, basicamente, conflitos entre Estados.

A emergência de forças não-estatais no período da chamada Guerra Fria
respondeu a dinâmicas próprias das guerras revolucionárias e de
libertação nacional e dos constrangimentos do equilíbrio nuclear entre
Estados Unidos e \versal{URSS} que impedia um embate direto entre as
superpotências. Ainda assim, o número de ``guerras civis'' ou
``conflitos domésticos'' cresceu entre as décadas de 1950 e 1980, em
tendência inversa à das guerras entre Estados (Pfetsch, 2003). As
chamadas ``guerras irregulares'', todavia, reportavam-se ainda ao
Estado, pois foram praticadas para resistir à invasão de um ou mais
Estados visando a defender a ``pátria'', atacar um determinado regime a
fim de ocupar o Estado, ou para destacar uma porção desse Estado com
vistas a formar uma nova unidade política soberana. Exemplo do primeiro
caso foram as forças vietcongues na resistência contra a presença
estadunidense no Vietnã, entre o início dos anos 1960 até sua retirada
total, em 1975. No segundo caso, há muitos exemplos entre as guerrilhas
e grupos militantes revolucionários, como as Forças Armadas
Revolucionárias da Colômbia, guerrilha ativa desde 1964, e as Brigadas
Vermelhas, organização que praticou atentados em nome da revolução
comunista na Itália entre 1969 e 1984. O terceiro campo contou com a
atuação de grupos como o Pátria Basca e Liberdade (\versal{ETA}), ativo entre
1959 e 2011 e que promoveu atentados contra o Estado espanhol visando a
independência política do País Basco (norte da Espanha e sudoeste da
França).

A partir dos anos 1990, discursos acadêmicos e diplomático-militares
passaram a conectar modalidades de enfrentamento provenientes dos tempos
da Guerra Fria com outras práticas que, se não eram propriamente novas,
teriam ganhado destaque numa variedade de ``novos alvos'' e ``novas
ameaças'' que passou a ocupar os planos estratégicos dos \versal{EUA} e de seus
aliados da \versal{OTAN}, mobilizando acessoriamente os Estados sob a sua
influência direta. Atividades ilícitas como o narcotráfico, que desde os
anos 1970 acionavam programas militarizados de combate apoiados pelos
\versal{EUA}, foram enfatizados na agenda de segurança estadunidense para a
América Latina e Ásia (Rodrigues, 2015), enquanto a ``nova grande
ameaça'' do terrorismo fundamentalista islâmico encontrava respaldo
naqueles mesmos discursos acadêmicos e diplomático-militares,
acompanhando atentados atribuídos não mais a guerrilhas ou grupos
revolucionários, mas a ``redes terroristas'', com destaque para a
Al-Qaeda (Laqueur, 1996; Degenszajn, 2006).

No campo das Relações Internacionais, foi a chamada ``teoria da
securitização'', primeiramente formulada e difundida por Buzan, Wæver e
De Wilde (1998), que ganhou maior aceitação. O argumento básico dos
autores é o que de não haveria tema que fosse essencialmente um
``problema de segurança'', sendo cada questão identificada como
``ameaça'' uma construção que passava pela emissão de um discurso (um
``ato de fala'' ou ``\emph{speech act}'') por um agente securitizador
que nomearia algo como tal. Para os autores, os Estados aparecem como
principal ``agente securitizador''. Para que o processo de securitização
fosse completo, o discurso emitido precisaria ser aceito por uma
audiência mais ampla, as forças presentes na sociedade civil, que, por
compreender a ``ameaça'' como um perigo à própria existência do Estado e
da sociedade, endossaria a tomada de medidas de segurança defendidas
pelo Estado como \emph{urgentes}, mesmo que implicassem na suspensão de
direitos civis e outras garantias constitucionais de uma democracia
liberal (Buzan, Wæver e De Wilde, 1998: 24).

Ainda que a ``teoria da securitização'' trabalhe para desnaturalizar o
conceito de ``segurança'', indicando sua produção em meio à emissão e
recepção de discursos, não há nela uma superação ou enfrentamento da
perspectiva contratualista, pois não contraria a definição de
``política'' como o conjunto de relações limitadas às instituições
estatais e às representações tradicionais (partidos e sindicatos, por
exemplo) em contraste com o espaço de ``ausência de política
institucional'' no plano internacional, sob a iminência constante da
violência por não possuir um poder centralizado superior. Tampouco essa
teoria deixa de entender o Estado como entidade que conecta e coloca em
movimento as políticas de segurança.

Desse modo, a identificação de ``novas ameaças'', ou mesmo a constatação
de que a identificação de ameaças é socialmente constituída, não impede
que o ``Estado'' seja o conector final das políticas de segurança, seu
principal ``agente securitizador'' e o fundamental objeto a ser
protegido. Com isso, a distinção entre ``dentro'' e ``fora'', entre o
``político'' e o ``não-político'', entre ``Estado'' e ``anarquia''
continuam orientando as análises sobre a guerra ou conflitos
contemporâneos. É preciso, então, um deslocamento de perspectiva para
além da tradição contratualista na qual se inscrevem as teorias
tradicionais das Relações Internacionais que permita compreender as
relações de poder e de conflito que se organizam por uma nova
governamentalidade que não é a da Razão de Estado, mas própria de uma
ecopolítica do planeta.

\emph{Purificam-se manchando-se com outro sangue, }

\emph{como se alguém, entrando na lama se lavasse.}

Heráclito de Éfeso, séculos \versal{VI}-\versal{V} a.C.~

Em obra recente, o filósofo francês Frédéric Gros (2009) voltou ao tema
das guerras na contemporaneidade para concluir que elas teriam chegado
ao seu fim. Para ele, as guerras como acontecimento militar de grandes
proporções, mobilizando forças armadas e pautado por regras de direito
público, cederam espaço ao que nomeia como ``estados de violência''.

Para o autor, as violências, genocídios, ilegalismos, guerras civis e
massacres que despontaram com força nos anos 1990 não seriam ``retornos
à barbárie'' ou supostas ameaças de um despontar de ``estados de
natureza'' no sentido hobbesiano da ``luta de todos contra todos''
diante da ausência de um poder estatal forte e disciplinador. Uma noção
de \emph{barbárie} pressuporia, para Gros, a crença prévia na existência
de um ``contrato social'' que, num momento histórico remoto reafirmado
cotidiana e espontaneamente por todos e cada um, teria fundado o Estado
como entidade capaz de manter a ``paz civil''.

Gros afirma que os ``estados de violência'' não são situações de
selvageria, mas, ao contrário, constituem novas práticas de violência
dotadas de racionalidades próprias vinculadas a uma ``nova distribuição
contemporânea das forças de destruição'' (2009: 232) que alteraram os
``princípios básicos de estruturação'' (Ibidem: 229) da guerra. Esses
princípios são quatro dimensões conectadas: \emph{tempo}, \emph{espaço},
\emph{regulamentação} e \emph{mobilização}.

No que diz respeito à dimensão tempo, os ``estados de violência'' são,
para Gros, de duração indeterminada, pois não se sabe precisamente
quando começam ou quando ― e se ― acabam. As formalidades de início e de
conclusão dos conflitos armados entre Estados, com seus protocolos,
cerimônias e formalidades, deixa de existir quando se trata de
acontecimentos como a chamada ``guerra ao terror'', declarada pelos \versal{EUA}
após os atentados de 11 de setembro de 2001 e que levou à imediata
invasão do Afeganistão. Naquele momento, o presidente George W. Bush
ordenou uma ocupação militar que se prolongou indefinidamente,
desdobrando-se com a presença de militares da \versal{OTAN}. Naquele contexto, em
2003, o Iraque também foi invadido por coalizão militar liderada pelos
\versal{EUA}. A justificativa de impedir a produção de ``armas de destruição em
massa'' (como armas químicas e bacteriológicas), ainda que infundada,
levou a uma ocupação continuada, entre promessas de desmobilização por
parte dos estadunidenses e tentativas sempre fracassadas de
estabilização de um governo iraquiano. Diante de problemas tidos como de
difícil solução pelos Estados interessados, como a sujeição das milícias
e terroristas, as ocupações e violências prolongaram-se sem demarcações
claras de início e término.

Demarcações também esmorecem na dimensão espacial dos ``estados de
violência''. Nas guerras entre Estados, por mais extensos e variados que
sejam os ``cenários'' ou as ``frente de combate'', é possível
identificar onde os enfrentamentos acontecem. Nos ``estados de
violência'' há a \emph{desterritorialização} das violências que podem
acontecer em qualquer parte do globo. Novamente, os atentados
terroristas do chamado ``fundamentalismo islâmico'' mostram a grande
mobilidade dos perpetradores dos ataques e a ampla variedade de alvos
pelo planeta. O mesmo acontece para as forças estatais (individuais ou
em coalizão) que enfrentam esses grupos: o rastreamento e a repressão
ganham, igualmente, expressão planetária, sob a justificativa de
combater um inimigo múltiplo e móvel.

Os ``estados de violência'' erodem, ao mesmo tempo, a regulamentação da
guerra. O conjunto de tratados e compromissos celebrados desde o século
\versal{XIX}, articulando humanitarismo com regulação e gerenciamento da guerra,
encontrou seu ápice nos pactos universais da Liga das Nações e da \versal{ONU},
com capítulos que buscaram formalizar o \emph{jus in bello} ― a conduta
na guerra ― e o \emph{jus ad bellum} ― o direito do Estado (e apenas do
Estado) de recorrer à guerra ― que foi, como mencionado anteriormente,
restrito aos casos da autodefesa estatal e da segurança coletiva.

Os grupos em confronto nos ``estados de violência'' não obedecem aos
pactos internacionais ou às leis nacionais, pois não as reconhecem
legítimas (como o caso dos terroristas do tipo Al Qaeda), ou porque
atuam em diferenciados graus de ilegalidade (como as gangues de rua,
milícias, máfias e organizações do narcotráfico). Do lado dos Estados, a
defesa da legalidade tampouco é uma constante, como atestam a utilização
da base estadunidense de Guantánamo, em Cuba, para a prisão de suspeitos
de terrorismo que não estão sujeitos nem ao sistema legal dos \versal{EUA}, nem
aos compromissos do direito humanitário internacional (Degenszajn, 2006;
Duarte, 2014).

Por fim, os ``estados de violência'' alteram os modos de convocação e
mobilização para os conflitos. A ``guerra pública'', recurso de força
dos Estados no choque com outros Estados, foi um dos pilares sobre os
quais se constituiu o sistema de Estados soberanos na Europa entre os
séculos \versal{XVI} e \versal{XVIII} (Foucault, 2008; Tilly, 1996). Nessas guerras, todo
cidadão homem, numa certa faixa de idade (geralmente entre finais da
adolescência e começo da maturidade) podia ser convocado a ingressar nas
fileiras das forças armadas, ser treinado, disciplinado, uniformizado e
enviado para uma frente de combate seguindo uma hierarquia clara e que
se reportava a uma autoridade central.

Essa modalidade de guerra, historicamente, principiou com a formação dos
exércitos estatais no início da Era Moderna, mas se cristalizou com a
constituição dos exércitos nacionais a partir da Revolução Francesa, na
passagem do século \versal{XVIII} para o \versal{XIX} (Keegan, 2002). Então, a convocação
se universalizou, pois a guerra não seria mais em nome do príncipe, mas
da ``Nação'', princípio transcendental a reunir todos os súditos de um
Estado numa mesma noção de ``comunidade nacional'' (Anderson, 1991;
Keegan, 2002). Além dos homens em idade de combater, as ``guerras
nacionais'' passaram a mobilizar toda a sociedade. Assim, as
necessidades de uma guerra da era do capitalismo industrial
transformaram a todos em ``combatentes'': soldados na linha de frente,
mulheres convocadas a trabalhar na indústria e homens fisicamente
incapacitados para o combate, nas funções logísticas (Rodrigues, 2010).

Esse padrão se alterou, segundo Gros, nos ``estados de violência''.
Hoje, os grupos em luta não são ``agentes públicos'', mas uma infinidade
de ``agentes privados'', tanto legais quanto ilegais. São milícias,
guerrilhas, grupos terroristas, máfias, organizações narcotraficantes;
mas também, as empresas privadas de segurança que atuam numa zona de
legalidade indefinida sendo convocadas para apoiar as ações dos
militares estatais, fazendo a segurança dos agentes públicos de
segurança (Singer, 2008; Hubac, 2005). Mesmo entre as forças armadas
nacionais, principalmente nos Estados com maior capacidade militar, os
modos de convocação se transformam, com maior ênfase às unidades
militares de elite, com pessoal (homens e mulheres) altamente preparado
para lidar com tecnologia de ponta e táticas especiais de combate,
enquanto os corpos de militares menos capacitados diminuem de tamanho e
atraem pessoas interessadas numa opção de carreira que lhes renda
seguridade social ou cidadania, como no caso dos estrangeiros ilegais
alistados nas forças armadas estadunidenses nesse início de século \versal{XXI}
(Kiras, 2010).

A reflexão de Gros sobre os ``estados de violência'' suscita uma análise
sobre as variadas e simultâneas dimensões nas quais os enfrentamentos
armados acontecem contemporaneamente. Os quatro elementos dos ``estados
de violência'' remetem à continuada atuação de guerrilhas formadas
durante a chamada Guerra Fria, mas que não encerraram suas atividades
com o fim do conflito Leste-Oeste, ou ainda às diferentes modalidades de
empresas privadas de segurança que operam numa zona tênue entre a
legalidade e ilegalidade, sendo contratadas para atuar em regiões sob a
ocupação militar de coalizões para fazer a segurança de negócios,
ocupações militares e atividades diplomáticas. É o caso, por exemplo, da
empresa de segurança privada estadunidense Blackwater, composta por
ex-militares de vários países que foi contratada pelo governo de George
W. Bush, nos anos iniciais da ocupação do Iraque, para fazer operações
de segurança para diplomatas e acusada de assassinar indistintamente
enquanto operou no país (Scahill, 2008).

Todavia, no âmbito desses grupos não-estatais, destaca-se a questão do
chamado ``terrorista fundamentalista islâmico'', por ter sido ele o
grande inimigo a ser securitizado desde 2001 pelas estratégias de defesa
dos \versal{EUA} e, por extensão, em maior ou menor grau, de todos os Estados do
globo conectados aos seus interesses. Essa modalidade de terrorismo
evidencia as quatro características do ``estado de violência'' conforme
descritas por Gros (2009). Pensando precisamente sobre esses terrorismos
é que se tornou possível ativar a noção de ``transterritorial'' a fim de
analisar os grupos privados ou estatais, legais ou ilegais, individuais
ou coligados em coalizões militares que transitam pelas dinâmicas e
velozes vias da globalização econômica com seus meios
computo-informacionais e facilidades logísticas (Passetti, 2007b).

Para compreender o trânsito global de empresas, dados, pessoas e
produtos, a literatura sobre globalização consagrou o termo
``transnacional'' desde que, a partir dos anos 1970, estudos no campo do
liberalismo nas Relações Internacionais definiram esse conjunto de
relações, ``com significante importância política que acontecem sem
controle governamental'' (Keohane \& Nye, 1971: 330). Interessava,
naquele momento, sustentar que, para além das relações entre Estados e
seus aparatos militares e diplomáticos, uma nova dinâmica havia
emergido, principalmente após a \versal{II} Guerra Mundial, conectando economias
nacionais, empresas, grupos de interesse, filiações identitárias e
políticas que, mesmo ainda disciplinadas e condicionadas pelos Estados
nacionais, acabavam por construir formas próprias de contato e
cooperação para além das tradicionais relações diplomáticas.

A expressão ``transnacional'', no entanto, ainda toma o ``nacional''
como referência e, por extensão, o ``estatal'', uma vez que essas duas
categorias foram plasmadas pelo pensamento político liberal a partir dos
séculos \versal{XVIII} e \versal{XIX} (Rodrigues, 2013). Todavia, os fluxos que se
dinamizaram a partir dos anos 1940, e as inovações tecnológicas que
permitiram ampliar e desdobrar conectividades entre pessoas, empresas,
Estados, grupos ilegais etc. vai além da referência exclusiva ao Estado,
sua soberania e controle territorial, como sugeriu o filósofo francês
Gilles Deleuze (1992) com as características próprias do que denominou
``sociedades de controle''. Nesse tempo histórico, para Deleuze, os
Estados não são superados, mas reconfigurados diante de muitos fluxos
conectados que tomam, por base e referências, novas formas de
territorializar e desterritorializar práticas sociais, táticas
políticas, conexões econômicas, formas de resistências.

O termo ``transterritorial'' para qualificar esses fluxos que perpassam
fronteiras políticas, regiões e espaços pelo planeta (e até mesmo fora
dele) sem ter o ``nacional'' ou o ``estatal'' como referenciais, decorre
da análise dos terrorismos para indicar que a Al Qaeda, diferentemente
de um grupo coeso e hierarquizado, apresentou-se como inovador porque
potencializou as táticas de descentralização provenientes das lutas
guerrilheiras e subversivas dos anos 1950, 1960 e 1970, projetando-as em
escala planetária. Mais intensamente descentralizado, o ``terrorismo
fundamentalista'' seria, de fato, uma multiplicidade de
\emph{terrorismos}, aproximados por um discurso geral partilhado, a
interpretação radicalmente excludente do Corão e a reação violenta à
influência e dominação ocidentais, mas desarticulados enquanto movimento
coeso voltado a uma \emph{libertação} nacional. Mais do que isso,
inaugura uma nova forma de terrorismo como um programa capaz de produzir
interfaces em qualquer lugar ou ocasião (Passetti, 2007b).

Assim, esses terrorismos se distinguem daqueles que abundaram durante o
século \versal{XX} porque acirram as características dos ``estados de
violência'', mas, principalmente, porque rompem com o referencial
``estatal/nacional''. Analisando discursos como o de Osama bin Laden
(2005) é possível notar que o ``fundamentalismo'' deveria lutar contra a
presença de ``infiéis'' (judeus, cristãos e muçulmanos a eles aliados)
no ``Mundo Islâmico'', em especial nas Terras Santas do Islã (nos atuais
Estados de Israel, Iraque e Arábia Saudita). Bin Laden afirmava que sua
luta era uma ``vingança contra os perpetradores do mal, transgressores,
criminosos e terroristas que aterrorizam os verdadeiros crentes'' (2005:
120). Essa reação só cessaria quando os ``{[}infiéis{]} se retirarem da
Península Arábica e cessarem sua interferência na Palestina e em todo
mundo islâmico'' (Ibidem: 127).

Classificando seus inimigos também como ``terroristas'', Bin Laden não
definia um projeto de Estado para isso que se poderia chamar aqui de um
\emph{programa de libertação religioso transterritorial}. As menções que
fizeram Bin Laden e seus seguidores à formação de um grande Califado que
se estenderia do Marrocos à Indonésia nunca foram precisadas, apesar da
referência à volta de um padrão de governo político desenvolvido nas
áreas de ocupação muçulmana no Oriente Médio, norte da África, sul e
sudoeste asiático e Península Ibérica por muitos séculos, e encerrados
com a ampliação europeia no século \versal{XIX} (Rodrigues, 2014).

Dos terrorismos do século \versal{XIX} à emergência do terrorismo
transterritorial, eles foram, no entanto, majoritariamente referenciados
ao Estado. Os revolucionários estatistas, como os de inspiração
bolchevista e, posteriormente maoísta, organizaram grupos ilegais para
atacar agentes e agências estatais, assim como instituições e indivíduos
partícipes do regime político e sistema econômico capitalista
(industriais, banqueiros, intelectuais etc.), visando abalar sua
autoridade, incitar uma sublevação geral e tomar o poder de Estado.
Foram muitos os grupos nessa chave, como as já citadas Brigadas
Vermelhas, na Itália, e o Baader-Meinhof, na Alemanha dos anos 1970;
além dos grupos da esquerda armada latino-americana, como o Movimento
Revolucionário 8 de Outubro (\versal{MR}-8) e a \versal{VAR}-Palmares, no Brasil do final
dos anos 1960 e começo dos 1970, os Montoneros, na Argentina, e
Tupamaros, no Uruguai dos anos 1970, Movimento 19 de Abril (M-19) na
Colômbia dos anos 1970 e 1980.

Os terrorismos nacionalistas como o Pátria Basca e Liberdade (\versal{ETA}), na
Espanha, e o Exército Republicano Irlandês (\versal{IRA}), no Reino Unido,
promoveram atentados para criar um novo Estado, no caso basco, ou para
desmembrar parte de um para anexá-lo a outro, como o exemplo irlandês. A
Frente de Libertação Nacional argelina, nos anos 1950 e princípio dos
1960, promoveu atentados contra alvos franceses, inovando nas técnicas
de ação urbana e de organização em ``células militantes'', motivando o
desenvolvimento de táticas repressivas por parte das forças armadas
francesas que incluíam tortura e pressão psicológica e que foram,
posteriormente, adotadas pelos regimes autoritários latino-americanos
(Haydte, 1990). A partir do final dos anos 1960, a Organização para a
Libertação da Palestina (\versal{OLP}) adotou a luta armada no enfrentamento com
Israel e anunciou uma mais intensa desterritorialização do terrorismo
contemporâneo quando passou a atacar alvos israelenses ou conectados a
Israel e aos \versal{EUA} longe da Palestina, como, por exemplo, o atentando
contra a delegação israelense nas Olimpíadas de Munique, Alemanha, em
1972, quando um dos dormitórios ocupados por atletas daquela
nacionalidade foi invadido por cinco membros do Setembro Negro, grupo
vinculado à \versal{OLP}. Após tentativa de fuga, uma emboscada fracassada da
força policial alemã levou à morte de todos os reféns e de três dos
sequestradores.

Na história política contemporânea, a única prática terrorista que não
teve o Estado como meta foi o chamado anarcoterrorismo que, na passagem
do século \versal{XIX} para o \versal{XX}, mobilizou indivíduos e pequenos grupos na
realização de atentados contra chefes de Estado e autoridades, além de
representantes do grande capital, em seus espaços de trabalho e lazer.
Diante da cerrada repressão que o movimento anarquista sofreu na Europa
e nas Américas no final do século \versal{XIX}, muitos decidiram pela ação
violenta como meio de expor as fragilidades da sociedade burguesa e do
aparato estatal que visavam superar (Passetti, 2007b; Rodrigues, 2014;
Augusto, 2006).

De todo modo, os terrorismos preponderantes ao longo do século \versal{XX} se
reportaram ao Estado, ocupando espaços de luta relacionados aos
territórios nacionais pelos quais lutavam ou contra os quais se opunham.
O terrorismo fundamentalista islâmico refere-se vagamente a uma unidade
política utópica ― o Califado ― mas inaugura práticas terroristas que
remetem a uma luta desterritorializada, que visa alvos ocidentais ou de
aliados em todas as partes do planeta. A ``guerra ao terror'', por sua
vez, também, se desterritorializa, levando à formação de coalizões
militares voltadas à ocupação de espaços usados para treinamento ou como
refúgio para terroristas. Essa guerra desterritorializada, no entanto,
vai além, ao redimensionar práticas de controle e vigilância nas mais
ínfimas situações cotidianas: rastreamento de mensagens eletrônicas,
quebras de sigilos e senhas, prisões sem mandado legal, buscas e
apreensões, vasculha e revista de corpos e objetos, entre outras formas
de governo sobre as condutas.

Se o terrorismo conecta espaços e fluxos sem ter o Estado como
referência básica, seria possível defini-lo como ``transterritorial''.
Essa noção pode ser estendida aos ilegalismos que perpassam fronteiras e
acoplam diferentes espaços. O narcotráfico, por exemplo, não é um
``fenômeno unitário'', mas um negócio transterritorial de grandes
proporções que conecta plantações ilegais, laboratórios clandestinos,
empresas legais de fachada ou condicionadas ao tráfico ilegal (como
companhias aéreas e marítimas), zonas urbanas em países pobres, guetos
em países ricos, além de outras máfias e diferentes ilegalismos (como o
tráfico de armas e de pessoas) posto que as drogas ilegais servem como
moeda em muitos negócios ilegais (Rodrigues, 2016; Paley, 2014).

Por sua vez, as forças mobilizadas para combater o tráfico de drogas
ilícitas combinam dimensões locais com transterritoriais, por conta dos
acordos militares, financiamentos internacionais para aparelhar forças
coercitivas, colaboração de agências de inteligência civil e militar.
Não obstante, cada Estado acopla-se de um modo à ``guerra contra as
drogas'', instituindo padrões repressivos próprios que rementem à
história da proibição das drogas em cada país e às suas arraigadas
práticas punitivas (Rodrigues, 2016). Assim, o impulso à repressão e ao
envolvimento de militares no combate ao narcotráfico vem, desde os anos
1970, produzindo uma crescente indistinção entre competências, táticas e
modos de ação das forças policiais e militares que promove, como será
analisado adiante, a \emph{militarização de forças policiais}
concomitante à \emph{policialização das forças armadas} de muitos países
comprometidos com a ``guerra às drogas'' (Balko, 2013; Saint-Pierre,
2015; Graham, 2016).

Por ora, interessa destacar que os ``estados de violência'' são
modalidades de conflito que não se desconectam do Estado, tampouco da
possibilidade de novas edições de guerras entre Estados, mas que abalam
a ``lógica clausewitziana'' da guerra em suas dimensões temporal, legal,
espacial e de mobilização. Os ``estados de violência'' têm
características transterritoriais que conectam ambientes de segurança
que antes pareciam claramente separados entre ``dentro'' e ``fora'' dos
Estados. O espaço internacional, ambiente no qual a guerra entre Estados
pode acontecer, não é mais claramente distinguível dos denominados
ambientes domésticos de segurança pois, entre eles, atravessam muitos
grupos transterritoriais, como os do terrorismo fundamentalista e do
narcotráfico. Se as guerras não terminaram diante de uma suposta vitória
do modelo democrático-liberal, novos ou reconfigurados conflitos
emergiram, acoplando espaços em disputa para além da lógica estatal e
transnacional. Na sociedade de controles, as ``ameaças'' e
``securitizações'' se projetam no planeta e assumem fluidez
transterritorial.

\section{planetárias}

Os ``estados de violência'' são acontecimentos violentos que se liberam
dos limites jurídico-políticos e dos conceitos acadêmicos tradicionais
das Relações Internacionais referentes à guerra e à segurança. Esses
acontecimentos promovem uma acentuada securitização para além dos
``temas clássicos'' referenciados no Estado, como destacaram Buzan,
Wæver e De Wilde (1998), mas vão além da identificação de ``novas
agendas de segurança''. O próprio referencial estatal passou a ser
questionado pela produção de noções de segurança voltadas não mais à
segurança do Estado, ou de uma sociedade, com seu meio ambiente e suas
questões econômicas demarcadas ou circunscritas ao Estado nacional.

No chamado período Pós-Guerra Fria, o conceito de \emph{segurança
nacional} e, com ele, o de \emph{segurança} \emph{internacional}
começaram a ser marcadamente modificados pela identificação de ameaças à
segurança do planeta como um todo. Categorias a serem protegidas se
destacaram da sujeição à segurança estatal, por meio da elaboração de
novos discursos securitizantes vinculados a um emergente conjunto de
táticas para o governo de tudo que está, vive ou circula no planeta
Terra. Essas circulações não são mais tão-somente \emph{transnacionais},
conectando agências subnacionais ou entidades privadas, menos ainda
meramente \emph{internacionais}, articulando Estados por meio de seus
dispositivos diplomático-militares, mas \emph{transterritoriais}, e
sugerem o despontar de uma segurança planetária gerida por um novo
dispositivo, o \emph{diplomático-policial}.

Um dos principais e primeiros indicativos dessa transformação foi a
elaboração do conceito de ``segurança humana'', pelo Programa das Nações
Unidas para o Desenvolvimento ― \versal{PNUD}, nos seus relatórios \emph{Human
Development Report} dos anos 1993 e, principalmente, 1994. Esse conceito
procurou deslocar o foco da segurança no Estado, passando-o para o
indivíduo. O Homem, categoria universal definida e investida por uma
plêiade de direitos, deveria ser o foco primordial da proteção, tanto
contra a violência física, que os relatórios definem como
``\emph{freedom from fear}'' (liberar-se do medo), quanto das privações
em geral, ``\emph{freedom from want}'' (liberar-se das carências)
(Rodrigues, 2012b; Weiss, 2007). O documento do \versal{PNUD} de 1993 afirma que
``novos conceitos de segurança humana devem enfatizar a segurança das
pessoas e não apenas das nações''\footnote{\emph{Human Development
  Report 1993}, p. 2\emph{.}

  \emph{http://hdr.undp.org/sites/default/files/reports/222/hdr\_1993\_en\_complete\_nostats.pdf}.},
enquanto o relatório do ano seguinte desenvolveu esse princípio,
demandando que os Estados-membros da \versal{ONU} encontrassem ``novas fundações
{[}no conceito de{]} segurança humana, que assegure a segurança das
pessoas através do desenvolvimento, e não pelas armas''\footnote{\emph{Human
  Development Report 1994}, p. 6.

  \emph{http://hdr.undp.org/sites/default/files/reports/255/hdr\_1994\_en\_complete\_nostats.pdf}.}.

Desse modo, na perspectiva do \versal{PNUD}, o foco da segurança no planeta não
deveria ser o do Estado, como objeto de referência primordial das
políticas de defesa militar, mas as ``pessoas'', o ser humano como
categoria ou sujeito de direito internacional. É interessante notar que
os relatórios referem-se a ``nações'' (``\emph{nations}'') como sinônimo
para Estado-nação e a ``pessoas'' (``\emph{people}'') como referência
para os seres humanos, atribuindo, assim, uma designação universal
próxima à de ``povos'' (``\emph{peoples}'') utilizada na \emph{Carta de
São Francisco}, tratado que institui a \versal{ONU} em 1945\footnote{O preâmbulo
  da Carta de São Francisco começa com a exortação ``Nós, os Povos das
  Nações Unidas...'' (``We the Peoples of the United Nations...'').

  \emph{https://treaties.un.org/doc/publication/ctc/uncharter.pdf}.}. Com
isso, evitou-se vincular os indivíduos a cidadanias específicas,
mantendo um princípio de pertencimento geral à espécie humana, sem
distinção de fronteiras nacionais, enquanto a própria palavra ``Estado''
não é mencionada.

A definição da ``segurança humana'' propôs um deslocamento da própria
existência, segurança e razão de ser dos Estados, condicionando-as à
segurança do indivíduo que, por sua vez, indica um caminho específico
vinculado ao respeito pelos direitos humanos. Desse modo, fica sugerida
a necessidade da conformação e universalização de um modelo
jurídico-político único, que seria a democracia liberal, capaz de
institucionalizar e salvaguardar os direitos humanos. Nesse contexto,
emergiram novos conceitos, instituições e normas no direito
internacional que intervieram diretamente sobre uma tensão já existente
na formulação dos grandes projetos universalistas do século \versal{XX}
cristalizados em torno da Liga das Nações e da \versal{ONU}. Segundo Thomas Weiss
(2007), essa tensão pode ser notada na delicada articulação entre
\emph{princípios westfalianos} (a configuração do mundo em Estados
nacionais, o respeito à soberania, à autodeterminação dos povos e à não
intervenção) e \emph{princípios cosmopolitas} (valores universais, os
direitos humanos, a paz entre os povos, o respeito ao direito
internacional).

Essa tensão, no entanto, começou a ser reequacionada nos anos 1990,
sendo o conceito de ``segurança humana'' seu ponto de clivagem. Isso
porque o conceito soberania passou por uma requalificação nesse período
pós-Guerra Fria deixando de ser formalmente ``absoluta'' (como nos
moldes westfalianos), para ser \emph{relativa} (Rodrigues, 2013). Mesmo
que a prática da soberania tenha sido historicamente desigual, seu
\emph{status} jurídico-político, reafirmado na \versal{ONU}, era o de igualdade e
inviolabilidade (Krasner, 1999). Inaugurou-se na entrada do novo milênio
a sua relativização formal. As proveniências mais remotas estão no
conjunto das novidades introduzidas pelos direitos humanos e pela
responsabilização de ``criminosos de guerra'' em Nuremberg e Tóquio,
após a \versal{II} Guerra Mundial, que, pela primeira vez, não responsabilizaram
``Estados derrotados'' pelas destruições da guerra, mas indivíduos que,
durante o conflito, estiveram à frente do comando do Estado (Rodrigues,
2010). As proveniências mais imediatas, no entanto, estariam
precisamente no contexto da formulação do conceito de \emph{segurança
humana}.

Naquele momento, o conceito de soberania passou a ser balizado pelo
princípio de que ``para ser legítima, a soberania deve demonstrar
responsabilidade'' (Deng et al., 1996: \versal{XVII}). \emph{Responsabilidade}
voltada para os indivíduos compreendidos, simultaneamente, como
``cidadãos'' de um país e ``sujeitos de direito internacional''
(``cidadãos cosmopolitas''), ou seja, duplamente qualificados como
sujeitos de direito, com primazia para o direito universal. Os conceitos
de ``soberania como responsabilidade'' e ``segurança humana'' emergem no
momento dos genocídios e violência étnica dos anos 1990
(Bósnia-Herzegovina, Ruanda, Kosovo) e ganham impulso diante dos
problemas jurídico-políticos colocados pelas intervenções humanitárias,
como ações militares destinadas a interromper violações dos direitos
humanos por grupos em luta ou por Estados contra parcelas de suas
populações.

Em especial, a intervenção de forças da \versal{OTAN} no Kosovo com bombardeios
aéreos, em 1999, foi um acontecimento importante para a revisão legal
dos parâmetros jurídico-políticos da \versal{ONU}. Isso aconteceu porque a
justificativa para a ação militar foi ``humanitária'', pois as tropas
sérvias foram acusadas de violar sistematicamente os direitos humanos
dos kosovares, principalmente os de origem albanesa que clamavam por
independência política com relação ao que restara da federação iugoslava
sob o comando sérvio. As relações diplomático-militares no Conselho de
Segurança da \versal{ONU} foram marcadas pela oposição entre britânicos e
estadunidenses de um lado, favoráveis a uma intervenção armada, e de
russos de outro lado, aliados históricos dos sérvios e contrários à ação
militar. Quando a missão se realiza, não pelos \versal{EUA} ou pelo Reino Unido
isoladamente, mas pela \versal{OTAN}, confere aos ataques uma dimensão
multilateral, ainda que ilegal do ponto de vista do direito
internacional, pois não foram aprovados pelo Conselho de Segurança.
Ainda assim, foram considerados \emph{legítimos} pelos seus defensores,
enquanto reação humanitarista às alegadas violações dos direitos
humanos. O problema entre o que valeria mais, a soberania iugoslava ou
os direitos humanos dos kosovares, explicitou a mencionada tensão entre
princípios westfaliano e cosmopolita, provocando uma revisão das bases
do direito internacional.

Essa revisão veio no mesmo ano de 1999, com a formação, autorizada pelo
secretário geral da \versal{ONU}, o ganês Kofi Annan, de uma comissão chamada
International Comission on Intervention and State Sovereignty (\versal{ICISS}),
que reuniu diplomatas, profissionais do direito humanitário
internacional, intelectuais e políticos para dois anos de reuniões que
culminaram com a publicação do relatório \emph{The Responsibility to
Protect}\footnote{\emph{The Responsability to Protect}.
  http://responsibilitytoprotect.org/ICISS\%20Report.pdf.} (R2P) (Evans,
2008).

O relatório apresentou o conceito de mesmo nome que pretendeu equacionar
os dois polos de tensão expressos no próprio nome da Comissão
Internacional: ``soberania'' e ``intervenção''. O princípio de R2P
encaminhou, assim, uma proposta para recondicionar as relações entre
soberania, segurança humana, segurança do Estado e segurança
internacional (Kenkel, 2010; Serrano, 2011). Seguindo a linha do
conceito de ``soberania como responsabilidade'', a R2P reconhece que a
soberania dos Estados deve ser respeitada, desde que os titulares do
Estado salvaguardem os direitos humanos. Caso não o façam por
incapacidade ou propositadamente, a ``comunidade de Estados'', reunida
na \versal{ONU} e sob a chancela do seu Conselho de Segurança, tem o dever de
intervir (Evans, 2008).

Desse modo, a soberania não é desqualificada, mas ao contrário,
\emph{requalificada} ou adjetivada: objetiva-se modular uma definida
forma de exercício do poder de governar pessoas, coisas e fluxos que
siga parâmetros democrático-liberais e cosmopolitas. Nesse ponto estaria
um dos elementos principais do emergente gerenciamento da segurança
planetária: a R2P explicita o estabelecimento de um \emph{padrão de
conduta} para os Estados a ser defendido e, no limite, imposto a Estados
com procedimentos de governo desviados do padrão universal. Esse padrão,
por sua vez, emergiu de uma comissão composta não apenas por diplomatas,
mas por eles e um conjunto de especialistas, acadêmicos e profissionais
de \versal{ONG}s que, por sua vez, ofereceu suas recomendações ao Secretário
Geral da \versal{ONU}, uma instância multilateral interestatal que, desde os anos
1990, vinha se abrindo às parcerias e à observação de ``grupos da
sociedade civil''. Essa conexão entre a diplomacia estatal, os foros
multilaterais e os agentes individuais ou não-governamentais deliberando
sobre regras de alcance universal baseadas em princípios universais
caracteriza a nova dimensão ``diplomática'' do dispositivo de segurança
da ecopolítica. Em suma, na discussão em torno da R2P, a \emph{prática
diplomática} (na negociação, definição de consenso e redação de
princípios universais) não ficou mais sob a exclusividade dos
diplomatas, como durante a primazia do dispositivo diplomático-militar.

Estaria em jogo, portanto, a construção de regras, instituições e
práticas que estabeleçam uma \emph{conduta esperável} dos Estados e não
somente deles, mas de todo ``corpo social'' (cidadãos, organizações
civis, \versal{ONG}s) e empresas, todos articulados em torno de um modelo
político-econômico transterritorial único e inquestionável: os valores
da democracia liberal. No entanto, do mesmo modo que o edifício
jurídico-político da soberania não foi meramente substituído, mas
reconfigurado, pela emergência da biopolítica no século \versal{XIX}, o despontar
da ecopolítica como conjunto de práticas para o governo do planeta não
prescinde de elementos do dispositivo diplomático-militar, mas os
redimensiona e atualiza. Os ``estados de violência'' e o caráter
transterritorial dos fluxos que desafiam o funcionamento do capitalismo
global e das instituições jurídico-políticas que lhes permitem fluir
atualizam as formas de definir ``ameaças'', de ``securitizá-las'', assim
como os meios para contê-las ou acomodá-las em níveis manejáveis. Nesse
sentido, as intervenções humanitárias obedecem à mesma lógica das
intervenções recomendadas pelos economistas neoliberais, ou seja, as
ações emergenciais, capitaneadas pelo Estado (ou por coalizões) para
gerenciar as inevitáveis crises econômicas, políticas e sociais
provocadas pelas desigualdades inerentes ao capitalismo.

Os ``estados de violência'', produzidos ou acirrados nesse momento
histórico, precisam, portanto, ser contidos, controlados, governados.
Esse controle se dá com a prescrição de condutas universais que
qualifica um Estado a jogar na economia global e a ser apoiado para
tratar com as ``ameaças'' internas ou transterritoriais, que podem
abalar sua autoridade, já bastante afetada pela sua própria inclusão nos
fluxos da sociedade de controle. Assim, na ecopolítica do planeta, o
governo dos ambientes de segurança ``nacionais'' e planetário se
interconectam e interfecundam. Para além da discussão entre o ``dentro''
e o ``fora'' do Estado, as porosidades aumentam estabelecendo
continuidades e comunicações constantes entre ``ameaças'' e, também,
entre práticas de securitização.

Segundo Didier Bigo, ``os conflitos contemporâneos não são mais
majoritariamente guerras interestatais ou grandes revoluções {[}mas{]}
múltiplas formas de conflito'' (2010: 337) compatíveis com a noção de
``estados de violência''. Para o autor, existiria um ``\emph{continuum}
conflitivo'' (Ibidem: 336) que mescla práticas de violência dentro, fora
e através das fronteiras estatais, conectando ambientes de segurança que
anteriormente, na lógica westfaliana, estariam separados. Hoje, o
terrorismo transterritorial, o narcotráfico, o tráfico de pessoas, o
contrabando de produtos, entre outras atividades fluindo entre os canais
legais e ilegais da economia global, atravessam fronteiras e articulam
espaços domésticos e internacionais. Ao mesmo tempo, as iniciativas de
construção de Estados nos moldes cosmopolitas, a partir de intervenções
armadas humanitaristas e de operações de estabilização e produção de
instituições, começam a dar forma a um ``ambiente de segurança
planetária''. Assim, a dicotomia \emph{dentro/fora} do discurso
jurídico-político e das teorias das Relações Internacionais, que
traduzem \emph{dentro} como ``ordem/paz'' e \emph{fora} como
``guerra/caos'', fica difícil de ser sustentada analiticamente (Walker,
2013). Os conflitos, traduzidos em múltiplos ``estados de violência'',
atravessam e modelam os espaços políticos, produzindo securitizações,
quer seja no designado campo doméstico, quer seja no ainda chamado de
internacional.

As próprias noções de ``nacional'' e ``internacional'' são erodidas,
ainda que não superadas, pois os ambientes de segurança se conectam em
diferenciadas gradações e \emph{continuums}. Em tempos de \emph{fluxos
transterritoriais} que incluem, também, grupos armados, terrorismos e
organizações ilegais, regiões, segmentos de centros urbanos e imensidões
desocupadas podem fazer parte de uma mesma série continuada, como, por
exemplo, o narcotráfico articulando plantações ilícitas em regiões de
difícil acesso, laboratórios clandestinos na selva, favelas com suas
organizações ilegais de varejo das drogas ilegais e rotas transoceânicas
controladas por distribuidores internacionais dessas substâncias. Não se
trata, portanto, de indicar a existência de vínculos ``internacionais'',
tampouco ``transnacionais'', pois não é mais o ``Estado'' ou o
``nacional'' que operam como referência. Os fluxos transterritoriais
acionam outras conexões entre pessoas, produtos, dados eletrônicos.
Nesses fluxos se atualizam os denominados conflitos, a definição de
``ameaças'', o que é definido como ``ameaçado'' e quais as formas de
atacar, conter ou administrar essas ``ameaças''. Sobre a dimensão da
\emph{segurança internacional}, alterando-a e transformando-a, emerge a
da \emph{segurança planetária}.

\chapter{O dispositivo diplomático-policial}

Os meios pelos quais a segurança planetária é exercida combinam novas ou
renovadas instituições de alcance regional ou global, com destaque para
a \versal{ONU}, atualizada sob a bandeira da proteção dos direitos humanos e da
promoção da democracia. Junto à \versal{ONU}, há organizações regionais de cunho
econômico e comercial, mas também militar, que se responsabilizam pela
gestão local e pela co-gestão dos problemas planetários, como a União
Europeia. Países como o Brasil, no campo das chamadas ``potências
emergentes'', são simultaneamente convocados e se voluntariam a
participar dessa gestão do planeta que, mesmo sendo coletivamente
articulada, não é igualitariamente colegiada. Assim, as hierarquias e
centralidades se recompõem, enquanto populações tidas como perigosas
porque resistentes ao capitalismo e à democracia liberal (como
anarquistas e ativistas antiglobalização) ou porque provocadoras de
pressões nos centros do capitalismo global (como os imigrantes ilegais e
refugiados) são alvo de programas de contenção e aprisionamentos, mas
também, são clientes de projetos de proteção aos direitos humanos e de
promoção do chamado desenvolvimento do econômico e social.

Na ecopolítica do planeta emerge um novo dispositivo de segurança que
denominamos \emph{dispositivo diplomático-policial}. Nele, os elementos
militares do dispositivo diplomático-militar não são completamente
ultrapassados, mas redimensionados em novas modalidades de exercício da
violência que conectam forças militares e policiais, coalizões militares
de Estados sob mandatos diplomáticos multilaterais, conexões entre
forças privadas legais (empresas de segurança) e ilegais (milícias,
grupos terroristas, guerrilhas) em práticas de lutas que não respeitam
as divisões jurídico-políticas das fronteiras nacionais. As formas de
exercício da violência combinam-se, por sua vez, como uma atualizada
\emph{polícia das condutas} dividuais e estatais que se projeta
globalmente, visando manter a segurança de todos e cada um num planeta
governado por decisões diplomáticas tomadas em foros que procuram
legitimidade democrática em valores universais e na participação de
governos, \versal{ONG}s, empresas, indivíduos.

O dispositivo de segurança na era da biopolítica das populações foi o
diplomático-militar; na ecopolítica do planeta emerge o
\emph{diplomático-policial}. Acompanhando os redimensionamentos da
guerra e a emergência de novas institucionalidades que visam gerir
``ameaças'' e ``violências'' no globo, será possível compreender como o
governo do planeta produz, para além da divisão estanque entre
``nacional'' e ``internacional'' e do equilíbrio de poder entre Estados,
novos ``ambientes de segurança'' na sociedade de controle?

Um dos conceitos que ativam novas práticas para o governo das condutas e
da segurança planetária é o de Responsabilidade de Proteger que, após
sua apresentação em 2001 recebeu severas críticas de países islâmicos e
do antigo ``Terceiro Mundo'', como a Venezuela e a Argélia, que
consideraram insuficientes as garantias de que o princípio não seria
utilizado para encobrir intervenções militares e sanções financeiras e
comerciais dos países do antigo ``Primeiro Mundo'', como os \versal{EUA} e
Estados na União Europeia, motivadas por seus interesses
diplomático-militares e econômicos. Um dos principais argumentos dessa
relutância foi a desconfiança de que o Conselho de Segurança, por ser
controlado pelas cinco potências (\versal{EUA}, Reino Unido, França, China e
Rússia), autorizaria intervenções baseadas na R2P apenas quando seus
interesses nacionais fossem beneficiados.

No entanto, a discussão em torno da R2P foi eclipsada pelos efeitos da
chamada ``guerra ao terror'' lançada pelos \versal{EUA} após os atentados de
setembro de 2001 e que fizeram com que os debates na \versal{ONU} se retivessem
no problema da legalidade e legitimidade da invasão do Iraque pelos \versal{EUA},
Reino Unido e aliados. Esses debates tomaram as cessões do Conselho de
Segurança no segundo semestre de 2002, até que finalmente a invasão foi
realizada, em março de 2003, à revelia de uma decisão daquela instância.

O tema da R2P, todavia, não foi abandonado, retornando nas sessões da
Cúpula Mundial (World Summit)\footnote{\emph{http://www.un.org/womenwatch/ods/A-RES-60-1-E.pdf}}
da \versal{ONU}, em outubro de 2005, evento revestido de importância especial
pelo Secretário Geral Kofi Annan por comemorar os sessenta anos da
organização, e no qual se esperava registrar mudanças importantes no seu
funcionamento, com espaço renovado para a promoção dos direitos humanos,
segurança humana e democracia. No documento produzido como resultado
dessa Cúpula (Resolução 60/2005 da Assembleia Geral), dentre os muitos
compromissos com metas para o desenvolvimento do planeta, figurou a
aceitação da R2P como um dos princípios gerais da \versal{ONU}, ou seja, como um
valor norteador sem obrigatoriedade explícita.

Numa sessão específica sobre o tema, a Resolução afirma que ``cada
Estado tem a responsabilidade de proteger suas populações de genocídios,
crimes de guerra, limpeza étnica e crimes contra a Humanidade'' (Res.
60/2005, § 138). A ``comunidade internacional'', continua o documento,
deve apoiar os Estados a alcançar esses objetivos, seguindo os
procedimentos estabelecidos pela \emph{Carta da \versal{ONU}} e dando preferência
aos meios ``diplomáticos, humanitários e de outros meios pacíficos''
(Res. 60/2005, § 139). No entanto, o compromisso dos Estados deve ir
além, preparando-se para ações coletivas, apoiados por organizações
regionais (como, por exemplo, a \versal{OTAN} e a União Africana), para agir, sob
o comando do Conselho de Segurança, sempre que ``os meios pacíficos se
mostrem inadequados e as autoridades nacionais explicitamente falhem na
tarefa de proteger suas populações'' (Ibidem). Para tanto, a Assembleia
Geral assumiu formalmente que a R2P deveria orientar o monitoramento da
conduta dos Estados, indicando, no § 138 da Resolução, a necessidade de
instaurar mecanismos de acompanhamento constante da situação dos
direitos humanos capazes de acionar alertas internacionais que deem
publicidade a possíveis violações, possibilitando sanções ou medidas
reparadoras. Esses mecanismos, chamados de \emph{early warning} (alerta
preventivo), são constituídos enquanto programas de monitoramento de
países com sérias crises político-sociais ou que passam por períodos de
transição pós-conflitos (como guerras civis e transições de
autoritarismos para regimes democráticos).

A prática do \emph{early warning} se dá pela conexão entre o trabalho de
associações e \versal{ONG}s locais, institutos de pesquisa locais e
internacionais, \versal{ONG}s internacionais, fundações internacionais
financiadoras de projetos em direitos humanos, agências especializadas
de organizações intergovernamentais (da própria \versal{ONU} ou de organizações
regionais) e agências dos Estados monitorados (secretarias, ministérios,
conselhos). Essa rede de instituições produz relatórios, aceleram e
promovem encontros de avaliação e recomendam políticas sociais e ajustes
legais. Com isso, desde os mais altos níveis institucionais e
normativos, no plano da \versal{ONU}, por exemplo, passando pela colaboração
intensa de \versal{ONG}s, agências estatais e institutos de pesquisa privados e
estatais, chega-se aos locais e situações mais pontuais nos países ou
regiões consideradas ``em crise'', onde, por sua vez, organizações
locais contribuem com dados e informações que ascendem na cadeia de
intermediários até os mais elevados planos decisórios, influenciado na
formulação de recomendações mais amplas e possivelmente replicáveis em
outras partes do planeta. No sentido oposto, descendente, tais
recomendações retornam capilarmente às regiões-alvo sendo exercitadas
pelas organizações locais que, desse modo, colaboram tanto para a
elaboração de modelos de conduta quanto para sua efetivação, e,
horizontalmente, conectam gentes a todas elas.

Esse funcionamento previsto da R2P explicita uma das dimensões do
funcionamento do \emph{dispositivo diplomático-policial}, pois mostra a
conexão entre ``agentes'' privados, estatais e intergovernamentais, que
fazem circular informações, recomendações, propostas de reforma e
práticas de governo das condutas em sentido \emph{ascendente} ― das
localidades tidas como ``problemáticas'', ``inseguras'' ou em ``crise''
para as mais elevadas instâncias decisórias dos Estados e da \versal{ONU} ― e em
sentido \emph{descendente} ― quando recomendações e políticas mais
amplas são elaboradas tornando-se padrões de conduta que retornam à
localidade com o intuito de remodelar costumes, práticas sociais e
conectar partes. Desse modo, as técnicas diplomáticas da negociação, da
representação de interesses e da busca de soluções por meio de
concessões deixam de ser uma prática apenas entre os diplomatas a
serviço dos Estados para se disseminar entre funcionários de \versal{ONG}s,
líderes comunitários, pesquisadores de institutos, missionários
humanitaristas, burocratas de agências governamentais e
intergovernamentais e o divíduo. Visa-se disseminar condutas a modular
práticas não apenas dos divíduos nas localidades, aqueles a quem se
destinam as políticas de proteção dos direitos humanos, mas também
coordenar a conduta de agências estatais, \versal{ONG}s, institutos de pesquisa
etc. Todos se conectam numa mesma \emph{confiança} nos procedimentos
burocráticos, nos valores universais em jogo e na legitimidade das
políticas e princípios produzidos nesse processo participativo no qual
muitos colaboram para propor, elaborar, aplicar e monitorar normativas e
padrões de conduta tolerantes. Eis a dimensão policial do dispositivo de
segurança planetária: a produção de condutas em escala local, nacional,
regional e global por meio da participação ativa de ``agentes''
privados, estatais e intergovernamentais em consonância com os divíduos,
todos conectados pela circulação de informações, recomendações,
controles e práticas em canais ascendentes, descendentes e horizontais.

O princípio da R2P, no entanto, não figurou sozinho naquele momento de
reformulação da \versal{ONU}. Das negociações preparatórias à Cúpula de 2005, e
dos arranjos que dela se desdobraram, surgiram novos organismos
vinculados à Assembleia Geral da \versal{ONU}, como o Fundo para a Democracia
(United Nations Democracy Fund, criado em julho de 2005)\footnote{\emph{http://www.un.org/democracyfund/}},
a Comissão para a Construção da Paz (Peacebuilding Comission, instituída
em dezembro de 2005)\footnote{\emph{http://www.un.org/en/peacebuilding/}}
e o Conselho de Direitos Humanos (United Nations Human Rights
Council\footnote{\emph{http://www.ohchr.org/en/hrbodies/hrc/pages/hrcindex.aspx}},
implantado em março de 2006).

Dentre essas novas institucionalidades produzidas em 2005, destaca-se o
Fundo para a Democracia, não por sua potencialmente maior efetividade
diante dos demais, mas pela introdução de uma novidade importante em
termos de governamentalidade planetária. Sua criação, apoiada pelo então
presidente dos \versal{EUA}, George W. Bush, incluiu a ``promoção da democracia''
como uma das bandeiras da \versal{ONU}. Essa promoção está ausente da \emph{Carta
de São Francisco}, de 1945, que indicou a ``paz e segurança
internacionais'' como meta da Organização e o ``desenvolvimento'' como
meio, mas evitou o termo ``democracia'' diante das disputas entre os
conceitos de democracia liberal, encabeçados pelos \versal{EUA}, e de
``democracia popular'', defendidos pela \versal{URSS} e demais países associados
ao socialismo de Estado. Do ponto de vista formal, a \versal{ONU} assumiu
definitivamente que um determinado regime político, a democracia,
deveria ser o modelo ideal de organização política de todos os povos e
que, portanto, seria papel da Organização promovê-la mundialmente. O
documento final da Cúpula de 2005 reservou espaço para confirmar a
defesa da democracia numa seção imediatamente anterior àquela dedicada a
R2P intitulada simplesmente ``Democracy''.

O texto procura desassociar a ``democracia'' de um padrão específico de
regime, buscando evitar uma defesa explícita do modelo da democracia
liberal de proveniência europeia e estadunidense, ajusta-se aos
imperativos da racionalidade neoliberal que funciona por modulações nos
diversos domínios. A Resolução registra que ``a democracia é um valor
universal baseado na livre expressão da vontade popular para determinar
seus sistemas político, econômico, social e cultural, e sua participação
em todos os aspectos de suas vidas'' (Res. 60/2005, § 135). Na
sequência, o parágrafo sustenta que ``não há um modelo apenas de
democracia'' e que a ``soberania e a autodeterminação'' devem ser
respeitadas, mas ressalva que ``democracia, desenvolvimento e respeito a
todos os direitos humanos e liberdades fundamentais são interdependentes
e se reforçam mutualmente'' (Ibidem). Sabendo que os conceitos de
``liberdades fundamentais'' têm proveniência no pensamento liberal
produzido na Europa ocidental entre os séculos \versal{XVI} e \versal{XVIII} e que,
depois, foram encampados e difundidos pelos \versal{EUA} a partir de finais do
século \versal{XVIII}, fica explicitada a tensão minimizada entre a defesa dos
valores universais de corte liberal e cosmopolita e a tentativa de
dissimular essa padronização forçada a partir de valores ocidentais com
a lembrança ao particularismo, tanto por meio do respeito à soberania e
autodeterminação, quanto pela sugestão de que pode haver mais modelos
para a democracia que o liberal. A democracia encontra o espaço político
renovado, conectado às práticas disseminadas nas empresas e na sociedade
civil organizada.

A forma com que a democracia emerge como recomendação universal no
discurso da \versal{ONU}, em 2005, é uma indicação adicional da nova dimensão de
``polícia das condutas'' da emergente governamentalidade planetária: a
determinação de um específico regime político serve de base para
reforçar a produção de expectativas de que ele se dissemine como padrão
de conduta para os Estados, mas não apenas, pois o princípio das
``liberdades fundamentais'' remete aos direitos de associação, de
liberdade de expressão, de credo e, sobretudo, de garantias da
propriedade privada que devem orientar movimentos sociais, partidos
políticos, corporações privadas e, na base de tudo, os divíduos em suas
condutas cotidianas. Quando reparamos na definição da democracia como
ideal, meta e parâmetro de conduta (estatal, associativa, corporativa e
dividual), é possível notar como passa a operar o \emph{dispositivo
diplomático-policial} ao disseminar padrões de conduta moduláveis a
serem monitorados, promovidos e, no limite, impostos por táticas de
governamentalidade planetária como a R2P.

Além das declarações formais nos documentos da \versal{ONU}, a primeira vez que,
historicamente, o conceito de R2P foi evocado se deu durante a crise na
Líbia em 2011, instaurada pela sublevação de forças políticas contra o
regime comandado por Muamar Kaddafi que levou o país a uma guerra civil.
Em março daquele ano, o Conselho de Segurança da \versal{ONU} aprovou a Resolução
1973\footnote{\emph{http://www.nato.int/nato\_static/assets/pdf/pdf\_2011\_03/20110927\_110311-UNSCR-1973.pdf}},
estabelecendo uma ``zona de exclusão aérea'' sobre o país que governava
a proibição para que a força aérea líbia sobrevoasse o próprio
território. A justificativa para essa medida foram as acusações de que o
governo de Kaddafi ataca sua própria população, qualificando-o como um
violador do compromisso universal com a defesa dos direitos humanos (St.
John, 2012).

A Resolução, em seu preâmbulo, afirma: ``\emph{Reiterando} a
responsabilidade das autoridades líbias de proteger a população líbia e
\emph{reafirmando} que as partes armadas no conflito arcam com a
responsabilidade primária de tomar todas as providências factíveis para
assegurar a proteção de civis...'' (Res. 1973/2011, p. 1; grifos no
original). Essa consideração, somada a outras condenações iniciais com
relação à violência no país, prepara o argumento a autorizar que
``Estados Membros, tendo notificado o Secretário Geral, agindo
isoladamente ou por meio de coalizões ou organizações regionais, e
agindo em cooperação com o Secretário Geral, tomem {todas as medidas
necessárias} (...) para {proteger civis} e áreas povoadas por civis que
estejam sob ataque na República Árabe Jamahiriya Líbia, incluindo
Benghazi, ainda que {excluindo a ocupação por força estrangeira} de
qualquer forma e em qualquer parte do território líbio'' (Res.
1973/2011: 3, grifos nossos).

O trecho é interessante para pensar a R2P como elemento de um emergente
dispositivo de segurança planetário, pois é uma autorização pelo
Conselho de Segurança da \versal{ONU} para que uma ação armada (significado da
expressão ``todas as medidas necessárias'') justifique-se por argumentos
humanitaristas voltados a ``proteger civis''; argumentos, portanto,
universalistas e não particulares ao ``interesse nacional'' de qualquer
Estado. Essa missão, por sua vez, poderia ser realizada por Estados
individualmente, coalizões ou organizações regionais. Não ocorreu uma
chancela para que um Estado ou uma aliança de Estados atacasse em nome
de interesses geopolíticos clássicos ou visando garantir diretamente sua
própria segurança. Os ataques que, em tese, visavam apartar as forças de
Kaddafi e as milícias rebeldes das áreas civis, foram assumidos pela
\versal{OTAN} e se resumiram a bombardeios aéreos, respeitando a proibição para
ocupação militar (Keating, 2013).

Essa proibição, vinculada ao estabelecimento da mencionada zona de
exclusão aérea, teve um duplo propósito: de um lado, visava a evitar que
a ação militar fosse acusada de violar a soberania líbia com fins de
ocupação territorial; de outro lado, afinava-se ao princípio da ``zero
morte'' que, desde tempos da primeira Guerra do Golfo, em 1991, fazia
parte dos discursos diplomático-militares ocidentais centrados na ideia
de que a alta tecnologia propiciava ``ataques cirúrgicos'' (precisos
sobre alvos militares) e que os militares ocidentais não seriam
colocados em situação de risco máximo (Press, 2001).

Guerra cirúrgica, ação pontual: a Resolução resume princípios da
``intervenção'' militar inerentes ao \emph{dispositivo
diplomático-policial}, pois baseados na legitimidade no direito
internacional e nas instituições multilaterais, a ser conduzida por
coalizões de Estado e voltada ao gerenciamento de ``crises'' (como
guerras civis, genocídios, violações dos direitos humanos). A ditatura
de Kaddafi fora isolada entre os anos 1980 e 1990 diante de acusações de
apoio ao terrorismo antiocidental. No entanto, a partir de finais dos
anos 1990 houve uma reaproximação com Estados europeus por conta da
privatização de suas empresas de extração de petróleo, pela contratação
massiva de empresas de segurança privadas ocidentais e pela disposição
daquele regime autoritário em colaborar para a contenção do fluxo de
imigrantes ilegais africanos em direção à União Europeia. Não obstante,
Kaddafi caiu novamente em descrédito diante das sublevações na África do
Norte a partir de 2010, conhecidas como Primavera Árabe e que, começando
na Tunísia e tendo um epicentro no Egito, registraram manifestações
populares expressivas contra os regimes autoritários desses países
levando a diferentes graus de modificação nos arranjos
político-institucionais entre tunisianos e egípcios e à deflagração de
violentos conflitos entre líbios e sírios (Hehir, 2013; Brancoli, 2014).
No caso líbio em particular, os argumentos favoráveis à defesa dos
direitos humanos preponderaram sobre o princípio westfaliano do respeito
à soberania, desbloqueando o princípio cosmopolita favorável à
``intervenção'' que, do ponto de vista jurídico-político, encontrava
chancela para ser evocado pelo fato da R2P ter sido incorporada ao
conjunto de valores da \versal{ONU} em 2005.

A consequência dos bombardeios militares promovidos pela Organização do
Tratado do Atlântico Norte (\versal{OTAN}) sobre a Líbia foi um desequilíbrio de
forças entre as milícias mobilizadas contra Kaddafi e as forças
militares a serviço do ditador que culminou na sua captura e imediato
assassinato em outubro de 2011, levando a um período no qual um novo
governo, sob o beneplácito da União Europeia e dos Estados Unidos,
começou a ser formado sem que a guerra civil pudesse ser interrompida
(St. John, 2012).

Diante da morte de Kaddafi e do apoio de europeus e dos \versal{EUA} à transição
política na Líbia, houve uma reação de Estados contra o que foi
interpretado como uma intervenção cuja motivação velada teria sido
retirar o ditador do governo. A principal dessas reações veio pela
manifestação do Brasil, quando sua delegação diplomática no Conselho de
Segurança apresentou um breve documento intitulado \emph{Responsibility
while Protecting: elements for the development and promotion of a
concept} (\emph{Responsabilidade ao Proteger: elementos para o
desenvolvimento e promoção de um conceito})\footnote{\emph{http://responsibilitytoprotect.org/concept-paper-\_rwp(1).pdf}}.

No documento, a diplomacia brasileira exigia mais clareza nos mandatos
do Conselho para intervenções amparadas na R2P e defendia a construção
de instrumentos de acompanhamento e avaliação internacionais de missões
militares de caráter humanitário. Para a delegação brasileira, a
``comunidade internacional {[}deveria{]} ser rigorosa nos esforços para
esgotar todos os meios pacíficos para a proteção de civis sob a ameaça
da violência''\footnote{\emph{Responsibility while Protecting: elements
  for the development and promotion of a concept}, item b.} antes que
uma intervenção armada fosse autorizada. Se ela, por fim, viesse a
ocorrer, deveria ser ``limitada nos seus elementos legal, operacional e
temporal''\footnote{Ibidem, item d.}, causando a ``menor violência e
instabilidade possível''\footnote{Ibidem, item e.} .

A preocupação externada pela diplomacia brasileira era a de que
intervenções baseadas no R2P pudessem ocultar intenções geopolíticas sob
a roupagem de humanitarismo (Rodrigues, 2013). Interessa destacar que o
conceito de Responsabilidade ao Proteger (RwP) reconhece a legitimidade
da R2P, mas indica a necessidade de elaborar meios para \emph{monitorar
a conduta} de quem intervém, de modo a manejar os ``objetivos
humanitários'' e controlar os resultados posteriores no país ou região
alvo. Dessa forma, o documento brasileiro encerra suas recomendações ao
Conselho de Segurança defendendo que esse organismo garantisse
instrumentos para ``monitorar e acessar a maneira pela qual as
resoluções {[}do próprio Conselho{]} são interpretadas e implementadas
com vistas a garantir a responsabilidade ao proteger''\footnote{Ibidem,
  item h.}.

Em suma, a RwP pleiteia o controle e monitoramento sobre a implementação
da R2P, demandando a elaboração e aplicação de mecanismos para evitar
\emph{excessos} na conduta dos Estados que assumam a aplicação de
resoluções do Conselho de Segurança baseadas nesse princípio. Os
excessos seriam, principalmente, a produção de mais violações aos
direitos humanos do que aquelas cometidas pelo regime político que se
pretende coagir e inibir e, finalmente, a substituição de um regime por
outro, ação que violaria o princípio da não-intervenção presente na
\emph{Carta da \versal{ONU}}. Os debates provocados após a apresentação do
conceito pelo Brasil levaram à elaboração de mais comissões de
analistas, publicações acadêmicas e posicionamentos diplomáticos
favoráveis ao estabelecimento de controles e limites para a R2P e
outros, encampados principalmente pelos \versal{EUA}, que consideraram uma
tentativa de burocratizar decisões a tomadas em momentos emergenciais
(Pattinson, 2016).

Nos limites dessa reflexão, aos propósitos do \emph{dispositivo
diplomático-policial}, interessa pensar a RwP para além da sua
eficiência ou não como elemento de regulação legítima da R2P, mas em sua
proposta de se constituir como um mecanismo de aperfeiçoamento que
introduz um \emph{sobregoverno}: diante da R2P como tentativa de
introduzir padrões de condutas para os Estados que rompam com o
compromisso de respeitar os direitos humanos, a RwP abre a possibilidade
de que no Conselho de Segurança se constituam outros mecanismos
destinados a monitorar os Estados que assumam a tarefa de intervir com
base na R2P. Em suma, se a R2P é um princípio para a \emph{polícia das
condutas} dos Estados, a RwP serve à polícia das condutas de quem
intervém para zelar pela conduta dos Estados. A R2P e a RwP são
princípios que podem acionar táticas de governo e sobregoverno das
condutas, explicitando as contínuas reformas institucionais que
configuram uma das características principais da ecopolítica do planeta:
a condição sempre inacabada das normas, práticas e instituições que
demandam infindáveis ajustes e aperfeiçoamentos e que também dependem
das repercussões entre os divíduos organizados na sociedade civil local.

Nota-se, aqui, outro importante elemento desse emergente dispositivo de
segurança planetária que é o da produção de regulamentações voltadas à
defesa de valores universais (os direitos humanos) que redunda na ação
concertada da chamada ``comunidade internacional'' reunida na \versal{ONU} e
regida pelo Conselho de Segurança, agindo com o propósito declarado de
impedir tais violações e de reprimir uma prática (ou uma conduta) do
Estado violador, retificando-a pela força das armas ou das coerções
econômico-comerciais. A RwP, por sua vez, seria um instrumento para
monitorar a aplicação do princípio da R2P, inaugurando a articulação
entre duas práticas de governo das condutas: uma sobre o Estado violador
dos direitos humanos e outra sobre os Estados que a colocam em
movimento.

Os meios de aplicação da R2P, conforme o relatório de 2001 que
apresentou o princípio e o documento da Cúpula da \versal{ONU} de 2005 que o
incorporou à Organização, não são apenas os militares. O rol de opções
coercitivas reconhece um espectro que está em conformidade com as
sanções previstas no \emph{Carta de São Francisco} e que vai das sanções
econômicas aos embargos comerciais, tendo a intervenção militar como
última opção (Evans, 2008). As opções são todas punitivas e têm como
meta retificar a conduta de um Estado considerado, pelos membros do
Conselho de Segurança, como desviados de seus compromissos
internacionais. A intervenção militar é apenas a mais espetacular das
punições previstas. Não obstante, sua possibilidade de aplicação, diante
da incorporação da R2P em 2005 e do precedente na Resolução sobre a
Líbia, em 2011, abriu espaço para a atualização da prática da guerra nas
relações internacionais.

A R2P redimensiona formalmente o conceito de \emph{guerra justa} (o
\emph{jus ad bellum}) pela terceira vez desde a formação do sistema de
Estados modernos: do século \versal{XVI} a inícios do século \versal{XX}, cada Estado era
formalmente livre para decidir recorrer à guerra, seguindo apenas um
cálculo utilitário (aquele recomendo por Clausewitz) sobre as vantagens
ou desvantagens de buscar a realização de seus ``interesses nacionais''
pela violência; no século \versal{XX}, desde o Pacto da Liga das Nações, em 1919
(antecedido pelas primeiras conferências pacifistas desde finais do
século \versal{XIX}), passando pelo Pacto Briand-Kellogg de proscrição da guerra
de agressão, de 1928, e chegando à \emph{Carta da \versal{ONU}}, em 1945, a
\emph{guerra justa} passou a ser apenas a de autodefesa e a de segurança
coletiva. No início do século \versal{XXI}, o \emph{jus ad bellum} ganha um
elemento a mais: o da intervenção humanitária, onde o princípio da R2P
opera o conceito que articula moralidade/legitimidade e legalidade para
a execução de operações militares. A soberania dos Estados, modulada a
partir do respeito (ou não) aos direitos humanos, deixou de ser, ao
menos formalmente, um anteparo a impedir sanções econômicas, comerciais
e, no limite, militares.

A tensão entre ``soberania'' e ``cosmopolitismo'' chegou, então, a um
ponto superior. Interessante, assim, reparar que o Brasil, com tradição
diplomática associada ao não-intervencionismo, admite que haja
intervenções humanitárias desde que controladas, gerenciadas, governadas
por instrumentos exteriores operados pela \versal{ONU}. Desse modo, as
intervenções seriam aceitáveis pelos padrões da \versal{ONU} que se alinham à
tradição diplomática e obrigações constitucionais brasileiras. Nesse
sentido, o equacionamento para a tensão ``soberania/cosmopolitismo''
viria por um encaminhamento de novo tipo, próprio da ecopolítica:
instâncias multilaterais e regras universais determinariam como, onde e
quando intervir, enquanto instrumentos de monitoramento e avaliação
acompanhariam as intervenções como uma \emph{auditoria} ou uma
\emph{corregedoria}. As intervenções corretoras da conduta de Estados
``degradados'' de seus compromissos serão, também, monitoradas em nome
do seu aperfeiçoamento e controle. Em todo caso, está sempre em jogo a
retificação e produção de condutas esperadas, ou seja, uma \emph{polícia
das condutas} que também espera encontrar conexão em cada divíduo.

A partir do momento em que violações aos direitos humanos e os chamados
``crimes contra a humanidade'' são considerados ameaças à ``paz e
segurança internacionais'' (expressão-chave da \emph{Carta de São
Francisco}, que instaura e rege a \versal{ONU}), observa-se um amplo processo de
securitização, conforme o conceito de Buzan, Wæver e De Wilde (1998): o
``objeto de segurança'' primordial, o que deve ser protegido, passa a
ser, nominalmente o ``Homem'', categoria universal, mas o Estado segue
como entidade primordial, tanto para agir em nome da proteção do
``Homem'', quanto para ser retificado e resgatado para a ``justa
conduta'', em uma ordem democrática e capitalista planetária. O conceito
de segurança humana e os \emph{Human Development Report} de 1993 e 1994,
conforme exposto anteriormente, marcaram esse novo enfoque para a
segurança estatal a partir da ênfase na proteção dos indivíduos e de
seus direitos humanos. Por sua vez, o conceito de R2P veio para
equacionar a segurança do Estado e das relações internacionais
relacionando a proteção da soberania do Estado em conexão com a
segurança do ``ser humano''.

As múltiplas securitizações (do meio ambiente, do Homem, da energia, dos
alimentos etc.) não deveriam ocultar o fato de que se pretende construir
um ``ambiente de segurança planetário'' para a preservação e
disseminação de modalidades novas e antigas de poder político
centralizado (o Estado, os consórcios de Estado, as alianças
diplomático-militares, os acordos comerciais) e do regime de propriedade
público-privado: ``toda a descentralização provocada pelas novas
relações de poder em fluxo {[}na ecopolítica{]} permanece vinculada a
centralidades e às soberanias'' (Passetti, 2011a: 135). Os Estados não
são superados, tampouco a soberania é ultrapassada, mas ambos são
redimensionados diante de novos problemas para o governo de pessoas,
fluxos, produtos, ``ameaças'' transterritoriais, que são tanto grupos
não-estatais como os do terrorismo transterritorial ou dos ilegalismos
como o narcotráfico, até os contingentes de refugiados que se deslocam
vindos de regiões em guerra ou que enfrentam graves situações de penúria
climática ou crises econômicas.

Para governar essas ``crises'', que não mais se limitam às fronteiras
dos Estados e tampouco implicam nas relações entre dois ou mais Estados,
emerge um novo aspecto de uma governamentalidade planetária que combina
\emph{intervenções} ― ações corretivas, pontuais, estratégicas ― e
ocupações territoriais mais ou menos abrangentes e duradouras que
transformam não só as ações militares chanceladas pela \versal{ONU}, como o
caráter das próprias missões de paz dessa organização que mudam de
perfil conectando-se de modo inédito com programas de segurança
nacionais nos Estados a ela afiliados.

\section{pacificações}

As características das missões de paz da \versal{ONU} também mudaram
significativamente com o final da Guerra Fria. Segundo Hernández García
(1994), o marco dessa mudança foi a formação de uma missão de
acompanhamento da independência da Namíbia, em 1989. As missões até
então tiveram como objetivo a interposição entre forças em luta (guerras
civis, guerras de libertação nacional), de modo a permitir que o diálogo
diplomático preponderasse. Os capacetes azuis, militares de
países-membros oferecidos para compor cada missão, tinham mandatos
limitados, amparados no Capítulo \versal{VI} da \emph{Carta de São Francisco},
podendo apenas defender-se de ataques sofridos, portanto, não tinham
autorização para usar a força em nome da missão. A missão na Namíbia
teria introduzido um elemento novo: não foi formada apenas para dar
conta de uma guerra civil, mas para apoiar a construção de um novo
Estado nos moldes democrático-liberais.

Entre o ano da missão na Namíbia e 1993, ano que antecede a apresentação
do conceito de ``segurança humana'', o Conselho de Segurança da \versal{ONU}
aprovou dezesseis missões, igual número autorizado entre 1945 e 1989
(Hernández García, 1994: 150-151). O número das missões cresceu e seu
escopo também: os processos de construção institucional passaram a
caminhar ao lado do caráter propriamente militar de cada operação. Após
os ``desastres humanitários'' da primeira metade da década de 1990,
foram elaborados conceitos como o de ``segurança humana'' e emergiu a
questão das intervenções humanitárias, assim como a discussão sobre a
``passividade'' ou ``ativismo'' das missões de paz. O debate passou a
girar, então, sobre o ``uso da força'' pelas missões de paz, algo
permitido pelo Capítulo \versal{VII} da \emph{Carta da \versal{ONU}}, mas nunca colocado
efetivamente em operação (Idem).

A mencionada intervenção da \versal{OTAN} no Kosovo, em 1999, justificada por
questões humanitárias, levou a uma revisão do papel da \versal{ONU} e a
consequente produção do conceito de R2P (Kenkel, 2010; Serrano, 2011;
Rodrigues, 2013). Nesse mesmo movimento, esteve em discussão a
autorização para uso da força pela \versal{ONU} para ``impor a paz''. Segundo
Kenkel, as forças de paz da \versal{ONU} receberam, em 1999, ``em Serra Leoa,
pela primeira vez, a tarefa explícita de assegurar a proteção de civis''
(Kenkel, 2010: 61), o que redundava na prática diplomática, na
possibilidade de usar a força contra grupos ou indivíduos considerados
``hostis''. Essa proveniência é importante para compreender um dos
acontecimentos que indica a emergência do dispositivo de segurança
planetária na ecopolítica: a formação da Missão das Nações Unidas para a
Estabilização do Haiti ― \versal{MINUSTAH}.

A \versal{MINUSTAH} foi autorizada em 2004, após um novo período de confrontação
entre grupos fiéis ao presidente Jean-Bertrand Aristide e milícias
opositoras que levaram à destituição de Aristide após uma ação de forças
especiais militares estadunidenses que o retiraram do país. Desde 1989,
ano das primeiras eleições democrático-liberais do país após a ditadura
dos Duvalier (1954-86), a \versal{ONU} havia autorizado três missões de
acompanhamento eleitoral e ocupação temporária, acompanhando os períodos
de mais intensa violência política no Haiti.

A nova missão, no entanto, trouxe inovações importantes. A primeira
delas foi seu comando operacional dividido entre dois países
latino-americanos (Chile, na face diplomática, e Brasil, na militar). A
segunda inovação foi quanto à composição da maioria das forças militares
de capacetes azuis sendo de países do Sul ou ``em desenvolvimento''
(maior contingente de brasileiros acompanhados de jordanianos,
uruguaios, paraguaios, argentinos, guatemaltecos, nepaleses). A terceira
foi a ação integrada de agências da \versal{ONU} para o desenvolvimento e para a
construção e reforma de instituições como o poder judiciário, o sistema
penitenciário, a polícia nacional e a justiça eleitoral. A quarta
voltou-se para o investimento público de Estados-membros da \versal{ONU} e o
aporte privado para a instalação de infraestrutura (portos, aeroportos,
estradas, sistema de esgoto, iluminação pública), sob a justificativa de
preparar o país para receber unidades produtivas. A quinta e última
inovação foi a participação de \versal{ONG}s de atuação internacional em
atividades sociais e de fomento econômico, afinadas com o propósito de
``(re)construir o país''.

A missão no Haiti colocou em movimento um modelo de gestão baseado em
\emph{responsabilizar} e convocar à participação países chamados
``emergentes'', atribuindo-lhes o comando das operações diplomáticas e
militares. Um destaque especial foi registrado pelo pronunciamento
oficial brasileiro que, durante o governo de Lula da Silva (2003-2010),
tomou a \versal{MINUSTAH} como instrumento não só para projetar influência
brasileira no cenário internacional (a fim de demonstrar sua capacidade
para colaborar ativamente no desenvolvimento e gestão de crises), mas
também como fluxo de aproximação entre países no contexto
latino-americano e, especialmente na América do Sul.

Em suma, é possível afirmar que a \versal{MINUSTAH} tem objetivos associados ao
que a literatura liberal das Relações Internacionais caracteriza como
``construção de Estados'' (``\emph{state-building}'') (Fukuyama, 2005),
entendida como parte fundamental da ``construção da paz''
(``\emph{peace-building}''), cuja comissão especializada da \versal{ONU}, como
citado acima, foi inaugurada em 2005. Essa nova conformação é importante
para compreender o \emph{dispositivo diplomático-policial}, pois indica
como as definições de ``paz'' e ``segurança'' assumem uma dimensão que
vai além da mera cessação de violência física entre grupos em luta para
encampar, também, dimensões de ``desenvolvimento humano'' associados à
``segurança humana'', tal como definidos nos \emph{Human Development
Report} de 1993 e 1994. Assim, ``desenvolvimento humano sustentável'',
tal como definido pelo relatório \emph{Nosso Futuro Comum}, de 1987,
como a ``capacidade de suprir as necessidades do presente sem
comprometer a capacidade das futuras gerações de suprir as
suas''\footnote{\emph{Our Common Future.}
  \emph{http://www.un-documents.net/our-common-future.pdf}, p. 16.},
passaria a se conectar com ``segurança humana'', condicionando em
conjunto a ``segurança estatal'' e, por extensão, a ``segurança
internacional'', pois a estabilidade de um país ou região contribuiria
para a estabilidade e segurança do sistema internacional como um todo.
Estar em ``paz'', desse modo, seria viver sob o modelo de Estado
democrático-liberal integrado aos fluxos do capitalismo global.

A \versal{MINUSTAH}, nesse sentido, foi constituída como uma missão de amplo
espectro, chamada de ``missão multidimensional'' (Kenkel, 2010) por ser
destinada a ocupar, pacificar e promover a constituição de instituições
acopláveis à dinâmica política, diplomática, militar e econômica
planetária. O mandato da \versal{MINUSTAH}, estabelecido pela Resolução 1542/2014
do Conselho de Segurança\footnote{\emph{http://www.un.org/en/ga/search/view\_doc.asp?symbol=S/RES/1542(2004)},},
conferiu às forças de paz um ``mandato robusto'', ou seja, a autorização
para ``atuar no processo de pacificação, reconstrução do país e proteção
aos direitos humanos, ou seja, o uso da força não estava limitado à
autodefesa'' (Correa, 2014: 151). Com base nessa autorização, as tropas
militares brasileiras receberam treinamento especial baseado em material
da \versal{ONU} e ministrado principalmente por meio do Centro de Operações de
Paz do Exército Brasileiro\footnote{Para informações sobre o Centro de
  Operações de Paz no Exército Brasileiro ― \versal{CCOPAB}, ver
  \emph{http://www.defesa.gov.br/relacoes-internacionais/missoes-de-paz/centro-conjunto-de-operacoes-de-paz-do-brasil-ccopab}},
formalizado em 2004 precisamente para dar conta da formação dos
batalhões destinados ao Haiti. Além de aulas sobre os fundamentos da \versal{ONU}
e seus princípios básicos (respeito aos direitos humanos, por exemplo),
os militares foram instruídos em rudimentos de francês e \emph{créole}
(idioma mais falado no Haiti), informações sobre a história e situação
política haitiana, além de (e principalmente) ``táticas específicas de
guerra urbana (...) e de alta interação com a população local'' (Kenkel,
2010: 133).

O objetivo diplomático brasileiro, declarado pelo então presidente Lula
da Silva e o ministro das Relações Exteriores Celso Amorim, era projetar
o Brasil como agente diplomático-militar capaz de apoiar as novas
missões da \versal{ONU}, demonstrando a pertinência não só dessas novas missões,
como da presença do Brasil como membro permanente do Conselho de
Segurança (Correa, 2014; Kenkel 2010). A ênfase brasileira na diplomacia
multilateral teria suplantado, inclusive, as reticências
jurídico-políticas levantadas pelo fato da \versal{MINUSTAH} ser potencialmente
considerada uma ``intervenção'', o que violaria princípio constitucional
e tradição diplomática brasileira. No entanto, em outro sentido, seu
caráter ``solidário'' e ``multidimensional'' estaria afeito ao discurso
diplomático em que investiu o governo nos anos de Lula da Silva e que
redundou na imagem de apoiador das relações solidárias com os países
``em desenvolvimento'', postura acoplável ao mais antigo e arraigado
\emph{ethos} pacifista e negociador da diplomacia brasileira.

A \versal{MINUSTAH} é em termos do \emph{dispositivo diplomático-policial} uma
ampla iniciativa simultaneamente de \emph{pacificação militar} ―
contenção de ``instabilidades'' consideradas empecilhos para a
confirmação de um ``ambiente seguro'' para investimentos ― e de
\emph{criação institucional} nos marcos da democracia liberal. Missões
como esta não seriam mais ``de paz'' no sentido convencional, mas de
\emph{imposição e construção da paz}, pois visariam incorporar Estados
sem conexões lucrativas com o capitalismo global ou que, por sua
fragilidade, representassem ameaças à segurança da circulação de
produtos e capital à própria segurança de outros Estados por serem
considerados espaços seguros para terroristas, ilegalismos
transterritoriais ou celeiro para doenças e levas de refugiados. Visando
dar protagonismo às chamadas ``potências emergentes'' como o Brasil,
essas novas missões ``multidimensionais'' servem como vetores para
disseminar instituições afeitas ao modelo da democracia liberal com a
intenção de produzir, num futuro incerto, países ``estáveis''.

Está em jogo a produção de novas modalidades de governo de regiões,
fluxos, conflitos, populações e espaços que atualizam metáforas teatrais
que identificam os Estados e organizações internacionais, empresas e,
até mesmo, movimentos sociais, como ``atores'' num ``cenário''
internacional, cumprindo ``papéis'' como protagonistas, antagonistas ou
coadjuvantes. Como numa peça dramática, o ``desempenho'' dos ``atores''
corresponde a uma trama a partir da qual se ``encenam'' mudanças ou
continuidades dentro de um espaço demarcado e limitado pelas próprias
regras e marcações do ``texto''. Assim, o discurso sobre o suposto
aumento do protagonismo brasileiro, que teria elegido a participação na
\versal{MINUSTAH} como uma meio para incrementar sua projeção internacional como
``ator'' relevante, acomoda a participação militar do Brasil na missão
da \versal{ONU} como uma \emph{contribuição} para a pacificação e reconstrução do
Haiti, evitando que se evidencie o componente político-estratégico de
co-gestão de um espaço e de uma população considerados problemáticos
para a segurança regional, demandando uma ação de construção
institucional.

O próprio conceito de ``construção de Estados'' sumariza esse objetivo
amplo de modelar um país, com sociedade civil organizada, partidos
políticos em funcionamento, instituições de Estado operacionais, polícia
nacional formada no respeito aos direitos humanos, infraestrutura
produtiva instalada, níveis mínimos de ``desenvolvimento humano''
alcançados, conectividade com a economia global efetivada e certa
contenção e gerenciamento de ilegalismos. Em outras palavras, a \versal{MINUSTAH}
visa à \emph{produção e manutenção de condutas} no Haiti, mas também dos
Estados que dela fazem partes, condutas estas associadas à lógica
democrático-liberal e voltadas à produção de ``ambientes seguros'' para
o capitalismo global e pacificados enquanto espaços identificados
seletivamente como vulneráveis para a disseminação de ``ameaças''
(terrorismos, tráficos, refugiados, desastres ambientais etc.).

A \versal{MINUSTAH}, todavia, interessa ainda como tática no \emph{dispositivo
diplomático-policial} quando se analisa sua acoplagem a processos de
securitização no Brasil. Sua principal conexão nesse campo está nos
programas de \emph{pacificação} implementados desde 2008 na cidade do
Rio de Janeiro. A força de ocupação dos ``complexos'' de favelas do
Alemão e da Penha, na Zona Norte do Rio de Janeiro, atuantes entre
dezembro de 2010 e julho de 2012, e, posteriormente no Complexo da Maré
(na mesma região), entre junho de 2014 e abril de 2015, contou com
batalhões veteranos da \versal{MINUSTAH} na condução de uma missão que guardou
características muito próximas da realizada no Haiti.

A Força de Pacificação do Exército Brasileiro foi autorizada pelo
presidente Lula da Silva em novembro de 2010 após solicitação do
governador do Rio de Janeiro Sergio Cabral. Desde 2008, o governo do
estado começara a ocupar favelas com a Polícia Militar, implementando o
programa das Unidades de Polícia Pacificadora (\versal{UPP}). As invasões e
ocupações de favelas no Rio de Janeiro não foram, todavia, uma inovação
do governo Cabral; ao contrário, desde os anos 1990 multiplicaram-se os
episódios de ocupação ou isolamento temporário de favelas, inclusive com
a participação do Exército e dos Fuzileiros Navais (Marinha), durante
eleições e realização de grandes eventos, como a \versal{ECO}-92 e os Jogos
Pan-Americanos de 2007 (Coimbra, 2001; Barreira e Botelho, 2013).

As incursões em favelas, por sua vez, já eram procedimento corrente da
Polícia Militar, fazendo uso de soldados do Batalhão de Operações
Especiais (\versal{BOPE}), com treinamento militarizado, equipamentos de guerra e
blindados de combate. No entanto, essas entradas visavam combater alguma
facção narcotraficante ou restituir certo equilíbrio de forças entre
facções em disputa territorial. A ``inovação'' das \versal{UPP}, inspiradas em
programas semelhantes levados a cabo na Colômbia desde o final dos anos
1990, era ocupar, fixar bases policiais, manter o território.

O modelo das \versal{UPP} foi montado na combinação de táticas militares de
ocupação territorial e modelos de policiamento comunitário ou de
proximidade que, supostamente, instalariam uma nova relação entre
moradores e policiais centrada no respeito aos direitos humanos e na
cidadania. A conformação espacial das \versal{UPP} foi seguindo os contornos
econômicos e turísticos do Rio de Janeiro. Desde a primeira delas, no
Morro Dona Marta, em Botafogo, Zona Sul, as \versal{UPP} foram instaladas nas
favelas do centro da cidade (zona financeira, porto e centro
administrativo), Zona Sul (bairros turísticos e de classe média e
burguesia), na chamada Grande Tijuca (início da Zona Norte e ligação com
Barra da Tijuca e Alto da Boa Vista, bairros de classe média e
burguesia). Em 2010, faltava ao Programa das \versal{UPP} consolidar o controle
sobre a entrada do Rio de Janeiro: a Avenida Brasil e a Linha Vermelha
(conexão com a Rodovia Dutra e com o aeroporto internacional).

Os complexos do Alemão e da Penha, ocupados por grupos vinculados à
organização ilegal Comando Vermelho, eram considerados de difícil
controle pelo fato de serem labirintos de vielas, com milhares de
habitantes espalhados por grande área, de ocupação principalmente
horizontal (diferentemente da maioria das favelas da Zona Sul, contidas
pelo recorte geológico dos morros). Após ataques a ônibus e veículos,
entre setembro e outubro de 2010, atribuídos ao Comando Vermelho, o
governo estadual solicitou apoio federal para a ocupação do Alemão. Essa
solicitação era legalmente viável porque o governo federal havia
publicado a Lei Complementar 136/2010\footnote{\emph{http://www.planalto.gov.br/ccivil\_03/leis/lcp/lcp136.htm}},
em agosto de 2010, que regulamentava o dispositivo constitucional da
``Garantia da Lei e Ordem'' que regia os casos em que as Forças Armadas
poderiam ser convocadas a atuar em missões de segurança pública
(Rodrigues, 2012a).

A inclusão desse dispositivo no Art. 144 da Constituição de 1988 foi, à
época da Assembleia Constituinte, muito conturbada e discutida em meio a
debates sobre qual seria o papel das Forças Armadas após a ditadura
civil-militar. Segundo Hunter (1997), durante o processo de redação da
nova Constituição houve grande pressão do estamento militar para que o
texto mantivesse intacta a possibilidade de intervenções militares em
nome da manutenção da ordem e das instituições. O primeiro esboço geral
da Constituição, no entanto, não previa o controle do Exército sobre as
Polícias Militares, deixando-as sob a autoridade dos governos estaduais.
Como indica Hunter (1997), houve uma significativa reação do generalato
que levou a uma solução de consenso, deixando o comando das Polícias
Militares a cargo de coronéis da \versal{PM} indicados pelos governadores, mas
preservando uma relação de subordinação das \versal{PM} com o Exército enquanto
força auxiliar em caso de graves conturbações à ordem interna ou de
situação de guerra externa. Não obstante, a regulamentação de como os
governos estaduais poderiam solicitar a atuação das Forças Armadas ficou
para decisão futura.

Essa regulamentação veio com a citada Lei Complementar 136/2010, sendo
antes apenas esboçada, em 1999, no governo de Fernando Henrique Cardoso.
O decreto do governo de Lula da Silva reunia decisões tomadas no seu
primeiro mandato que foram, gradualmente, aumentando a possibilidade de
o governo federal utilizar as Forças Armadas em missões repressivas e
policiais no território brasileiro. A primeira delas foi a chamada Lei
do Abate, prevista pela primeira vez na reforma do Código Aeronáutico de
1986\footnote{\emph{https://www.planalto.gov.br/ccivil\_03/Leis/L7565.htm}.},
emendada em 1998\footnote{\emph{https://www.planalto.gov.br/ccivil\_03/Leis/L9614.htm}.},
mas apenas regulamentada em 2004. Essa decisão autorizou o presidente a
ordenar a derrubada de aeronaves suspeitas que invadissem o espaço aéreo
nacional (Feitosa e Pinheiro, 2012). Na sequência, sucessivas decisões
do governo federal conferiram poder de polícia às Forças Armadas na
fronteira e nas águas territoriais brasileiras, o que significa a
chancela para dar voz de prisão aos suspeitos e apreender veículos
(Rodrigues, 2012a). Com a lei de agosto de 2010, foram estabelecidos os
procedimentos legais para que os governadores solicitassem apoio federal
para operações de segurança pública, o que abriu a possibilidade de uso
das Forças Armadas, da Polícia Rodoviária Federal, da Polícia Federal,
da Agência Brasileira de Inteligência (\versal{ABIN}) e da Força Nacional.

No que se refere às Forças Armadas, a lei impõe os padrões de conduta
esperados dos militares e especifica que cada missão terá um território
e uma temporalidade definidos (Idem). Foi significativo, no entanto, que
durante os períodos em que as intervenções federais durassem, a
autoridade sobre o território delimitado ficasse em mãos de generais do
Exército que se revezariam no comando das Forças, respondendo
diretamente ao Ministro da Defesa e ao Presidente da República.

A missão da Força de Pacificação no Alemão e na Penha recebeu o codinome
de Operação Arcanjo e foi organizada em fases de aproximadamente três
meses, seguindo o padrão de revezamento de tropas praticado pelas
missões de paz da \versal{ONU}. O primeiro contingente a entrar nas favelas, o
Batalhão de Paraquedistas lotado no Rio de Janeiro, fora treinado e
havia atuado na pacificação de favelas em Porto Príncipe, capital do
Haiti. Nos vinte meses de Operação Arcanjo, batalhões de veteranos do
Haiti passaram pelas favelas, alternando com batalhões sem essa
experiência prévia. Uma análise das ``Regras de Engajamento'' da Força
de Pacificação, documento que estabelece objetivos, base jurídica,
procedimentos operacionais e conduta tática, e das ``\emph{Rules of
Engagement}'' da \versal{MINUSTAH}\footnote{No site da \versal{ONU} dedicado à \versal{MINUSTAH},
  encontra-se a série de documentos e relatórios que dão embasamento
  jurídico-político à missão.}, indica muitas semelhanças e trechos
traduzidos diretamente. Dentre as semelhanças, há a autorização para
agir preventivamente interceptando, prendendo ou respondendo com fogo à
ação ou ataque de ``elementos hostis'' (Rodrigues, 2012a). O objetivo
geral das operações também se aproxima, na medida em que ambas visam
criar um espaço de controle do Estado ― no caso do Haiti, das forças
multilaterais na ausência de forças estatais competentes ― para que
cheguem serviços públicos e privados e as metas mais gerais de
``desenvolvimento humano'' sejam atingidas.

Nesse contexto, o uso do termo ``pacificação'' deve ser analisado. Uma
das proveniências do termo é militar, significando a ocupação e controle
sobre determinado território e sua população (Kenkel, 2010). Portanto,
seu uso militar redunda em conquista militar e imposição da ordem do
vencedor que faz cessar os combates diante da superioridade de sua força
física. No caso brasileiro, todavia, o termo tem antigas proveniências
na sua história política. Segundo Gomes (2014) são quatro as principais:
a primeira se refere às pacificações das revoltas regionais que
eclodiram desde a independência do Brasil, em 1822, até a vitória sobre
a secessão republicana no Sul, em 1845. Um dos líderes militares da
vitória na Guerra dos Farrapos, o Duque de Caxias, ganhou o epíteto ``O
Pacificador'', tornando-se patrono do Exército brasileiro e nome de uma
alta comenda militar. A segunda proveniência está relacionada à
pacificação das revoltas milenaristas (Canudos e Contestado), entre
finais do século \versal{XIX} e a segunda década do século \versal{XX}, e às
insubordinações militares ocorridas no período, com destaque para a
Revolta da Armada (1891) e a da Chibata (1910). A terceira proveniência
está na repressão a greves e manifestações políticas organizadas por
movimentos de trabalhadores, principalmente anarquistas, nas décadas
iniciais do século \versal{XX}. A quarta e última, aos aldeamentos voltados à
aculturação de populações indígenas no Brasil central e amazônico,
principalmente pelo trabalho do marechal Cândido Rondon, fundador e
primeiro diretor do Serviço de Proteção ao Índio ― \versal{SPI}, criado em 1910
--e desenvolvido seguindo seu programa de integração nacional pela via
da conexão física dos sertões brasileiros ao litoral e do
assistencialismo humanitarista.

A expressão ``pacificação'', desse modo, faria parte do \emph{ethos} e
da missão autoatribuída pelos militares brasileiros, em especial o
Exército, como força civilizadora e reguladora da estabilidade política
e social do país. Essa mesma atribuição esteve presente no golpe
liderado por Getúlio Vargas contra a oligarquia da Velha República em
1930, depois no autogolpe do Estado Novo, em 1937, ressurgindo no golpe
de 1964 e na ditadura que dali despontou para reaparecer uma vez mais
nos mencionados debates sobre o papel das Forças Armadas com o retorno
da democracia formal nos anos 1980. O chamado ``mito da excepcionalidade
do Exército'' (Figueiredo, 1980), ou seja, a crença na sua superioridade
diante das disputas político-ideológicas e na sua ligação direta com a
pátria e a nacionalidade ressurge em argumentos favoráveis à atuação das
Forças Armadas em missões de pacificação de favelas (Lima, 2012). Nesse
caso, as Forças Armadas seriam convocadas para \emph{intervir} em sérios
problemas de segurança pública que não conseguiriam ser solucionados
pelas polícias estaduais.

A Força de Pacificação conduziu, ao longo do primeiro semestre de 2012,
uma transição para que bases de \versal{UPP} fossem instaladas e, gradativamente,
substituíssem os postos do Exército. Em abril de 2014, cerca de dois
meses antes do início da Copa do Mundo da \versal{FIFA}, o Complexo da Maré,
contíguo ao aeroporto internacional do Rio de Janeiro e ao primeiro
trecho da Linha Vermelha, um conjunto de favelas vizinho aos do Alemão e
da Penha, foi ocupado por uma segunda Força de Pacificação do Exército,
com apoio de blindados da Marinha. Tendo como referência a mesma Lei
Complementar 136/2010, com regulamento próprio e delimitação de funções
e territórios, a Força de Pacificação na Maré obteve prorrogação de seu
mandato, seguindo ativa após o término do torneio de futebol, terminando
sua ocupação apenas em abril de 2015, após processo de transição para
\versal{UPP}s.

Associado ao Programa das \versal{UPP}, sob a responsabilidade do governo
estadual, a prefeitura do Rio de Janeiro inaugurou, em 2009, um programa
denominado \versal{UPP} Social, voltado à introdução de aparelhos sociais
(escolas, clínicas, postos de pronto atendimento) nas favelas
pacificadas. Articulada às \versal{UPP} e aos projetos da \versal{UPP} Social proliferou a
atuação de \versal{ONG}s e fundações culturais patrocinadas por empresas ou
institutos de empresas com projetos de ação social. Parte dessas \versal{ONG}s é
de atuação transterritorial como o Comitê da Cruz Vermelha
Internacional, que testa no Brasil novas formas de atuação voltadas às
chamadas zonas de conflito urbano ``não declarado'' (ou seja, que não
são guerras civis ou guerras interestatais), ou são organizações
brasileiras com experiência internacional, com destaque para a \versal{ONG} Viva
Rio, que atua há dez anos no Haiti e conecta, por exemplo, projetos de
desarmamento em favelas haitianas a práticas desenvolvidas no Brasil
(Hamman, 2015).

Interessa destacar aqui que o conceito de ``pacificação de favelas''
traz consigo a prática da \emph{polícia das condutas} fomentada pela
articulação entre ações estatais e privadas, com ativa participação de
associações de moradores, que se motiva pelo objetivo de controlar os
ilegalismos (como o narcotráfico e o oferecimento de serviços como \versal{TV} a
cabo, transporte e gás de cozinha), conter a violência explícita das
disputas entre grupos narcotraficantes e milicianos e \emph{integrar} as
populações das favelas pelo consumo de bens e pela \emph{participação
cidadã} (eleições, eventos cívicos, atuação em programas de governança
local, colaboração com a polícia para ``combater a criminalidade'').
Modulações de governo das condutas se instalam, assim, por meio da
disseminação das práticas de delação e monitoramentos praticadas não
apenas pelas forças estatais, mas pelos próprios cidadãos, portadores de
direitos inacabados que consideram que seu \emph{status} de cidadania
depende da participação ativa como cogestores das táticas de governo
(Augusto, 2013).

Assim, as técnicas do pastorado atualizadas pela biopolítica (Foucault,
1999) são alteradas pela articulação entre táticas governamentais
agenciadas pelo Estado e outras colocadas em marcha por \versal{ONG}s, fundações
privadas, associações de bairro e cidadãos participativos que vigiam,
monitoram, denunciam, colaboram com as práticas punitivas em nome da
melhoria das condições de vida no bairro, na favela, no país, no mundo.
Esse \emph{pastorado horizontal} não prescinde das antigas táticas de
governamentalidade, mas instaura novas modulações de governo de si e
sobre si próprias da ecopolítica (Passetti, 2013).

A conexão entre as \versal{UPP}, a Força de Pacificação e a \versal{MINUSTAH} situa como
funciona o \emph{dispositivo diplomático-policial}, produzindo práticas
voltadas para a construção de ``ambientes de segurança'' que operam
transterritorialmente de modo análogo e simultâneo, nos planos local e
no ainda chamado de internacional. No caso da \versal{MINUSTAH}, todo um país é
alvo de uma grande missão chancelada e patrocinada pela \versal{ONU} para conter
e pacificar uma sociedade convulsionada pela violência política e
decênios de autoritarismos, desigualdade social e invasões estrangeiras.
O Haiti é tido como origem de fluxos de imigrantes ilegais
(principalmente para os \versal{EUA} e Canadá, mas agora, também, para o Brasil e
depois para o Chile), espaço de trânsito para o narcotráfico em direção
aos \versal{EUA}, foco de doenças como a \versal{AIDS} e de epidemias tropicais, além de
ser terra arrasada do ponto de vista ambiental pela devastação
ininterrupta do seu solo desde tempos coloniais. Esse país é contido e
\emph{pacificado} por uma missão que tem autorização para matar em nome
da paz e da estabilidade e que combina ações militares com ações civis
no campo diplomático e no campo privado, incluindo empresas e \versal{ONG}s.
Humanitarismo e lucratividades conectam-se na construção de um
\emph{ambiente de segurança}, ou seja, um \emph{espaço modulável}
(favelas, porções de cidades, regiões de um país, países inteiros) no
qual populações, seus movimentos e suas dinâmicas política e econômicas,
entre o legal e o ilegal, são gerenciadas por meio de técnicas que
incluem a participação dos próprios governados, orientados por valores
universais (direitos humanos, democracia liberal) em articulação com
organizações internacionais e regionais, Estados, empresas e \versal{ONG}s locais
e estrangeiras.

Nas favelas cariocas, a pacificação pela Polícia Militar ou pelo
Exército segue a lógica da militarização repressiva combinada com uma
nova filantropia e novas formas de conexão e integração produtiva entre
moradores das favelas e a economia capitalista global: grandes
multinacionais e bancos financiam projetos sociais, como a Coca-Cola e o
Itaú, em nome da formação de cidadãos e da produção de um futuro melhor
e sustentável. A própria dimensão militar não é mais apenas repressiva.
Ela se remodela como policiamento comunitário e formalmente atualiza a
relação entre policiais militares e de militares com a população civil.
Nesse sentido, \emph{militarização} não poderia ser compreendida somente
como \emph{repressão}, mas sim, como uma composição de força letal
autorizada e legitimada com serviços no campo da defesa civil e do
fornecimento de infraestrutura básica balizados pelo respeito aos
direitos humanos.

A própria distinção entre polícia e força militar fica mais difusa, como
batalhões especiais como o \versal{BOPE} que recebem treinamento nivelado ao das
forças especiais militares de Estados como os \versal{EUA}, França, Reino Unido e
Israel e que operam armas de grosso calibre (fuzis, metralhadoras,
submetralhadoras, lança-foguetes) e equipamentos com blindagem especial
(com destaque para os ``caveirões'', blindados sobre rodas para invasão
de favelas, e helicópteros blindados). Segundo Alves e Evanson (2013:
4), a ``\versal{PM} {[}fluminense{]} age mais como infantaria do Exército em
missões de \emph{search-and-destroy} {[}busca e destruição{]} do que
como polícia, enquanto a tropa de elite do \versal{BOPE} tem características de
uma unidade blindada''. Os autores, numa perspectiva liberal, indicam
uma diferenciação entre a atuação que se esperaria da polícia (como
força de controle social parametrada por critérios como os direitos
humanos e a manutenção da ordem pública) e de forças armadas (como
forças militares especializadas em combates para defesa do território e
soberania). Mas o que interessa notar aqui de um ponto de vista
analítico é que polícias e forças armadas convergem na ecopolítica do
planeta como elementos de um dispositivo de segurança que alinhava
funções repressivas, filantrópicas e de governo das condutas.

Essa militarização teria mais conexão com o que Graham (2016) denomina
``militarismo urbano'' que é o trânsito de tecnologias produzidas com
finalidades militares (como o \versal{GPS}, as câmeras de vigilância, os sensores
infravermelhos, drones, sensores biométricos, cercas elétricas etc.)
combinado com novos esquadrinhamentos do espaço urbano que conectam
especulação imobiliária em zonas depauperadas ou decadentes, processos
de gentrificação em favelas (especulação imobiliária que expulsa as
populações mais pobres das áreas urbanas em ``revitalização''),
ampliação de zonas de investimento imobiliário e proliferação de
condomínios fechados com sofisticados sistemas de segurança. Tecnologias
de controle do espaço sideral passam a incidir diretamente sobre o
controle dos fluxos de pessoas, com o georreferenciamento das regiões
habitadas, o mapeamento do solo e do subsolo, o monitoramento do clima,
dos focos de desmatamento, das ocupações urbanas irregulares, das zonas
a serem controladas e ocupadas por programas sociais ou policiais em
função de vulnerabilidades. Os controles, vigilâncias e monitoramentos
voltam-se, assim, para o ``governo de vulnerabilidades'' que identificam
``situações de risco'' nas quais vivem crianças e jovens, populações
miseráveis, bairros, favelas, cidade, e também regiões e países inteiros
(Oliveira, 2011). Mapas de vulnerabilidades produzidos por
monitoramentos de Estados, \versal{ONG}s e organizações internacionais apoiam a
seleção de populações e espaços sobre os quais deverão incidir políticas
de contenção de conflitos latentes e misérias, desimpedindo a circulação
de investimentos, produtos e forças produtivas. Nesse contexto, o uso
das tecnologias computo-informacionais é de importância central.

Os controles computo-informacionais próprios da sociedade de controle
(Passetti, 2003a) estão relacionados às tecnologias de transmissão de
dados e monitoramento por satélites, fibras óticas e redes de
computadores. Nas proveniências desses controles está a conquista do
espaço sideral, iniciada por \versal{EUA} e \versal{URSS} no final dos anos 1950, que se
atualiza hoje em exploração econômica, científica e político-militar do
entorno da Terra, do sistema solar e além, com o envio das sondas
intergalácticas. O \emph{Tratado sobre Espaço Exterior}\footnote{\emph{http://www.unoosa.org/pdf/publications/STSPACE11E.pdf}},
de 1967, por exemplo, foi celebrado no auge da conhecida corrida
espacial entre as superpotências e estabeleceu o espaço sideral como
área aberta à livre exploração da humanidade. Esse discurso é similar ao
dedicado à Antártida, cuja internacionalização, firmada em tratado da
\versal{ONU} de 1959\footnote{\emph{http://www.ats.aq/e/ats.htm}}, fia-se na
igualdade de condições legais para a exploração de um território cuja
concreta possibilidade de presença se restringe a poucos países com
recursos financeiros, tecnológicos e militares.

De todo modo, o interesse desses tratados em internacionalizar e
desmilitarizar o espaço sideral e a Antártida traz em comum o fato de
que a exploração econômica, o monitoramento eletrônico e o
desenvolvimento de pesquisas científicas são possíveis por conta de
recursos tecnológicos e táticos de proveniência militar e contam com o
chamado ``uso dual'' (simultaneamente ``militar'' e ``civil''), como os
que se dissiminaram na vida cotidiana: telefonia celular, radares,
câmeras de vigilância, aparelhos de micro-ondas, redes de computadores,
utilização de senhas, códigos de acesso e leituras biométricas acionadas
sempre em nome da maior segurança do usuário e das instituições estatais
e privadas (Graham, 2010). Desse modo, na ecopolítica do planeta, a
distinção entre uso ``civil'' e ``militar'' fica cada vez mais
indistinta, na medida em que técnicas e práticas de governo das condutas
se espraiam não apenas pelos dispositivos de segurança do Estado, mas
também, pela combinação de forças estatais (policiais e militares),
empresas privadas de segurança (compostas, em larga medida, de
ex-militares e ex-policiais) e por iniciativas menos explícitas de
monitoramento e controle por associações de moradores, institutos de
pesquisa, agências especializadas de organizações internacionais em
projetos locais (como as muitas ações do Programa das Nações Unidas da
\versal{ONU}) e \versal{ONG}s (financiadas por empresas, fundações privadas, Estados, pela
\versal{ONU}) que coletam, analisam e fornecem dados para seus financiadores
servindo de fontes para a elaboração de programas de segurança pública e
para programas de investimento social, tudo em função do acesso
democrático à comunicação, o que fortalece a continuidade dos controles
na sociedade operando em relações verticais e horizontais.

Assim, esse ``novo militarismo'' não se confundiria mais com a ação
exclusiva de militares das forças armadas dos Estados: as forças
militares instituídas no processo de formação dos Estados Modernos, na
passagem da Idade Média à Moderna (Foucault, 2008; Tilly, 1996; Keegan,
2002). Essas forças, associadas à guerra clausewitziana, são
contemporaneamente atualizadas diante de novas emergências da violência
que, lembrando Frédéric Gros (2009), não seriam mais a ``guerra'', mas
múltiplos e diferenciados ``estados de violência''.

Nesse contexto, as forças repressivas estatais se reconfiguram, dando
lugar a novas institucionalizações no exercício da violência. As forças
armadas de Estados investem em tropas de elite, com uso intensivo de
tecnologia e treinamento especializado, cuja proveniência vem dos
\emph{commandos} especiais britânicos e estadunidenses formados durante
a \versal{II} Guerra Mundial para executar operações secretas atrás das linhas
inimigas e que não eram chanceladas pelo direito da guerra (praticando,
portanto, atentados, assassinatos de personalidades etc.) (Singer,
2008). Hoje, grupos especiais como os \versal{SEALS} (\emph{See, Air and Land})
da Marinha dos \versal{EUA} atuam como tropas altamente treinadas e
profissionalizadas, em contraste com as tropas regulares cada vez menos
profissionais e compostas por voluntários que buscam remuneração ou
benefícios como os vistos de permanência para estrangeiros, distribuídos
aos imigrantes nos Estados Unidos (Owen, 2012). Foi um comando \versal{SEAL} que
invadiu o Paquistão, localizou e assassinou Osama bin Laden, em 2011.

No antigo ``Terceiro Mundo'' aconteceram outras atualizações. Na América
Latina, em especial, a Doutrina de Segurança Hemisférica instituída
pelos \versal{EUA} no imediato pós-\versal{II} Guerra Mundial estabeleceu que a defesa do
continente contra eventuais ataques militares extrarregionais
(soviéticos, notadamente) seria contraposta pelas forças armadas
estadunidenses e que caberia aos militares latino-americanos e
caribenhos combater os ``inimigos internos'' comunistas. Um conjunto de
documentos firmou esse compromisso hemisférico, com destaque para o
\emph{Tratado Interamericano de Assistência Recíproca} ― \versal{TIAR}, de
1947\footnote{\emph{http://www.bibliotecajb.org/Portals/0/docs/seguridadcolectiva/4.pdf},.},
que instituiu o princípio de que todos os Estados do continente
americano se apoiariam em caso de ataque por uma potência extraregional,
e a \emph{Carta de Bogotá}, de 1948, que criou a Organização dos Estados
Americanos\footnote{\emph{http://www.oas.org/dil/port/tratados\_A1\_Carta\_da\_Organiza\%C3\%A7\%C3\%A3o\_dos\_Estados\_Americanos.htm},}.

Desde o esmorecimento da Guerra Fria, na segunda metade dos anos 1980, o
discurso diplomático-militar estadunidense identificou os grupos
narcotraficantes como a principal ameaça à segurança dos Estados e do
continente como um todo (Rodrigues, 2006; 2012a). Na passagem para o
século \versal{XXI}, forças militares e batalhões especiais em Estados do ``Sul''
são financiados e preparados, principalmente pelos \versal{EUA} e União Europeia,
para \emph{colaborar} com a ``guerra ao terror'' ou ``guerra às
drogas'', enquanto tropas policiais em todo globo passam a receber
treinamento e equipamentos militarizados, contando com suas próprias
tropas de elite e aprofundando a indistinção entre forças armadas e
forças policiais (Bayley e Perito, 2010; Balko, 2013; Graham, 2016).

Os ``estados de violência'' ensejam uma nova e multifacetada prática da
\emph{guerra justa}: contra o crime, contra o terrorismo, contra os
genocídios, contra os ``Estados falidos'' ou os que violam os direitos
humanos de seus cidadãos. Estabelece-se o ``continuum conflitivo''
descrito por Bigo (2010) que atravessa fronteiras, reforçando a conexão
entre ``dentro'' e ``fora'' dos territórios nacionais (Walker, 2013) e
que indica um dispositivo de segurança próprio à ecopolítica do planeta.
Nesse contexto, há um crescente processo de \emph{policialização das
forças armadas} acompanhado da \emph{militarização das polícias}. Assim,
na sociedade de controle, o ``inimigo exterior'' da guerra
clausewitziana (outro Estado com suas forças armadas regulares) e o
``inimigo interno'' da luta antisubversiva da Guerra Fria são
redimensionados. Hoje, os Estados e os novos agenciamentos de poder
centralizado (\versal{ONU} e União Europeia, por exemplo), definem ``inimigos
transterritoriais'' que combinam e alteram características dos
``inimigos externos'' e ``internos'', pois atravessam fronteiras
nacionais, mantêm pontos de apoio e conexão dentro dos Estados e
movimentam-se nos fluxos eletrônicos e de pessoas que são acelerados na
contemporaneidade. São grupos e indivíduos de atuação transterritorial
porque têm simultaneamente \emph{expressão local} e
\emph{transfronteiriça}, com grande mobilidade e capacidade de se
desterritorializar e se reterritorializar (Rodrigues, 2012c).

Desse modo, seria possível compreender conflitos como os agenciados
contra o terrorismo transterritorial, o narcotráfico e suas conexões
locais e distendidas no planeta como \emph{guerras-em-fluxo}: embates
transterritoriais que não são as ``guerras interestatais'', tampouco as
``guerras civis'' previstas no direito internacional humanitário como os
confrontos entre grupos organizados dentro de um Estado, em torno da
disputa pelo controle do Estado (Rodrigues, 2010). As guerras-fluxo não
substituem as guerras entre Estados, não superam as guerras civis, mas
articulam variados ``estados de violência'' que se conectam de forma
transterritorial. Não interessa mais, então, discutir o ``dentro'' e o
``fora'' das fronteiras, pois esses fluxos articulam-se por meio de
grupos que realizam suas atividades de forma transterritorial. As
modalidades de securitização e as táticas de controle, monitoramento,
gestão e combate que são elaboradas por Estados e consórcios de Estados
assumem, também, expressões transterritoriais. A definição dos
``inimigos internos'' ganha dinâmica que reconfigura a definição da
``política como guerra continuada por outros meios'' (Foucault, 1999).

Para Michel Foucault (1999; 2013), as lutas a propósito do poder
político e travadas pelo estabelecimento de uma dada ordem social
levaram, na história moderna, a uma recondução permanente de uma guerra
entre grupos sociais dominantes e grupos sociais sujeitados. Essa guerra
foi mantida por meio de técnicas repressivas desenvolvidas pelos Estados
Modernos desde seus inícios e aperfeiçoadas com o desenvolvimento das
táticas disciplinares e biopolíticas na passagem do século \versal{XVIII} para o
século \versal{XIX}. Para Foucault (1999; 2013), diferentemente do que defendiam
os autores contratualistas como Thomas Hobbes, a instituição do poder
político não teria calado uma guerra primordial de todos contra todos,
mas ativado uma guerra constante que se dá pelas instituições e em nome
da paz civil. Para Foucault, desse modo, a política poderia ser
analisada em termos de uma guerra interna constante, ou seja, em termos
de uma ``guerra civil'' permanente.

Essa ``guerra civil'' foi acionada e travada, para o filósofo, pela
penalização ou pelo sistema penal. Em outras palavras, foi pela
definição de ``crime contra a sociedade'' e, por conseguinte, da
produção da categoria de criminoso como ``aquele que rompe o pacto
social'' (Foucault, 2013: 34), que foi possível perpetrar essa guerra
constante voltada para dentro do próprio território e sobre a própria
população. Ao criminoso, segundo Foucault, é associada a figura da
``hostilidade social'' o que remete a Clausewitz e ao conceito de
``intenção hostil'' como atitude que define o ``inimigo'' (e que
reaparece, usando termos semelhantes, nas Regras de Engajamento da
\versal{MINUSTAH} e das Forças de Pacificação no Rio).

Na biopolítica, o ``inimigo externo'' deixou de ser formalmente aceito
quando da proibição da guerra de agressão enquanto o ``inimigo interno''
seguiu alvo da criminalização e repressão quer fosse como
\emph{subversivo} ou como \emph{criminoso comum} (sem vinculação direta
com grupo ou ideias políticas). Na sociedade de controle, o criminoso de
guerra se transforma em genocida (criminoso contra a Humanidade), o
terrorista transforma-se, ao mesmo tempo, em criminoso contra o direito
nacional e contra a ordem mundial, o narcotraficante é classificado como
criminoso ao atentar contra a saúde pública, a segurança pública e a
segurança nacional e internacional, ao passo que os tradicionais
``criminosos'' que atentam contra a propriedade privada e a vida
individual seguem sendo assassinados pela polícia ou lotando prisões por
todo o planeta.

As penalizações se desdobram em crimes contra o planeta (crimes
ambientais) e contra a Humanidade, tanto no \emph{ambiente planetário}
(com os tratados internacionais e instituição do Tribunal Penal
Internacional, criado em 1998), quanto nos \emph{ambientes}
\emph{domésticos} (cada Estado) e \emph{regionais} (como a União
Europeia), nos quais ``crimes'' como genocídio e terrorismo são
tipificados e incorporados às legislações locais, afinando normas
internacionais e leis nacionais.

Nesse sentido, se para Foucault a ``guerra civil'' não foi abolida, mas
instaurada pelo Estado, coexistindo com a ``guerra exterior'', em tempos
de ``estados de violência'', a guerra atravessa fronteiras nacionais,
como \emph{guerras-fluxo}, conectando ambientes que os discursos
jurídico-políticos ainda situam como \emph{domésticos} e
\emph{internacionais}. Quando as definições de inimigos são atualizadas
e se um conjunto de táticas instaura práticas de governo planetário, a
guerra civil também se planetariza seguindo as novas governamentalidades
em escala global. Em suma, na ecopolítica a guerra entre Estados cede
espaço para distintas modulações de guerras-fluxo nas quais Estados e
coalizões de Estados (na \versal{ONU}, na \versal{OTAN}, na União Europeia etc.)
identificam ``inimigos'' dentro, fora e através de suas fronteiras,
apontando-os como ``ameaças'', simultaneamente, à ordem política, social
e econômica nos espaços nacionais e no plano global.

É nesse contexto que proliferam as práticas de securitização
identificando ``ameaças existenciais'' não só ao Estado, mas a outros
``bens'' (como os direitos humanos e o meio ambiente). A produção
incessante de ``ameaças'' e ``inimigos'', própria da política (Foucault,
1999), supera os limites da divisão estanque demarcada por fronteiras
nacionais (o inimigo interno) e pelo ``estado de natureza''
internacional (o inimigo externo) para conectar ``dentro'' e ``fora''
num \emph{continuum de segurança} que configura um ambiente planetário a
ser gerido pelas táticas governamentais da ecopolítica.

A \emph{segurança} desse ambiente planetário significa, assim, a
\emph{segurança} \emph{dos fluxos produtivos e inteligentes pelo globo}
compreendidos como os fluxos de comunicação eletrônica (mídias sociais,
mídias alternativas, transmissão de dados de inteligência entre Estados
e de gestão para empresas, novos serviços ― legais e ilegais ―
oferecidos no mercado ``virtual''), a transferência de capitais, os
fluxos de transporte de produtos e pessoas, a abertura e liberalização
de mercados, a garantia de segurança em pontos estratégicos como a
Chifre da África, o Canal de Suez, o Cáucaso, o Estreito de Málaga, o
Canal do Panamá; a \emph{contenção de revoltas, levantes e guerras
civis} em regiões estratégicas para o capitalismo global; a
\emph{contenção, ocupação e tutela de regiões ou governos} que apoiem ou
não controlem a atuação de grupos do terrorismo transterritorial; a
\emph{contenção, ocupação ou tutela de países ou regiões} que não
garantam a fixação de suas populações por conta de miséria extrema,
violência ``étnica'' ou ``religiosa'' e destruição do meio ambiente; a
\emph{contenção e governo de bairros, periferias, favelas} em
cidades-chave para o capitalismo global; o \emph{controle sobre os
fluxos transterritoriais de ilegalismos} de modo a garantir que se
realizem para potencializar certas dinâmicas vitais ao capitalismo
global, evitando excessos nos desequilíbrios econômico-financeiros e nas
manifestações explícitas de violência; e a \emph{ampliação das
securitizações de temas globais} que trazem problemas para o governo das
populações, fluxos e territórios, mas que são dificilmente equacionados
isoladamente por cada Estado devido a sua expressão transterritorial,
como a mudança climática, os fluxos migratórios, o provimento de energia
e alimentos. Essas securitizações combinam ações individuais dos Estados
e mobilizações coletivas no âmbito da \versal{ONU}, com alianças militares e
atuação de organizações regionais de segurança. Nos seus terminais
locais, por sua vez, associações de moradores, \versal{ONG}s e cidadãos colaboram
com a produção de dados, acolhida de institutos e pesquisadores e
participação na aplicação de políticas com as quais colaboraram na
elaboração.

A biopolítica da população contou com seu dispositivo de segurança ― o
diplomático-militar ―, que no plano interno combinou táticas
disciplinares com estratégias de gestão da vida interessadas na produção
de um adicional de vida a fim de produzir indivíduos úteis e dóceis, e
no plano externo produziu efêmeros equilíbrios de poder agenciados pelas
forças militares e pelo dispositivo diplomático dos Estados. Esses
equilíbrios foram entrecortados por grandes guerras mundiais e
atravessados por infindáveis guerras locais, guerras civis, levantes
guerrilheiros e de libertação nacional. A biopolítica configurou-se como
prática espacial, interna a cada Estado ainda que com crescentes e
incessantes comunicações com outras práticas biopolíticas de outros
Estados.

Na ecopolítica do planeta, as conflitividades próprias da biopolítica
não desaparecem, mas são redimensionadas por questões de segurança de
projeção global. O desafio de governar as condutas, propósito das
práticas de poder agenciadas pelas centralidades de governo, alcançou
outra magnitude com a dinamização geral dos fluxos transterritoriais da
sociedade de controle. Na ecopolítica do planeta, um novo dispositivo de
segurança emerge compondo com o conjunto geral das novas
governamentalidades que visam o governo de fluxos, coisas, pessoas,
inteligências no globo terrestre e no entorno sideral. Trata-se do
\emph{dispositivo diplomático-policial}. Das práticas mais ínfimas, no
governo sobre as condutas individuais, até a conduta de Estados no
ambiente global, esse novo dispositivo não articula apenas táticas
provenientes das centralidades políticas, mas agencia práticas de
governo acionadas por indivíduos participativos, moderados e que se
(auto) configuram como cidadãos-polícia, enquanto ativa modalidades
planetárias para a gestão ``confiável'', ``protocolada'',
``contra-protocolada'' e ``segura'' para os fluxos de capital, dados
eletrônicos, produtos e fluxos inteligentes.

\section{ambientes de segurança}

Em entrevista concedida em 1977, Michel Foucault foi questionado sobre o
que compreendia por ``dispositivo'' a partir da noção de ``dispositivo
da sexualidade'' que acabava de apresentar no primeiro volume de
\emph{História da Sexualidade}, intitulado \emph{A vontade de saber}. Ao
responder a seu entrevistador, Foucault esboça uma definição que destaca
alguns elementos. Em primeiro lugar, para Foucault, dispositivo
{[}\emph{dispositif}{]} é ``um conjunto absolutamente heterogêneo que
implica discursos, instituições, estruturas arquitetônicas, decisões
regulamentares, leis, medidas administrativas, enunciados científicos,
proposições filosóficas, morais e filantrópicas {[}...{]} O dispositivo
é ele mesmo a rede que se estabelece entre esses elementos'' (1977a:
299). Logo depois, o filósofo complementa sua resposta afirmando que
dispositivos se formam num dado momento histórico com ``a função
principal de responder a uma urgência'', por isso, ``o dispositivo tem
uma função preponderantemente estratégica'' (Ibidem). Apenas após sua
formulação, sua emergência para responder a questões urgentes, é que o
dispositivo ganha corpo, sofisticação, abre conexões com outras
práticas, produzindo efeitos ``positivo{[}s{]} ou negativo{[}s{]},
pretendido{[}s{]} ou não'' (Ibidem).

Assim, antes de serem formulados por uma ``inteligência superior'' que
pretendesse constituir de propósito um grande mecanismo de dominação, os
dispositivos surgem como respostas heterogêneas a problemas específicos,
problemas políticos que desafiam as práticas de governo. Desse modo, é
possível compreender o dispositivo diplomático-militar, produzido pelos
Estados europeus entre os séculos \versal{XVI} e \versal{XVII} como um conjunto
heterogêneo que combinou novas táticas militares associadas à tecnologia
das armas de fogo, a constituição de grandes exércitos permanentes, a
atualização das práticas diplomáticas que ganharam forma permanente
estabelecendo relações ininterruptas entre os príncipes, a elaboração da
estatística como saber do Estado voltado para inventariar, organizar e
dispor as forças materiais dos reinos, a elaboração de uma filosofia
política em torno do Estado interessada em justificar sua existência e
sua necessidade para a manutenção da paz e da ordem interna ao reino.

Esse dispositivo produziu efeitos de equilíbrio de poder entre os
Estados mais poderosos da Europa Moderna, constituindo-se, desse modo,
como um mecanismo de e para a segurança dos Estados. Com ele, foi
possível a cada príncipe lidar com certa previsibilidade nas ações e nas
capacidades de poder dos Estados, movimentando-se para equiparar-se ao
rival num ambiente aberto, pois não era organizado e disciplinado por
uma força político-militar superior. Esse dispositivo funcionou nas
relações de poder entre europeus e, depois do século \versal{XIX}, entre as
dezenas de novos Estados que foram surgindo do desmonte dos impérios
coloniais ibéricos. No século \versal{XX}, após as duas guerras mundiais, as
poucas dezenas de Estados independentes que havia antes de 1914 foram
multiplicadas atingindo mais de uma centena no início dos anos 1970.
Cada um desses novos Estados emulou o modelo jurídico-político dos
Estados Modernos europeus, suas antigas metrópoles coloniais, elaborando
suas próprias versões do dispositivo diplomático-militar. No entanto, as
relações internacionais já não eram as mesmas de quando os primeiros
dispositivos diplomático-militares se conformaram na Europa.

A guerra entre Estados atingira seu ponto máximo com a incrível
destruição das duas guerras mundiais, encerradas, em 1945, com a
possibilidade real de um apocalipse nuclear. A conservação dos Estados,
meta da política moderna, viu-se diante de situação paradoxal: as
armadas mais poderosas já criadas poderiam estabelecer a segurança total
daqueles que as possuíssem, mas também poderia anunciar sua completa
eliminação; e não apenas a sua, mas a de todo planeta.

Além disso, o arranjo diplomático que emergiu das guerras mundiais
introduziu elementos novos de universalidade e cosmopolitismo que se
conectaram aos discursos da soberania e da autonomia jurídico-política
de todos os povos: a fórmula westfaliana se universaliza com a \versal{ONU} ―
organização formada por Estados e comandada pelos vitoriosos da \versal{II}
Guerra Mundial ―, mas não se trata simplesmente da projeção da Europa do
século \versal{XVII} para o planeta como um todo. As transformações nas dinâmicas
econômicas, políticas, tecnológicas e sociais apresentavam, em meados do
século \versal{XX}, novos desafios para o governo de pessoas, coisas e
territórios. O capitalismo, de expressão planetária desde sua
emergência, alcançou novas dimensões, conectando todas as partes do
globo, a produção de bens se planetarizou e a circulação de produtos e
capital foi dinamizada pelos novos meios de comunicação e transporte. As
modalidades de lutas acompanharam o descolamento das formas tradicionais
de enfrentamento entre Estados legado pela Europa westfaliana. As
energias guerreiras, que a busca do monopólio da coerção física procurou
submeter à autoridade central do Estado, irromperam de diversas
maneiras, nas guerrilhas de libertação nacional, nos terrorismos e nos
grupos armados dedicados aos negócios ilegais acelerados pela
globalização do capital e abertura dos mercados.

Os ``estados de violência'' emergiram voláteis e de difícil contenção
pelos Estados com seus aparatos de segurança. Com velocidade crescente,
os conflitos violentos começaram a se desassociar do enfrentamento entre
Estados, com sua racionalidade, institucionalidade e códigos jurídicos.
A guerra entre Estados foi codificada e mesmo proscrita por tratados,
mas não deixou de existir diante dos choques entre Razões de Estado. No
entanto, os grandes embates entre forças armadas passaram a ser cada vez
menos frequentes do que os amorfos conflitos que despontaram dentro e
através das fronteiras estatais. A separação jurídica, política e
conceitual entre espaços ``domésticos'' supostamente pacificados pelo
monopólio estatal da coerção física contraposto a um espaço
``internacional'' sem ordem e destinado a ser um eterno ``estado de
natureza'' foi se confirmando como discurso caro aos estudos das
Relações Internacionais, interessados que são em justificar, conservar e
legitimar o Estado-Nação como única alternativa para a organização
política dos povos. Mas mesmo os Estados encontraram outras formulações
e arranjos jurídico-políticos e diplomático-militares para seguir no
trabalho de conservar-se e governar.

A sociedade de controles colocou novos desafios para o exercício do
poder desde centralidades políticas. Então, uma nova governamentalidade,
de expressão e alcance planetário, emergiu não como um ``governo do
mundo'', instaurando um centro imperial, mas como a articulação de
múltiplos dispositivos compreendidos como, acima, definiu Foucault.
Dentre esses dispositivos, o de segurança. Não mais a segurança dos
Estados tal qual na Europa de princípios do que se convencionou chamar
de era moderna, mas um redimensionado dispositivo que não abandona as
táticas e mecanismos dos dispositivos diplomático-militares, mas os
atualiza, modifica, recondiciona.

A dimensão ``diplomática'' desse novo dispositivo transborda os limites
da diplomacia dos embaixadores e das chancelarias estatais e se
reconfigura, simultaneamente, em vários níveis. Em primeiro lugar, os
próprios Estados produzem e investem em organizações multilaterais,
foros para a produção de discursos de alcance universal, para a
negociação de compromissos que estabeleçam balizas para o comportamento
dos Estados. Nessas organizações, uma nova métrica dos comportamentos
vai sendo moldada, desde as prescrições pacifistas impressas na Liga das
Nações, nos anos 1920 e 1930, até a tensão entre cosmopolitismo e
westfalianismo que modelou a \versal{ONU}, a partir de 1945. A elaboração de
códigos de conduta, contendo a previsão de meios para punir os Estados
transgressores, elaborada a partir de 1945, com base em princípios
universais, como os ``direitos humanos'', visando a ``paz e segurança
internacionais''. A aplicação seletiva das regras previstas na
\emph{Carta de São Francisco} e em todo o conjunto de tratados
celebrados desde então sob os auspícios da \versal{ONU}, não fugiu à sobejamente
conhecida seletividade de todo o direito nacional ou internacional
formulado como ``universal'', mas aplicado sobre alguns em detrimento de
outros. As relações de poder legislam, alertou o anarquista
Pierre-Joseph Proudhon quando, ainda em meados do século \versal{XIX}, afirmou
que ``o direito faz sua entrada no mundo através da força (...) o
direito do mais forte (...) é o mais antigo de todos, o mais elementar e
o mais indestrutível'' (Proudhon, 2011: 42).

Liberados de crer na formulação do direito internacional, da
constituição de organizações internacionais e das regulamentações sobre
a guerra como emanações de um princípio metafísico e transcendental de
``Justiça'', permanecemos livres para compreender a produção das novas
institucionalizações da ecopolítica do planeta em sua dimensão
estratégica, ou seja, como respostas histórico-políticas interessadas em
atualizar práticas para manter o governo sobre coisas, pessoas, espaços,
ideias, produtos. Essas respostas se dão, hoje, em escala planetária e
projetam suas táticas para o espaço sideral. São configurações
heterogêneas de discursos, práticas, instituições que visam assegurar
práticas de governo que já não se limitam aos territórios nacionais e às
populações nacionais. Está em jogo o governo do planeta e, para tanto,
constitui-se um novo dispositivo de segurança.

Esse dispositivo faz da ``diplomacia'' uma prática disseminada desde os
locais onde incidem as práticas de governo ― favelas, bairros,
comunidades, regiões, países ― envolvendo cidadãos, associações de
moradores, militantes políticos, agentes estatais, burocratas,
filantropos, polícias e forças armadas, empresas de segurança: todos
conectados a instâncias de representação e participação, contribuindo
com suas experiências para a constituição de ambientes mais seguros para
o exercício de seus muitos direitos. A diplomacia dos embaixadores
articula-se, então, com a das delegações diplomáticas nas organizações
internacionais e deles como as \versal{ONG}s com assento nessas organizações que
se conectam com fundações privadas que, por sua vez, são ouvidas por
governos nacionais. A ``diplomacia'' está na prática de produzir
acordos, confeccionar consensos, negociar as justas medidas e a
satisfação das necessidades e demandas. Essa prática diplomática, por
sua vez, é alimentada e, ao mesmo tempo, aciona práticas de polícia das
condutas num movimento ascendente, descendente e horizontal: desde a
favela até as cúpulas da \versal{ONU}, uma cadeia de feixes conduz experiências,
recomendações e conceitos, todos eles alinhavados por valores universais
(direitos humanos, conservação ambiental, paz, desenvolvimento,
democracia) que disciplinam a diversidade numa unidade múltipla.

A polícia das condutas está, assim, no cidadão local que sabe dos seus
direitos e é assistido por \versal{ONG}s locais financiadas por fundações
internacionais e pelo seu próprio Estado (programas municipais,
estaduais ou federais). Esses programas, por sua vez, acontecem em
comunicação, conectando a \emph{pacificação} de regiões (como o Darfur
no Sudão) ou de países inteiros (como o Haiti), com o controle e a
gestão de espaços urbanos como as favelas cariocas. Nesses programas,
atuam forças estatais reconfiguradas ― polícias recondicionadas por
doutrinas de ``policiamento cidadão'' e forças armadas policializadas ―,
\versal{ONG}s humanitaristas, fundações privadas voltadas à promoção de
atividades culturais e de formação de capital humano, forças de
segurança privadas contratadas por moradores, por empresas ou pelo
Estado, associações de moradores chamadas a participar no governo dos
seus próprios bairros e favelas, segundo a lógica da pacificação, além
de milícias e grupos do chamado crime organizado que não deixam de
existir, mas alteram práticas de modo a compor com as novas forças
presentes nos seus antigos espaços. As práticas de segurança nos grandes
espaços planetários conectam-se, então, às mais locais táticas de
controle.

Essas práticas e táticas, nas mais variadas dimensões, têm como elemento
comum a busca pela constituição de \emph{condutas}. No plano mundial,
princípios como a R2P, as intervenções humanitárias e as missões de paz
da \versal{ONU} voltadas à estabilização e ``construção de Estados'' estabelecem
modelos de conduta centrados na obediência aos direitos humanos e à
democracia liberal com conectividade ao capitalismo globalizado. Em nome
da ``segurança humana'', meios coligados de intervenção podem ser,
então, acionados para corrigir desvios, reeducar regimes políticos,
punir ``criminosos de guerra'', promover ``revoluções democráticas'',
ocupar regiões sem governo (nos Estados ditos ``frágeis'' ou
``falidos''), conter fluxos de refugiados e imigrantes, buscando
reprimir seu movimento transterritorial ao mesmo tempo em que se procura
cuidar, acolher e repatriar, contendo a degradação.

No plano local, as experiências de gestão de segurança conduzidas no
plano global encontram ressonância e se retroalimentam, não sendo nítido
quanto ou como uma missão como a \versal{MINUSTAH} influencia ou é influenciada
pelas \versal{UPP}s e Forças de Pacificação. E vice-versa. A conduta esperada de
cada cidadão na favela é similar à de cada cidadão haitiano, e os
programas são aplicados, por vezes, pelas mesmas instituições privadas,
estatais ou multilaterais.

Busca-se com o \emph{dispositivo diplomático-policial} a produção de
``ambientes seguros''. Foucault, em anotação para a aula de 21 de março
de 1978, em seu curso \emph{Nascimento da biopolítica}, indicou que o
pensamento liberal estadunidense, a partir dos anos 1970, e antes dele,
o ordoliberalismo alemão, nos anos 1930 e 1940, formularam uma noção de
``ambiente'' segundo o princípio de que a ``ação governamental'' deveria
limitar-se a manter as condições gerais de um espaço no qual os
indivíduos ― os agentes econômicos ― pudessem jogar. Jamais intervir
sobre a ``mentalidade dos jogadores'' (Foucault, 2008a: 304), mas apenas
sobre o ambiente, estabelecendo marcos legais ``bastante frouxos para
que ele possa jogar'', gerando ``não uma individualização unificadora,
identificadora, hierarquizante, mas uma ambientalidade aberta aos acasos
e aos fenômenos transversais'' (Ibidem). Um ambiente seguro para a livre
iniciativa, limitando o papel do Estado ao de um operador de uma
``tecnologia ambiental'' (Ibidem).

O \emph{dispositivo diplomático-policial} funciona como um operador de
uma tecnologia ― um conjunto de táticas, instituições e discursos ― que
visa produzir ambientes seguros para que os fluxos do capitalismo
circulem e realizem suas lucratividades, para que resistências políticas
a esses fluxos sejam contidas não apenas pela repressão e
criminalizações (que seguem muito ativas), mas também pelos
recondicionamentos e capturas de inteligências produtivas e
potencialmente contestadoras em práticas participativas nas quais se
exercitam as ``diplomacias'' contemporâneas. Esses ambientes, por sua
vez, organizam-se e se tornam seguros pela participação ativa dos seus
próprios moradores, articulados por direitos e por programas de inserção
econômica e educação formal para permitir tal inserção. Os ``estados de
violência'' são combatidos ou acomodados por novos arranjos entre forças
repressivas estatais, privadas ou coletivas (como a \versal{OTAN} ou as missões
da \versal{ONU}) e o policiamento das condutas se exerce desde as mais altas
instâncias das organizações internacionais até o morador da favela. A
conexão entre ``estados de violência'' explicita o acontecimento de uma
guerra-fluxo.

Trata-se de produzir precauções, os conflitos são evitados ao máximo,
precisamente pela proliferação de direitos e pela difusão de protocolos,
negociações, conferências, consultas públicas. Viver com cidadania, ou
seja, com direitos e responsabilidades, significa, então, dominar
minimamente \emph{práticas diplomáticas}: saber negociar, contemporizar,
cooperar e compor alianças. Esse comportamento, esperado tanto de
Estados quanto de empresas, \versal{ONG}s, das próprias organizações
multilaterais e dos indivíduos é o próprio estabelecimento de
\emph{condutas seguras}, pois previsíveis e acopláveis, exercitadas com
moderação.

Por esse motivo, o dispositivo de segurança planetário que emerge na
ecopolítica é um \emph{dispositivo diplomático-policial}, pois associa a
disseminação das técnicas de representação e negociação para além da
dimensão diplomática do antigo dispositivo diplomático-militar analisado
por Foucault com o propósito de estabelecer e gerenciar condutas
esperadas, um gerenciamento policial que redimensiona o antigo
dispositivo de polícia. Agora, a \emph{polícia} não é uma intervenção do
Estado com o objetivo de controlar direta e detalhadamente todos os
fluxos, atividades e condutas. A meta continua sendo \emph{governar} ―
e, como definiu Foucault (1995), governar é ``conduzir condutas'' ―, mas
esse governo se dá contemporaneamente com a ativa participação de quem é
governado, seja ele um indivíduo, um Estado ou coalizões de Estados.

A \emph{polícia das condutas} se projeta no planeta e não tem um centro
único de onde emane, ainda que reconduza (Passetti, 2011a), a novas
centralidades. Essas \emph{centralidades sem centro} se unificam na
condução das condutas e na adesão aos protocolos e táticas que as regem.
Há um ``ambiente de segurança'' planetário que interconecta o local e o
global, abalando as distinções entre ``dentro'' e ``fora'', tão caras à
literatura das \versal{RI}, mas que soçobram ante uma análise genealógica que
mostra como as práticas políticas e as lutas em torno do governo das
condutas não respeitam as divisões jurídico-políticas das fronteiras
nacionais. A ecopolítica do planeta explicita não haver um espaço da
política (dentro das fronteiras) confrontada com uma ausência da
política no além-fronteiras das chamadas relações internacionais. O
conjunto heterogêneo de táticas do \emph{dispositivo
diplomático-policial} visa produzir \emph{ambientes} que não estão
sujeitos a um ``governo mundial'', mas à conexão entre múltiplas táticas
governamentais que visam assegurar o governo do planeta.

\chapter{O dispositivo monitoramento}

Monitorar é verbo transitivo direto que, em sua definição dicionarizada,
está relacionado à ação de vigilância, verificação e acompanhamento, de
algo ou alguém visando um determinado fim. Não é o elemento central de
uma ação, mas um meio específico que se coaduna a um alvo subsequente.
Pode, também, qualificar um objeto ou uma pessoa, funcionando assim como
substantivo masculino ou feminino: o monitor, a monitora. Nesta forma,
ele também se apresenta com função auxiliar ou meio para um fim. Seja o
monitor do computador ― meio pelo qual se estabelece interface dos
comandos emitidos diretamente pelo teclado ―, seja o monitor de uma
atividade educacional qualquer, que auxilia o professor ou o mestre.
Monitorar é uma ação que serve para controlar; o monitor é algo ou
alguém que serve.

Hoje, o ato de monitorar se tornou tão ordinário quanto os monitores
computo-informacionais presentes na vida das pessoas em seu cotidiano em
diversos lugares e formatos: dos computadores pessoais e
\emph{smartphones} à robótica e aos monitores de circuitos internos de
vigilância patrimonial em residências privadas, nas empresas e nas ruas,
onde se dá a circulação de coisas e pessoas. Somos convocados a
monitorar initerruptamente, assim como somos monitorados de forma
extensiva. Pessoas monitoram suas dietas como meio de obter ganhos
estéticos e/ou de saúde, seguindo recomendações adaptadas a situações
específicas, como o tratamento de uma doença; a moléstia a ser
combatida, ou, mesmo como forma de prevenir o aparecimento de
complicações com a saúde no futuro. Há uma pretensão de antecipação
verificável no ato de monitorar, de precaução antes mesmo de qualquer
prevenção, por meio de práticas de monitoramento. Empresas monitoram
suas atividades regulares como forma de atender exigências de
legislações e/ou protocolos ambientais, bem como para controle contínuo
de desempenho na produção, distribuição e descarte de rejeitos.
Monitorar permite acompanhar uma atividade, conduta ou ambiente sem a
necessidade de interferir em sua pretensa continuidade infinita.

O monitoramento pode ser preventivo (como esforço de antecipação de
situações críticas), profilático (como orientador ou meio de
administração de tratamentos ou modificador de condutas), regenerativo
(como meio para se organizar a recomposição de uma forma desejada), ou
as três coisas ao mesmo tempo. O ato de monitorar é a maneira pela qual
se busca governar elementos e agentes variados, vivos e não vivos,
naturais ou não, visíveis ou microscópicos em terra, mar e ar.

Os satélites governamentais monitoram regiões do planeta para acompanhar
ou corrigir riscos ambientais, preocupados com a produtividade e, por
conseguinte, com a saúde do planeta\footnote{O exemplo mais perene desta
  prática específica de monitoramento governamental no Brasil é o \versal{PRODES}
  (Programa de Monitoramento da Floresta Amazônica Brasileira por
  Satélite), realizado pelo Ministério do Meio Ambiente em parceria com
  o Instituto de Pesquisas Espaciais (Inpe), do Ministério da Ciência e
  Tecnologia. Este programa funciona desde 1988 para ``calcular as taxas
  anuais de desflorestamento, fazer projeções e produzir um banco de
  dados geográfico ao longo do tempo''. A partir de 2004 ele foi
  complementado pelo Deter (Detecção de Desmatamento em Tempo Real),
  ``que mapeia mensalmente as áreas de corte raso e de processo
  progressivo de desmatamento por degradação florestal. Trata-se de um
  levantamento ágil de identificação das áreas de alerta para as ações
  rápidas de controle de desmatamento''.

  \emph{http://www.mma.gov.br/florestas/controle-e-preven\%C3\%A7\%C3\%A3o-do-desmatamento/inpe-monitora-amaz\%C3\%B4nia}}.
Funcionam para registrar e quantificar danos já causados, evitar que
novos danos ocorram, e servem, eventualmente, como referência para punir
quem provoca danos anunciados como ilegais e para selecionar áreas
degradadas que necessitam de restauração. Portanto, não diferencia
urbano, rural e área de preservação, acompanhando pelo alto a extensão
desses espaços. São também utilizados como forma de seguir atividades
aceitas, mas que possuem risco ambiental virtual, como atividades
extrativistas de recursos naturais orgânicos ou minerais. Também nesse
caso, o monitoramento, seja por meio de satélites georeferenciados,
mapeamentos biológicos de espécimes ou de microrganismos, é preventivo,
profilático e/ou restaurativo, ou os três ao mesmo tempo. Sua ação se
define mais pela racionalidade específica e estratégica do ato de
monitorar do que propriamente pelo uso do satélite ou qualquer
equipamento de registro para constituição de bancos de dados.

A coleta de dados para um monitoramento ambiental, por exemplo, lança
mão tanto de dados capturados do espaço por satélites quanto de dados
coletados e fornecidos por agentes das comunidades locais a ser
monitorados. O monitoramento não se resume ao impacto causado por uma
tecnologia específica ou uso de determinados equipamentos como satélites
ou drones, mas trata-se de uma estratégica prática discursiva de coleta
que produz meios variados para fins de governo.

De um ponto ao outro, monitorar é um fluxo, ato contínuo, não tem começo
(ou não se funda num ato inaugural) nem fim (está sempre conectado ou em
interface com um outro fluxo), por isso seu itinerário será sempre
inacabado. E na condição de onda contínua, institui-se apenas como ação
meio em relação a um ambiente determinado. Não é um fim em si, mas serve
a objetivos variados, reprogramáveis; é policresto, mas não como
instrumento. É como uma tecnologia de conduta que ao monitorar serve aos
vários usos e a muitos objetivos condensados em um único objetivo
estratégico por excelência: governar tudo o que é ou está vivo no
planeta e aplicar penalizações sem a necessidade de deslocamento do
ambiente em que se encontra o alvo dessa pena. É por meio do
monitoramento que se recolhe informações para penalizar a céu aberto.
Pune-se uma conduta sem interromper o fluxo de atividade ou de
informação, podendo decidir, por meio de um cálculo de risco, pela
necessidade ou não de retirada do alvo da penalização do ambiente
monitorado. Monitorar é a tecnologia específica de distribuição
horizontal e democratizada das penalizações a céu aberto voltadas para o
bem do planeta. Pretende-se, com ele, intervir para melhorar o ambiente
e modular os viventes que neles se encontram.

É possível descrever esse conjunto polimorfo dos monitoramentos como um
dispositivo característico da sociedade de controle? A despeito de sua
viabilização e desbloqueio possibilitado pelas tecnologias
computo-informacionais, é pertinente pensar no monitoramento como uma
tecnologia política específica das penalizações a céu aberto e da
governamentalidade planetária própria da ecopolítica?

\section{mutações }

Quando Michel Foucault aponta para o panóptico de Jeremy Bentham como um
dispositivo, salta aos seus olhos a pluralidade do uso, a multiplicidade
funcional de tal obra arquitetônica, dessa máquina de vigilância que
serve ao governo disciplinar. E é nas cartas escritas por Bentham na
Rússia czarista (ou a Rússia branca, como ele a chamava) que Foucault
encontrará a multifuncionalidade desse dispositivo da qual se gaba o
próprio utilitarista inglês. Como está redigido no prefácio às cartas
fictícias escritas a partir da observação dos experimentos de seu irmão,
Samuel Bentham, e compiladas posteriormente: ``a moral reformada; os
encargos públicos aliviados; a economia assentada, como deve ser, sobre
uma rocha; o nó górdio da Lei sobre os Pobres não cortado, mas desfeito
― tudo por uma simples ideia de arquitetura!''. E após expor as variadas
aplicações da máquina de vigilância que projetou para uso público, trata
de sublinhar sua substância: ``um novo modo de garantir o poder da mente
sobre a mente, em um grau nunca antes demonstrado; e em um grau
igualmente incomparável, para quem assim o desejar, de garantia contra o
exagero'' (Bentham et al., 2008: 17). Trata-se, para Bentham, de uma
máquina arquitetônica, com variadas funções de aplicabilidade, mas com
uma substância bem definida: um meio de governo da mente de outrem capaz
de produzir frugalidade, moderação; uma forma de combater os exageros,
os desperdícios, a inutilidade. Uma máquina que serve e ensina a servir
pela introjeção de valores morais e normas de conduta.

Esta noção reaparece em diversos momentos de suas análises. É por meio
da noção de ``dispositivo'' que Foucault refuta a hipótese até então
largamente utilizada e difundida pelas teorias críticas do poder
político de que o poder tinha como função principal inibir, conter,
reprimir indivíduos e coletividades. Em seu capítulo sobre o método em
\emph{História da Sexualidade I. A vontade de saber} (Foucault, 1977a:
88-97), o filósofo, antes de expor sua tese sobre a biopolítica, não só
descreve o poder como uma ``determinada situação estratégica'', como
sublinha sua ``positividade'', ou seja, o exercício de poder na medida
em que é capaz de incitar, suscitar, produzir valores e condutas em
relação a algo, a alguém ou em relação a um conjunto determinado de
pessoas e sobre suas ações. Se as tecnologias disciplinares para a
produção de corpos úteis e dóceis correspondem à ativação do dispositivo
de vigilância do panoptismo, o ``fazer viver e deixar morrer'' da gestão
biopolítica da população passa pelo dispositivo da sexualidade como
produtora da vida e das normas que a moldam. A norma, a capacidade de
normalização, é a amálgama que liga disciplina e biopolítica, ou as
relações individualizantes e totalizantes de poder e de governo no que
se convencionou chamar de era moderna.

Esta breve retomada de características bastante conhecidas da analítica
genealógica pretende demarcar um campo de entrada na mutação dos
dispositivos. Não se trata de produzir analogias ou princípios de
desenvolvimento, mas demarcar mutações do dispositivo por meio de
acoplamentos e redimensionamentos nas relações de poder e nas
tecnologias de governo das condutas. Tecnologias de governo são tomadas
aqui como um conjunto de técnicas, de saberes, de estratégias de poder
que não podem ser reduzidas a uma instituição específica, ideologia de
grupo, classe ou casta ou a uma racionalidade científica; mas sim aos
exercícios de poder polimorfos que encontram sua especificidade segundo
os embates que estão inseridos e segundo as forças historicamente
dispostas (Cf. Foucault, 2008a).

A noção de dispositivo é alvo de variados usos e interpretações.
Pretende-se, por uma questão de método, estancar a noção de dispositivo
como elemento analítico-descritivo que dispõe forças em determinada
situação estratégica de disputa e eventual embate. Evita-se, assim, o
trato desta noção como conceito filosófico, seja enquanto linhas gerais
de um diagrama do poder, conforme a interpretação de Gilles Deleuze
(2015: 78-100), seja como forma de positividade capaz de operar
processos de subjetivação e dessubjetivação, que guardam relações com a
noção hegeliana de positividade religiosa, conforme a intepretação de
Giorgio Agamben (2009: 25-51). Não se trata de refutar ou referendar
tais intepretações, porém, procura-se deslocar o uso de uma formação
conceitual para uma utilização analítico-descritiva das relações de
poder como tecnologia de governo das condutas. O alvo é a mutação que se
pode descrever nas formas de vigilância dos corpos e das coletividades
em uma sociedade disciplinar para as práticas modulares de
monitoramento. E como toda mutação, guardando elementos de sua forma
anterior.

Nomear o panóptico como um dispositivo disciplinar é colocá-lo além de
uma máquina arquitetônica que possui uma multiplicidade de funções
operando cálculos de sofrimento, numa polícia das identidades (Miller,
2008: 89-124). Mas, também, tomá-lo como um registro
histórico-intelectual do autoritarismo fundante no capitalismo liberal,
na figura do inspetor Bentham e sua utopia de transformação social pelo
controle (Perrot, 2008: 125-170). Em \emph{Vigiar e punir}, Foucault
escreve sobre o panóptico como um ``dispositivo importante, pois
automatiza e desindividualiza o poder''. Ele não se funda numa pessoa,
mas ``numa certa distribuição concertada dos corpos, das superfícies,
das luzes, dos olhares''. Enfim, ``uma máquina maravilhosa que, a partir
dos desejos mais diversos, fabrica efeitos hegemônicos de poder''
(Foucault, 1977: 167). Nesse sentido, certamente ``é o diagrama de um
mecanismo de poder'', mas, sobretudo, ``é na realidade uma figura de
tecnologia política que se pode e se deve destacar de qualquer uso
específico'' (Ibidem: 170). Tecnologia política que desindividualiza e
automatiza o exercício de poder, capaz de fabricar \emph{efeitos}
hegemônicos de poder e, por isso, referência para se registrar amplas
transformações históricas, econômicas e políticas que darão forma à
sociedade disciplinar. É, portanto, um dispositivo de governo que
funciona como uma máquina de vigilância, mas que condensa, em seu
projeto específico, um conjunto de práticas sociais, conceitos
filosóficos, preceitos morais, uma gama de práticas discursivas que
operam na produção de corpos úteis e dóceis.

Em um escrito de 1978, ainda sobre o dispositivo panóptico, Foucault
sublinha que ``o Panopticon é a utopia de uma sociedade e de um poder
que é, no fundo, a sociedade que atualmente conhecemos ― utopia que
efetivamente se realizou. Vivemos em uma sociedade onde reina o
panoptismo (...). Vigilância permanente sobre os indivíduos por alguém
que exerce sobre eles um poder ― mestre-escola, chefe de oficina,
médico, psiquiatra, diretor de prisão ― e que, enquanto exerce esse
poder, tem a possibilidade tanto de vigiar quanto de constituir, sobre
aqueles que vigia, a respeito deles, um saber'' (Foucault, 2003: 87-88).
Qualquer um que tenha entrado em uma escola, uma fábrica, um hospital,
uma prisão, um hospital psiquiátrico ou um quartel não teria dificuldade
em constatar essas afirmações de Foucault, sobretudo no momento em que
foram escritas, ao final dos anos 1970. Vigilância constante e, por meio
dela, coleta de informações para o funcionamento da norma (parâmetro de
uma normalização dos indivíduos e da sociedade), para estabelecimento do
que é normal ou anormal; para a constituição dos saberes sobre quem é o
indivíduo perigoso que deve ser contido, encarcerado, reformado, curado
e, no limite, eliminado, entregue à morte.

Nos dias de hoje, mesmo que essas tecnologias de vigilância prossigam
operando nos espaços disciplinares austeros de reclusão ou em suas
variantes, a figura do perigoso não se apagou. Está modulada e
estratificada, virtualmente ampliada e democraticamente disposta e
distribuída de maneira horizontal. E a vigilância, ainda que subsista
como técnica de uma estratégia de poder que visa disciplina e extração
de saberes cede lugar aos monitoramentos, que operam por uma
racionalidade diferente da disciplinar.

As tecnologias computo-informacionais não são apenas a ampliação da
capacidade de vigilância, um dispositivo panóptico com maior capacidade
do campo de ação e de amplitude de alcance, que responde aos mesmos
objetivos políticos de vigilância (Gros, 2009). Para Gros, os estados de
violência reordenam os dispositivos de segurança como ``um princípio de
regulação interna e contínuo'' (Ibidem: 249). Dessa maneira, borra-se a
figura do cidadão como correspondente ao sujeito de direito do Estado,
que poderia se tonar alvo dos dispositivos de segurança ao assumir a
face do perigo como delinquente, prostituta, revolucionário, criança
indisciplinada ou anarquista, em favor de um vivente que deve ser
assegurado pela pluralidade dos dispositivos de segurança em todo
planeta face à exposição de suas vulnerabilidades: ``o indivíduo vivo em
lugar do sujeito de direito. Tomado em sua dimensão de vivente, o
indivíduo tem menos direitos ou deveres do que pontos de vulnerabilidade
a superar e capacidades de desenvolvimento a melhorar. Uma só comunidade
de viventes integrados: \emph{continuum} da segurança, do policial ao
militar, \emph{continuum} das ameaças, do risco alimentar ao risco
terrorista, \emph{continuum} da violência da catástrofe natural à guerra
civil, \emph{continuum} da intervenção, da agressão armada contra um
Estado sacripanta ao socorro humanitário, \emph{continuum} das vítimas,
do refugiado à criança maltratada'' (Ibidem: 247). Há, para Gros, uma
ampliação da vigilância e uma diversificação dos alvos dos dispositivos
de segurança, entretanto ele não situa as mutações em sua forma de
atuação ou estratégia como tecnologia de governo.

Por sua vez, os estudos sobre o controle computo-informacional também
costumam tratar a ingerência de governos e empresas sobre as informações
das pessoas nos incontáveis bancos de dados de computadores, em especial
nas chamadas nuvens, como práticas de vigilância. O caso Edward Snowden,
o ex-consultor da Agência de Segurança Nacional (\versal{NSA}), e os vazamentos
viabilizados pelo jornalista australiano Julian Assange, no início dessa
década, mostram isso. Apesar do imenso conhecimento técnico seletivo da
coleta de dados e suas formas de tratamento para espionagem por meio de
algoritmos que permitem buscas específicas, politicamente ainda tratam
essas ações de governo de Estado e de empresas como ações de vigilância
e militarização, tomando o monopólio das empresas e as tecnologias
governamentais de coleta de dados, como uma forma outra de
super-panóptico capaz de esmiuçar o mais íntimo de cada cidadão onde
quer que ele esteja.

Assange chegou a dizer em um programa de televisão no Brasil, de forma
irônica, que um celular é um mecanismo de vigilância que também faz
ligações e, em outra ocasião, que o aparato de coleta de dados dos
governos faz com que cada cidadão tenha um militar embaixo de sua cama.
Afirmações seguramente verdadeiras, mas que limitam a análise, pois, ao
tratar politicamente as tecnologias computo-informacionais, esquece-se
de lidar com a própria política como uma tecnologia que possui uma
racionalidade específica em interface com as demais tecnologias. Nesse
sentido, também, é que os monitoramentos não podem ser reduzidos às
funções de um instrumento ou máquina. Seu clamor por ``maior privacidade
para os fracos e transparência para os poderosos'', restringe as
possibilidades de quebra com a forma de governo que corresponde a essa
tecnologia política.

Mesmo que esse não seja seu objetivo político imediato, as ações de
vazamentos de dados governamentais e a consequente busca de
regulamentações, que visam proteger o cidadão comum da vigilância
governamental e do monopólio empresarial, acabam alimentando o jogo de
atualizações e refinamentos das formas de monitoramento contemporâneas.
Por conseguinte, o caso Snowden e as intervenções públicas de Assange
são constantemente reivindicados por grupos organizados da chamada
sociedade civil como marcos da necessidade de novas regulamentações
governamentais sobre os usos dos bancos de dados. Em reposta ao
\emph{continuum} monitoramento de governos e empresas sobre a vida
eletrônica dos cidadãos, reivindica-se maior controle social sobre essas
ações. Os casos mais conhecidos desse embate são, no Brasil, as
mobilizações bem-sucedidas para criação do Marco Civil da internet, e,
nos \versal{EUA}, o impedimento do projeto de lei no congresso estadunidense que
visava criar um Stop Online Piracy Act (pare com a pirataria on-line),
conhecido como Sopa, e Protect \versal{IP} Act (ato para proteção da propriedade
intelectual), chamado de Pipa, companha que também foi exitosa.

A importância do uso analítico-descritivo da noção de dispositivo em
Michel Foucault endereça ao mapeamento da mutação do dispositivo de
vigilância disciplinar em \emph{dispositivo monitoramento} como
tecnologia de governo das condutas na governamentalidade planetária da
\emph{ecopolítica}. Isso exige acompanhar seus deslocamentos
estratégicos e as formas como respondem a efeitos imprevistos produzidos
pelo funcionamento de um dispositivo específico. Segundo Foucault, ``por
um lado, processo de \emph{sobredeterminação funcional}, pois cada
efeito, positivo ou negativo, desejado ou não, estabelece uma relação de
ressonância ou de contradição com os outros, e exige uma rearticulação,
um reajustamento dos elementos heterogêneos que surgem dispersamente;
por outro lado, processo de perpétuo {preenchimento estratégico}''
(Foucault, 1979: 245, grifos do autor).

A mutação da sociedade disciplinar para a sociedade de controle refaz e
compõe antigos itinerários em formas diversas. Os bancos de dados e os
controles a céu aberto caracterizam-se menos pela sua infinidade
eletrônica e mais pela sua inacabada ampliação, gerando um regime de
dívida infinita e, assim, atualização constante. Esta dívida pode ser
econômica, mas, sobretudo é moral e, portanto, política. Esta política
conecta os viventes e não viventes aos programas (não só os
computo-informacionais), atualizando as tecnologias de poder pastoral na
produção de assujeitamentos e na constituição de uma vida policial,
menos pelo seu traço repressivo e mais por suas buscas por práticas de
zelo (cuidados, \emph{cares}) variadas na produção de uma vida moderada.
Neste sentido, não há mais uma vigilância sobre os que são classificados
como potencialmente perigosos em vistas de uma normalização, mas uma
\emph{normalização do normal} (Passetti, 2004). É esta outra forma de
normalização que opera na busca por segurança global que instala, assim,
um quadro de suspeição e de cuidados generalizados. Não mais a
vigilância contínua sobre os indivíduos construídos como perigosos e
separados em espaços destinados para esse fim como forma regenerá-los,
mas o monitoramento modular de todos os ambientes para virtuais
intervenções de restauro e produção constante de melhorias: ``a
sociedade de controle é uma sociedade de segurança que se pauta num
triângulo formado pela reafirmação da incerteza assentada no
aperfeiçoamento do inacabado --- característica marcante do trabalho
intelectual ---, pela confiança nos programas --- de governo,
organizações e computação --- e pela tolerância como maneira de lidar
com assimetrias e dissemetrias. Funda a era da democracia, da
\emph{convocação à participação} redimensionando a representação por uma
pletora de direitos que suprimem os específicos direitos sociais,
anteriormente conseguidos. Constrói-se uma vida em fluxos regidos
segundo protocolos, uma vida diplomática em que não prepondera mais o
Estado diante do exterior, mas em que se afirma o exterior organizado
segundo o modelo estatal sobre o interior: era do cosmopolitismo, da
hospitalidade aos assemelhados, da crença na paz perpétua, do empírico,
da comparação, do pluralismo e do relativismo cultural. Nem Hegel, nem
Marx, mas era de Kant'' (Ibidem: 154). É diante de tamanho fluxo
político comunicacional que se forma a algaravia indiscernível dos
falantes. O poder pastoral, antes hierarquicamente disposto e muitas
vezes representado pela figura geométrica de uma pirâmide em pé, opera
de forma horizontalizada, distribuindo triangulações entre tolerância,
segurança e confiança nos círculos concêntricos entrecortados e
democráticos de governo. Essa imagem de projeção da horizontalidade
produzida pelos monitoramentos borra a moderna distinção entre
superfície e profundidade.

Esta é uma projeção estratégica triangular de preenchimento de ambientes
governados e funções hierarquicamente definidas muito semelhante aos
contemporâneos esquemas táticos do chamado ``futebol moderno'', no qual
as peças (leia-se jogadores) devem ser moduláveis a uma multiplicidade
de situações para quais os técnicos (leia-se comandantes), e suas
equipes de auxiliares, procuram ter o maior controle possível da disputa
exercendo o monitoramentos de seus adversários, mas sobretudo de seus
comandados. Para isso, há uma variabilidade de equipamentos, desde mapas
de calor (ao estilo militar contemporâneo), ocupação do campo de jogo e
a \versal{GPS} individualizados para registro de movimentação de jogadores, mas,
sobretudo, há uma racionalidade específica que depende principalmente da
capacidade intelectual de jogadores e de técnicos, orientada sempre por
metas ajustáveis e análises racionais de programas. A referência
expositiva ao futebol não é metafórica, trata-se da projeção de
características das tecnologias de governo por meio de uma das
atividades mais praticadas no planeta e um dos negócios mais rentáveis
também, que envolve fatores políticos, econômicos e sociais. Um outro
traço a ser ressaltado, nesse sentido, é a transformação da linguagem.
Cada vez mais técnicos de equipes e jornalistas especializados abandonam
as metáforas militares em favor de conceitos e recursos explicativos
referentes à gestão, controle de riscos e cumprimento de metas, o que
também é praticado em outras modalidades esportivas. O jogo físico, seja
em termos individuais ou coletivos, cede espaço ao governo de
inteligências. Mesmo a moldagem corporal passa a ser submetida aos
monitoramentos médicos e otimização de recursos genéticos dos atletas.
Isso somado ao governo de fatores externos da atividade fim, como gestão
de imagem e cálculos de marketing que envolvem desde o corte de cabelo
até jornadas de \emph{media training}, das quais o sucesso do atleta
(sempre medido em termos de ganho econômico e êxito midiático) está
totalmente dependente.

De modo diferente de Gros, essa análise situa o interesse explícito nas
práticas de resistências e produção de antipoderes e indica o abandono
da concepção de poder em rede e da primazia das vigilâncias. Para esta
análise, as tecnologias de governo em fluxo, operacionalizadas pelos
dispositivos de segurança, produzem no trinômio segurança, tolerância e
confiança, uma inclusividade contínua e inacabada como
institucionalizações, continuamente renovadas pela convocação à
participação que opera os controles a céu aberto na distribuição de
penalizações. Não reduz o monitoramento à função eletrônica das câmeras
de circuito interno ou das tecnologias de informação e comunicação.
Considera a produção de governo das condutas moderadas como
\emph{vida-polícia} na distribuição democrática do pastorado
horizontalizado. Estas ações indicam os traços descritivos da
constituição contemporânea do \emph{dispositivo monitoramento}. Se há
esgotamento do sujeito de direito, como colocado por Gros, o vivente
ainda é malhado no fluxo de negociações de minorias coordenadas pelas
\emph{elites secundárias}, que vão de lideranças indígenas às lideranças
locais de comunidades e associações de bairro ― situação na qual ocupa
lugar estratégico o \emph{intelectual modulador} (que oscila entre a
figura do \emph{ativista} e do \emph{agente social}), e que no fluxo de
produção imaterial da sociedade de controle abarca desde técnicos
formados por centros universitários tidos como de segundo ou terceiro
escalão, até intelectuais que garantem a sua sobrevivência e
continuidade nos chamados centros de excelência ranqueados prioritários
pelas agências financiadoras (privadas e estatais), vinculando suas
pesquisas a minorias específicas, como negros, indígenas, mulheres e
demais variações inclusivas de participação política (Passetti, 2011a:
205-220). Assim, a figura do ativista local ou do agente social se
condensa no \emph{intelectual modulador} como a forma que toma o
reformador na sociedade de controle. Essa configuração estratégica não
apenas faz de cada vivente alvo e agente dos dispositivos de segurança,
segundo o princípio interno regulador dos \emph{estados de violência},
mas transforma cada vivente em agente dos monitoramentos, de maneira que
``intelectuais e populações transitam como \emph{portadores de
liberdades} pelos monitoramentos que os dispõe como forças reativas
atuantes neste \emph{campo de concentração a céu aberto}'' (Ibidem:
220).

Não está em jogo apenas o borramento do sujeito de direito, como mostra
Fédéric Gros, e tampouco estamos lindando com a ampliação da capacidade
de vigilância dos dispositivos de segurança para além das fronteiras do
Estado-Nação. Uma outra racionalidade política se configura na
transformação das tecnologias biopolíticas de controle da população:
``esta é a sociedade dos conservadores moderados, articulados em fluxos
que atraem empresas e seus empregados, \versal{ONG}s, \versal{PPP}s e \versal{OSCIP}s e governos
transterritoriais'' (Passetti, 2007: 31). Considera-se, portanto, essas
metamorfoses não apenas da sociedade disciplinar para uma sociedade de
controle, mas também de uma política planetária que aponta para uma
transformação da biopolítica em \emph{ecopolítica}, na qual os cuidados
com o planeta e o desejo de uma democracia global anunciam novas
conformações dos governos das condutas numa \emph{governamentalidade
planetária}. A disciplina e seu dispositivo de vigilância agem sobre o
corpo vivo moldando-o para extrair energias de sua condição de vivo; os
monitoramentos, por sua vez, produzem ambientes móveis, moduláveis, e
seu funcionamento se intensifica com as convocações à participação,
produzindo condutas variadas e efeitos de governo sobre todos viventes e
não viventes.

O \emph{dispositivo monitoramento} é composto por práticas discursivas
que se servem de postulados científicos para produzir e difundir valores
morais, como a prática do bom governo. Forja condutas de zelo,
acolhimento e intervenções violentas para produzir melhorias, refaz o
campo da política como meio pelo qual se equaciona interesses
individuais e coletivos, funda novas institucionalizações que produzem e
monitoram ambientes governados e governáveis em busca de melhorias
possíveis, elastifica e distende o jogo de representações pela
participação em papéis móveis e plásticos como o do \emph{ativista}, do
\emph{agente social} ou do \emph{empresário responsável} (condensados na
figura do \emph{intelectual modulador}), enfim, anima a vida do
\emph{cidadão-polícia}, que cuida de si, dos outros e do ambiente em que
vive. Responde à convocação contínua, no jogo governo-verdade, na gestão
dos viventes que não mais se restringe ao conjunto da população, mas às
interações dispostas horizontalmente de forma hierárquica entre tudo que
vive e respira no planeta, distribuindo penalizações a céu aberto sem
prescindir dos espaços de confinamento, cuidados e responsabilizações de
vidas dignas e autônomas, respondendo aos cálculos e metas da
racionalidade neoliberal, que operam por medições de capacidades a serem
investidas e virtualidades elásticas de perigo a serem contidas.

Essa é a mutação que ocorre no dispositivo de vigilância panóptico para
formação do \emph{dispositivo monitoramento}. Não se trata de ampliação
e/ou diversificação do dispositivo de vigilância, e também não se
relaciona diretamente com as mutações nos dispositivos de segurança, mas
de uma reconfiguração estratégica das tecnologias de governo das
condutas que se inscrevem num registro da \emph{ecopolítica}, diversa
das formas de regulação e regulamentação do governo da população
próprios da biopolítica, sem se preocupar em buscar origens ou pontos de
ruptura, e considerando que diversos aspectos tanto da disciplina quanto
da biopolítica seguem operando ainda hoje. Trata-se de apresentar
registros contemporâneos do funcionamento deste dispositivo. Colocadas
as referências analíticas dessa mutação, passemos à mesma mutação em
contraste de acoplamento e sobreposição destas práticas com as práticas
de vigilância.

\section{práticas de monitoramento}

O projeto técnico-social-arquitetônico de Jeremy Bentham, inspirado nos
experimentos de seu irmão Samuel, engenheiro de profissão, foi um
retumbante fracasso. Bentham arruinou quase toda a fortuna herdada do
pai em tentativas de fazer vingar politicamente, via o parlamento inglês
e com cópias do projeto traduzidas para o francês e apresentadas à
assembleia francesa, essa panaceia social com pretensões que iam além
dos muros dos cárceres (Bentham et al., 2008). Tal fracasso, amplamente
registrado pela historiografia, não impediu que Michel Foucault
localizasse nas cartas fictícias do projeto utilitarista a emergência de
uma nova tecnologia política, de caráter disciplinar. Na mutação do
dispositivo de vigilância em monitoramentos, duas proveniências são
decisivas para emergência da sociedade de controle e da
governamentalidade planetária: primeiro, os acontecimentos em torno da
\versal{II} Guerra Mundial com as negociações de Estados após seu término e a
emergência simultânea de tecnologias de informação e comunicação para
compilação e armazenamento de dados em larga escala.

Ao escrever sobre a biopolítica Foucault ressalta que essa tecnologia de
poder, emergente no final do século \versal{XVIII}, inverte a equação do poder
soberano: em vez de fazer morrer e deixar viver, ela faz viver e deixa
morrer. Para isto, era necessário um saber que pudesse exercer controle
e estabelecer uma norma para a população. Mais que isso, saberes que
tornassem o fator população tangível. Medicina social, economia política
e estatística darão coerência e convergência às artes de governar
dispersas para formar uma ciência política como ciência do governo, com
estratégias variadas para produzir uma vida regulada pelas normalizações
e regida pelos direitos. Essa tecnologia encontrará seu paroxismo na
Alemanha nazista, assim como a resposta ao que se classificou como
atrocidades do excesso poder do Estado com a formulação de direitos
universais que vazam as fronteiras da nação e do nascimento para
ligarem-se à condição do humano, apontando para uma cidadania
planetária.

Desses direitos, celebrados pela \emph{Carta de São Francisco} de 1945 e
na \emph{Declaração Universal dos Direitos Humanos ―} \versal{DUDH} de 1948, como
fim da guerra entre as nações, derivam as inúmeras recomendações e
normativas produzidas e geridas pela \versal{ONU}, que terão como alvo tudo que é
humano no planeta e, logo em seguida, tudo que é vivo e deve ser
preservado para as futuras gerações. Por um lado, a \versal{DUDH} tem um caráter
de reparação, mas, de outro lado, anuncia novas configurações do
exercício do poder. Agora, uma ameaça passa ser entendida como ameaça a
todo planeta e a capacidade de Estados em destruir este planeta, exibida
ao final da guerra pelas bombas nucleares, deve ser regulada. Se o
Estado foi governamentalizado precisamente na emergência da biopolítica
tendo os direitos como princípios regulatórios do poder estatal sobre os
cidadãos, agora está em jogo uma governamentalidade planetária. Não há
mais dentro e fora e abre-se caminho para as penalizações a céu aberto.
Afinal se há algo até hoje incrustado na cultura é a crença no castigo
como educação, cultura e produção.

Outros saberes vão se constituindo para além da medicina social e dos
dados estatísticos censitários restritos ao território nacional.
Novamente é em meio aos acontecimentos da \versal{II} Guerra Mundial que
encontramos esses saberes capazes de exercer um controle tão extenso e a
céu aberto. Segundo Uehara (2013: 17-87), os cartões perfurados para
controle populacional censitários, utilizados já na década de 1910 para
realização do censo nos \versal{EUA}, foram importados pela Alemanha nazista que
contrata a recém-criada empresa \versal{IBM} para transferir tecnologia dos
cartões perfurados criados por Hollerith para controle das entradas e
mortes nos campos de concentração. Dessa inicial técnica de contagens
modulares e úteis deriva a criação dos contemporâneos bancos de dados
computo-informacionais. Conectados em redes, eles sãocapazes de cruzar,
mesclar e compartilhar dados modulares de fluxos inteligentes em todo
planeta, não apenas das populações, mas de todos os viventes e das
aéreas mais remotas do planeta, possibilitados pela tecnologia do
pós-guerra, os mapeamentos via satélite.

A própria internet, como também mostra Uehara, foi desenvolvida como
estratégia militar da nação que saiu vitoriosa militarmente da \versal{II} Guerra
Mundial para compartilhamento de dados em larga escola. Campos de
concentração, guerra e criação da internet associada à \emph{Carta de
São Francisco} que abre caminho para variabilidade de direitos são
proveniências decisivas para a mutação do dispositivo de vigilância em
\emph{dispositivo monitoramento} e compõem técnicas e saberes
importantes para a limitação do controle disciplinar em espaços fechados
e operacionalizando as modulações dos controles a céu aberto. Do ponto
de vista do Estado, há uma elastificação das fronteiras que reconfigura
os controles estatais sem prescindir da forma Estado como referencial
dos direitos a serem garantidos aos humanos e do Estado como categoria
do pensamento que organiza racionalmente a ação política de grupos
ligados ou não a um Estado específico. A lógica do compartilhamento
derivada dos computadores e da formação dos bancos de dados também se
institui como prática política.

Há outro fracasso inglês, embora de outra dimensão, derivado também do
acontecimento \versal{II} Guerra Mundial, que indica outras precedências dos
monitoramentos: a história do matemático e criptoanalista Alan Turing. A
história de Turing, que ficou mais conhecida após ser contada no cinema,
registra a conduta dos serviços de inteligência em situação de guerra,
durante a \versal{II} Guerra Mundial, para decodificar as mensagens da marinha
alemã ― uma máquina para mensagens de guerra chamada ``Enigma''. Hoje é
amplamente aceito que a máquina criada pela equipe de Turing, a ``Hut
8'', conseguiu desvendar os segredos do exército nazista, e que é um dos
primeiros esforços que levaria a criação dos computadores ― desde os
primeiros, para uso militar, até os utilizados hoje. Além disso, é muito
sugestivo, para além da máquina em si, o esforço político de um país que
liderava a campanha militar dos Aliados na Europa contra o chamado
``Eixo do Mal'' para decodificar mensagens nazistas, o que contribuiria
para a vitória do chamado ``mundo livre''. É de conhecimento público,
também, que Gilles Deleuze (1992) localiza o ponto de virada de uma
sociedade disciplinar para uma sociedade de controle nas transformações
da \versal{II} Guerra Mundial. De maneira diversa, mas no mesmo registro
histórico, encontra-se a emergência da \emph{ecopolítica} no lançamento
das duas ogivas nucleares contra Hiroshima e Nagasaki (Passetti, 2003a:
19-53). Portanto, a criação dos computadores e o período do pós-\versal{II}
Guerra Mundial são proveniências marcantes na emergência do
\emph{dispositivo monitoramento}. Menos pelo chamado ``progresso
técnico-científico'' e mais pelas transformações das tecnologias
políticas de controle que ocorriam naquele momento, das quais,
evidentemente, os computadores e seus bancos de dados compõem fluxos
integrantes: promovem o controle através de fluxos inteligentes
compartilháveis e não mais apenas os corpos são alvos individuais das
disciplinas e da biopolítica como espécie.

Ironia dessa história é que o homem que ajudou a derrotar o horror
nazista, que enviava para os campos de concentração e extermínio os
classificados como anormais, foi entregue à morte no regime democrático
parlamentar liberal por sua suposta anormalidade. Em 1952, Turing passou
por um processo criminal no qual fora acusado de práticas homossexuais,
um crime descrito no código penal inglês da época. Condenado, ele
aceitou trocar a pena de prisão por um ``tratamento'' que envolvia
administração de hormônios e castração química, que o levaria à morte em
1954. Apesar do perdão concedido pela Rainha Elizabeth \versal{II}, em 2013, o
governo inglês sustenta que Turing se suicidou, enquanto sua família
alega que a morte foi um ``acidente'' decorrente da administração dos
remédios. Importa, nesse caso, que Turing mesmo sendo decisivo para a
vitória dos aliados, foi entregue à morte devido a sua identificação
pelas técnicas políticas da época como anormal, um perigoso portador de
uma utilidade excepcional em tempos de guerra; deveria ser afastado do
convívio ou ``curado'' para ser digno de vida no chamado mundo livre
(Leavitt, 2007). O projeto obsessivo de Turing foi um sucesso militar e
de governo, mas não evitou sua ruína pessoal diante do governo das
condutas.

Essas indicações de proveniências e emergências das mutações das
tecnologias disciplinares de poder para novas forças que configuram a
sociedade de controle, permitem traçar as linhas que formam o
\emph{dispositivo monitoramento} e que disparam as práticas de
penalização a céu aberto, ressaltando-se que interessa menos a máquina
em si e mais o conjunto de ações, preceitos morais e filosóficos,
práticas e formas de governo das condutas que se tornam possíveis a
partir dela, suas tecnologias políticas. Para além dos saberes e
tecnologias computo-informacionais há um conjunto de práticas ordinárias
distribuídas segundo os imediatismos e preenchimentos estratégicos do
dispositivo que caracterizam os monitoramentos.

As práticas de monitoramento são operacionalizadas por meio de um
exercício horizontal do poder pastoral que caracteriza a democracia
contemporânea, na qual todos são convocados a ser pastores laicos de si
e dos outros, em busca do bem comum pela realização de seus interesses
particulares e identitários. Essas práticas produzem processos modulares
de subjetivação\footnote{Indicar, por meios de práticas ordinárias, as
  formações subjetivas produzidas no entrechoque dos monitoramentos não
  dever ser confundido com formas microscópicas que formalizam e
  institucionalizam em formas que poderiam ser nomeadas de
  macropolíticas. Aqui se trata de seguir, metodologicamente, uma
  indicação de Foucault a respeito das relações entre sujeito e poder,
  que produzem processos a um só tempo individualizantes e totalizantes.
  Segundo Foucault, ``não se trata de negar a importância das
  instituições na organização das relações de poder. Mas de sugerir que
  é necessário, antes, analisar as instituições a partir das relações de
  poder, e não o inverso; e que o ponto de apoio fundamental destas,
  mesmo que elas incorporem e se cristalizem numa instituição, deve ser
  buscada aquém'' (Foucault, 1995: 245).}, um regime político de
legalismo e institucionalizações, no qual as contestações são inscritas,
de forma constante e inacabada, no campo das disputas políticas que
distribui reconhecimentos aos que buscam protagonismo. Uma prática que
se insinua na interface entre o \emph{ativista} e o \emph{agente
social}, cumprindo o itinerário de indignação-participação
responsável-autonomia, que exige uma conduta pacífica e zelosa,
participativa, inovadora e \emph{resiliente}. Deve-se atender às
convocações admitindo as regras estabelecidas e buscar protagonismo para
influir em suas modificações, que serão sempre inacabadas e
insuficientes, formando um arco de governo que se retroalimenta nas
formas democráticas de controle monitoradas pela participação dentro e
fora das instituições estatais, mas orientadas por uma forma-Estado.

A forma subjetiva mais imediata produzida pelas práticas de
monitoramentos por meio das convocações a participação é a do
\emph{ativista}. Sua ação é, em geral, reativa, intermitente e
inacabada, pois reponde a uma determina decisão governamental, e atua
contra determinada política institucional. Por vezes, salta de uma
``pauta'' a outra na agenda e, não invariavelmente, traduz sua
contestação em termos de reconhecimento na formalização de direitos que
redundam em mais monitoramentos e penalizações. A prática mais comum é a
de busca por criminalização de condutas, movida por movimentos de
minorias, e a reivindicação de regulação estatal de certas práticas,
movida por grupos ambientalistas ou associações civis voltadas para
regulações econômicas e fiscalizações de gastos governamentais. Nesse
sentido, o ativista é tomado aqui em seu sentido amplo, como o que
participa de forma ativa de variados movimentos de contestação,
reivindicação, protestos de rua e/ou, o que é mais comum, por meio das
tecnologias computo-informacional, via redes sociais digitais e petições
on-line dirigidas às autoridades. A figura do ativista é resultante das
convocações à participação, e cumpre um preenchimento estratégico dos
monitoramentos com as pautas do movimento (que podem ser entendidas como
metas), atuando em ações de pressão, como manifestações de rua ou ações
via tecnologias computo-informacionais. Esse ativista é o componente de
enclaves pontuais como o sujeito que compõe o movimento e busca
protagonismo, ou seja, participa do jogo político tentando ocupar, mesmo
que de maneira efêmera, o centro da cena. A figura do ativista se faz de
forma plástica e modular, nesse sentido ele pode tanto desviar desse
roteiro reivindicativo e irromper em ações que afirmem uma recusa
própria à \emph{antipolítica}, quanto se metamorfosear em uma figura
mais plasmada do \emph{agente social}.

O \emph{agente social} também está relacionado às convocações à
participação e atua de forma reativa. No entanto, suas ações se realizam
de forma menos intermitente. Está voltado, sobretudo, ao monitoramento
de políticas já instituídas ou como fiscalizador do cumprimento de leis
específicas ou tratados e convenções internacionais. A maior constância
do \emph{agente social} se deve ao fato de que ele também pode ser
entendido como aquele que acumulou capital humano enquanto ativista e
hoje vincula sua ação política à sua atividade profissional. Não que
isso condicione sua existência, mas também por isso, seu ambiente mais
característico são \versal{ONG}s, Fundações e Institutos de defesa dos direitos
humanos ou mesmo instâncias governamentais e na conexão entre uma e
outra. Por essas características, pode-se situar como estratégica a
definição de agente social de Amartya Sen: ``o agente às vezes é
empregado na literatura sobre economia e teoria dos jogos em referência
a uma pessoa que está agindo em nome de outra (talvez sendo acionada por
um `mandante'), e cujas as realizações devem ser avaliadas à luz dos
objetivos de outra pessoa (o mandante). Estou usando o termo agente não
nesse sentido, mas em sua acepção mais antiga --- e `mais grandiosa' ---
de alguém que age e ocasiona mudança e cujas realizações podem ser
julgadas de acordo com seus próprios valores e objetivos, independente
de as avaliarmos ou não também segundo algum critério externo'' (Sen,
2010: 34). Esta definição ao valorizar o próprio envolvido na ação de
governo se aproxima da conduta produzida pelos monitoramentos. Ela foi
preparada por um dos elaboradores do principal instrumento planetário de
medição do desenvolvimento social e humano, o \versal{IDH} (Índice de
Desenvolvimento Humano), meio pelo qual, no âmbito da \versal{ONU}, se produz \versal{RDH}
(\emph{Relatório de Desenvolvimento Humano}), aplicado desde 1990 e um
dos principais instrumentos das políticas voltadas para investimento em
capital humano, que pontificam que ``as pessoas são a verdadeira riqueza
das nações''. A efetividade discursiva dessa estratégia nas quais atuam
o agente social é reafirmada pela noção de autonomia e busca do bem
comum que a própria \versal{ONU} ressalta em sua apresentação: ``o Relatório de
Desenvolvimento Humano (\versal{RDH}) é reconhecido pelas Nações Unidas como um
exercício intelectual independente e uma importante ferramenta para
aumentar a conscientização sobre o desenvolvimento humano em todo o
mundo. A publicação tem autonomia editorial garantida por uma resolução
da Assembleia Geral das Nações Unidas''\footnote{\versal{PNUD}. ``O que é \versal{RDH}''.
  \emph{http://www.pnud.org.br/IDH/RDH.aspx?indiceAccordion=0\&li=li\_RDH}.}.
A evidente preocupação em mostrar independência de interesses
específicos e de governos de Estado garante a fluidez das práticas.

Nesse jogo de participações, negociações e busca por protagonismo
horizontalizado, cada um é \emph{agente} de penalizações, sobre si e
sobre os outros, atuando de formas variadas nos monitoramentos. Detalhe
ínfimo e monumental: hoje o carcereiro passou a ser denominado de
agente. Não é fortuito que a linguagem predominante é jurídica, seja em
termos de sanções a serem aplicadas, seja em termos de ajuste
circunstanciado ou mesmo recomendações e diretrizes baseadas em tratados
internacionais. Ao final, tudo acaba em algum tribunal, local ou
internacional, e mesmo a racionalidade das ações, medida entre riscos,
perdas e ganhos, segue uma linguagem punitiva. Um exemplo bem evidente é
o reconhecimento pela Suprema Corte Estadunidense da união civil
igualitária entre pessoas do mesmo sexo. O movimento gay estoura nos
\versal{EUA}, politiza-se, busca reconhecimento por meio de direitos e,
finalmente, tem causa ganha diante do tribunal, quando sua potência de
contestação já havia sido pacificada pela política. Essa lógica de
judicialização não apenas amplia as penalizações como também forma, nas
negociações colaborativas e compartilhadas dos monitoramentos, o sujeito
caraterístico do \emph{dispositivo monitoramento}: o
\emph{cidadão-polícia}. Conformado em situações em que visam a melhoria
dos ambientes em que vivem, agindo em função dos investimentos e ganhos
possíveis, segundo um princípio de inteligibilidade econômica em
consonância com a racionalidade neoliberal, suas ações compõem com a
série planetária de investimentos monitorados que são guiados pela
aferição de metas articuladas pelas elites principais e
\emph{secundárias} (Passetti, 2003a), traçando um amplo arco de governo
que vai das ações mais ordinárias de contestação, passando por políticas
nacionais, até tratados internacionais que agregam os direitos humanos,
celebrados pela Carta de São Francisco de 1948, e a série de
recomendações relativas aos cuidados sustentáveis com o planeta.

Nesse jogo se dá a captura de práticas com a descentralização das
decisões, os processos de produção que se pretendem autogeridos, a
valorização da liberdade individual e um maior nível de autonomia nas
ações. A valorização das potencialidades e a flexibilização das formas
de exercício de autoridade resultam de uma crítica contundente à rigidez
das antigas vigilâncias, e formam as características dos controles e
monitoramentos atuais. Pautadas na valorização da autonomia e nas
potencialidades individuais associadas às novas tecnologias
computo-informacionais, mesmo as práticas de contestação acabam se
configurando e formatando o campo de dissiminação da racionalidade
neoliberal, como assinalado por Foucault (2008a). O \emph{homo
economicus}, referência na literatura do liberalismo clássico,
redimensionado pelo neoliberalismo, apresenta-se como o empreendedor de
si, o sujeito que joga no mercado com a sua liberdade, capaz de
capitalizar suas potencialidades pelo autocontrole racional entre risco
e segurança. Faz-se, com isso, tábula rasa do cidadão pleno de direitos
e depositário de conquistas coletivas, sociais e políticas,
metamorfoseando-o em \emph{cidadão-polícia}, cuja liberdade de existir e
de consumir é julgada pelas regras do mercado e regulada pelo
monitoramento de direitos em suas comunidades pela forma de aplicação
elástica da lei penal por meio da formação racional de um quadro
jurídico estatal fiscalizador, que formata a nova forma de intervenção
do Estado, muitas vezes solicitadas pela ação de \emph{ativistas} e
\emph{agentes sociais.}

A preponderância da racionalidade neoliberal, portanto, atravessa,
simultaneamente, as tecnologias de governo e algumas práticas dos
movimentos de contestação. Em ambos há uma recorrência à moldura
jurídica capaz de arbitrar diante de interesses divergentes. No caso dos
movimentos de contestação, ora se recorre à própria Constituição Federal
de 1988 (nos casos específicos do Brasil), ora ao quadro jurídico
planetário, seja como normativa específica relativa às minorias, seja,
no limite, à própria \emph{Declaração Universal dos Direitos Humanos} de
1948 ― conduta que favorece a proliferação de penalizações, novas
institucionalizações e busca por ativação reparadora de direitos para
grupos minoritários, em grande medida produzidos pela busca de
empoderamento como capacidade de autonomia e decisão. Esta leitura
diverge de algumas pesquisas realizadas recentemente no Brasil, que
identificam a predominância da racionalidade neoliberal, nos termos
legados pela análise de Michel Foucault (2008a), como uma supressão da
atividade política em favor de um governo despolitizado e puramente
econômico (Cabanes; Geoges; Rizek; Telles, 2011)\footnote{Optou-se por
  essa referência por ser uma coletânea que congrega diversas pesquisas
  na área das Ciências Sociais que partem dessa hipótese que poderíamos
  chamar de economicista de crítica à racionalidade neoliberal.}.
Entretanto, há uma intensificação da gestão econômica dos viventes como
princípio de inteligibilidade, a força de lei com princípio regulador e
como meio da política contemporânea a convocação à participação que
inclui uma ampla variedade de objetos e móveis de governo: das pessoas
ao ambiente e seus recursos a serem heterogestionados por meio de
monitoramentos contínuos. Compreende-se heterogestão, pela chave crítica
de Pierre-Joseph Proudhon, como regime de gestão política e econômica
que se opõe e impede a possibilidade de práticas mutualistas e
federativas (Cf. Resende \& Passetti, 1986), ou seja, práticas de ação
direta e autogestão. Não há determinação de um sobre o outro, mas um
jogo de trocas compartilhadas que produzem verdades e assujeitamento.
Não se trata apenas de um combate à dominação e à sujeição, mas de
produzir compartilhamentos, dimensionando o amor à obediência, agora,
também, por meio de judicializações.

Há, nesta heterogestão econômica dos viventes, uma ativação e reativação
da política como ciência do governo, conforme indicado por Foucault
(2008a) e enquanto capacidade de condução das condutas do poder
pastoral, agora funcionando de maneira horizontal não mais restrita à
incorporação racional como biopolítica. Se a racionalidade neoliberal
orienta para que o mercado (espaço econômico) seja o princípio de
inteligibilidade da governamentalidade como governo das condutas,
operando como ações sobre ações dentro de um quadro no qual se respeita
as regras do jogo, é a política que comanda as ações \emph{do} e
\emph{no} interior do jogo e da forma democrática para além dos
princípios do regime político. Há uma outra forma de fazer e imaginar
dos comandos políticos e as formas de participação que não se resumem às
transformações meramente econômicas que correspondem às práticas de
monitoramento como tecnologia política.

Assim, os contemporâneos movimentos de contestação (iniciados com o
movimento \emph{antiglobalização} de Seattle, em 1999), que nascem
contra a propagação planetária do neoliberalismo e rompendo com a lógica
de busca por reconhecimento pela conquista de direitos por meio da
participação democrática, como mostra Richard J. F. Day (2005), são em
parte capturados ou se afirmam em explícita recusa da política. A
captura ocorre quando decidem compor, mesmo em enclaves pontuais ou
temáticos, com os novos movimentos sociais (aqueles que lutam por
direitos de minoria, como movimentos feministas ou de direitos civis),
operando assim uma ativação de campos não tradicionalmente reconhecidos
de atuação política. Ao aderirem à disputa ou mesmo engrossarem a luta
por hegemonia, levam com eles temas e pessoas para esses campos
ampliando as práticas de governo até o limite em que emergem novas ou
renovadas institucionalidades.

Esse é o itinerário que se estabelece para a constituição de uma
\emph{nova política}, afinada com a racionalidade neoliberal, porém
construída a partir do campo político contestatório e mesmo
discursivamente de oposição ao neoliberalismo. Dos atuais movimentos de
rua procedentes do \emph{movimento antiglobalização} emerge tanto o
\emph{ingovernável}, que se expressa como \emph{antipolítica}, quanto
práticas capturadas pelas tecnologias de governo, que se expressam como
\emph{nova política.} É deste jogo que emerge tanto as novas formas de
\emph{convocação à participação}, ampliando o campo de atuação da
política como ciência do governo, quanto a recusa em participar,
recolocando a pertinência da \emph{cultura libertária} como
\emph{antipolítica}.

Os monitoramentos, portanto, operam segundo a lógica da racionalidade
neoliberal que operacionaliza um campo de cálculos racionais de riscos
em busca da segurança, sob o pano de fundo da proliferação dos direitos,
maior controle social do Estado e liberdade qualificada como
desenvolvimento do capital humano sob a dominância de um aparelho de
Estado que se orienta por tratados e protocolos internacionais. Nesse
campo, capturam-se contestações antes mesmo delas oferecerem qualquer
risco à ordem estabelecida. Assim, o \emph{dispositivo monitoramento}
funciona por meio de reiterados investimentos na produção de relatórios
capazes de medir e mensurar ações, em especial as relacionadas com
práticas violentas, como forma de monitorar as condutas, sejam
individuais ou institucionais, para acompanhamento e intervenção em
favor do fortalecimento do Estado. Essas medições, expressas em bancos
de dados georeferenciados, podem ser encontradas em relatórios anuais
que alimentam e direcionam ações tanto de governos como da chamada
sociedade civil organizada. No Brasil, destacam-se relatórios como o
\emph{Mapa da Violência Brasil}\footnote{\emph{http://www.mapadaviolencia.org.br/}},
produzido todos os anos e que, em 2014, foi complementado pelo relatório
J\emph{uventude Viva: os jovens do Brasil}\footnote{\emph{http://www.mapadaviolencia.org.br/pdf2014/Mapa2014\_JovensBrasil.pdf}},
ambos produzidos pelo governo federal. Mas há também relatórios de
agências independentes, de atuação internacional, como o relatório anual
da Human Rights Wacth (\versal{HRW}) e o \emph{Relatório da Anistia
Internacional}\footnote{\emph{https://anistia.org.br/direitos-humanos/informes-anuais/o-estado-dos-direitos-humanos-mundo-20142015/}}.
Ambos, em 2015, trouxeram dados sobre a violência que clamavam por maior
atuação do Estado e seus órgãos competentes, assim como sugeriram
reformas institucionais como, por exemplo, a desmilitarização das
polícias no Brasil. Mapas como estes cumprem a função de monitorar
tendências e orientar intervenções governamentais e de organizações
internacionais, assim como da chamada sociedade civil, na medida em que
são também produzidos em parcerias entre governos, agências
internacionais e institutos universitários.

Objetivam produzir moderação tanto do Estado e das suas instituições,
quanto dos grupos políticos que atuam em meio à sociedade civil. Nesse
sentindo, confirmam a relação de monitorar como produção de
subjetividade moderada. Esse trabalho de monitoramento é completado pela
publicização dos dados e resultados que ativam o que se chama de opinião
pública, expressa por meio da mídia, seja a grande mídia, sejam as redes
sociais digitais, promovendo a vulgarização e a orientação dos
resultados em direção à efetivação de intervenções que vão da criação de
leis à necessidade de programas sociais e/ou equipamentos de governo que
por sua vez retroalimentam as novas ações de monitoramento.

Esses relatórios e mapas reúnem um conjunto de características
vinculadas ao \emph{dispositivo monitoramento} na medida em que fomentam
a prática de acompanhamento contínuo de instituições, grupos políticos e
pessoas, que produzem medições e mediações para aferição desses mesmos
acompanhamentos. Por meio dos resultados, professam valores de ``boa
governança'' e condutas governadas tais como responsabilidade e
instrumentos de ``\emph{accountability}''; respeito, defesa de minorias,
solidariedade, enfim, uma série de prescrições direcionadas às pessoas e
às instituições voltadas para as metas de produção da \emph{cultura de
paz} ― principal recomendação da \versal{ONU} para o novo milênio. Os produtores
desses relatórios são, no fluxo entre institutos e universidade, os
\emph{intelectuais moduladores} (Passetti, 2011a). É nesse campo que o
\emph{agente social}, em sua forma já profissionalizada, atua de maneira
contínua e que, por vezes, tem na mobilização de \emph{ativistas} uma
linha auxiliar ou mesmo antagonista em disputa que funciona, ao final,
como o duplo que reforça a importância social das questões levantadas em
relatórios ou tornadas objetos de petições públicas.

No entanto, é na capilaridade mais macabra dos controles citadinos e de
combate à violência, sobre o uso de drogas e pequenas incivilidades, que
os monitoramentos se disseminam com capacidade inclusive de intervenções
violentas autojustificadas, combinando repressão, acolhimento, cuidados
médicos e ambientais. No Brasil, o uso de crack é considerado por
governos e especialistas como uma epidemia. Ao seu redor, além do
combate militar às drogas que ainda se orienta pela utopia de extinção
do tráfico do qual se nutre, há também o gerenciamento disso que é
qualificado como epidemia por meio de programas institucionais de
redução de danos, em geral operados em regime de parceria
público-privada. Assim, forma-se um conjunto amplo de programas de
intervenção, projetos de leis e novas ações de monitoramento para a
produção de condutas governadas, extraindo positividade política e
produção de verdades até mesmo de corpos carcomidos pela miséria e que
vagam pelas zonas mais execráveis da cidade. Programas como estes de
combate ao crack mostram como ações de monitoramento vão muito além de
seus alvos declarados.

No Brasil, um programa pioneiro de cuidados direcionados aos usuários de
crack, os Consultórios de Rua no estado da Bahia\footnote{Sobre a
  história dos Consultórios de Rua no estado da Bahia, ver
  ``Consultórios de Rua: Trajetória e Perspectivas'' in Observatório
  Baiano sobre Substâncias Psicoativas.

  \emph{http://twiki.ufba.br/twiki/bin/view/CetadObserva/Noticia201002051}.},
desdobra-se, em seguida no programa federal Crack, É Possível Vencer, e
chega a São Paulo, numa parceria entre administração municipal, estadual
e federal, junto ao projeto Nova Luz\footnote{Sobre o Programa ``Crack,
  é possível vencer'', coordenado pelo Ministério da Justiça, ver
  ``Crack, É Possível Vencer'' in Ministério da Justiça\emph{.}
  \emph{http://www.justica.gov.br/sua-seguranca/seguranca-publica/programas-1/crack-e-possivel-vencer}.}.
O programa ambulatorial baiano foi tomado como referência nacional de
tratamento contínuo e complementar à repressão policial ao tráfico,
promovendo o controle a céu aberto dos usuários e um monitoramento
contínuo das ruas e da circulação por elas, sob o pano de fundo dos
cuidados médicos em práticas de redução de danos. Na cidade de São
Paulo, onde os números de usuários são vistos como alarmantes, emerge um
outro programa, entre 2013 e 2014, que também ganhará status de
referência de programa de tratamento aos usuários de forma não
repressiva e acolhedora, o programa De Braços Abertos\footnote{O
  programa De Braços Abertos da prefeitura da cidade de São Paulo foi
  instituído em janeiro de 2014 e elaborado pela Secretaria Municipal da
  Saúde. Está regulamentado pelo Decreto Municipal nº. 55.067 de 28 de
  abril de 2014.}. Contando com grande aporte do governo federal, via
Crack, É Possível Vencer, o programa da prefeitura de São Paulo combina
internação, intervenções pontuais da Guarda Civil Metropolitana nos
locais de concentração de uso e abrigo em hotéis deteriorados do centro
da cidade, vinculados a serviços, como varrer as ruas, e aos usuários
que aderem ao programa.

Cunhados em uma retórica de erradicação e renovação, esses programas se
sobrepõem e se retroalimentam no interior das práticas de monitoramentos
como prognósticos do problema seguido pela aferição de metas e novas
intervenções corretivas. Projetos como estes articulam vigilância
eletrônica, com câmeras pelas ruas, saber psiquiátrico, por meio dos
\versal{CAPS}-\versal{AD} (Centro de Atendimento Psicossocial ― Álcool e Drogas), técnicos
em ciências humanas, assistência social e religiosa, negociações com a
polícia repressiva e os agentes dos ilegalismos, governança de agentes
privados e estatais por meio de secretarias, conselhos locais e
subprefeituras, e atuação de \versal{ONG}s e institutos ligados aos
\emph{negócios sociais}\footnote{Sobre o termo \emph{negócios sociais}
  ou \emph{inclusive business,} que no interior desta pesquisa designa
  ações sociais que pretendem produção de lucros e fomento de
  empreendedorismo social, ver Augusto, 2012a.}. Além do monitoramento
como cuidado do ``público-alvo'', produzindo responsabilidades sobre sua
própria condição de assistido e carente, envolvem a participação destes
na aplicação e funcionamento dos programas, a partir do fomento do
empreendedorismo como forma de restaurar e superar a assistência e a
filantropia.

De forma complementar, os agentes privados, estatais e público-privados
são formados \emph{como} e \emph{nas} \emph{elites secundárias},
compondo uma série muito expressiva de trabalhadores formados e
certificados nas universidades com papel decisivo no desenvolvimento e
aplicação destes programas. Desse modo, reitera-se como as práticas de
monitoramento não se restringem ao uso de tecnologias
computo-informacionais, mas também estão direcionadas à aplicação de
tecnologias sociais\footnote{O termo \emph{tecnologias sociais} é aqui
  utilizado em sentido amplo, mas pode também ser tomado no sentido que
  é dado pelos agentes sociais que replicam essas tecnologias de
  contenção. Um exemplo é dado pela Fundação Banco do Brasil e pelo
  Instituto Polis, que definem tecnologia social como ações que
  ``compreendem produtos, técnicas ou metodologias reaplicáveis,
  desenvolvidas na interação com a comunidade e que representem efetivas
  soluções de transformação social. É um conceito que remete para uma
  proposta inovadora de desenvolvimento, considerando a participação
  coletiva no processo de organização, desenvolvimento e implementação.
  Está baseado na disseminação de soluções para problemas voltados a
  demandas de alimentação, educação, energia, habitação, renda, recursos
  hídricos, saúde, meio ambiente, dentre outras. As Tecnologias Sociais
  podem aliar saber popular, organização social e conhecimento
  técnico-científico. Importa essencialmente que sejam efetivas e
  reaplicáveis, propiciando desenvolvimento social em escala''. Fundação
  Banco do Brasil. ``Tecnologia Social para superar a pobreza''.
  \emph{http://www.fbb.org.br/tecnologiasocial/tecnologia-social}.} de
contenção, acompanhamento e empreendedorismo como algo para além da
assistência social direta.

Ademais, o funcionamento de programas como este reitera a lógica
inacabada dos monitoramentos na sociedade de controle. Uma prática local
ou um pequeno projeto citadino por tecnologia social suficiente para se
disseminar como programa a ser generalizado e até mesmo exportado.
Monitorados por instrumentos de aferição de metas, esses programas nunca
chegam ao seu termo, mas tampouco são abolidos, eles se renovam, mudam
de nome ou são transferidos sempre com renovados planos de metas a serem
cumpridos, e assim seguem inacabados sob constantes atualizações. A
atenção, especial às áreas de assistência à saúde e de formação e
capacitação profissional confirmam, mais uma vez, a predominância da
racionalidade neoliberal. Como ressalta um dos principais teóricos do
capital humano, Theodore W. Shultz (1973), o Estado, segundo a
racionalidade neoliberal, deve fomentar o capital humano investindo em
saúde, educação, pesquisa de desenvolvimento e garantir, por meio dos
dispositivos de segurança a propriedade, em especial a propriedade
intelectual. Nesse sentido, não é fortuito que se fale em negócios
sociais como uma forma de desenvolver patentes relacionadas a programas
como estes. Além disso, explicita-se uma outra lógica de tratamento dos
indesejáveis, da criança tida como vulnerável ou perigosa ao usuário de
crack, em vez de retirá-los do convívio social. Estes programas permitem
extrair o máximo de lucratividades políticas e econômicas diretas e
indiretas, promovendo empregabilidades e docilidade política pela adesão
participativa aos programas. Os refratários e tidos como irrecuperáveis
são devolvidos à morte.

As práticas de monitoramento também podem ser verificadas em formas que
se pretendem dar às cidades com a utilização de conceitos moduláveis de
projetos urbanos como de cidades sustentáveis, resilientes ou
inteligentes (\emph{smartcities}\footnote{O conceito de
  \emph{smartcities} é relativamente novo, mas já conta com experiências
  em países da Ásia e em cidades da Alemanha. Entende-se por cidades
  inteligentes organizações urbanas com alta conectividade, programas de
  desenvolvimento sustentável e ampla rede de serviços, empregos,
  moradia e transportes de fácil acesso e modulação. O maior congresso
  sobre o assunto, \emph{Smart City Expo World Congress}
  (\emph{http://www.smartcityexpo.com/en/}), é realizado desde 2010 em
  Barcelona, e conta com mais de 9.000 visitantes, 400 cidades, 3.000
  congressistas e mais de 160 empresas, na maioria de comunicação
  computo-informacional e construção civil. Entre os patrocinadores
  estão a \versal{IBM}, Microsoft, Cisco Systems e Telefônica.}). Na construção
de condições sustentáveis para cidades, combinam-se cuidados e produção
constante de relatórios avaliativos, na mesma lógica de retroalimentação
para melhorias vistas nos relatórios sobre violência. Os investimentos
em sustentabilidade se orientam, segundo os relatórios de órgãos
nacionais e internacionais\footnote{No Brasil, as principais conexões
  para promoção e gestão de cidades sustentáveis estão em torno da \versal{AVINA}
  local (http://www.ecodesenvolvimento.org.br). Esta se conecta via o
  Programa Cidades Sustentáveis (http://www.cidadessustentaveis.org.br),
  realizado pelo Instituto \versal{ETHOS}, à Rede Nossa São Paulo
  (www.nossasaopaulo.org.br) e à Rede Social Brasileira por Cidades
  Justas e Sustentáveis (http://rededecidades.ning.com). O programa e
  toda a rede de promoção recebem apoio da \versal{AVINA} e do Instituto Grapyaú.
  Entretanto, a designação de ``cidades resilientes'' não aparece
  relacionada a nenhuma dessas organizações, apesar de estar anunciada
  como proposta destinada à construção de cidades capazes de absorver
  traumas e desastres naturais e sociais como área da Secretaria de
  Defesa Civil do Ministério da Integração. Defesa Civil. ``Cidades
  resilientes''.
  \emph{http://www.defesacivil.gov.br/cidadesresilientes}.},
cada vez mais voltados para a produção de uma capacidade resiliente das
cidades em relação às grandes catástrofes, sejam elas naturais ou
sociais.

O discurso em torno da capacidade resiliente das cidades por parte das
ações de governo e dos institutos voltados à sustentabilidade apareceu,
inicialmente, em um documento produzido pelo Ministério da Defesa Civil
voltado para atendimento de emergência às chamadas ``catástrofes
naturais''. Todavia, verificou-se, posteriormente, um forte investimento
de institutos e agências internacionais, como o Instituto de Tecnologia
de Massachusetts (\versal{MIT}, na sigla em inglês), dos \versal{EUA}, em disseminar
programas de monitoramento de catástrofes voltados à produção da
capacidade resiliente de Estados em repostas a elas. E isso com
especiais atenções aos chamados países emergentes, na medida em que essa
capacidade resiliente, ainda que se justifiquem pela resposta às
catástrofes ambientais derivadas das alegadas mudanças climáticas,
depende do alcance de metas socioambientais, em especial a redução da
pobreza, orientada pelos \versal{ODM} (Objetivos do Milênio) e hoje
redimensionada como \versal{ODS} (Objetivos do Desenvolvimento Sustentável).
Também, nesse caso, os monitoramentos realizados por um conjunto variado
de agentes são decisivos, tanto para os prognósticos, quanto para a
implementação e acompanhamento de intervenções pontuais e/ou contínuas,
em geral relacionadas à mudança de conduta de populações locais diante
das mudanças ambientais.

Assim, a resilência, que aparece neste caso, primeiramente, vinculada
apenas à capacidade institucional de responder às situações das chamadas
catástrofes ambientais, passa a ser uma conduta produzida, por meio de
discursos educacionais e programas de intervenção especialmente voltados
às pessoas e aos ambientes qualificados como vulneráveis ― abrangendo de
ocupações de encostas aos aglomerados urbanos sub-normais (favelas),
objetivando a produção de condutas resilientes das comunidades
ribeirinhas aos moradores das periferias das grandes cidades. Quanto à
questão ecológica, a ocorrência ou eminência de catástrofes ambientais
se mostra como elemento catalizador poderoso capaz de empastelar
diferenças e produzir uma comoção geral, produzindo um imediato
consenso, com adesão e colaboração comuns, mesmo entre forças que se
mostram inicialmente antagônicas. Aqui, o \emph{cidadão-polícia} governa
a sua conduta e a conduta dos outros com vistas à salvação do ambiente e
do bem comum, convocado a ser copartícipe da gestão de catástrofes, como
um simples pescador ou empresa causadora da catástrofe. Nesse ponto,
destaca-se a característica do \emph{dispositivo monitoramento} como
governo de si, dos outros e dos ambientes: um cuidado como controle e
produção de condutas moderadas de monitoramento contínuo e atualização
constante.

A produção de relatórios e seus correspondentes prognósticos derivam, de
forma direta ou indireta, em intervenções estatais variadas,
especialmente relacionadas à questão da violência. Destas intervenções
diretamente relacionadas ao governo de Estado e como resposta aos
diagnósticos realizados pelos mapas de violência, tomemos duas ações
governamentais: uma do governo do estado e outra da prefeitura de São
Paulo. Elas expõem as formas específicas de funcionamento da
racionalidade neoliberal como operacionalização da produção de
monitoramentos mútuos sempre considerando que estes, apesar de
incluí-las, não se resumem às ações de vigilância eletrônica. Se
entendidos apenas como ampliação tecno-científica de vigilância por meio
de bancos de dados e câmeras de vigilância digitais, perde-se a dimensão
da produção de condutas que os monitoramentos produzem, pois a par do
desenvolvimento tecnológico há o investimento em capital humano próprio
da racionalidade neoliberal. Os relatórios aferem, orientam e
estabelecem metas, mas por meio da produção de subjetividades afeitas
aos controles e disponíveis às convocações à participação se produzem
elementos decisivos para os monitoramentos. Deriva daí a importância, no
campo dos investimentos em educação, como destacado a partir de Shultz
(1973), a valorização de programas culturais que fomentam o que é
genericamente nomeado como cultura popular, um campo tanto de
restauração e produção de laços locais, quanto de investimentos em
oportunidades de empreendedorismo de si.

A prefeitura da cidade de São Paulo mantém um Programa de Valorização de
Ações Culturais (\versal{VAI}) que, conforme anuncia o seu edital, ``apoia
financeiramente, por meio de subsídio, atividades artístico-culturais,
principalmente de jovens ou adultos de baixa renda e de regiões do
Município desprovidas de recursos e equipamentos culturais, e objetiva
estimular a criação, o acesso, a formação e a participação do pequeno
produtor e criador no desenvolvimento cultural da cidade, promover a
inclusão cultural e estimular dinâmicas culturais locais e a criação
artística em geral''\footnote{\versal{EDITAL} Nº 10/2014/\versal{SMC}-\versal{NFC} -- \versal{CHAMAMENTO} \versal{DE}
  \versal{PROJETOS} \versal{DO} \versal{PROGRAMA} \versal{PARA} A \versal{VALORIZAÇÃO} \versal{DE} \versal{INICIATIVAS} \versal{CULTURAIS} --
  \versal{VAI} -- 2015 -- 12ª \versal{EDIÇÃO} 2014-0.351.700-9. No caso do estado de São
  Paulo e em relação às chamadas ``intervenções culturais'', utiliza-se
  como referência \versal{IVJ} (Índice de Vulnerabilidade Juvenil) para localizar
  as áreas ditas mais carentes. Esse indicador, forjado no interior do
  projeto Fábrica de Cultura, da fundação \versal{SEADE}, orientava a geografia
  de instalação dos antigos Telecentros da prefeitura, e agora funciona
  como instrumento para ações culturais e sociais. Sobre o índice e para
  visualização do mapa de vulnerabilidade juvenil na cidade de São
  Paulo, ver Fundação \versal{SEADE}. ``\versal{IVJ}''.
  \emph{http://www.seade.gov.br/produtos/ivj/}.}. O enunciado do edital
deixa clara a função restauradora do programa ao ter como objetivo a
inclusão e estimular a criação cultural local. Isso produz um duplo
efeito: de um lado, convoca os agentes locais a serem agentes sociais do
próprio ambiente em que vivem, produzindo apreço ao local; de outro
lado, fornece capital cultural a ser agregado ao capital social já
acumulado pelo agente, formando assim, uma política de investimento em
capital humano para formar, por meio do investimento cultural,
empreendedores em zonas da cidade qualificadas como vulneráveis ou de
risco.

Ainda que inscrito num plano da política cultural, é apenas uma, dentre
as diversas ações de governo que, por meio do fomento do
empreendedorismo social especialmente voltado aos jovens, responde aos
planos de metas traçados por relatórios nacionais e internacionais,
produzindo o que se nomeia como política social individualizada na
proliferação, entre camadas mais pauperizadas, de uma racionalidade
neoliberal. Nessas formas de política social individualizada, que
correspondem à racionalidade neoliberal, as políticas e programas
voltados sempre com especial atenção para crianças e jovens
constituem-se como amálgama do que pode, num mesmo programa, reunir
política social, econômica e de segurança. As práticas de
sustentabilidade e a capacidade resiliente fazem o papel aglutinador no
cuidado com o ambiente. As campanhas, sempre de caráter cultural,
educativo e formador procuram atingir, antes, a maneira como as pessoas
vivem, trabalham e se relacionam em seus ambientes. A segurança é tomada
como pré-condição para o desenvolvimento social e econômico sustentável
em seus três pilares (econômico, social e ambiental). Desta maneira,
conectam-se pessoas e instituições em diversos níveis por meio de uma
cultura que modula resilientemente as condutas em moderação e obediência
em ambientes tidos como vulneráveis. Funciona também como instrumento de
captura e formação de \emph{elites secundárias} que mantêm interfaces
com as elites principais, ambas hoje especialmente alocadas nas direções
dos \emph{negócios sociais}. Assim, as práticas de monitoramento se
mostram como investimentos de variadas formas em produção de condutas
voltadas para melhorias locais que redundam na contenção de possíveis
contestações derivadas das condições sociais e econômicas que podem
surgir das faixas das populações localizadas pelo critério de renda e
acesso ao consumo.

Ações de governo como estas, somadas ao que já foi exposto a respeito
dos \emph{negócios socais}, produzem uma indistinção entre ações de
Estado e intervenções da chamada sociedade civil organizada, como é
possível notar pela trajetória de empresários eminentes na disseminação
da Responsabilidade Social Empresarial (\versal{RSE}) com seus Institutos e
Fundações e livre trânsito no governo de Estado, em especial a partir de
2002. São os casos, em especial, dos empresários Oded Grajew, Sérgio
Mindlin (já falecido) e Hélio Mattar, todos com passagens por órgãos do
governo e integrantes do Grupo de Instituições, Fundações e Empresas
(\versal{GIFE}) e do Instituto Ethos de Responsabilidade Social\footnote{Segundo
  site oficial, o \versal{GIFE} (Grupo de Institutos, Fundações e Empresas)
  consiste em ``uma organização sem fins lucrativos que reúne associados
  de origem empresarial, familiar, independente ou comunitária, que
  investem em projetos de finalidade pública''. O Grupo nasceu em 1989
  como grupo informal que discutia principalmente filantropia, e se
  institucionalizou em 1995 por meio de 25 organizações, tornando-se
  referência no Brasil em investimento social privado.
  \emph{http://www.gife.org.br/}.}. A suposta oposição entre Estado e
sociedade civil ― como confirmam a observação de Michel Foucault no
final dos anos 1970 (2008) e reiterada por Pierre Bourdieu (2014) no
começo dos anos 1990 ―, é meramente teórica e política; a relação entre
ambas evidencia o \emph{continuum} entre Estado e sociedade civil num
jogo de reforço mútuo na gestão do social.

No entanto, o documento que melhor expressa essa relação entre Estado e
sociedade civil, elite principal e \emph{secundária}, é a pesquisa,
publicada na forma de livro, \emph{Um País Chamado Favela} (Meirelles e
Athayde, 2014). Autodenominada como ``a maior pesquisa já feita sobre a
favela brasileira'', é uma descrição da conduta empreendedora e zelosa
dos habitantes das regiões mais pauperizadas das cidades. Escrito em
linguagem acessível, o livro demonstra em números as décadas que vão da
produção da estabilidade monetária nos governos Itamar Franco e Fernando
Henrique Cardoso até o primeiro governo de Dilma Rousseff, que se seguiu
aos dois governos de Lula da Silva. Período este no qual a favela teria
se modificado, segundo os autores, para ``refavela'': consumidora,
conectada, empreendedora, orgulhosa de seu local, participativa e
desejosa de melhorias. Conforme apresentam: ``76\% das pessoas opinaram
que a vida melhorou no período imediatamente anterior à pesquisa. No
entanto, poucas atribuem esse avanço às políticas públicas ou aos
empregadores. Para 14\%, a família é a principal responsável pela
evolução. Deus é citado por 40\%. Segundo 42\%, a ascensão é resultado
do próprio esforço'' (Ibidem: 32).

A exposição da pesquisa mostra como a produção musical da favela
abandonou o tom de denúncia para expressar desejos de consumo, e como
seus habitantes viram com desconfiança as \emph{jornadas de junho de
2013} (Ibidem: 95-101). Concluem que, assim como os ricos, os habitantes
da favela querem felicidade e mobilidade. Vão além ao sugerir, via
economia baseada no empreendedorismo social, um novo contrato social:
``A economia, de verdade, faz-se a partir de tramas que promovem
benefício compartilhado e felicidade somada. Quando o ganho é
unilateral, trata-se de simples e perversa exploração. E toda
exploração, no curto ou no longo prazo, conduz ao conflito e à dor. O
setor F {[}classificação de renda na qual estão os moradores das
favelas{]} espera, pois, a economia da cortesia, do entendimento, do
esforço honesto pela redação de um novo e justo contrato social''
(Ibidem: 84). Em investimentos deste tipo, nos quais a conduta
empreendedora é associada a um conservadorismo que visa o sucesso
material, ainda que modesto, encontram-se os efeitos do
\emph{dispositivo monitoramento} como governo das condutas e moderador
de possíveis contestações.

Além disso, guardam, ainda, certa relação com os cálculos utilitaristas
de frugalidade e produção da felicidade como forma de evitar a dor na
produção de um contrato social resultante não mais do reconhecimento de
direitos naturais, mas de uma equação econômica do cálculo de
interesses. O fato de a racionalidade neoliberal operar por um princípio
de inteligibilidade econômica não denota nem uma determinação do
econômico sobre o político e o social, nem a produção de um governo
puramente econômico. Trata-se da inscrição numa racionalidade específica
que se insere na política como ciência do governo de fluxos específicos
e de interesses comuns a uma determinada composição populacional. Em
favor dessa descrição está o fato de que, recentemente, um dos
realizadores desta pesquisa, Celso Athayde, apresentou o projeto de
criação do Partido Negro, que conta com mais de 15 mil assinaturas, que
seria o representante oficial dos moradores da favela \footnote{Ana
  Cláudia Guimarães. ``Vem aí o Partido dos Negros'' in \emph{O Estado
  de S. Paulo}.
  \emph{http://blogs.oglobo.globo.com/ancelmo/post/vem-ai-o-partido-dos-negros.html}.}.

Essas descrições selecionadas e expostas aqui poderiam ser desdobradas
em outros programas similares, mas são suficientes para descrever as
características do \emph{dispositivo monitoramento}: a implicação do
divíduo em sua adequada melhoria por meio das convocações a
participação. Ao participar, como \emph{agente social}, empreendedor e
\emph{ativista} que pressiona as autoridades para garantia de direitos,
o portador de direitos busca suas melhorias que se dão pela
elastificação de seu ambiente possível, especialmente no que se refere à
capacidade de consumo e ao nível de influência política em questões de
interesses locais, imediatos e, por vezes, identitários. Assim, produz
capacidade de consumo, do ponto de vista econômico, e empoderamento, do
ponto de vista jurídico político.

Ambos dependem da capacidade de cada um de julgar e fazer escolhas, bem
como de denunciar e ativar penalizações aos que ameaçam o ambiente para
a realização de interesses. Isso leva o agente a investir em si como
capital humano e a zelar pelo outro como eventual adversário ou
potencial risco para as suas conquistas e/ou de sua comunidade real e
virtual. Portanto, ele monitora e avalia suas ações e a dos outros como
forma de ampliar melhorias e garantir um futuro melhor para a sua prole,
inserida nas ``futuras gerações'', e também reitera a importância da
forma democracia vinculada ao Estado como categoria do entendimento para
o funcionamento da comunidade e, nessa projeção de futuro, o inacabado
dos programas e ações, que operam pelo princípio da incerteza.

Como elemento dos monitoramentos enquanto vigilância eletrônica
repressiva destaca-se um outro programa levado adiante pelo governo do
estado de São Paulo. Trata-se do programa de Big Data para a Polícia
Militar e Polícia Civil nomeado como Detecta. Este programa entrou em
funcionamento em abril de 2014, viabilizado por meio de uma parceria
entre governo do estado, prefeitura de Nova Iorque e a empresa de
computadores e softwares Microsoft. Conforme o site da Secretaria de
Segurança Pública de São Paulo: ``Com o \emph{Detecta}, serão emitidos
alarmes automáticos para ajudar no trabalho policial. Isso permite que
\versal{PM}s e policiais civis recebam informações de inteligência sem que seja
necessário operar o sistema a todo momento. Por exemplo, um suspeito
foge em um carro vermelho em que só se sabe parte do número da placa.
Com apenas isso, o sistema pode ser configurado para localizar todos os
veículos com aquele número parcial, da mesma cor, e apresentar essas
localizações em um mapa''\footnote{Secretaria de Segurança Pública do
  Estado de São Paulo. ``\versal{SP} ganha nova etapa do Detecta, sistema de
  monitoramento criminal'', 16/04/2014.

  \emph{http://www.ssp.sp.gov.br/noticia/lenoticia.aspx?id=33930}.}. Não
se trata apenas de um circuito de câmeras, mas de um banco de dados
capaz de cruzar informações e produzir resultados georreferenciados para
ações de busca e apreensão. Este programa que surgiu como ferramenta
para ações de contraterrorismo, após os atentados de 11 de setembro de
2001, logo foi ampliado para utilizações em ações de segurança pública
na cidade de Nova Iorque.

Esses exemplos de ações pontuais que articulam Estado e sociedade civil
em favor da expansão dos monitoramentos podem se multiplicar, seja no
campo da educação e cultura, cuidados com o meio ambiente e promoção da
sustentabilidade, saúde pública e saúde mental ou mesmo relacionados
diretamente ao campo do da segurança pública. No entanto, em todos está
em jogo a questão de regras da produção para um ambiente monitorável e
em que, portanto, seja possível inferir cálculos de riscos. Todos estão
direcionados para a geração de um espaço seguro sob controle, porém, o
mais importante é que a segurança na operação dos monitoramentos tenha a
capacidade de prever e controlar virtuais riscos (muitas vezes
inevitáveis) que uma situação pode suscitar. Nisto está incluída a
capacidade de criar interações e modos de captura que atinjam também os
que contestam a ordem, produzindo modulações tanto de participação
quanto de repressão, como exposto acima na modulação entre o
\emph{ativista} e o \emph{agente social}. Uma situação que demonstra o
hibridismo de contestação, acionamento de leis e expansão de
monitoramentos se mostrou na onda de protestos ocorridos no Brasil entre
junho de 2013, as \emph{jornadas de junho}, e na realização da Copa do
Mundo da \versal{FIFA}.

Há três movimentos, desdobrados a partir das \emph{jornadas de junho},
que apontam para as transformações no campo dos monitoramentos na
interação entre segurança repressiva e mecanismo de captura pela
ampliação e convocação a participação. A primeira referência é a imensa
mobilização das Forças Armadas, da Força de Segurança Nacional e novos
equipamentos para forças policias (militares e civis), como uniformes e
armas ditas não-letais. Esse crescimento se voltou principalmente à
contenção de manifestações de rua que tem na empresa brasileira que
fabrica esse tipo de munição, a Condor Não-Letal, um dos seus maiores
exemplos\footnote{Condor -- tecnologias não letais\emph{.}
  \emph{http://www.condornaoletal.com.br/}.}, reiterando o crescimento do
que Nils Christie (1998) chamou, nos anos 1990, de indústria do controle
do crime. Agora, em níveis planetários e com forte investimento, além
dos equipamentos e armas, em formação de capital humano policial,
explicita os constantes enfretamentos com multidões e recebe treinamento
de técnicas antidistúrbios voltadas para garantia da ordem urbana. Isso
demanda uma colaboração com outras polícias do planeta, como, por
exemplo, o treinamento dado à polícia militar do Rio de Janeiro por
equipes francesas\footnote{``Policiais franceses dividem técnicas com a
  \versal{PM} do Rio para Jogos de 2016''.

  \emph{http://mais.uol.com.br/view/f4d5g8hwtbxo/policiais-franceses-dividem-tecnicas-com-a-pm-do-rio-para-jogos-de-2016-04024D993666E0B15326?types=A\&}}.
Mas esse investimento em capital humano das forças de segurança não se
restringe às técnicas de contenção, em consonância com as metas e
tratados internacionais de respeito aos direitos humanos. Combina
técnicas repressivas com recomendações de moderação, vinda de críticos
acadêmicos aos dispositivos de segurança. Um exemplo singelo: em
paralelo aos treinamentos, como o indicado acima, a Academia de Dom João
\versal{VI}, da Polícia Militar do Estado do Rio de Janeiro, recebeu o filósofo,
também da França, Frederic Gros; o evento foi organizado pelo coronel
Íbis Pereira, em outubro de 2013, logo após as maiores manifestações já
vistas no Brasil; os parceiros do coronel na realização do evento foram
o jornalista Adauto Novaes e a \versal{FLUPP} (Feira Literária das \versal{UPP})\footnote{Folha
  de S. Paulo. ``Evento com filósofo traz alternativa ao 'pé na
  porta''', 20/10/2013.
  \emph{http://www1.folha.uol.com.br/cotidiano/2013/10/1359418-evento-com-filosofo-traz-alternativa-ao-pe-na-porta.shtml}}.

Antes de passar à segunda referência, é importante ainda anotar a
confluência de fatores, além das lucratividades e empregabilidades da
indústria do controle do crime, que cercam essa pequena
palestra-acontecimento a respeito das práticas componentes do
\emph{dispositivo monitoramento}. Ela mostra, em primeiro lugar, a
capacidade de reposta à uma urgência pela convocação à participação na
incorporação do exercício da crítica ao chamar um filósofo
reconhecidamente crítico dos dispositivos de segurança. Cabe lembrar que
as corporações da polícia militar em todo país foram alvo de duras
críticas às suas condutas em relação aos manifestantes que saíram as
ruas a partir de junho de 2013 e, de maneira intermitente, até a
realização da Copa do Mundo em 2014; além da coincidência do filósofo
provir do mesmo país que forneceu, meses antes, treinamento aos mesmos
policiais. A articulação do evento a partir da própria corporação contou
com a parceria de um jornalista, afeito ao iluminismo e também graduado
em filosofia na França, reconhecido pela chamada opinião pública e pela
universidade brasileira. Por fim, o outro parceiro é uma feira literária
voltada para moradores de favelas pacificadas pelo programa de segurança
pública mais festejado do país até então, as Unidades de Polícia
Pacificadora. Assim, temos em um exclusivo evento Estado, sociedade
civil, opinião pública, universidade e população diretamente atingida
pelas políticas de segurança agindo como parceiros na formulação de
práticas, preceitos morais e filosóficos, postulados científicos e alta
tecnologia trabalhando para tornar os monitoramentos toleráveis de cima
a baixo e na horizontal.

Uma segunda referência está no acionamento e criação de leis. Em outubro
de 2013, um delegado utilizou a Lei de Segurança Nacional 7170/1983 ―
outorgada ainda na ditadura civil-militar como adaptação da anterior de
1969, que embasou o período mais duro da repressão durante o governo de
Garrastazu Médici ― para prender dois estudantes durante uma
manifestação na cidade de São Paulo\footnote{Folha de S. Paulo.
  ``Estudante de moda e Humberto Baderna são presos em protesto em \versal{SP}'',
  09/10/2013.
  \emph{http://www1.folha.uol.com.br/cotidiano/2013/10/1353846-estudante-de-moda-e-humberto-baderna-sao-presos-em-protesto-em-sp.shtml}.}.
Em seguida, desencadeou-se uma série de projetos de lei que visavam
responder às situações de manifestações públicas, como as leis estaduais
de proibição das máscaras promulgadas nos estados de São Paulo, Rio de
Janeiro e Minas Gerais. Em seguida, o governo federal, ainda em 2013,
apresentou a revisão da Lei Antiterrorismo (Lei 499/2013), proposta pelo
Ministério da Justiça. A proposta do governo federal não avançou e
transformou-se na \versal{PL} 2016/2015, já aprovada na Câmara dos Deputados e no
Senado Federal e assinada pela presidenta Roussef no início de 2016.
Além desses atos voltados para a criação de leis, também foram criados
mecanismos de intervenção pontuais e vias de participação da sociedade
civil como forma de monitorar as manifestações e capturar os
contestadores ao conjunto de decisões estatais.

A mais significativa das produções institucionais desse período foi a
criação de uma central de inteligência e monitoramento de manifestações
pelo Ministério da Defesa. Além disso, houve um crescimento exponencial
dos gastos em segurança pública para a realização do evento Copa do
Mundo da \versal{FIFA}, que envolveram desde a compra de equipamentos eletrônicos
até a renovação dos equipamentos e fardamento das tropas. O governo do
estado do Rio de Janeiro, seguindo o mesmo itinerário, criou o \versal{CEIV}
(Comissão Especial de Investigação de Atos de Vandalismo em
Manifestações Públicas), visando monitorar e investigar as ações dos
manifestantes identificados como ``vândalos'' e, não raras vezes, como
``terroristas''. A criação dessa central de investigação gerou diversas
críticas, mas se instalou e segue funcionando até quando for
suficiente\footnote{Exame. ``Governo do Rio altera decreto de criação da
  \versal{CEIV}, 26/07/2013.
  \emph{http://exame.abril.com.br/tecnologia/noticias/governo-do-rio-altera-decreto-de-criacao-da-ceiv}.}.

No entanto, nem todas as ações de governo produzidas como resposta aos
atos de rua das \emph{jornadas de junho} foram voltadas para
intervenções investigativas e/ou repressivas. Além de uma série de
decisões que não avançaram, como o Plano Nacional de Mobilidade Urbana
que visava responder às reivindicações do \versal{MPL} (Movimento Passe Livre) em
todo Brasil, o governo federal criou a Política Nacional de Participação
Social, pelo Decreto-Lei 8.284, assinado pela presidente Dilma Rousseff
em 23 de maio de 2014\footnote{\emph{\textbf{.}}
  \emph{http://www.planalto.gov.br/ccivil\_03/\_Ato2011-2014/2014/Decreto/D8243.htm}.}.
Foram tomadas medidas que agem de forma complementar às decisões de
controle e repressão às manifestações, pois visam capturar as ações
políticas e criar formas de monitoramento das reivindicações dos
movimentos sociais que fiquem acolhidas e legitimadas em estruturas,
instituições e procedimentos controlados pelo Estado. São formas pelas
quais as instâncias estatais, em diversos níveis, procuram se aproximar
das novíssimas maneiras de ação política inauguradas pelos movimentos
que não se disciplinavam nos modelos tradicionais (partidos, sindicatos,
movimentos sociais antigos como o \versal{MST} ou \versal{ONG}s) e que se propagaram a
partir das \emph{jornadas de junho} acondicionando-os, de certo modo, ao
que se espera de uma conduta democrática moderada em parâmetros de
participação e protocolos oficializados para a expressão de demandas e
geração de expectativas de cumprimento dessas sempre referenciadas ao
Estado. As indicações dessas respostas governamentais às manifestações
das \emph{jornadas de junho} sugerem um campo de investigação a ser
alargado. O resultado imediato das respostas às \emph{jornadas de junho}
é a virtual identificação de qualquer um como terrorista, indicando a
adoção de um conceito local de ``terrorismo'' com capacidade de
movimentar políticas de segurança após mais de dez anos da declaração de
``guerra ao terror'' pelos \versal{EUA}.

Além disso, essas ações são complementadas por iniciativas de
alargamento da participação da sociedade civil, como a lei que cria a
Política Nacional de Participação Social. Entre os atos de rua, as
decisões governamentais, as variadas participações, institucionais ou
não, e as repostas que os próprios movimentos dão a essas ações,
constitui-se um espaço de indeterminação. Nesse espaço pouco discernível
abre-se um campo de investigação que indica a constituição contemporânea
do \emph{dispositivo monitoramento}. Uma tecnologia de governo que passa
não apenas pelas sobejamente conhecidas táticas biopolíticas, mas pela
conexão inusitada entre rastreamento, registro, controle e
acompanhamento constantes propiciados pelas tecnologias eletrônicas
(vídeos, bancos de dados, acessos parametrados por senhas, equipamentos
biométricos, georreferenciamento) e a permanência das táticas mais
tradicionais repressivas voltadas a prender, cercear e, no limite,
eliminar fisicamente pessoas que provoquem ruído à plena realização das
lucratividades do capital.

Essas táticas combinadas, que vão dos monitoramentos eletrônicos à
repressão continuada por outros meios, são atravessadas por adensadas
práticas de atração e acondicionamento das demandas populares
manifestadas pelo chamado esgotamento das vias tradicionais de
organização política. As práticas de governo não atuariam mais pela
lógica da cessão de direitos somada à repressão aos dissidentes, mas
pela readaptação dessa fórmula com a inclusão ativa dos próprios
governados que são convocados a participar de níveis variados de
cogestão da coisa pública, enquanto são instigados a monitorar seus
concidadãos desde suas práticas mais banais (se respeitam a lei do
silêncio ou se obedecem às regras da coleta seletiva de lixo), até as
denúncias contra suspeitos de crimes (pelos Disque Denúncia) ou, ainda,
pela vigilância da ação de políticos e empresários a fim de expô-los
diante de casos de corrupção. Trata-se de afirmar a governança onde está
em jogo dissipar pelo compartilhamento a relação governante-governado.
Trata-se, enfim, de encontrar formas inacabadas em curso para responder
aos fluxos de poder que devem estar parametrados pelas práticas de
governança.

Recomenda-se, portanto, uma ``conduta moderada'' (Passetti, 2011), de
cada um para o bem-comum e a segurança de todos e de cada um. Nesse
caso, as manifestações de rua, desde 2013, passam a ser julgadas em
termos positivos ou negativos: a revolta contra a propriedade estatal ou
o grande capital é tida como intolerável, e os atos de vandalismo logo
estão ameaçados de ser tipificados como ``terrorismo'', abrindo brechas
para penalizações e monitoramentos ainda mais intensificados (câmeras,
banco de dados, prisões preventivas). Por sua vez, as demonstrações de
descontentamento que seguem os parâmetros da ``conduta democrática
esperada'' são celebradas por autoridades, intelectuais e pela grande
mídia como provas da maturidade política democrática dos brasileiros.

Assim, novos canais de participação, excitando as práticas de democracia
participativa acentuadas desde a presidência de Lula da Silva
(2003-2010), condicionam para uma moderada expressão de descontentamento
voltada à melhoria e reforma da política, da economia, das questões
urbanas etc. Todos são considerados copartícipes do governo de tudo e
todos, numa espécie de reforço da utopia democrática definida como o
governo de todos por todos. Nesse contexto, ganhou fôlego e difusão o
conceito de ``governança'' que, como explica Deneault (2013), é uma
noção procedente do mundo empresarial dos anos 1970 e 1980 que visava
aumentar os índices de produtividade e eficiência corporativas a partir
de uma relativa dissolução das hierarquias no ambiente de trabalho
herdadas do fordismo em que o trabalho intelectual (concentrado nos
gestores) era separado do manual (próprio dos geridos). Para o autor, a
noção de governança procura despolitizar as relações de mando e
obediência, dando a percepção de que todos, juntos, governam a empresa.
Transposta para o campo da política, a expressão é amplamente difundida
pelos discursos governamentais do Estado, sempre indicando um princípio
de coparticipação no governo de tudo e de todos, sem evidenciar quais
valores e quais forças sociais e econômicas efetivamente governam essa
democracia.

A produção dessa interação entre aqueles que contestam, as ações de
governo e o acionamento de monitoramentos não está no campo da
consciência ou da cooptação, mas se dá como situação estratégica
produzida no choque de forças. Um momento assim se mostrou no caso
composto por 23 mandatos de prisão temporária e dois de busca e
apreensão, cumpridos no dia 12 de julho de 2014. Esta ação quase
produziu um incidente diplomático. Uma das pessoas autuadas por esse
mandato refugiou-se na embaixada do Uruguai, no Rio de Janeiro, para
pedir asilo político, alegando perseguição do governo brasileiro. Pedido
prontamente negado pelo governo uruguaio que alegou reconhecer o Brasil
como um Estado democrático de direito, observador das recomendações
internacionais de respeito às liberdades individuais, como liberdade de
expressão e manifestação política, assim como o direito à ampla defesa e
presunção de inocência\footnote{Folha de S. Paulo. ``Justiça do Rio
  solta 5 ativistas, mas torna réus 23 acusados de violência'',
  18/07/2014.
  \emph{http://www1.folha.uol.com.br/poder/2014/07/1488328-ministerio-publico-denuncia-23-por-formacao-de-quadrilha-armada-no-rio.shtml}.
  Folha de S. Paulo. ``Ativista pede asilo no Consulado do Uruguai no
  Rio e foge após negativa''.
  \emph{http://www1.folha.uol.com.br/poder/2014/07/1489031-procurada-advogada-de-ativistas-pede-asilo-no-consulado-do-uruguai-no-rio.shtml}.}.
De outro lado, o governo federal não hesitou em acionar as Forças
Armadas aprofundando as prerrogativas para uso interno dos militares
para a chamada ``garantia da lei e ordem'' regulamentada pela Lei
Complementar 136/2010. A presença dos militares foi apresentada
internacionalmente como garantia da segurança para a realização dos
eventos da \versal{FIFA} (a Copa das Confederações, em 2013, e a Copa do Mundo,
em 2014) e das Olimpíadas e Paralimpíadas (em 2016) no interior de suas
fronteiras\footnote{Globo. ``Forças Armadas exibe tropa e começa
  operação para a Copa no \versal{DF}''.
  \emph{http://globoesporte.globo.com/df/copa-do-mundo/noticia/2014/06/forcas-armadas-exibe-tropa-e-comeca-operacao-para-copa-no-df.html}.
  O Estado de S. Paulo. ``Mobilização das Forças Armadas para a Copa
  começa na sexta-feira''.
  \emph{http://esportes.estadao.com.br/noticias/futebol,copa-do-mundo,mobilizacao-das-forcas-armadas-para-a-copa-comeca-na-sexta-feira,1507850}.},
mesmo que as tropas já estivessem sendo largamente utilizadas nos
processos de pacificação das favelas cariocas. Como destacamentos de
apoio na implementação de \versal{UPP}s (Unidades de Polícia Pacificadora), os
dois episódios reunidos apontam para uma ampliação da ação combinada
entre repressão militarizada e monitoramentos.

Descrever como se produz essa interação, esse entrechoque de governo de
Estado com organizações da sociedade civil, leva-nos a compreender como
se formam, modificam e articulam os discursos em torno da segurança, com
especial atenção ao impasse que a situação produziu para o Estado e para
os grupos que o contestam. De um lado, o Estado, no cumprimento de seu
dever de manutenção da ordem pública e defesa da propriedade privada e
estatal, colocou-se como agente violador de direitos constitucionais e
recomendações internacionais de garantia de direitos humanos, gerando
variadas críticas à ação do Estado, expressas sobretudo na imprensa
diária. De outro lado, os manifestantes se viram entre defender seus
direitos constitucionais e reivindicar, perante o Estado, o respeito aos
direitos humanos e liberdades fundamentais, envolvidos em manifestações
que só se mostraram capazes de chamar a atenção na medida em que
romperam com os limites da lei (pela prática da desobediência civil),
promovendo confrontos com as forças policiais e ataques diretos à
propriedade. Nesse entrechoque entre ações de contenção, formalizações
jurídico-políticas e movimentos de contestação se instalam os
monitoramentos, que podem ser reivindicados tantos para moderar ações de
governo, bem como são largamente utilizados contra os que contestam a
ordem.

Nas novas tecnologias de governo os monitoramentos funcionam de forma
combinada e contínua na produção de relatórios-prognósticos,
estabelecimento e aferição de metas, intervenções de inclusão e
vigilância, produzindo um campo de governo das condutas que opera na
produção de assujeitamentos e capacidade de controle a céu aberto. Elas
expõem o \emph{dispositivo monitoramento} como produtor de
contenção/dissuasão, mais do que a vigilância em si, pois esses fluxos
computo-informacionais e georreferenciados do \emph{dispositivo
monitoramento} visam, de um lado, aumentar o cálculo de risco de quem
pretende infringir a lei e, de outro lado, fornecer elementos precisos
para melhor intervenção e punição por parte do Estado, que passa a
investir em punir mais e melhor, contando, em muitos casos, com a
participação dos próprios alvos dos monitoramentos.

Deste entrechoque, porém, também se anunciam novas configurações de
ações políticas que produzem estilos de recusa inéditos, legados da
\emph{cultura libertária} como \emph{antipolítica}, e constituição de
novas formas de participação que, mesmo contestando a ordem vigente,
indicam novas institucionalizações, chamadas aqui de \emph{nova
política}.

O \emph{dispositivo monitoramento} é composto por práticas discursivas
que se servem de postulados científicos para produzir e difundir valores
morais, como a prática do bom governo. Forja condutas de zelo,
acolhimento e intervenções violentas para produzir melhorias, refaz o
campo da política como meio pelo qual se equaciona interesses
individuais, coletivos e identitários, funda novas institucionalizações
que produzem e monitoram ambientes governados e governáveis em busca de
melhorias possíveis, elastifica e distende o jogo de representações em
papéis móveis e plásticos como o do \emph{ativista}, do \emph{agente
social} ou do \emph{empresário responsável}, enfim, anima a vida do
divíduo como \emph{cidadão-polícia}, que cuida de si, dos outros e do
ambiente em que vive. Responde ao imeditao, no jogo de governo-verdade,
na gestão dos viventes que não mais se restringe ao conjunto da
população, mas às interações dispostas horizontalmente de forma
hierárquica entre tudo que vive e respira no planeta, distribuindo
penalizações a céu aberto sem prescindir dos espaços de confinamento,
cuidados e responsabilizações de vidas dignas e autônomas, respondendo
aos cálculos e metas da racionalidade neoliberal, que operam por
medições de capacidades a serem investidas e virtualidades de perigo a
serem contidas em ambientes.

\chapter{Penalizações a céu aberto, nova política e antipolítica}

A polícia é a forma ordinária de exercício da política. Conforme mostrou
Michel Foucault ao analisar a Razão de Estado, a polícia emerge como uma
forma específica, uma tecnologia própria do Estado. Destarte que a
questão contemporânea das penalizações a céu aberto está ligada o
crescimento da atividade policial, uma elastificação desta técnica
característica do Estado moderno. Considera-se polícia não como um
instrumento do poder judiciário, mas como uma forma determinada de se
fazer política, uma tecnologia que funciona segundo uma racionalidade
específica e promove o vigor do Estado, não pelo cumprimento rigoroso da
lei, mas pela vulgarização das práticas de normalização.

Nesse sentido, as penalizações a céu aberto não se referem
especialmente, embora estejam em relação, aos processos de
judicialização da vida e/ou da política, que são alvo de diversas
análises críticas hoje. Ela está conectada a uma configuração
contemporânea das tecnologias de subjetivação que produzem o
cidadão-polícia, liga, condicionalmente, o exercício moderno da vida
cidadã ao domínio das tecnologias de governo policial que, no limite,
apelam à intervenção do Estado. Trata-se de uma tecnologia produtora de
assujeitamentos. O crescimento e a elastificação da atividade policial,
hoje, encontra contestações, mas é precisamente na tensão entre grupos
que buscam reconhecimento ou tentam limitar a atividade do Estado e as
formas de atuação do governo, que a atividade policial se apoia. Isso se
dá pela triangulação entre Estado, demanda legal por reconhecimento e a
prática de polícia, que hoje se encontram, agora, além e aquém da
existência de um aparelho repressivo estatal.

Em sua genealogia do Estado, realizada no curso ``Segurança, território,
população'', Michel Foucault (2008) conclui que o conjunto de práticas e
saberes que tornou possível tal emergência não está ligado a uma
estatização da sociedade, mas à governamentalização do Estado. Isto será
possível pela emergência de um novo objeto de governo, a população. Este
acontecimento funda um conjunto complexo e coerente de práticas e
saberes que dão forma coesa as variadas artes de governar existentes até
o momento, fundando uma racionalidade específica. No decorrer dessa
instauração o saber policial tem um papel decisivo e realiza, junto à
Razão de Estado e ao dispositivo diplomático-militar, a forma do Estado
como o conhecemos. No entanto, diverso do que é colocado ainda hoje, a
atividade policial não se reduz a função repressiva e de contenção de
forças, mas a um amplo escopo de cuidados com a população que se traduz
como manifestação ordinária da racionalidade estatal com um amplo leque
de intervenções diretas. Como define na última aula desse curso de 1978:
``a polícia é a governamentalidade direta do soberano como soberano.
Digamos ainda que a polícia é o golpe de Estado permanente. É o golpe de
Estado permanente que vai se exercer, que vai agir em nome e em função
dos princípios da sua racionalidade própria, sem ter de se moldar ou se
modelar pelas regras de justiça que foram dadas por outro lado.
Específica, portanto, em seu funcionamento e em seu princípio primeiro,
a polícia também deve sê-lo nas modalidades da sua intervenção.''
(Foucault, 2008: 457).

Dos dispositivos de segurança do Estado, a polícia, desde sua
emergência, apresentava-se como o mais plástico; o que lhe permite se
ocupar do miúdo, do ordinário da vida população para garantir o
esplendor do Estado. Ao passo que as leis e a justiça, menos plásticas e
mais modelares, se ocupam dos grandes atos, das coisas que se julga
importantes. Essa distinção, entre a política como razão de Estado e as
atividades de governo como atividade policial, corresponde à
diferenciação entre \emph{Die Politik}, uma tarefa negativa, que
``consiste, para o Estado, em se defrontar com seus inimigos, tanto
internos como externos'', e \emph{Polizei}, tarefa positiva, que
``consiste em favorecer ao mesmo tempo a vida dos cidadãos e o vigor do
Estado'' (Foucault, 2003: 383). Distinção que Foucault retira dos
manuais de polícia prussianos do final do século \versal{XVIII}, que orienta o
funcionamento dos dispositivos estatais para além de sua configuração
institucional.

Ao longo dos séculos \versal{XIX} e \versal{XX} essas intervenções específicas das
tecnologias de governo policial cindiram-se em duas, no que se refere à
atividade estatal, como resposta às disputas em torno do Estado e como
funcionalidade adaptativa dos dispositivos de segurança. Essa cisão não
se traduz como oposição, mas como complementariedade. Assim, os cuidados
policiais são divididos em segurança repressiva, realizados por
destacamentos de contenção especializados e armados que levam o nome de
polícia propriamente dito, e em cuidados com seguridade social, por meio
de instituições e programas sociais voltadas à gestão e majoração da
saúde da população, denominadas por políticas públicas ou, mais
precisamente, políticas sociais. O auge dessa dupla intervenção do
dispositivo se dará nos períodos subsequentes da Segunda Guerra Mundial,
na combinação entre repressão e cuidados regulatórios do Estado de
Bem-Estar Social. A distinção dessas duas funções de um mesmo
dispositivo de segurança aparece de forma mais clara em língua inglesa.
Nela encontramos, de um lado, a \emph{policy}, que se refere a um amplo
conjunto de atividades regulatórias e intervenções específicas e não
normatizadas voltadas para majoração da saúde da população como
normalização, ou seja, políticas sociais voltadas para saúde e educação,
traduzida para o português, comumente, por políticas públicas. E, de
outro lado, complementarmente, a \emph{police}, como polícia repressiva
e de contenção para manutenção da ordem e captura seletiva de sujeitos
tidos como perigosos, também com vista à normalização dos sujeitos
individualmente e da população, coletivamente, extirpando os elementos
ameaçadores de seu seio ou isolando-os em instituições austeras.

No auge dessa forma dual de exercício do poder de polícia, precisamente
na metade do século \versal{XX}, ela começa a sofrer mutações significativas que
modularam a tecnologia policial para a forma como a conhecemos hoje. Uma
primeira referência é Carta de São Francisco, de 1948, que prescreve uma
série de recomendações norteadoras das práticas policiais, independente
dos controles modelares fronteiriços do Estado-Nação. Em reposta aos
horrores da guerra entre as nações e à violência sistematizada do
fascismo e do nazismo, que levaram as tecnologias bipolíticas ao seu
paroxismo, a \versal{DUDH} estabelecerá referenciais norteadores e não
impositivos que sugerem uma cidadania planetária vinculada não mais ao
nascimento, à nação, mas a condição de humano no planeta.

Esta é uma proveniência decisiva da atual cidadania policial que se
projeta planetariamente, ainda que seu desbloqueio amplo só tenha
ocorrido no início dos anos 1990, com o final da chamada Guerra Fria.
Não é fortuito que os índices de aferição de metas, as tecnologias
sociais voltadas para melhorias possíveis e os cuidados ambientais, as
técnicas de gestão planetária e as práticas de monitoramento se refiram
hoje, quase que exclusivamente, aos tratados, protocolos e normativas da
\versal{ONU}, sem tomá-los, necessariamente como modelos. Isso se dá num amplo
arco de governo que vai dos cuidados e controles com recursos naturais,
passando por ajudas emergenciais aos sujeitos classificados como
vulneráveis, até intervenções humanitárias realizadas por enclaves de
exércitos nacionais sob o comando do Conselho de Segurança da \versal{ONU}.
Assim, cuidados, controles e contenções se espargem, a partir de um
referencial modularmente comum, para além de um contingente populacional
determinado, como governamentalidade planetária altamente investida de
tecnologias policiais e de práticas de monitoramentos. Mais do que isso,
como dispositivo de segurança, a tecnologia policial não se restringe ao
controle e cuidado dos cidadãos e do conjunto populacional no meio em
que se encontram, mas passam a operar como produtoras de ambientes
seguros, ou cogeridos em direção à produção de segurança em sentido
amplo e polimorfo, como contenção da desordem e como seguridade
orientada pela preservação da vida do planeta e das futuras gerações que
irão habitá-lo. Vê-se como a ambiguidade entre segurança e seguridade dá
lugar ao que alguns teóricos das Relações Internacionais chamam de
processos de securitização, processos contemporâneos e transterritoriais
que transformam objetos antes alvos de cuidados especiais e assistência
estatal em objetos a serem segurados e defendidos, no limite, com o uso
sistemático de violência (Buzan \& Hansen, 2014).

Gradualmente a distinção, que se pode estabelecer com uso da língua
inglesa, entre \emph{policy} e \emph{police}, se turva e torna, sob
condições absolutamente outras, a formar uma nova versão do que os
prussianos chamavam de \emph{polizeiwissenschaft}, que ``é ao mesmo
tempo uma arte de governar e um método para analisar uma população
vivendo em um determinado território'' (Foucault, 2003: 384). Enquanto
tecnologia que possui no poder pastoral uma proveniência decisiva, a
polícia segue como arte de governar e método de análise, mas não mais
voltada para o mesmo objeto, a população como corpo-espécie, mas para o
conjunto dos viventes como \emph{corpo-planeta}, o que não significa
apenas uma ampliação do campo de atuação ou mesmo uma ampliação do
objeto-alvo, mas uma reconfiguração, uma metamorfose nas variadas formas
de intervenção, guardando, entretanto, proximidades com as tecnologias
policiais tal como elas se estabeleceram em sua emergência.

Trata-se de produzir ambientes seguros (reais e virtuais, naturais e
artificias, dentro e fora do planeta Terra, enfim, eliminando essas
diferenciações na contínua produção de segurança), aos quais corresponde
uma forma de subjetivação sobredeterminada pelo \emph{dispositivo
monitoramento} que é o cidadão-polícia, alvo e partícipe das
penalizações a céu aberto. Estas, sem abrirem mão do circuito do sistema
penal e suas instituições austeras, se espargem como prática ordinária e
comum de empresas, cidadãos e grupos sociais como \versal{ONG}s e Institutos
ligados à empresas e setores da chamada sociedade civil organizada.
Hoje, temos um crescimento exponencial da polícia como efetivo de
contenção e repressão, com efetivos de policiais cada vez maiores e uma
gigantesca indústria do controle do crime que investe fortemente em
contingente de seguranças privados e, combinado a esse crescimento, uma
dilatação dos programas estatais, privados e público-privados voltados
aos cuidados de pessoas consideradas vulneráveis ou em situação de risco
que são convocadas a participar desses mesmos programas para a produção
de ambientes seguros.

\section{cidadão-polícia}

Uma política voltada para contenção de jovens pegos por condutas
classificadas pelo Estatuto da Criança e do Adolescente --- \versal{ECA} como ato
infracional (tudo que se chama crime no código penal) dá a dimensão do
amplo arco de práticas policiais que fazem funcionar as penalizações a
céu aberto por meio do \emph{dispositivo monitoramento}. Nas políticas
voltadas à contensão de jovens encontra-se a combinação de uma série de
controles como cuidados levados a diante por \versal{ONG}s, Institutos, setores
da chamada sociedade civil organizada e uma série de técnicos em
humanidades, com políticas de repressão que não abrem mão da ativação de
internações em instituições austeras e da violência caraterística dos
agentes policiais do Estado. No fluxo entre sistema criminal,
prisão-prédio, cuidados racionalizados e convocações a participação,
instalam-se as penalizações a céu aberto, forma contemporânea das
tecnologias policiais.

A captura de jovens em ato chamado infracional é feita pela polícia. No
Brasil, esta é a principal função do policiamento militar ostensivo de
rua. O itinerário da internação provisória é este: polícia (que não faz
parte do sistema judiciário e defende a ordem o apresenta ao promotor)
--- promotor ou Ministério Público (não faz parte do sistema judiciário
e atua no caso de jovens considerados infratores como Guardião do Bem
Comum e da Ordem Pública e apresenta representação ao juiz) --- juiz
(faz parte do sistema judiciário e acolhe a representação do promotor e
decreta a internação provisória). A administração da aplicação da medida
socioeducativa, em geral ativada no momento seguinte à detenção
implicando envio à internação para uma \versal{UIP} (Unidade de Internação
Provisória), é responsabilidade da Secretaria de Justiça dos governos
dos estados. Em paralelo, esse jovem será submetido a julgamento em
alguma Vara da Infância e da Juventude (\versal{VIJ}), que estabelecerá, conforme
o artigo 112 do \versal{ECA}, uma medida socioeducativa segundo a gradação
estabelecida pelo referido artigo: ``I. Advertência; \versal{II}. Obrigação de
reparar o dano; \versal{III}. Prestação de serviço à comunidade; \versal{IV}. Liberdade
assistida; V. Inserção ao regime de semiliberdade; \versal{VI}. Internação em
estabelecimento educativo; \versal{VII}. Qualquer uma das previstas no artigo
101, \versal{I} ao \versal{\versal{VI}}''\footnote{Estatuto da Criança e do Adolescente. Lei nº
  8.069, de 13 de julho de 1990.
  \emph{http://www.planalto.gov.br/ccivil\_03/leis/L8069Compilado.htm}}.

Após a internação provisória, que pode durar meses a despeito da
determinação legal em não ultrapassar 45 dias, ele é enquadrado em umas
das medidas listadas pela lei. A prática comum entre juízes da Vara de
Infância e Juventude é determinar a internação, embora o jovem possa ser
encaminhado para outra medida como semiliberdade ou liberdade assistida.
A administração das medidas socioeducativas em meio aberto é
responsabilidade dos municípios, segundo o processo de regionalização e
descentralização administrativa estabelecida pelas reformas contidas na
Constituição Federal de 1988. Devido a isso, as medidas em meio aberto
são operacionalizadas por meio de parcerias público-privadas (não
necessariamente submetidas e regulamentadas pela Lei 11.079/2004) que
envolve \versal{ONG}s, prefeituras e financiamento de Fundações e Institutos
empresariais. Essas parcerias viabilizam programas sociais de inclusão e
capacitação profissional recomendados, incentivados e regulados pelo
Conselho Nacional dos Direitos da Criança e do Adolescente ---\versal{CONANDA} e
são monitorados pelo Sistema Nacional de Atendimento Socioeducativo ---
\versal{SINASE}, criado em 2006, mas instituído como política federal de Estado
pela Lei Ordinária 12.594/2012. Esses controles federais visam
consolidar as recomendações de atenção aos Direitos Humanos emitidas
pela \versal{ONU} no que diz respeito à proteção integral do Estado sobre
crianças e adolescentes. Ocorre que esses programas locais,
administrados por prefeituras, operacionalizados por \versal{ONG}s e financiados
por Institutos e Fundações, produzem um envolvimento dos próprios
apenados no cumprimento de suas penas nomeadas como medidas
socioeducativas.

Segundo o mapeamento em pesquisa realizada entre 2006 e 2009 sobre o
programa Pró-Menino da Fundação Telefônica (Augusto, 2013), a convocação
à participação leva os jovens que cumprem penas em meio aberto a se
tornarem monitores das atividades atribuídas aos novos ingressos no
mesmo programa, como aplicação de questionários para avalição e aferição
de metas. Assim, desenha-se um arco completo de governo como um circuito
que se retroalimenta da captura de um sujeito considerado infrator pela
polícia de contenção e repressão à participação do próprio apenado nas
práticas policiais de controle como cuidado dos programas de atendimento
e assistência social. Assim, a pena se torna algo contínuo e, ao mesmo
tempo, inacabado. O sujeito pego em chamado ato infracional, ou mesmo os
que não são pegos mas, por serem classificados como vulneráveis ou em
situação de risco, participam desses programas, encontram-se em relação
de dívida que nunca é quitada, mas que pode ser amortizada na medida em
que participa dos controles policiais, tornando-se policial de si e dos
outros. Some-se a isso a racionalidade policial que circula nos meios
governados pelo regime dos ilegalismos, nos quais as empresas ligadas ao
comércio de substâncias classificadas como ilícitas também depuram um
saber policial nas relações. Assim, os enquadramentos jurídicos, sempre
estabelecidos a posteriori, apenas conformam o quadro móvel e modular da
atividade policial ordinária, que também é praticada pelo conjunto amplo
de técnicos em humanidades envolvidos nesses programas. Sem abrir mão
dos regimes de internações, distendem-se as penalizações a céu aberto,
que ativam práticas de monitoramento que vão muito além do controle
social formal ao qual os jovens tidos como infratores são submetidos.

No entanto, a expansão de tecnologias mais sutis e sofisticadas de
controles policiais não se traduz como uma redução da presença e ação da
polícia de contenção e repressão. Ao contrário, a configuração do
cidadão-polícia fomenta uma vontade de polícia ampliada, voltada para o
Estado e em busca de reconhecimento, uma ânsia por inclusão no arco das
penalizações a céu aberto e uma fé ampliada na eficácia dos
monitoramentos. Basta notar o que se passa com demandas específicas de
grupos voltados aos direitos de minorias. Referenciados nos artigos I e
\versal{II} da \emph{Declaração Universal dos Direitos Humanos}\footnote{``Artigo
  I: Todas as pessoas nascem livres e iguais em dignidade e direitos.
  São dotadas de razão e consciência e devem agir em relação umas às
  outras com espírito de fraternidade.

  Artigo \versal{II}: Toda pessoa tem capacidade para gozar os direitos e as
  liberdades estabelecidos nesta Declaração, sem distinção de qualquer
  espécie, seja de raça, cor, sexo, língua, religião, opinião política
  ou de outra natureza, origem nacional ou social, riqueza, nascimento,
  ou qualquer outra condição''.}, grupos da chamada sociedade civil
organizada demandam ao Estado ações de fiscalização e monitoramento do
cumprimento dos deveres correspondentes a esses direitos humanos em
busca de reconhecimento da dignidade sem distinções.

Essas demandas seguem um itinerário mais ou menos regular. Por meio de
convocação à participação, acumulam-se forças para demandar órgãos de
Estado especiais para monitoramento relativo a alguma minoria
identitária ou prática social específica, que é atendido pela criação de
algum programa social ou secretaria especial de Estado lotados por
ativistas, agentes sociais e \emph{intelectuais moduladores} ligados a
essas demandas. Em seguida, trabalha-se em torno da criminalização de
alguma conduta para institucionalização de resposta protetiva dos
direitos anunciados nos artigos \versal{I} e \versal{II} da \versal{DUDH}, e nas demais legislações
específicas que se orientam por essa referência universal. Os exemplos
mais comuns atualmente são as demandas por criminalização da homofobia,
do racismo e de práticas de intolerância religiosa. A sequência desse
processo de institucionalização da conduta policial não poderia acabar
em outro lugar que não fosse uma delegacia.

No Brasil, estados como São Paulo, Rio de Janeiro e Distrito
Federal\footnote{Sobre a criação da delegacia especial no Distrito
  Federal, ver Jéssica Nascimento. ``\versal{DF} cria 1ª delegacia para
  investigar crimes de intolerância religiosa'', notícia no Portal G1,
  de 21 de janeiro de 2016.
  \emph{http://g1.globo.com/distrito-federal/noticia/2016/01/df-cria-1-delegacia-para-investigar-crimes-de-intolerancia-religiosa.html}.}
já possuem delegacias especializadas, chamadas \versal{DECRADI} (Delegacia
Especial de Crimes Raciais e Delitos de Intolerância). E, seguindo um
itinerário lógico-institucional, o próximo passo é a criação de uma Vara
especial com juizado que se ocupe da matéria. Importante ressaltar que
esse itinerário da proteção à repressão, ou da demanda por direitos à
ampliação da ação policial, tem nas políticas de atenção a crianças e
jovens, no Brasil pós-Constituição Federal de 1988, seu laboratório
político inaugural, na medida em que fez esse caminho, no que diz
respeito aos jovens classificados como vulneráveis ou em situação de
risco, já nos anos 1990. Não é fortuito que o corolário legal que
enquadra essa proteção policial dos direitos se encontre na criação de
estatutos muito semelhantes ao \versal{ECA}, que vão do Estatuto do
Idoso\footnote{Criado em 1 de outubro de 2003, pelo Decreto-Lei 10.741.

  . \emph{http://www.planalto.gov.br/ccivil\_03/leis/2003/L10.741.htm} .}
ao Estatuto do Torcedor\footnote{Criado em 15 de maio de 2003, pelo
  Decreto-Lei 10.671.

  \emph{http://www.planalto.gov.br/ccivil\_03/leis/2003/L10.671.htm}.}.

As penalizações a céu aberto, primeiro se colocam como prática policial
ordinária de demandas específicas orientadas por valores universais
para, finalmente, receberem o enquadramento jurídico necessário que abre
caminho para a proliferação e aplicação das penas sob o controle do
Estado. E para além do acionamento e controle direto do Estado, as leis
de criminalização, os estatutos específicos ou mesmo leis restritivas,
que não demandam, necessariamente, em proibição total, como a lei que
restringe o uso de tabaco em algumas áreas públicas, ao produzirem um
efeito de empoderamento de determinados grupos, ou de um grupo amplo em
determinada situação, fomentam a conduta policial nos cidadãos que
passam a monitorar, voluntariamente e em nome da lei, a conduta dos
outros pelas ruas, praças, prédios e demais espaços públicos onde eles
presenciem uma conduta normativa condenável e normalizável.

Duas referências encontradas em torno da criação e funcionamento das
\versal{DECRADI}s de São Paulo e Rio de Janeiro evidenciam como essa formalização
da conduta do cidadão-polícia em delegacias especializadas se relaciona
com a participação da chamada sociedade civil organizada, tanto em sua
criação, quanto em seu funcionamento. A \versal{DECRADI} de São Paulo é a
primeira delegacia especial contra crimes de intolerância a ser criada
no Brasil. Sua interface protocolar com entidades e grupos da chamada
sociedade civil organizada é destacada no próprio site da Polícia Civil
do Estado de São Paulo: ``Em 2006, por meio do Decreto nº 50.594, o
grupo se tornou a 2ª Delegacia de Polícia de Repressão aos Crimes
Raciais e Delitos de Intolerância (Decradi), vinculada à Divisão de
Proteção à Pessoa do Departamento Estadual de Homicídios e de Proteção à
Pessoa (\versal{DHPP}). Atualmente, a equipe é formada pela delegada titular
Daniela Branco, três escrivães e nove investigadores, e além de
investigar crimes, busca também, estreitar laços com a sociedade civil,
organizações não governamentais (\versal{ONG}s), Ordem dos Advogados do Brasil
(\versal{OAB}), centros de referência, conselhos estaduais e demais órgãos
relacionados as suas atividades''\footnote{\emph{http://www.policiacivil.sp.gov.br/portal/faces/pages\_noticias/noticiasDetalhes?rascunhoNoticia=0\&collectionId=358412565221001826\&contentId=UCM\_015005\&\_afrLoop=16466579062752425\&\_afrWindowMode=0\&\_afrWindowId=null\#!\%40\%40\%3F\_afrWindowId\%}}.

A reprodução deste trecho é suficiente para expor a relação ascendente e
descendente na produção de penalizações com participação do
cidadão-policial por meio da criminalização de condutas específicas em
nome da defesa de direitos universais e orientados por uma moral da
tolerância, e por conseguinte, a horizontalidade das relações de poder
neste novo pastorado. Também mostra como a referência estatal não está
na formalização jurídica ou na regulamentação institucional, mas no
Estado como maneira de pensar e a forma de ação política para
organização das demandas de grupos, inclusive, ou talvez
preferencialmente, aqueles tidos como excluídos, vulneráveis ou em
situação de risco, que não se veem reconhecidos pela organização estatal
formal e protegidos pelo imperativo da lei. Trata-se do funcionamento do
Estado não como um local de poder disputado por forças antagônicas que
anseiam pelo seu poder como um recurso escasso, mas o Estado como
categoria do entendimento que regra e normaliza as relações, estejam
esses grupos dentro ou fora do que se chamada aparelho estatal ou
edifício jurídico.

Alguém poderá objetar que esse reconhecimento oficial da polícia e a
convocação à participação das atividades policiais (e nessa altura
destaca-se que isso vai muito além da mera prática de denúncia, que
segue operando, mas como convocação invertida, do cidadão para o Estado)
seja mera retórica ou mesmo uma ação de cooptação do Estado em direção
aos grupos de contestação. Não é isso que a criação do \versal{DECRADI} em São
Paulo mostra. Como toda lei e instância estatal, em seus baixos começos,
a criação da delegacia foi escrita com sangue. Em fevereiro de 2000, o
adestrador de cães Edson Neris foi brutalmente executado por um grupo de
Carecas do \versal{ABC}, um agrupamento de jovens identificados com ideais
fascistas\footnote{Um breve resumo do processo policial consta em
  matéria de 25 de fevereiro de 2002 na versão on-line do jornal Folha
  de S. Paulo
  \emph{http://www1.folha.uol.com.br/folha/cotidiano/ult95u46666.shtml} .}
(nacionalismo, orgulho operário e exaltação da masculinidade e da
violência\footnote{Diferentes do Skins Heads ou White Powers, os Carecas
  do Brasil ou Carecas do \versal{ABC}, referência à região metropolitana de São
  Paulo de forte industrialização e, consequentemente, com grande
  presença operária, não são neonazistas, nem partidários da supremacia
  branca, como em geral são descritos pela grande imprensa.
  Identificam-se com as ideias corporativistas do fascismo ou suas
  variações como o Integralismo, no caso brasileiro. Por essa razão, sua
  atuação como milícia juvenil tem como principais alvos os punks, por
  representarem fatias degeneradas do operariado como \emph{lumpens}, e
  os gays, por atentarem contra a moral fortemente referenciada na
  masculinidade. Seus inimigos de morte são os anarco-punks, pois além
  de ostentarem abertamente e de forma militante a relação com a
  anarquia, o que é uma afronta a devotos do poder coorporativo do
  Estado, eles encarnam como grupo de rua, para os Carecas do \versal{ABC}, os
  dois grandes demônios: são gays e são \emph{lumpen}.}). Neris foi
morto na noite de 6 de fevereiro de 2000, por decorrência de hemorragia
provocada por pancadas e golpes de soco inglês desferidas por cerca de
20 homens (a polícia apreendeu 18 suspeitos e, em 2002, a justiça
condenou um homem e uma mulher pelo ato) que o abordaram por ele estar
namorando com seu parceiro na Praça da República, localizada no centro
da cidade de São Paulo. Imediatamente, anarco-punks ligados a um grupo
denominado \versal{ACR} (Anarquistas Contra o Racismo), iniciaram uma campanha
chamada ``Alerta Antifascista!'', com panfletos de esclarecimento sobre
a existência de grupos de Carecas do Brasil, Skin Heads e White Powers
pelas ruas da cidade caçando os alvos de seu ódio político, social e
racial, além de fornecerem orientações sobre táticas de autodefesa e
referências das regiões da cidade onde a probabilidade de se encontrar
com esses grupos era maior. Os panfletos, shows e pequenos atos de rua
(que reuníam não mais que 50 pessoas) lançavam mão de referências
históricas da luta antifascista de anarquistas no Brasil e no mundo, e
do conhecimento acumulado pelos anarco-punks no enfretamento direto com
esses grupos fascistas nas ruas de São Paulo desde o final dos anos
1980\footnote{As informações sobre a atuação dos anarco-punks neste fato
  resumem relatos encontrados em panfletos do arquivo pessoal do
  pesquisador e em entrevistas não-formais feitas com pessoas que
  participaram das jornadas antifascistas realizadas até hoje pelos
  anarco-punks a partir da campanha criada em resposta ao que ficou
  conhecido como ``caso Edson Neris''. Sobre a história do \versal{ACR} e a
  criação das jornadas, ver \emph{http://anarcopunk.org/antifa/}.}.

A campanha iniciada pelos anarco-punks despertou o interesse de grupos
ligados ao que na época era conhecido como movimento \versal{GLS} (Gay, Lésbicas
e Simpatizantes), inclusive a presidência da Parada Gay, hoje Parada
\versal{LGBT} (Lésbicas, Gays, Bissexuais e Transexuais), e a Comissão de
Direitos Humanos da Câmara de Vereadores de São Paulo, presidida pelo
vereador Ítalo Cardoso, do Partido dos Trabalhadores --- \versal{PT}, artido que
havia recém assumido o executivo municipal com a vitória, em outubro de
2000, da candidata Martha Suplicy. Os grupos \versal{GLS}, em reuniões com
anarco-punks, insistiam em se iniciar uma campanha de criminalização da
homofobia e dos atos de intolerância, enquanto os anarco-punks e outras
associações e pessoas ligadas ao anarquismo convidavam o movimento \versal{GLS} a
engrossar a campanha do ``Alerta Antifascista''. Diante da
impossibilidade de um acordo, realizou-se dois atos paralelos, o dos
anarco-punks, de um lado da Praça da República, e o do movimento \versal{GLS}, de
outro lado, com a presença de partidos e personalidades políticas. O
desacordo entre os grupos não era total, na medida em que se encontram
nos panfletos anarco-punks reivindicação de direitos, além de o
movimento continuar frequentado a Parada Gay nos anos subsequentes. Foi
decisivo para a recusa dos anarco-punks em participar a presença de
partidos políticos, a exclusividade da pauta que não se abria para uma
luta antifascista e a reivindicação por criminalização de condutas.

Este último ponto foi vetado, menos por uma questão de princípios e mais
pela experiência empírica de conivência da polícia com grupos de Carecas
e White Powers\emph{,} largamente conhecida entre os punks de variadas
proveniências. Ocorre que o caso Edson Neris tornou-se mote, até os dias
de hoje, das reivindicações em torno da criminalização da homofobia,
presente em quase todas as Paradas Gays desde então. Tempos depois, como
resposta às reivindicações e demandas colocadas por diversos movimentos
que não apenas o movimento \versal{GLS}, criou-se um grupo de investigadores
dedicados à apuração do que se classifica como crime de intolerância,
grupo este que, em 2006, se tornará uma delegacia especial, a \versal{DECRADI},
como vimos acima, segundo o site oficial da Polícia Civil de São Paulo.

Evidente que esta não é \emph{a} história da criação da \versal{DECRADI} em São
Paulo, que possui uma narrativa institucional passível de verificação no
site da própria Polícia Civil de São Paulo. No texto do site, alega-se a
preocupação do governo do estado de São Paulo, na época governado por
José Serra (\versal{PSDB}), com a garantia dos direitos de minorias, em
particular, e na defesa dos Direitos Humanos, em geral. A reprodução
desse relato aqui tem por objetivo descrever as demandas por
reconhecimento como prática política do cidadão-policial, mostrando como
a \emph{nova política} ligada a grupos específicos que constatam certa
crise de representação e atuam de forma mais ou menos apartidária (o que
não implica o rompimento de relações com partidos políticos, desde que
estes colaborem em favor de suas demandas específicas), terminam por
reforçar o Estado a partir de uma lógica punitiva, abrindo espaço e
participando das penalizações a céu aberto por meio de seus agentes
sociais e ativistas. Este episódio também expõe como o Estado codifica
essas demandas provenientes do ativismo da \emph{nova política}. Segundo
relatos dos anarco-punks, os policiais recrutados pela \versal{DECRADI} são
ex-integrantes de grupos de Carecas, algo que, admitamos, é de difícil
verificação para além de quem travou batalhas diretas com eles. No
entanto, o dado que mais interessa na relação \emph{nova política},
penalizações e práticas \emph{antipolíticas} é que precisamente a
\versal{DECRADI}, como destacamento policial de investigação, realiza
acompanhamento, registro e infiltração em manifestações de rua, e tem
como seus alvos preferenciais os punks e grupos identificados como
anarquistas, na medida em que, para a lógica policial, anarquistas são
um grupo político tão intolerante quanto os fascistas.

A \versal{DEGRADI} participou, sob a coordenação do \versal{DEIC} (Delegacia Estadual de
Investigações Criminais), do processo aberto em novembro de 2013 para
investigar os chamados atos de vandalismo (entendidos como crimes de
intolerância) nas manifestações iniciadas em junho do mesmo ano, no caso
que ficou conhecido como processo dos \emph{black blocs}. Este
investigação correu, por muito tempo, de forma sigilosa. Mesmo advogados
de pessoas convocadas a depor tinham dificuldades ou eram simplesmente
impedidos de acessar o inquérito policial. Seu desfecho, para além de
alimentar regulares reportagens nos variados meios da imprensa, deu-se
com duas prisões, decretadas em julho de 2014, dirigidas a sujeitos
identificados como lideranças dos \emph{black bloc} (algo que
absolutamente não existe, conforme se demonstrou depois), e mais de cem
intimações de pessoas para prestar depoimento na sede do \versal{DEIC}, na zona
norte de São Paulo. Hoje, segundo relato de 2015 no site da Polícia
Civil do Estado de São Paulo, a maior demanda de investigações decorre
dos atos classificados como crimes de racismo, que representaram, no ano
de 2015, 68,2\% dos casos apurados pela \versal{DECRADI}. Em geral são casos de
injúrias raciais publicadas em redes sociais digitais, como o caso da
apresentadora da previsão do tempo do jornalismo do sistema Globo de
comunicações, Maria Júlia Coutinho, que levou à prisão de um jovem de 15
anos que teve o \versal{IP} do computador mapeado pela polícia por ser apontado
como autor da injúria\footnote{Polícia Civil de São
  Paulo.\emph{http://www.ssp.sp.gov.br/noticia/lenoticia.aspx?id=36514}.}.
Enfim, uma volta completa da participação em nome da defesa de diretos
ao inquérito policial e o estabelecimento de culpado a ser punido.

A \versal{DECRADI} fluminense é mais recente que a paulista, foi criada em 22 de
março de 2011, a partir do Projeto de Lei Estadual 1.609/2008\footnote{\emph{http://alerjln1.alerj.rj.gov.br/scpro0711.nsf/f4b46b3cdbba990083256cc900746cf6/797984842c4bcc0883257464006ce2d8?OpenDocument}.}.
No entanto, sua criação segue um itinerário semelhante ao da delegacia
de São Paulo, mesmo que motivado por grupo específico e demanda de
criminalização diversa. A referência segue escorada nos artigos \versal{I} e \versal{II}
da \versal{DUDH}, porém, seu mote foi um relatório enviado à \versal{ONU} pela Comissão de
Combate à Intolerância Religiosa, da Assembleia Legislativa do Estado do
Rio de Janeiro. O envolvimento e convocação à participação pela chamada
sociedade civil organizada é notável por meio da organização que saudou
a criação da delegacia: o Observatório de Favelas. Em seu site oficial
há um texto de apresentação sobre a organização que a define como ``uma
organização da sociedade civil de pesquisa, consultoria e ação pública
dedicada à produção do conhecimento e de proposições políticas sobre as
favelas e fenômenos urbanos. Buscamos afirmar uma agenda de Direitos à
Cidade, fundamentada na ressignificação das favelas, também no âmbito
das políticas públicas''\footnote{Observatório de Favelas.
  ``Apresentação''. \emph{http://of.org.br/apresentacao/}.}. Trata-se de
uma iniciativa não ligada diretamente ao Estado, orientada pela defesa
de direitos humanos e voltada para um público específico, os moradores
das favelas cariocas. Atua, sobretudo, em projetos sociais e culturais
que visam ampliar a participação política e a representatividade
democrática dos habitantes dessas áreas da cidade. Embora não possua
conexão direta com o Estado, salvo na condição de parceira em campanhas
e programas sociais, declara em sua apresentação o objetivo em
contribuir para criação de políticas públicas (\emph{policies}) voltadas
aos interesses da favela que, também segundo o Observatório de Favelas,
deve ser ressignificada. Em suma, trata-se de um instrumento de
monitoramento, construído desde a participação da sociedade em
colaboração com pesquisadores oriundos da própria favela, para a
produção de melhorias locais orientadas pelos valores universais dos
direitos humanos e suas recomendações de políticas planetárias para o
cuidado com as futuras gerações.

É sabido o quanto os habitantes das favelas são alvo preferencial da
seletividade do sistema penal. Pela justificativa do combate ao comércio
varejista de substâncias classificadas como ilícitas, seus moradores são
também alvo regular da violência perpetrada pelo policiamento ostensivo
da Polícia Militar. No caso das favelas cariocas isso se tornou mais
evidente e sistemático desde que os territórios foram alvo de
pacificação pelo maior programa de segurança urbana do país, as \versal{UPP}s,
saudado por diversas organizações num primeiro momento e alvo de
objeções pontuais, sob a égide dos direitos humanos, após a repercussão
do que ficou conhecido como caso Amarildo, desaparecido por policiais de
uma dessas unidades, em 2013. Posto isso, ver uma organização que se
dedica à produção de políticas que visam à melhoria das favelas saudar
uma delegacia pode soar paradoxal ou contraditório, mas apenas mostra
como \emph{policy} e \emph{police} atuam em parceria, de forma
complementar e combinada na produção de pacificações, estando ligadas
diretamente às atividades do Estado ou não. Em seu site, o Observatório
de Favelas publica um texto assinado por Thiago Ansel, que celebra a
criação da \versal{DECRADI} no Rio de Janeiro como um meio de pacificar as
perseguições, que ele relata ao longo do texto, de praticantes das
religiões neopentecostais contra adeptos do candomblé e demais práticas
religiosas de matriz africana.

Ansel argumenta citando: ``o autor da proposta, deputado Átila Nunes
(\versal{PSL}), afirmou que a grande incidência de delitos de intolerância no Rio
foi o que motivou a criação da \versal{DECRADI}-\versal{RJ}. `Sobretudo, os casos de
intolerância religiosa no estado atingiram índices alarmantes nos
últimos três anos. Os ataques verbais tornaram-se ataques físicos. Os
crimes de racismo continuam a ser denunciados. Outros tipos de
intolerância vêm surgindo: contra os obesos, o bulling e o cyberbulling,
por exemplo', explica Nunes''\footnote{Thiago Ansel. ``Delegacia
  especializada para intolerância'', 30/03/2011.

  \emph{http://of.org.br/noticias-analises/delegacia-especializada-para-intolerancia/}}.
Uma pletora de direitos, a partir de demandas identitárias específicas,
abre caminho, pela ação dos próprios alvos das práticas de policiamento,
controle e penalizações a céu aberto.

A criação da \versal{DECRADI} fluminense expõe uma outra dimensão das
penalizações a céu aberto pelas convocações à participação e a atuação
de ativistas e agentes sociais. O mais explícito na produção de uma
subjetividade policial ligada ao exercício da cidadania é o fomento à
prática da denúncia como meio direto de garantia da convivência
tolerante em uma comunidade determinada. Além disso, há uma convocação
da polícia, como destacamento especial de investigação, criminalização e
repressão, para solucionar os conflitos (no caso, motivado por
divergências religiosas) entre pessoas e grupos sociais diferentes,
admitindo a impossibilidade de soluções singulares que escapem da lógica
punitiva e do estabelecimento de agressores, vítimas e culpados,
corroborando a necessidade de um medidor externo ao qual é dado a
legitimidade para intervir de forma violenta. Para além desses
desdobramentos explícitos vê-se, em organizações como o Observatório de
Favelas e sua relação positiva com a criação da \versal{DECRADI} fluminense, como
uma série de investimentos políticos, culturais e sociais, articulados
por \emph{intelectuais moduladores}, operam para fomentar uma educação
política por meio da ressignificação de espaços antes tidos como
degradados e da convocação à participação que, combinados com a
violência sistemática das polícias (legais e ilegais), atuam na produção
de ambientes seguros. Ambientes nos quais mesmo aqueles que vivem sob as
condições mais adversas devem participar, zelar por eles, por si e pelos
outros para a produção de melhorias, avaliadas segundo um cálculo de
probabilidades e de riscos. Assim, produz-se o assujeitamento (Foucault,
1995) aos variados governos (do Estado, das empresas, das \versal{ONG}s, dos
Institutos, das polícias) produzindo a si mesmo como um policial:
participativo, tolerante, cidadão e, sobretudo, com capacidade de
empreender no local em que vive, transformado em ambiente seguro.

O efeito imediato desse superinvestimento em participação, democracia,
direitos e denúncias, além das condutas assujeitadas e de um
\emph{conservadorismo moderado} (Passetti, 2007), é a intensificação da
sociabilidade autoritária, traço marcante da distenção das penalizações
a céu aberto. Hoje, quase todo desacerto ou diferença guarda em si a
possibilidade de virar um boletim de ocorrência policial. Uma injúria
não é mais respondida com outra de igual impacto ou mesmo, no limite,
com uma agressão física ou um tapa; a injuria se torna passível de ser
esquadrada na ampla definição de crimes de intolerância, ou seja, é
transformada pelo ativamento dos próprios governados em conduta
criminalizada. O trânsito ambíguo das leis de proteção ao assédio (moral
ou sexual) entre o Direito Civil e o Direito Penal pode transformar uma
ação descuidada ou um movimento impensado em uma conduta criminalizável.

E como a lei carece sempre de interpretação, todos ficam à mercê de um
denunciante, um policial, um promotor, um juiz, a prática do poder
soberano democratizada por meio da ativação de denúncias. Desta maneira,
acumulam-se os processos, as dívidas impagáveis e se ampliam as
penalizações, dentro e fora do sistema penal, cumpridas dentro e fora da
prisão. A penalização a céu aberto é a produtora de ambientes seguros
monitorados pelo cidadão-polícia. O efeito imediato: em um ambiente no
qual todos são policiais, o correlato lógico é que todos se tornem
suspeitos e, logo, todos sejam monitorados como virtuais perigosos. A
ironia, talvez não tão paradoxal como possa parecer, é que esse estado
de suspeição ampliado e elastificado se produziu por meio da busca de
proteção e garantia de direitos aos que se acreditava serem os mais
vulneráveis diante dos poderes políticos e econômicos. A proliferação
das vítimas fez com que se ampliasse o campo de atuação dos algozes,
dando a possibilidade a todos de ocupar um ou outro papel, a depender da
circunstância em que se encontra. Somos divíduos sob suspeição.

\section{democracia e polícia}

A política da suspeição virtual ampliada, elastificada e generalizada
encontra sua formalização nas inúmeras legislações de combate ao
terrorismo que se multiplicaram pelo planeta desde o ano 2000.
Inglaterra, \versal{EUA}, Espanha, enfim, diversos países do Norte e do Sul
aprovaram novas leis de controles mais rígidos e regulamentações de
monitoramento eletrônico a partir dessa data. O Brasil, por ocasião da
realização das Olimpíadas de 2016, também sancionou uma nova lei
antiterror (Lei Federal 13.260/2016) com suas vagas definições de
manutenção da ordem, controle de fluxos, monitoramentos e intervenções
ambientais. No entanto, mais decisivo do que a rigidez da lei, para as
práticas de penalizações a céu aberto, é a plasticidade das ações
policiais que se tornaram extremamente investidas. Seria possível
imaginar que, ao fomentar uma conduta policial entre os cidadãos, a
necessidade de intervenção policial direta diminuísse, todavia ela só
aumentou, em quantidade, regularidade e amplitude.

Hoje, qualquer um que caminhe por uma grande cidade do planeta
experimenta a inevitável sensação de estar numa zona de guerra ou sob
ocupação de algum país estrangeiro. Homens e mulheres fortemente
armados, com roupas e equipamentos de alta tecnologia e assessórios
letais e não-letais de contenção, são componentes inexoráveis das
paisagens urbanas contemporâneas de São Paulo, Johanesburgo, Paris,
Londres, Nova Iorque ou Tel Aviv. A capacidade de destruição em massa e
de destruição mútua anunciada com o início da era nuclear é hoje
rotinizada pela presença de forças ostensivas de segurança policial em
qualquer lugar e por virtuais ameaças.

Ademais, outras \emph{ameaças} não diretamente humanas, como terremotos,
cataclismos e acidentes climáticos habitam corações e mentes dos
viventes nos dias de hoje. No entanto, o estado perene de monitoramento
e a ostentação contínua do poder de destruição e eliminação do Estado se
mostram impotentes diante de regulares ataques de pessoas isoladas ou
pequenos grupos de afinidades que agem em nome de alguma autoridade
programática do terrorismo transterritorial, ou mesmo de perdedores
radicais que agem por motivações pessoais, em práticas de destruição
nesses grandes centros, instalando o terror pela violência praticada de
forma exemplar ou por vingança, face macabra das penalizações a céu
aberto que funcionam pela lógica punitiva. Mais regularmente, esse
enorme contingente policial é mobilizado para conter deslocamentos
multitudinários, seja de protestos de rua, seja de espectadores de
grandes eventos (Copa, Olimpíada, Cúpulas Internacionais), ou, mais
comumente, de ambos ao mesmo tempo, tentando controlar o jogo entre
espetáculo e contra-espetáculo. Os governos democráticos de hoje
possuem, tanto em termos legais, quanto em termos tecnológicos, uma
capacidade de destruição e de ataque aos cidadãos, uma capacidade de
destruição e uso da violência infinitamente maior do que possuíam os
governos autoritários e totalitários do século passado.

No Brasil, a recente criação de uma força policial sob o comando do
governo federal oferece uma visão bem clara dessa ampliação dos
controles policiais, com a vantagem política de desonerar as forças
militares de praticar intervenções militares no meio civil para
manutenção da ordem. Esse novo destacamento é assim descrito no site
oficial do Ministério da Justiça: ``A Força Nacional de Segurança
Pública foi criada em 2004 para atender às necessidades emergenciais dos
estados, em questões onde se fizerem necessárias a interferência maior
do poder público ou for detectada a urgência de reforço na área de
segurança. Ela é formada pelos melhores policiais e bombeiros dos grupos
de elite dos Estados, que passam por um rigoroso treinamento no Batalhão
de Pronta Resposta (\versal{BPR})''\footnote{\emph{http://www.justica.gov.br/sua-seguranca/forca-nacional}}.

O lema de guerra desse destacamento especial de forças federais traduz
de forma cristalina o que se espera da polícia na era das penalizações a
céu aberto. Em seu brasão encontra-se a inscrição: ``Preparados para
tudo''. No entanto, sua ativação e intervenção, anunciada como
extraordinária no site do Ministério da Justiça, tornaram-se,
imediatamente após a sua criação, um fato ordinário. Basta consultar o
noticiário recente para constatar o protagonismo dessas forças na
ocupação militar de favelas no Rio de Janeiro e nas atividades de
contenção e investigação de protestos contra a realização da Copa do
Mundo em 2014, contra a construção da Usina de Belo Monte, na
resistência dos índios Tupinambá no sul da Bahia ou, mais recentemente,
na garantia da realização das Olimpíadas de 2016, na cidade do Rio de
Janeiro, todas em parceria com a Polícia Federal, o Exército Nacional
(sua dispensa se deu apenas em tese) e as polícias locais. É apenas sob
a rubrica de polícia repressiva e investigativa que, hoje, no Brasil, a
Força Nacional de Segurança, a Polícia Federal, as Polícias Civis dos
estados da federação e suas Polícias Militares e, por fim, as Guardas
Metropolitanas dos municípios existem. Some a isso o fato de que o
exército vem assumindo funções policiais cada vez mais regularmente,
como no caso da Ocupação do Complexo da Maré no Rio de Janeiro, e já
temos uma verdadeira muralha armada sob o comando do Estado.

Se admitirmos nessa descrição todo o contingente ligado ao comércio
varejista de substâncias classificadas como ilícitas, que atua como
polícia nas favelas, inclusive, portando arsenal de grosso calibre, e o
contingente de segurança patrimonial privada lotados em prédios de
bancos ou de empresas e em casas de espetáculos e de particulares, é
seguro afirmar que no exato momento em que você lê essas linhas há um
policial ao seu lado, ou talvez ele seja você mesmo.

A polícia, diferente do que havia mostrado Foucault (2008) em seu curso
de 1978, não é mais um meio de governo, um dispositivo de segurança
voltado para o controle e promoção da saúde da população, ela se tornou
a forma mesma dessa atividade de governar a si e aos outros. Polícia e
política, \emph{politic}, \emph{policy} e \emph{police} se tornaram um
bloco único que se encontra não mais restrito ao controle da população,
mas que visa controlar emergencialmente tudo que é vivo no planeta. A
ironia, ou preenchimento estratégico dessa forma-dispositivo, é que a
mesma discursividade ativada para a produção dessa forma estendida de
policiamento e essa produção de subjetividades policiais, os direitos e
mais universalmente os direitos humanos, é utilizada como forma de
conter o poder de polícia.

Um relatório de 2011, do Programa das Nações Unidas para o
Desenvolvimento --- \versal{PNUD}, condena o Brasil por possuir uma razão de 213
Policiais Militares e 60 Policiais Civis por 100 mil habitantes, estando
à frente apenas da Guatemala (um país que vive praticamente uma ocupação
militar dos \versal{EUA} até hoje) no quesito entre países da América Latina,
algo que o programa da \versal{ONU} aponta como fator de insegurança. Um
levantamento feito pelo Portal de notícias G1 em 2015\footnote{Tahiane
  Stochero. ``Mesmo com alta de efetivo no país, sobe nº de habitantes
  para cada \versal{PM}''.

  \emph{http://g1.globo.com/politica/noticia/2015/07/mesmo-com-alta-de-efetivo-no-pais-sobe-n-de-habitantes-para-cada-pm.html}}
aponta uma média de 471 policiais militares por habitante no país, num
total 430,8 mil \versal{PM}s em todo o Brasil. No Distrito Federal, com a maior
média nacional, essa razão é de 190 \versal{PM}s por habitante, com uma taxa de
homicídios de 24 óbitos por 100 mil habitantes. São Paulo, que está
acima da média nacional, possui uma razão de 491 \versal{PM}s por habitante, num
total de quase 90 mil policiais militares. Cabe ressaltar que esses
números não incluem dados de outros efetivos policiais, como a Guarda
Civil Metropolitana, que no caso da cidade de São Paulo tem poder de
polícia e possui porte de arma, e batalhões especiais de choque, como a
\versal{RONDA}. Sobressai nesse levantamento que os especialistas e as
autoridades ouvidos pela reportagem, todos defensores dos direitos
humanos e referenciados em recomendações da \versal{ONU}, avaliam a necessidade
de aumento do contingente. Mesmo entre os movimentos de ativistas que
podem ser incluídos no amplo leque chamado aqui de \emph{nova política},
não vão muito além de pedidos de desmilitarização da polícia e acréscimo
de políticas sociais como forma de promoção de maior segurança, todos
referenciados, também, em recomendações de instâncias da \versal{ONU}.

Chega-se, assim, a um ponto da análise que uma indicação metodológica de
Foucault nos coloca em uma dimensão outra. Ao concluir seu curso
\emph{Segurança, Território, População}, ele faz a seguinte observação:
``A história do Estado deve poder ser feita a partir da própria prática
dos homens, a partir do que eles fazem e da maneira como pensam. O
Estado como maneira de fazer, o Estado como maneira de pensar. Creio que
essa não é, {[}certamente{]}, a única possibilidade de análise que temos
quando queremos fazer a história do Estado, mas é uma das
possibilidades, a meu ver, suficientemente fecunda, fecundidade essa
ligada, no meu entender, ao fato de que se vê que não há, entre o
micropoder e o macropoder, algo como um corte, ao fato de que, quando se
fala num {[}não{]} se exclui falar no outro. Na verdade, uma análise em
termos de micropoderes compatibiliza-se sem nenhuma dificuldade com a
análise de problemas como os do governo e do Estado'' (Foucault, 2008:
481). No caso, a formação de subjetividade policial corresponde a um
crescimento e a um superinvestimento, em termos planetários, nos
efetivos policiais propriamente ditos e nas tecnologias sociais e
computo-informacionais de monitoramento. Desde logo, não se está falando
de uma micropolítica, mesmo que a análise se dê em termos microfísicos,
mas de uma política tal como ela se configurou modernamente e como ela
se apresenta hoje em dimensões planetárias, vazando os enquadramentos
disciplinares e legais, as fronteiras dos Estados nacionais e o contorno
do planeta propriamente dito. No que diz respeito ao que foi apresentado
até aqui, trata-se de uma política como polícia: polícia institucional e
polícia como a forma de vida nas democracias contemporâneas.

O cidadão-polícia encontra-se sob um Estado democrático de direito
altamente policiado, seja com polícia repressiva e de contenção, seja
com polícia enquanto cuidados com os ambientes. A prática policial,
assim, não apenas vaza as fronteiras do território do Estado e de sua
população específica, como também se projeta, desde o local, como uma
tecnologia de poder com contornos planetários, tendo os variados
ambientes deste planeta como objeto de sua produção de segurança e
seguridade. Assim, a polícia não é mais ``a governamentalidade do
soberano direta do soberano como soberano'', como Foucault apontara em
relação à governamentalidade moderna.

Na ecopolítica, as tecnologias de governamentalidade planetária
empoderam o cidadão que atua democraticamente como polícia e a
convencional distinção política de disputa pelo Estado entre esquerda e
direita encontra, nesse início de século \versal{XXI}, sua unidade em torno da
busca por segurança e da defesa dos direitos humanos. É nesse sentido
que as resistências que efetivamente colocam um problema insuportável às
tecnologias de governo hoje são as que atentam contra a segurança, não
estão decalcadas no discurso dos direitos humanos e atacam os
controlares policiais. Por isso, seja de forma positiva ou negativa (a
depender de quem enuncia), essas resistências são chamadas de
\emph{antipolíticas}. Não por acaso, estão relacionadas com a anarquia e
anunciam uma forma de vida outra, a produção de uma vida militante
distante do ativismo e da agência social (macros e micros) das formas
policiais da nova política.

\section{nova política e \emph{antipolítica}}

O esgotamento e as crises dos espaços disciplinares anuniciaram novas
configurações nas tecnologias de governo mostradas até aqui pela mutação
do dispositivo de vigilância em \emph{dispositivo monitoramento}, com as
sobreposições e simultaneidades configuradas pelo exercício de poder em
determinadas situações estratégicas. A governamentalização do Estado se
projeta planetária e sideral na medida em que as tecnologias
computo-informacionais e os cuidados de cada um com o planeta em busca
de melhorias se amplificam. Novas práticas de resistências também tomam
forma, próximas às forças que também enfrentaram as tecnologias
disciplinares de forma radical, os anarquistas. No entanto, a emergência
de outras de resistências também terá que lidar com as capturas e as
forças que buscam aplacá-las, assim como associações operárias,
sabotagens e greves gerais tiveram que enfrentar a reação da ordem na
sociedade disciplinar. O limiar dessas novas práticas de contrapoder foi
o acontecimento \emph{1968}, que anunciou de maneira contundente o
esgotamento das tecnologias disciplinares, no capitalismo e no
socialismo. No entanto, a fissura contemporânea aconteceu com o
movimento antiglobalização, que fez sua aparição espetacular e
planetária a partir dos protestos de rua contra a \versal{OMC} (Organização
Mundial do Comércio) em novembro de 1999, na cidade de Seattle,
antecedido pelo \versal{EZLN} (Exército Zapatista de Liberação Nacional), no
México.

A partir do \emph{movimento antiglobalização} opera-se uma ruptura.
Antes dele os conservadores acreditavam ter enterrado os projetos de
emancipação humana e declaravam que \emph{68} havia acabado e sido
capturado pelo novo capitalismo em sua forma neoliberal. O
\emph{movimento antiglobalização} trouxe renovação nos discursos de
emancipação e será interpelado, na Grécia, pelo intempestivo da revolta.
Em meio a este embate reaviva-se o interesse na anarquia e nas práticas
da cultura libertária, ao mesmo tempo em que um conjunto de práticas
vinculadas às lutas históricas da anarquia é capturado como maneira de
fazer funcionar o movimento pela constituição de uma cidadania global.
Entre o ativista e o agente social, essa cidadania global se consolida e
se constitui, ora como contestação, ora como força da ordem, compondo os
traços de uma \emph{nova política} que ganha espaço de formas variadas e
em meio a processo bastantes específicos em todo planeta.

A revolta anuncia a recusa a essas capturas ao se colocar como
\emph{antipolítica}, que na produção de uma \emph{cultura libertária}
desgoverna os monitoramentos pelo ingovernável de uma vida outra que se
afirma como militantismo: ``um militantismo aberto que constitui a
crítica real e do comportamento dos homens e que, na renúncia, no
despojamento pessoal, trava o combate que deve conduzir à mudança do
mundo inteiro. (...) Não simplesmente como escolha de uma vida
diferente, feliz e soberana, mas como a prática de uma combatividade no
horizonte do qual há um mundo outro'' (Foucault, 2011: 253).

Se, de um lado, há uma fartura de comunicações e expansão planetária de
monitoramentos que atravessam os espetaculares protestos de rua no
começo do século \versal{XXI}, nos quais a tática \emph{black bloc} emerge como
escudo de proteção contra a polícia, força ativa que abre caminhos e
desperta a atenção produzindo um contra-espetáculo, de outro lado, a
Conspiração das Células de Fogo, uma associação de terroristas
anarquistas gregos, é o escândalo que se interpõe ao espetáculo e ao
contra-espetáculo, afirmando uma vida outra em combate perpétuo como
forma de militantismo \emph{antipolítico}.

Ao cartografar essas resistências a partir do \emph{movimento
antiglobalização}, chegou-se a um mapa de acomodação que remete às
experiências dos governos de esquerda na América do Sul do começo da
primeira década do século \versal{XXI}, pós-\emph{movimento antiglobalização}, e
aos movimentos de contestação na Grécia no final da mesma década até a
culminância de um novo governo dirigido pela coalização de esquerda
radical (\versal{SYRIZA}) na metade desta década. Esta \emph{nova política}
também se manifesta na Espanha com o Podemos e, mais recentemente, com o
Barcelona em Comum, vencedor das eleições municipais em 2015. Estas
seguem apresentando variações em toda Europa e nos \versal{EUA}, com destaque
para a utilização afinada das tecnologias computo-informacionais, algo
decisivo em movimentos como os da Praça do Sol com os Indignados de
2011.

As coalizões europeias, mesmo se apresentando como uma ``nova
esquerda'', ainda preservam formas de mobilização relacionadas aos
movimentos históricos do socialismo no continente, algo que na Espanha é
um pouco mais elástico, pois o Podemos se vincula a uma política que
saiu dos bancos universitários. De qualquer forma, referência
contemporânea para todos esses movimentos, partidos e partidos-movimento
é o \emph{movimento antiglobalização}, seja em suas pautas mais gerais,
como denúncia do poder financeiro se sobrepondo às democracias
nacionais, seja nos modos de atuação, como bloqueios de rua e nomeação
dos atos pela data em que são realizados, como foi o S26, em 2000 nos
\versal{EUA}, e o 15M de 2011, conforme se autonomearam os que depois foram
chamados de Los Indignados.

O \emph{movimento antiglobalização} teve sua emergência, atuação e
ampliação vinculadas às lutas políticas na América Latina. Sua aparição,
com o uso de tecnologias computo-informacionais, remete aos comunicados
enviados por e-mail pelo movimento zapatista no México, a partir de 1994
(Di Felice \& Muñoz, 1998; Holloway, 2003). Sua atuação
institucionalizada inicia-se com a realização, em 2001, na cidade de
Porto Alegre, do Fórum Social Mundial --- \versal{FSM}, que modificou o caráter
anticapitalista inicial do movimento para a busca de um altermundialismo
e ligações com os governos da região (Bringel \& Muñoz, 2010)\footnote{Sobre
  a relação entre o \versal{FSM} e os governos nacionais na América Latina, ver:
  Fórum Social Mundial.
  \emph{http://www.forumsocialmundial.org.br/main.php?id\_menu=19\&cd\_language=1}}.
Sua disseminação, retomada radical e transformação, inicia-se com a
revolta grega em dezembro de 2008 (Augusto, 2013), e logo encontrou
ressonância em práticas de contestação radicais em meio às jornadas de
junho de 2013 no Brasil (Passetti, 2013c). Em todas as modulações o
movimento sempre esteve atravessado, direta ou indiretamente, por
enunciados e práticas derivados dos anarquismos (Newman, 2011).

Em meio ao que se passou a conhecer como oposição à globalização e ao
neoliberalismo, o \versal{EZLN} (Exército Zapatista de Libertação Nacional) foi
de imediato apontado como a força atualizada e mais expressiva desse
combate por duas razões: veio a público no dia 1º de janeiro de 1994,
dia em que entrava em vigor o \versal{NAFTA} (Acordo de Livre Comércio da América
do Norte) --- acordo que, para eles, simbolizava a forma do capitalismo
neoliberal e globalizado ---; e suas declarações eram enviadas por
e-mail pelo subcomandante Marcos, nas quais o guerrilheiro afirmava que
seu \emph{lap top}, para o objetivo do \versal{EZLN}, era uma arma mais decisiva
do que os fuzis.

De uma só vez, dispunham-se os elementos que guiariam quase todos os
movimentos posteriores de crítica ao capitalismo: o uso intensivo da
internet como canal de produção de um discurso contra hegemônico global
e a luta contra a nova configuração de poder globalizado. Mas havia
mais: os zapatistas diziam querer ``mudar o mundo sem tomar o poder''
--- expressão utilizada posteriormente pelo cientista político marxista
John Holloway (2003) como mote da emergência de uma nova esquerda não
mais orientada pela luta por ocupação do aparelho de Estado e de caráter
antissistêmico, mas que já anunciava uma certa disposição em participar
da luta democrática, sem disputar postos na burocracia estatal. Mas
também pode ser tomada como uma posição de recusa do Estado e, portanto,
relacionada aos anarquismos, como aponta Richard Day (2005). Algo que o
próprio John Holloway (2013: 180, nota 25) admitirá em resposta à
crítica de Day, porém reafirmando uma posição de defesa da democracia
como tal como campo possível de ação política, ou seja, afastando-se da
recusa radical e admitindo a disputa política.

Nesse momento, anunciava-se uma nova forma de ativismo político que
combinava certa recusa de disputa por ocupar o governo do Estado e o uso
das novas tecnologias computo-informacionais. No entanto, os zapatistas
ainda se encontravam atrelados a um discurso militar, de organização de
guerrilha, e seu anticapitalismo marcado pela crítica da economia
política clássica combinada com elementos de defesa da tradição
indígena. A ampliação dessa forma de ativismo e a ampliação das
formulações críticas cada vez mais próximas dos anarquismos e do
pensamento pós-estruturalista, relacionado em especial a Michel Foucault
e a Gilles Deleuze, levará ao que Richard Day nomeia como novíssimos
movimentos sociais em seu livro com o sugestivo título \emph{Gramsci is
dead} (Day, 2005). Esses movimentos se afastariam cada vez mais da luta
não só pelo poder, mas também de uma disputa em direção à busca de
influenciar as decisões de Estado, produzindo \emph{ações diretas} e de
experimentações éticas de relações que efetivamente operam segundo a
transformação que se quer para o presente e não projetada para o futuro.
Não é fortuito que será em meio à disseminação dessa forma de atuação e
de ativismo que temas como ecologia, luta contra as prisões e feminismo
ganharão mais impactos que temas antissistêmicos mais genéricos.

Foi com a propagação das tecnologias computo-informacionais e as
articulações para ações coordenadas em todo planeta no dia de reunião
das organizações internacionais, como o G8 (grupo dos sete países mais
ricos do mundo mais a Rússia) e o Banco Mundial, que o \emph{movimento
antiglobalização} ganhou forma e fez sua primeira aparição para o mundo
em 30 de novembro de 1999, contra a reunião da Organização Mundial do
Comércio --- \versal{OMC}, na cidade de Seattle. Nesse primeiro
protesto-espetáculo já se apresentavam características do que se nomeou
como \emph{movimento dos movimentos} --- designação utilizada para se
referir a uma forma de atuação que engloba desde pequenos sindicatos,
associações de bairro, ecologistas, feministas, \versal{ONG}s e toda sorte de
grupos da chamada sociedade civil organizada até associações
anarquistas, punks, anarco-punks e ecologistas radicais.

Em poucas palavras, via-se nas aparições destes grupos de luta por
direitos de minorias ou contra o recuo de direitos trabalhistas,
marchando por suas reivindicações específicas, até praticantes de uma
política radical, vinculada à anarquia, de ataque frontal ao Estado e às
empresas do capitalismo global como os \emph{black blocs}, como mostram,
de maneiras diferentes, as pesquisas de Francis Dupuis-Déri (2014) e A.
K. Thompson (2010). Não tardou para que essas manifestações se
repetissem por diversas capitais e grandes cidades do planeta, inclusive
na América Latina, em São Paulo, Buenos Aires, Cidade do México,
Montevidéu e Santiago do Chile, já em setembro de 2000.

Inaugurou-se, aí, um ciclo de disseminação dos protestos de rua que
tiveram sua marca comum na oposição ao capitalismo financeiro
globalizado, com nuances entre um anticapitalismo frontal e propostas de
regulação do sistema financeiro internacional para a produção de um
``capitalismo mais humano'', como era possível notar em organizações
como a \versal{ATTAC} (Associação pela Tributação das Transações Financeiras para
Ajuda aos Cidadãos), que teve papel decisivo na realização de diversas
dessas manifestações. Por isso, foram pautados por uma agenda que visou
atingir, protestar ou impedir as grandes reuniões de cúpula, fossem de
empresários ou de autoridades governamentais, com manifestações
simultâneas, articuladas via internet, em várias capitais do planeta.
Essas manifestações tinham como objetivo pontual impedir as reuniões de
cúpula e, paralelamente, atacar simbolicamente os grandes ícones do
capitalismo neoliberal globalizado.

No caso da manifestação de Seattle, em novembro 1999, esse duplo
objetivo foi alcançado. A reunião de cúpula teve de ser cancelada devido
aos inúmeros bloqueios que pararam a cidade, e se promoveu a crítica às
organizações internacionais de regulação da economia e do comércio
global. Além do impedimento pontual da reunião, o sucesso das jornadas
de Seattle evidenciou os efeitos deletérios das multinacionais e dessas
organizações às pessoas e ao meio ambiente --- temas que passaram a
fazer parte mais incisivamente da agenda política internacional. Esse
foi um dos mais evidentes resultados da atuação de inúmeros coletivos de
mídia independente e da repercussão na grande mídia, que, nesse caso,
deveu-se também à brutal repressão policial que atingiu os
manifestantes. Hoje, há uma série de estudos acadêmicos, relatos
pessoais, filmes e documentários que registraram e difundiram esses dias
iniciais de ação global contra o capitalismo.

Essa ``nova onda'' anticapitalista possui estreitas relações com as
formulações, práticas e críticas dos anarquismos, seja de maneira direta
ou indireta. Logo, a anarquia, força de contestação política que havia
sido declarada morta pela historiografia oficial desde a derrota da
Frente Popular na Espanha, em 1939 (Joll, 1964; Woodcok, 1981 e 2002),
reaparece surpreendentemente, como em \emph{1968}, de formas diversas em
meio a essas ações. No entanto, não tardará para que a radicalidade
\emph{antipolítica} e de crítica à representação, presente entre os
anarquistas, encontre outros adversários no interior do movimento dos
movimentos que visavam capturá-la em propostas de renovação
institucional e de representação compartilhada.

A forma mais direta e evidente de captura da potência anárquica de
contestação apareceu pela formação de um fórum, articulado pela
``esquerda tradicional'', lançando as bases para a formação de novas
formas de atuação política que pipocaram nos países do Norte, nomeadas
aqui de \emph{nova política}.

Em 2001, apenas a Venezuela possuía um governo que atuava segundo os
objetivos \emph{altermundialistas} postulados por parte dos realizadores
do primeiro Fórum Social Mundial. Seu líder militar e popular, Hugo
Chavez, aparecia como a imagem primeira do que ele mesmo denominou de
``socialismo do século \versal{XXI}'', a realizar-se por meio da solidariedade
entre os povos latino-americanos segundo o programa formulado pelo o que
se denominou de bolivarianismo. Combinava-se a manutenção de certa
democracia formal com a retórica antiestadunidense e a promoção de
políticas sociais compensatórias conformadas em torno de um possível
populismo com pretensões igualitaristas e orientado para o aumento do
consumo.

Com diferenças pontuais, mas convergindo em certa crítica ao mercado
financeiro global em favor de uma solidariedade no Sul, seguiu-se, a
partir daí a composição de governos na região afinados com esse
discurso. Ressalte-se a utilização do termo populismo aqui segundo seus
usos contemporâneos, a saber, como momento decisivo no campo político de
expressão da tomada de decisão de um povo fazendo uso do campo
institucional, mas ultrapassando-o como singularidade histórica que se
expressa em um líder quando preenche o significante vazio produzido pela
crise de representação democrática (Laclau, 2013).

A formulação filosófica do termo, que defende uma visão positiva do
populismo e se vale de formulações da teoria estruturalista na
psicanálise, nomeadamente a de Jaques Lacan, é uma referência largamente
utilizada para justificar os chamados governos de esquerda na América do
Sul, como na Argentina, Brasil, Venezuela, Equador, Bolívia, Uruguai e
mais recentemente no Chile. A noção de \emph{Povo} proposta por Laclau
neste estudo busca se opor à de \emph{multidão} de Negri \& Hardt,
carente, segundo Laclau (Ibidem), do ``sujeito histórico'' capaz de
efetuar o ato político por excelência, quando da tomada de decisão que
romperia com a ordem vigente; portanto, é imperativo um líder capaz de
institucionalidade. As críticas de Laclau às formulações de Negri \&
Hardt serão por estes rebatidas; embora não recusem a
institucionalidade, afirmam que a tarefa da \emph{multidão} é a produção
de um poder constituinte e não uma decisão transcendental de uma
liderança carismática.

Com a vitória de Lula da Silva no Brasil, em 2002, seguiu-se a
institucionalização do \emph{altermundialismo} proposto pelo \versal{FSM} por
meio de um governo que quitou a dívida brasileira com o \versal{FMI} e promoveu o
crescimento de programas sociais, como o Bolsa Família\footnote{Ministério
  do Desenvolvimento Social. ``Bolsa Família''.

  \emph{http://www.mds.gov.br/bolsafamilia}} e o Minha Casa, Minha
Vida\footnote{Sobre o Minha Casa, Minha Vida como programa de acesso à
  moradia, mas também de desenvolvimento econômico como fomento às
  empresas de construção civil no interior do \versal{PAC} \versal{II} (Plano de
  Aceleração do Crescimento) do governo de Dilma Roussef, ver
  \emph{http://www.pac.gov.br/minha-casa-minha-vida}.}. A composição
regional consolidou-se com a ocupação de outros governos de Estado na
região alinhados ao mesmo discurso e práticas políticas, como com a
vitória, em 2005, de Evo Morales na Bolívia e seu chamado Estado
Plurinacional formado em torno da noção de \emph{potência plebeia}
(Linera, 2011), proposta por seu vice Garcia Linera, e a vitória de
Rafael Correa, em 2007, no Equador. Este também saldou a dívida
internacional de seu país e seguiu na construção de seus conselhos de
segurança cidadã\footnote{Sobre o papel de regulação política das
  dívidas nos países, em especial a Grécia, e a forma encontrada por
  Rafael Correa para saldar a dívida equatoriana: \emph{Debtocracy},
  2011, de Katerina Kitidi.
  \emph{https://www.youtube.com/watch?v=qKpxPo-lInk}.}, bem como com o
governo dos Kirchiner na Argentina, a vitória de Mujica no
Uruguai\footnote{Sobre a repressão às manifestações de rua pelo governo
  uruguaio, ver o documentário produzido pela Plenaria Memoria y
  Justiça, em 2013: \emph{Todos en La Mira}.

  \emph{https://www.youtube.com/watch?v=Lp-ckPVgzZ8}} e a de Bachelet, no
Chile. Essa composição regional de governos seguiu marcada pela formação
de enclaves político-militares multilaterais, como a \versal{UNASUL}\footnote{A
  \versal{UNASUL} (União das Nações Sul-Americanas) foi criada em 23 de maio de
  2008, pelo \emph{Tratado Constitutivo da \versal{UNASUL}}. Trata-se de um bloco
  inspirado nos ideais de Simón Bolivar, composto por 12 países e que
  visa estreitar as relações políticas, culturais e militares entre os
  países da América do Sul. Atualmente é composto por Argentina,
  Bolívia, Brasil, Chile, Colômbia, Equador, Guiana, Paraguai, Peru,
  Suriname, Uruguai e Venezuela. \emph{http://www.unasursg.org/}.}, e
esforços de políticas voltadas para a cooperação sul-sul, encabeçada
pelo Brasil e expressa na formação de acordos e cooperações
interestatais como os \versal{BRICS}\footnote{Sigla que não corresponde a um
  bloco juridicamente constituído, ainda que recentemente tenha se
  configurado em torno de um Banco Comum. \versal{BRICS} é a expressão utilizada
  para designar as nações consideradas economicamente emergentes no
  mercado global, composta pelo Brasil, Rússia, Índia, China e África do
  Sul, na sigla em inglês.}. A vitória política alegada por esse enclave
de países manifestou-se na não realização do acordo multilateral de
livre comércio proposto pelos \versal{EUA}, conhecido como \versal{ALCA} (Área de Livre
Comércio das Américas) --- pauta comum do \emph{movimento
antiglobalização} no início do século \versal{XXI}.

Esta é, em linhas gerais, a forma de institucionalização governamental
do \emph{movimento antiglobalização}, por meio do primeiro \versal{FSM}, que além
de deslocar a inicial luta anticapitalista para uma crítica econômica da
mundialização do capital financeiro, serviu como fórum ou plataforma das
políticas de Estado na América Latina, produzindo um claro recuo das
contestações expressas pelo povo de Seattle e suas repercussões
planetárias por meio da \versal{AGP} (Ação Global dos Povos). Essa
institucionalização governamental se conectará na ampliação do mercado
consumidor interno desses países, com o crescimento do endividamento
individual e supervalorização das políticas de Estado, traduzida em
investimentos e ações cada vez mais volumosos no campo da chamada
segurança pública e no da criação de uma variedade de conselhos voltados
à participação popular diretamente atrelada ao Estado. De certa maneira,
será essa uma marca distintiva da nova política dos governos
latino-americanos que tomaram como plataforma o \versal{FSM} --- bastando lembrar
que os encontros seguintes foram realizados em países que respondem a
essa estratégia de cooperação sul-sul, como a Venezuela, Quênia, Índia,
Senegal, e de volta ao Brasil, em 2012.

Nesse processo de institucionalização, completou-se o ciclo de mutação
do \emph{ativista} anticapitalista em \emph{agente social} que milita
segundo valores e metas universais postuladas por organizações
internacionais, como a própria \versal{ONU} e os \versal{ODM}, sintetizados no combate à
pobreza, defesa do meio ambiente e populações tradicionais e garantia de
direitos humanos O agente encontrará seu protagonismo em partidos e \versal{ONG}s
envolvidas na própria criação do \versal{FSM}, como a \versal{ABONG} (Associação
Brasileira de \versal{ONGS}), o Instituto \versal{ETHOS}, o Instituto Polis, os partidos
de esquerda, as comissões da sociedade civil organizada, \versal{ATTAC} e demais
organizações similares (Leite, 2003). A cidadania global, buscada como
fim das mobilizações planetárias de uma emergente sociedade civil
globalizada, nomeada também como \emph{multidão}, encontrou seu campo de
ação no apoio aos governos locais e na profissionalização da militância
em organizações internacionais. No entanto, esse \emph{novo mundo
possível} está desenhado em meio à continuidade e até reforço do
capitalismo, e baseado na valorização do papel corretivo do Estado, além
de dura repressão a quem se opõe ao modelo dominante. Também no plano
conceitual vê-se que, apesar das críticas entre eles, a proposição de
uma \emph{democracia multitudinária} (de Negri \& Hardt) e de um
\emph{populismo igualitarista} (de Laclau) se mostram complementares
quando utilizadas como parâmetro para se analisar as experiências desses
governos de esquerda na América do Sul da primeira década de 2000.
Perfazem assim a lógica de captura dessas práticas no campo da
constituição de uma cidadania global por meio de valorização de
experiências locais, no jargão do próprio movimento, uma democracia
\emph{glocal}. Reinscreveram, assim, as contestações no campo da prática
política como ciência do governo, renovado e democratizado, mas longe da
produção de uma vida outra, afeito às negociações, avanços e recuos
pertinentes à política e das tecnologias de governo que buscam efeitos
de hegemonia.

No ano de 2011, os protestos planetários retomaram fôlego com os
movimentos de ocupação de praças em todo planeta. Atravessaram desde o
Norte da África, abalando governos autoritários locais, passando por
cidades como Londres, Paris, Nova Iorque e Madri. No entanto, as ações e
reivindicações em tais aparições ainda oscilavam entre o uso de táticas
radicais e de estratégias de garantia de direitos em busca de políticas
sociais compensatórias e empregabilidades. Entre a institucionalização
do \emph{movimento antiglobalização} e a retomada dos protestos
planetários, as táticas de sabotagem e de enfretamento radical do Estado
e do capitalismo reapareceram nas ruas de Atenas, na Grécia. Novamente
se recolocou uma contestação apartada da crítica da economia política
clássica e da formulação de proposições de políticas sociais a serem
incorporadas por governos para a afirmação de uma recusa radical, ética
e estética, às formas de vida possíveis sob o domínio do capitalismo e
do Estado.

Como é possível notar na declaração de uma associação dedicada à prática
da ação direta e da sabotagem em meio à revolta grega, a \versal{CCF}, que
expressa, na própria maneira de se pronunciar a firmação do
militantismo: ``depois de mais de três anos de ação intransigente e mais
de 200 ataques explosivos, seguimos acreditando que nossos atos não são
mais que uma gota frente ao imenso oceano do nosso desejo de
transformação (...). Portanto, pertencemos à corrente antissocial da
anarquia, a que rompe não apenas com o Estado, mas também com a
sociedade, porque vemos que o poder não se mantém unicamente com a
violência e as ordens dos quartéis do Estado, mas também pela aceitação,
reconciliação e renúncia da multidão silenciosa, que aprende a ovacionar
as vitórias nacionais, a festejar as vitórias de sua equipe de futebol,
a mudar de ânimo apertando o botão do controle remoto, a enamorar-se com
escapismos de lojas e de modelos artificiais, a odiar os estrangeiros, a
olhar apenas para si e fechar os olhos ante a ausência de uma vida de
verdade'' (\versal{CCF}, 2011: 308-309).

Em junho de 2013, as táticas de confronto radical apareceram no Brasil,
expressando o quanto não se estavam alinhadas com a \emph{nova política}
que anunciava um futuro próspero para o hemisfério Sul, em especial,
para governos de países latino-americanos. Neste caso, mostram-se mais
interessantes na medida em que, diferente da Europa, que assistiu aos
protestos em um cenário de crise e austeridade econômica e desemprego de
jovens, no Brasil eles irromperam em meio a um cenário em que se
acreditava haver prosperidade, desenvolvimento econômico, muito emprego
e projeção internacional do país. Ainda que no interior das
\emph{jornadas de junho} fosse possível encontrar todo o leque que
compõe a chamada \emph{multidão}, a singularidade irrompeu das ações da
tática \emph{black bloc}, que escandalizou um dos mais triunfantes
governos latino-americanos, o brasileiro, da política dita
\emph{altermundialista}, afirmando-se como revolta \emph{antipolítica}
(Passetti \& Augusto, 2014).

As idas e vindas da racionalidade neoliberal investem na fundamentação
de um Estado enxuto, por vezes compreendido como mínimo, apesar de não
haver evidências ou consenso sobre o que seja o mínimo. A própria razão
neoliberal necessita de capital humano, requer investimentos estatais em
educação e saúde, pesquisas científicas financiadas pelo Estado,
ordenações sobre os viventes que são mais do que regulações e
regulamentações. Não se trata de políticas compensatórias, ao contrário,
de conexões entre Estado e sociedade civil organizada em torno do
palpável, posto que a redução de riscos é uma tarefa pessoal do divíduo
portador de direitos inacabados.

É diante desta situação que o duplo composto pelo combate à impunidade e
à corrupção se constitui em práticas de garantias da segurança e bem
comum ao Estado e ao divíduo. Espera-se que as condutas devam ser cada
vez mais criminalizadas --- aceitando-se o déficit no controle dos
chamados crimes e a continuidade de encarceramentos com gestão
compartilhada nas prisões entre governo administrativo penitenciário e o
dos presos, por meio de moralidade e contenção de rebeliões, em torno
também de um quociente tolerável de ilegalismos inerentes. Mas essa
criminalização de condutas atinge outro patamar de seletividade penal
que ultrapassa a conhecida e voltada para populações pauperizadas não
capturadas como exército de reserva de poder pelas instituições
policiais repressivas e de inteligência. O novo combate à impunidade
situa as relações entre Estados, políticos e empresários que dilapidam o
chamado erário público (as riquezas monetárias do Estado) por meio de
negócios econômicos voltados para o empreendimento nacional. Combater a
corrupção passa a ser uma estratégia de redução e formalização normativa
e jurídica, como uma polícia o judiciário, na defesa da gestão dos
gastos públicos.

O \emph{movimento de junho de 2013}, no Brasil, não introduziu somente o
\emph{black bloc}, o fluxo insuportável da inteligência inominável.
Atraiu para seu interior segmentos explicitamente neoliberais que
gestaram de seu interior a disseminação do combate à corrupção de um
governo que se apresentava implicado nos gastos. Funcionando em duplos
fluxos, de formatação da conduta resiliente às adversidades produzidas
por esse governo e de conexão com os descontentamentos políticos
acumulados desde a eleição de 2014, gestaram também no interior daquele
movimento o que repercutirá amplamente em 2015 com o escudo protetor, a
Operação Lava Jato, iniciada em março do ano anterior, e suas variações,
produzindo a crença na Justiça (polícia da justiça) e abrindo
possibilidades para o impeachment da presidente em exercício no primeiro
semestre de 2016. Como o golpe de Estado é inerente ao próprio
funcionamento regular do Estado (o que os analistas enfatizam são apenas
as resultantes repressivas que desembocam em ditaduras, negligenciando
os dispositivos de exceção inerentes ao seu funcionamento) sua
legitimidade, obviamente, oscila entre regularidades e irregularidades
jurídico-políticas, circunscritas ao momento. Todavia, deve-se ressaltar
que não só no Brasil, mas planeta adentro, o combate à corrupção do
Estado e às impunidades se propagam seletivamente como condição
necessária e suficiente para uma normalização da política normal
neoliberal e de sua racionalidade econômica, observando-se o zelo pela
gestão da \emph{coisa pública}, que deve repercutir em redução de Estado
e potencialização do mercado capaz de produzir mais e mais ocupações.
Trata-se de viabilizar a sustentabilidade dos governos de Estado e, por
conseguinte, de empresários, bancos e principalmente do divíduo. Uma
nova política, portanto, em cujo pluralismo não cabe o inominável.

Não apenas os governos, mas também analistas identificados com a
esquerda logo correram aos palanques midiáticos para manifestar seu
horror com o modo \emph{antipolítico} das forças identificadas como
anarquistas, acusando-as de praticarem uma forma autoritária de
protesto, o que as tornaria, segundo esses intérpretes, forças próximas
ao fascismo. Foi o caso do pronunciamento da professora de filosofia da
\versal{USP}, Marilena Chauí, à Polícia Militar do Estado do Rio de Janeiro, e
das análises do cientista político e professor da \versal{UNESP}, Marco Aurélio
Nogueira, publicadas em jornal de grande circulação e depois compiladas
em livro. Este último, em livro sobre as manifestações de junho de 2013,
seguindo pronunciamento da filósofa Marilena Chauí, escreveu sobre os
\emph{black blocs}: ``Não são de esquerda, agem com táticas fascistas,
infiltram-se sibilinamente no meio das multidões para desmoralizá-las''
(Nogueira, 2013: 103).

As \emph{jornadas de junho de 2013} no Brasil, em especial as posições
adotadas em relação aos manifestantes que aderiram à tática \emph{black
bloc}, colocaram um fim à ambiguidade dos governos de esquerda da
América do Sul em relação às lutas que emergiriam com o \emph{movimento
antiglobalização}. Isto é mais evidente em relação ao governo
brasileiro, embora seja verdadeiro, por outras razões, em países como
Venezuela, Bolívia e Argentina. Embora esses governos remetam aos
movimentos sociais do final das ditaduras civis-militares, eles emergem
e se consolidam em meio à ascensão do \emph{movimento antiglobalização},
e logo mostram uma extrema irritação em lidar com essas novas formas de
formular contestações, reivindicações e protestos que valorizam as lutas
pontuais, com ações táticas, descontínuas e dispersas. Formas que são,
em diferentes gradações, anárquicas. Para além do escândalo produzido
pela tática \emph{black bloc}, ascendendo a sanha repressiva e obtusa
desses governos e seus apoiadores, o esgotamento da experiência
governamental se mostrou incontornável no segundo ciclo das lutas do
final da década de 1990.

Se a distância com a velha esquerda é clara e a recusa do governo é
latente, as relações com a \emph{nova política} que se revigora no novo
ciclo da segunda década do século \versal{XXI} mostra a nova ambiguidade. Isso se
torna mais claro quando hoje se afirma a emergência da \emph{nova
política}, que afronta o capital financeiro global em países europeus
como Grécia e Espanha, principalmente. As ascensões de partidos como a
\versal{SYRIZA}, na Grécia, e o Podemos, na Espanha, apontam para uma captura dos
movimentos de contestação na Europa. Em 20 de agosto de 2015, Alexis
Tsipras, primeiro ministro grego pela \versal{SYRIZA}, renunciou alegando que seu
mandato havia se esgotado (durou 7 meses) e a missão de barrar
austeridade havia falhado, mesmo ressaltando que seus esforços impediram
um desastre maior. As novas eleições o reconduziram à chefia de governo,
com acordos firmados junto aos nacionalistas dos Gregos Independentes. A
ativação do chamado populismo contemporâneo indica sinais de poder que
capturam as insatisfações populares na rotina eleitoral ou mesmo em
consultas plebiscitárias eleições \emph{ad hoc} que visam renovar a
confiança nos poderes estatais, ainda que estes cheguem a acordos
diversos dos que referendados em plebiscito. O mesmo vale para o Podemos
na Espanha que, após se tornar uma força eleitoralmente viável, já senta
à mesa com seus antigos alvos, como o \versal{PSOE}, para coalizações de oposição
parlamentar ao \versal{PP}, o partido conservador à frente do governo espanhol.

Bem diverso é o que se passa com os movimentos de ação direta,
vinculados diretamente à anarquia e que recusam a ação institucional.
Diante de jovens vestidos de negro no auditório da \versal{UERJ}, ao analisar o
embate aberto pelos praticantes da tática: ``há muito mais cinismo à
vista, combates e alvos certeiros mirados pelos orifícios dos olhos de
um rosto encapuzado \emph{black bloc}. Hoje é assim; no passado estes
corpos, ou seja, `braços erguidos, peitos intrépidos, pernas ágeis,
capacetes cintilantes em cima das cabeças' se apresentavam com as roupas
de operários. Há muito tempo, talvez desde 1968, enfrentam a ordem com
vestimentas anti-uniformes, anti-identitárias, antipolítica. Se eles são
muitos ou poucos, isso é irrelevante diante do \emph{insuportável}. Eles
trazem no corpo suas utopias e produzem heterotopias. Resistem: os
mascarados, na igualdade e na poética, com olhos firmes e corpos fortes,
utópicos e heterotópicos''\footnote{Trecho transcrito da apresentação de
  Edson Passetti no \emph{\versal{VIII} Colóquio Internacional Michel Foucault:
  Michel Foucault e os saberes do Homem}, realizado entre os dias 22 e
  25 de outubro de 2013 nos auditórios 91 e 93 do campus Maracanã da
  \versal{UERJ}. Ver também Conde (2015).}.

As análises de Michel Foucault (2008a) indicam que a governamentalidade
se faz em relações ascendentes e descendentes que instituem e incorporam
condutas e contracondutas; não se realizam pela centralidade do Estado,
mas encontram nele efeitos de institucionalização ou ``efeitos de
hegemonia''. Os protestos globais que buscam protagonismo e novas formas
de ação política constituem, em âmbito transterroritorial, elementos
dessa governamentalidade, agora planetária, atuando na renovação de
práticas e na produção de demandas políticas que animam governos.
Entretanto, a emergência da tática \emph{black bloc} como potência da
revolta \emph{antipolítica}, rasgando de negro a \emph{multidão}
multicolorida, alerta que diante do governo as práticas anarquistas,
históricas e atuais, seguem na afirmação resistente da \emph{cultura
libertária}.

Resistem aos governos e às capturas que buscam pacificar as lutas dos
libertários. Não se orientam por fins estratégicos, não separam meios e
fins, como certa vez sublinhou Emma Goldman, mas atuam segundo a dureza
das lutas. Não buscam a vitória. Os anarquistas são uma força difusa na
história que, por meio de práticas dispersas e descontínuas, lembram
sempre aos poderes que eles não estão sozinhos. Algo negro sempre estará
à sua espreita, seja para derrubar o velho, seja para não se deixar
acomodar pelo novo. É nesse sentido que são força \emph{antipolítica} em
meio aos movimentos, sejam velhos, novos ou novíssimos. Não se conformam
com políticas pontuais e contingenciais e tampouco estão disponíveis às
capturas.

Quanto às resistências e a produção de novas institucionalizações, este
jogo entre a antipolítica, que afirma uma recusa de governo, e a
produção de uma nova política, que renova práticas entre os governáveis,
apareceu como uma característica marcante das práticas de monitoramento.
Além da dinâmica das próprias manifestações e seus diversificados grupos
com reclames de aprofundamento da democracia e funcionamento em torno da
ideia de horizontalidade e participação, registrou-se, também, a
correção de estratégias, incremento de equipamentos e ampliação
legislativa das forças diretamente controlada pelo Estado.

Na tensão entre \emph{antipolítica} e a \emph{nova política}, o caso
grego expõe o jogo de governo que renova o campo da ação política como
condução das condutas e renovação da referência ao Estado como moldura
jurídica na conformação da dinâmica do jogo de interesses entre forças
díspares (Augusto, 2013). As turbulências e enfretamentos nas ruas da
Grécia e os enfretamentos nas \emph{jornadas de junho} no Brasil, como
exposto acima, possibilitaram a emergência do \emph{ingovernável} e a
caraterização das resistências na sociedade de controle como a atitude
\emph{antipolítica} componente da \emph{cultura libertária}. No entanto,
a conformação governamental conseguida pelo \versal{SYRIZA} e seu acordo de
governança junto às forças partidárias de extrema direita para condução
do Estado grego expõem como as proliferações dos movimentos sociais
contemporâneos ativam um campo da formação de novas institucionalidades.

Diferente do que foi apontado por diversos analistas como expressão de
uma crise política de representação, expressa no crescimento desses
movimentos como se deu no Brasil, em 2013, grande parte desses
movimentos são contabilizados como forças de freio e contrapeso às ações
do Estado (sendo, mesmo que indiretamente, agentes de monitoramentos
como contenção), ainda que não possuam representatividade e legitimidade
institucional no interior da arquitetura institucional do Estado. Eles
se inserem na dinâmica da política também como móvel da política. Ainda
que se apresentem como ações de resistência, cumprem uma função de
monitorar as ações do Estado dentro do contínuo Estado e sociedade
civil.

A emergência do \emph{ingovernável} retoma a pertinência dos anarquismos
nos movimentos contemporâneos, nos quais muitos dos instrumentos dos
monitoramentos computo-informacionais, como câmeras, celulares,
computadores e geolocalizadores, são voltados contra os poderes. Não há
vida libertária na internet (Uehara, 2013), na medida em que esta opera
por compartilhamentos e se constitui como campo privilegiado dos
monitoramentos. No entanto, os anarquistas não viraram simplesmente as
costas para as novas tecnologias de governo. Entraram no combate, sob o
risco da eliminação ou da captura, produzindo suas armas como equipes de
filmagens em manifestações, produção de softwares criptografados,
servidores e Data Centers próprios, como Risup\footnote{\emph{https://we.riseup.net/}}
e Nodo50\footnote{Sobre o Nodo50 e a construção de um Data Center
  próprio para hospedar sites anarquistas e de movimentos sociais fora
  do radar de governo e empresas, ver o Documentário \emph{Nodo50, error
  en el sistema}: \emph{https://vimeo.com/78492174}.}, aplicativos de
comunicação instantânea para usos em ações clandestina, práticas de
enfretamento e sabotagens de empresas de instituições governamentais.
Nada que garanta escapar aos sofisticados controles, mas utilizados como
táticas de uma guerrilha contemporânea. As penalizações a céu aberto
digerem-se, no limite, para contenção dessas formas de recusa. Mais do
que reprimi-las ou anulá-las, há um investimento, por meio dos
monitoramentos, em antecipá-las. Seja pela captura política, como jogo
de demandas e convocação à participação, seja com leis cada vez mais
minuciosas, que visam penalizar condutas ordinárias, como a recente lei
da mordaça na Espanha ou a lei antiterrorismo, sancionada recentemente
no Brasil.

Monitorar é servir à produção de ambientes pacíficos (físicos e
eletrônicos) nos quais cada cidadão é convocado a monitorar também, sob
a suspeição de conspirar contra o ambiente pacífico das cidades ou da
internet. E para se produzir a paz é preciso penalizações. E o próprio
cidadão, emparedado pela lei, ativa os monitoramentos extensivos, que
não visam apenas grupos organizados para fins políticos específicos;
trata-se de securitizar ambientes e, assim, qualquer um,
inadvertidamente, pode ser classificado como um terrorista, sejam grupos
artísticos que se apresentam nas ruas, torcidas de futebol, como ocorre
na Inglaterra, na Alemanha, na Espanha ou mais recentemente no Brasil.
Mas isso não se produz pelo negativo da lei: as penalizações a céu
aberto, escoradas em leis específicas que se refazem em todo planeta,
produzem três formas características da governamentalidade planetária: o
\emph{cidadão-polícia}, a expansão dos monitoramentos e a elastificação
da identificação de perigoso virtual para todos os cidadãos.
Compreende-se como as ações do \versal{GAC} (Grupos Anarquistas Coordenados), na
Espanha, estão voltadas ``contra la democracia''\footnote{Observatório
  Ecopolítica. ``O antiterrorismo específico e seletivo na Espanha''.
  Ano I, n. 7, março de 2016.
  \emph{http://www.pucsp.br/ecopolitica/observatorio-ecopolitica/n7.html}.}.
São os anarquistas que anunciam, de forma inequívoca, as formas
autoritárias das práticas de monitoramento na democracia contemporânea.
Seguem em suas lutas contra o regime dos castigos, colocando-se como
atiradores e alvos do sistema penal, que hoje se amplia para além da
prisão prédio e para além das salas dos tribunais pelas penalizações a
céu aberto.

A atitude de revolta que ativa a \emph{antipolítica}, porém, não se
apresenta apenas nesses momentos de enfretamento. A pertinência da
\emph{cultura libertária} indica a urgência da produção de uma vida
outra, na combatividade nem sempre visível do militantismo, que não
corresponda à gestão dos viventes e a produção do \emph{cidadão-polícia}
próprios ao \emph{dispositivo monitoramento}.

\chapter{Direitos e o dispositivo resiliência }

A noção de dispositivo, sugerida por Michel Foucault (1977, 1977a, 1979;
2001, 2008; 2008a), não se confunde com legitimidade, como já alertara
de forma precisa Deleuze (1988). Tampouco se reduz a uma noção restrita
à sociedade disciplinar, como é comum encontrar em leituras recorrentes
de Foucault que se fazem via sobreposições deleuzianas, pois neste caso,
em especial, um dispositivo estaria enredado ao modelo de rede e à
própria concepção de modelo.

A noção de dispositivo utilizada aqui se encontra, também, distante das
incursões de Agamben (2009) em torno dela, e das derivações de
aplicações realizadas por seu intermédio que trafegam no duplo
indissociável entre a profanação e a sacralização. Dito de outro modo,
nas proliferações de direitos, enquanto o que trafega na
complementaridade do duplo das significações e empoderamentos, ou ainda,
na perseguição do que foi separado e dividido (leia-se também alienado)
e deve (direito-dever) ser restituído (leia-se, também, nova emancipação
ou nova subjetivação ou contracondutas). O castigo, ao permanecer,
retraduz por outras vias a lógica do sacrifício, em função dos
\emph{agenciamentos de dispositivos} e \emph{de contradispositivos}, bem
como suas correspondentes atualizações de restituição do uso comum
restauradas, como quer Agambem.

Um dispositivo, tomado a partir da análise genealógica, dispõe, põe em
cena por meio de seu exercício de forças em luta, distante da gênese do
soberano, não circunscrita meramente à relação saber-poder, mas
interessada nas relações governo-verdade e ao que ela forma, dá forma e
põe a funcionar.

Não se trata de representação ou do enredamento entre os jogos de
legitimidade e legalidade, nem mesmo de relativizações em torno de
interpretações que retraçariam o conforto da hermenêutica jurídica,
tampouco do apaziguamento de resistências como pretendem as práticas
voltadas à fomentação da resiliência.

Um dispositivo dispõe e põe a funcionar conjuntos de práticas voltadas
ao próprio funcionamento da política. Não como um a priori, pois neste
caso estaríamos sob o império da lei à espera de legitimidade, como
requer o Estado democrático de Direito, e, simultaneamente, nos seus
interstícios, naquilo mesmo que subjaz a ele próprio, pelas conformações
de ditaduras, de genocídios e etnocídios, extermínios, subordinações,
assujeitamentos. E nessa esteira, um arcabouço infindável de normativas
e diretrizes, projetos e programas que pretendem conjurar o que é
insuportável para o direito e para a política, em compasso equivalente
ao que se mostra oportuno e capaz de suportabilidade.

Daí, mais uma vez, a relevância atual que assume o conceito de
resiliência como aquele que se mostra profícuo em transitar como o
inverso do que resiste e configura a expressão de capturas. O próprio
termo ``resiliência'' foi capturado da física do século \versal{XIX}. O conceito
foi inicialmente cunhado, em 1807, por Thomas Young a partir de seus
experimentos para medir a resiliência de materiais diversos submetidos a
um determinado impacto, sua suportabilidade e capacidade de absorção de
energia, deformação e elasticidade para voltar à sua condição anterior
ao impacto sofrido (Young, 1845).

Mas uma captura nunca é fortuita. A ida aos experimentos de Young se
mostrou conveniente, pois lá se encontra a possibilidade de se transpor
para os dias atuais a captura como capacidade de absorção e,
simultaneamente, superação de energia de forças em movimento. ``Vimos
que o espaço através do qual um corpo pode ser distendido ou comprimido,
sem qualquer alteração permanente da forma, constitui a sua dureza: a de
que a sua força, ou a resistência final que ela proporciona, depende da
magnitude conjunta de sua dureza e elasticidade ou rigidez, e que a sua
resiliência, ou o poder de superar a energia ou o impulso de um corpo em
movimento, é proporcional à força e tenacidade conjuntamente'' (Ibidem:
482).

Mas antes do termo, ou do conceito, a palavra.

A palavra resiliência é antes de tudo uma palavra religiosa. Sua
etimologia provém do latim e conjuga em uma só o substantivo
\emph{resílio} (o que cobre de proteção) e o verbo \emph{resálio}
(voltar atrás ou sobre si mesmo, ou voltar sobre o mesmo). Suas
derivações são inúmeras: desdizer-se, encolher-se, distender-se. Palavra
religiosa e renovada pela ciência, pela política. O resílio também é o
ligamento da valva de um molusco, ou o que pode ser designado como a
relação indissociável entre religião e política.

O resiliente e a resiliência são o que suporta, tolera, resigna,
restaura, protege, pacifica e ao mesmo tempo reconfigura o sujeito de
direito em portador de direitos inacabados, muitas vezes conjugando-os,
pois a própria resiliência se mostra como um catalizador. Tomado de
forma literal, é um elemento que acelera uma reação de adequações
moduláveis. Na perspectiva analítica, a resiliência se mostra como
característica reativa, e não resistente, por excelência.

O final do século \versal{XX} sinalizou para um relevante deslocamento voltado à
disputa pelo controle da segurança, retraduzida pelos direitos
medicalizados, respaldando a continuidade da propriedade, do Estado e da
vontade de governo sob o advento dos direitos de quarta geração.

Os investimentos políticos, as convenções, tratados, normativas e
protocolos que no início do século \versal{XXI} englobam do nano ao sideral,
encontram proveniências em uma série de cuidados histórico-políticos e
jurídico-políticos que, inicialmente, voltaram-se para as crianças, as
mulheres, os refugiados, os trabalhadores e o meio. Entretanto, hoje,
encontram-se distantes da vigilância disciplinar e diante de novas
convergências de controle conectadas por combinações sinérgicas,
designando um trabalho conjunto compartilhado entre o nano, o bio, o
info, o cogno, o neuro e uma novíssima velha psiquiatria.

Esta condição retraduz a vida da política, sob as condicionalidades de
certificação do cidadão pleno e em aperfeiçoamento (definição
jurídico-político-institucional), da conduta do portador de direitos
inacabados (noção analítica e histórico-política), combinada à ampliação
política contínua de gestões compartilhadas de direitos por seus
próprios portadores.

Como os cidadãos deixam de orientar-se pela iminência da guerra e pela
defesa da sociedade, para perseguirem a cultura de paz e da tolerância
na defesa de seus direitos e para o bom governo do planeta?

O conceito de qualidade de vida conecta-se ao de vulnerabilidade,
atravessados pela cultura da paz e tolerância, não pela necessidade ou
carência, mas pelo investimento na capacidade de cada um em conduzir sua
vida sabendo-se governado.

A primeira década do século \versal{XXI} marcou a proliferação ampliada de
programas de direitos e monitoramentos transterritoriais, conectados a
projetos regionais e locais voltados ao investimento político da
denominada cultura de paz. Redimensionou-se a guerra como paz, por meio
de monitoramentos de violências, sob a forma de combinações de direitos
e medidas de contenção de vulnerabilidades pela ampliação atual de uma
justiça planetária restaurada, que passa a enfatizar a necessidade de
resiliência. A resiliência, desdobrada de estudo inicial sobre
neurociências, crianças e jovens e psiquiatria do desenvolvimento, deve
ser compreendida pelo deslocamento que começa a ocorrer entre a
designação ``cultura da tolerância'' e ``cultura de paz'' para a de uma
inevitável ``cultura resiliente'': a resiliência, que passou a ser
chamada de ``ética do futuro'', sinaliza para adaptações que se
aproximam das adequações ao atual estado das coisas.

A resiliência hoje está presente em imenso escopo e cobre áreas e
práticas, tornadas campos vazados, que vão da física às neurociências e
à cultura de paz; da psiquiatria à psicologia; da biologia molecular e
mapeamentos genômicos à ecologia e gradações do desenvolvimento
sustentável; dos índíces de desenvolvimento humano (\versal{IDH}), elaborados
pelo Programa de Desenvolvimento das Nações Unidas --- \versal{PNUD}, à religião
isolada ou combinada com novos ajustes científicos; dos investimentos em
cidades resilientes ao reforço da resiliência voltado às medidas
protetivas de meio ambiente, das aferições e projeções de metas e novas
metas da \versal{ONU} que culminaram em 2012 no documento preparatório da Rio+
20, \emph{Povos Resilientes, Planeta Resiliente: um Futuro Digno de
Escolha.}

A resiliência apontou para um espraiamento em conjuntos dilatados de
vulnerabilidades e ambientes de direitos, que incluem mulheres, gays,
pretos, indígenas, famélicos, deficientes, velhos, ecossistemas de meio
ambiente, aferições de mudanças climáticas, avaliações internacionais de
``desastres naturais'' e de campos de refugiados.

O investimento político em torno das denominadas vulnerabilidades passa,
também, pela ênfase na segurança e congrega construções de ambientes
resilientes por vias restaurativas diante do degradado, adaptações
plásticas vinculadas à qualidade de vida e vulnerabilidades convertidas
em superações às denominadas ``adversidades nocivas'' que recobrem
variados espaços, preservando a continuidade do capitalismo e do Estado,
assim como redimensionamentos da política.

A relação entre política, direitos e resiliência aponta para efeitos de
seu funcionamento histórico-político em compasso com a proliferação de
direitos ao compor investimentos preferenciais em restaurações do
governo do vivo frente à iminência ou consecuções de sua
\emph{degradação}, que se inicia pela valorização positivada da
\emph{sobrevivência}. Reforça-se, assim, o itinerário de violação e
defesa de direitos por revestimentos protetores; reatualiza-se e
diversifica-se a ideia de prevenção colocada, também, como base da
resiliência; articulam-se normativas internacionais e programas
transterritoriais voltados à formação de resilientes vinculados à
cultura de paz.\footnote{Ver bibliografia e arquivos, com destaque para
  os principais temas que se encontram, em particular, no fluxo direitos
  no site ecopolítica.
  \emph{http://www.pucsp.br/ecopolitica}.}

A resiliência explicita-se como um catalizador para a contenção de
resistências e se articula de forma indissociável ao conceito de
\emph{vulnerabilidade} e \emph{adversidade} ao se mostrar como um
elemento imprescindível ao lado da \emph{sustentabilidade,} da
tolerância e do \emph{empreendedorismo}.

Uma das entradas na política do conceito de resiliência situa-se no
pós-\versal{II} Guerra Mundial, cujo estudo referência situa esta entrada
encontra-se em Werner \& Smith (1982). Werner, munida por arsenais
provenientes da psicologia, psiquiatria e antropologia, acompanhou por
mais de três décadas 698 crianças nascidas na ilha de Kuai, no Hawaí, e
sua pesquisa situa como se chegou ao conceito de resiliência a partir de
sua aplicação sobre crianças e jovens. Este extenso estudo sinaliza para
um possível deslocamento biopolítico da relação medo-contágio-risco para
a combinação ecopolítica entre resiliência, vulnerabilidade e proteção,
onde não se abdica do medo e do castigo e cujo corte inicial incide mais
uma vez em crianças. Os efeitos do conceito de resiliência ganharam
maiores contornos na década de 1970, destacando-se, em especial, os
estudos publicados por Edwin James Anthony (1987). Os estudos desse
neuropsiquiatria, a partir da década de 1970, são tomados regularmente
como uma entrada na resiliência pela concepção ainda marcada de
invulnerabilidade. Por sua vez, a introdução do conceito de resiliência
na literatura denominada ecológica, que posteriormente passa a ser
designada por ``contextos ecológicos'' e se consolida sob a nomenclatura
de ``sócio ecológicos'', vai se dar, como vimos, pelos estudos de
Holling, a partir do crivo liberal, simultaneamente à emergência do
neoliberalismo, quando ele procura mostrar a diferença entre sistemas
estáveis e sistemas resilientes. Estes contornos trouxeram avolumadas
amplificações que atravessam seu funcionamento político no planeta na
década de 1990 e na primeira década do século \versal{XXI}.\footnote{Ver, em
  especial, no site ecopolítica, referências em Documentos Resiliência.

  \emph{http://www.pucsp.br/ecopolitica/documentos/direitos/resiliencia.html}}

O pós-guerra trouxe o redimensionamento dos direitos humanos que se
consolidaram na \versal{DUDH}\footnote{Declaração Universal dos Direitos Humanos.
  Em especial, artigo \versal{XVII}: ``1. Toda pessoa tem direito à propriedade,
  só ou em sociedade com outros. 2. Ninguém será arbitrariamente privado
  de sua propriedade''.}, não repetindo, mas realocando alguns artigos
da declaração de 1789, dentre os quais se revestiu de proteção aquele
que já no século \versal{XVIII} designava a propriedade como direito sagrado na
Declaração dos Direitos do Homem e do Cidadão de 1789\footnote{Assembleia
  Nacional da França (1789). Declaração dos Direitos do Homem e do
  Cidadão, 1789.
  \emph{http://pfdc.pgr.mpf.gov.br/atuacao-e-conteudos-de-apoio/legislacao/direitos-humanos/declar\_dir\_homem\_cidadao.pdf}
  .Ver em especial artigo \versal{XVII} ``Sendo a propriedade um direito
  inviolável e sagrado, ninguém pode ser dela privado, a não ser quando
  a necessidade pública, legalmente reconhecida, o exige evidentemente e
  sob a condição de uma justa e anterior indenização''.}. E parece não
ser coincidência que até mesmo o número dos artigos em ambas as
Declarações coincidam.

Vale sublinhar a contundente demolição serial realizada por Proudhon ao
regime de propriedade como o que viria estabelecer o direito como
superação da injustiça, ao mostrar que o direito se funda sobre a
injustiça por uma inversão assimétrica. E que diferente do que se
apregoa a propriedade é o roubo, visto não existir acumulação natural,
mas sim expropriação de forças por meio da força e da astúcia. O Direito
é a expressão de que alguém esmagou alguém. A tentativa de dirigir a
resistência das forças pelo Estado e a conformidade letárgica produzida
pela religião como alertara Proudhon, encontra na suportabilidade
(resiliência e tolerância) sua tradução maior na \versal{DUDH}.

Se fosse possível indicar um território, tornado frase, capaz de
condensar o ponto de convergência dos discursos moderno e contemporâneo
acerca da tolerância, seria este: a tolerância é uma conquista. Se fosse
possível apontar um domínio no qual este território, do século \versal{XVII} ao
\versal{XXI}, refestela-se no discurso em defesa da tolerância, seria este: a
conquista de direito. Se fosse possível tocar no campo discursivo da
tolerância no qual o domínio se constitui a partir do território, seria
este: a natureza humana. Mas como na história não há ``se'', é preciso
ir de encontro ao espaço de enfrentamento deste território, domínio e
campo, lá onde eles se fazem rasteiros e brutais, imperceptíveis e
legíveis, ordinários e grandiloquentes: na educação para a obediência.
Os termos território: noção jurídico-política; domínio: noção
jurídico-política e campo: noção econômico-política, são compreendidos a
partir da sugestão fornecida por Michel Foucault. Isto não significa se
voltar para uma reflexão filosófico-jurídica, mas a uma análise
histórico-política travada no \emph{espaço} e distante, tanto do recorte
de períodos, etapas e idades temporais, quanto de uma hermenêutica do
direito. ``A descrição espacializante dos fatos discursivos desemboca na
análise dos efeitos de poder que lhe estão ligados'' (Foucault, 1979:
159).

O cultivo do medo ao castigo é a base da educação para a obediência. Ao
contestá-la William Godwin (1945), no século \versal{XVIII}, afirma que a questão
da punição talvez seja a mais fundamental da ciência política. Sua
análise mordaz sobre o castigo descreve como a prevenção assume o nome
de justiça penal, ou punição. A falácia da prevenção geral reside,
segundo ele, em seu próprio efeito reverso de eficácia, ao converter
quase todos em uma massa de covardes. A covardia tornada obediência.

O deslocamento do direito penal clássico para o moderno, além de compor
uma das proveniências da prevenção geral mostra-se como um dos efeitos
da humanização das penas, presente no discurso dos reformadores do
século \versal{XVIII}, ao defenderem a individualização e proporcionalidade da
pena ao delito, concomitante à gestação da prisão moderna e imediata
constatação de seu fracasso. Proudhon, no século \versal{XIX}, atento a estes
efeitos realiza uma crítica demolidora para o momento em que vivia e
presentifica a prática abolicionista como uma das atualidades vigorosas
da atitude anarquista. ``O crime faz a vergonha e não o cadafalso, diz o
provérbio. Apenas por isso, pelo fato do homem ser punido mesmo que o
mereça, ele se degrada: a pena o torna infame não em virtude da
definição do Código Penal, mas por causa da falta que motivou a punição.
O que importa, pois, a materialidade do suplício? O que importam todos
os sistemas penitenciários? O que fazeis deles é para satisfazer a vossa
sensibilidade, mas eles são impotentes para reabilitar o infeliz que
vossa justiça golpeia. O culpado, uma vez dobrado pelo castigo, é
incapaz de reconciliação; sua mancha é indelével e sua danação eterna.
Se as coisas pudessem ocorrer de outra maneira, a pena deixaria de ser
proporcional ao delito e não seria mais do que uma ficção, não seria
nada'' (Proudhon, 2003: 427).

A atualidade das análises de Godwin e Proudhon se encontra, também, em
incidir sobre o próprio princípio da tolerância que exige uma relação
assimétrica de comando do superior e obediência do inferior. Neste
sentido, ambos explicitam os efeitos de direitos, descobertas,
submissões e extermínios provenientes da tolerância como conquista.
Diante da alocação na \versal{DUDH}, dentre outros, dos artigos \versal{III}, \versal{IV} e \versal{V}
(``Artigo \versal{III}. Toda pessoa tem direito à vida, à liberdade e à segurança
pessoal. Artigo \versal{IV} Ninguém será mantido em escravidão ou servidão, a
escravidão e o tráfico de escravos serão proibidos em todas as suas
formas. Artigo \versal{V}. Ninguém será submetido à tortura, nem a tratamento ou
castigo cruel, desumano ou degradante''), Proudhon permanece atual. ``Se
eu tivesse de responder à seguinte questão: \emph{O que é a
escravidão?}, e com uma única palavra respondesse: \emph{É o
assassinato}, meu pensamento seria, em princípio compreendido. Eu não
precisaria de um longo discurso para mostrar que o poder de suprimir do
homem o pensamento, a vontade, a personalidade é um poder de vida e
morte, e que fazer o homem escravo é assassiná-lo. Por que, então, a
esta outra pergunta: \emph{O que é a propriedade?}, não posso responder
do mesmo modo: \emph{É o roubo}, sem ter certeza de ser compreendido,
ainda que essa segunda proposição só seja a primeira transformada?''
(Proudhon, In Passetti \& Resende; 1986: 32).

Entretanto, corroborou-se a base da própria liberdade capitalista,
consagrando o proprietário e mantendo intacta a concepção liberal de
existência do direito moderno e contemporâneo, a ser aplicável e
fiscalizável em qualquer lugar, como proteção da humanidade. Promoveu-se
espaço, ao mesmo tempo, para uma das entradas de referência nas
normativas internacionais dos direitos de minorias, fundamentados na
introdução do termo dignidade da pessoa humana na \versal{DUDH} (Preâmbulo, no
primeiro e quinto parágrafos: ``Considerando que o reconhecimento da
dignidade inerente a todos os membros da família humana e de seus
direitos iguais e inalienáveis é o fundamento da liberdade, da justiça e
da paz no mundo,'' (...) ``Considerando que os povos das Nações Unidas
reafirmaram, na Carta, sua fé nos direitos humanos fundamentais, na
dignidade e no valor da pessoa humana e na igualdade de direitos dos
homens e das mulheres, e que decidiram promover o progresso social e
melhores condições de vida em uma liberdade mais ampla''), pretende
funcionar como um receptáculo dos chamados direitos fundamentais, a
partir de um tríptico de direitos com ênfase em refugiados, mulheres e
crianças, enquanto efeito da guerra e do próprio funcionamento da
política\footnote{Para entrada ampla em documentos e bibliografia de
  Cultura de paz:

  \emph{http://www.pucsp.br/ecopolitica/documentos/cultura\_da\_paz/cultura\_paz.html}

  Para entrada específica, em normativas:

  \emph{http://www.pucsp.br/ecopolitica/documentos/cultura\_da\_paz/normativas.html}

  Para documentos emblemáticos sobre resiliência:

  \emph{http://www.pucsp.br/ecopolitica/documentos/direitos/resiliencia.html}
  .}.

No caso de crianças e jovens, em especial, é emblemática a Declaração
dos Direitos da Criança de 1959 (\versal{ONU}, 1959) como pretensão de
ultrapassar a Declaração dos Direitos da Criança de 1924 (Liga das
Nações, 1924), refazendo circuitos exegéticos de reforma da prevenção
geral para a condição da própria existência e permanência do direito, do
Estado e da política.

Seria possível estabelecer uma cronologia evolutiva de normativas
oriundas da \versal{DUDH} que apontam para iniciais investimentos de defesa de
direitos no pós-\versal{II} Guerra Mundial voltada para mulheres, crianças e
refugiados. Isto bastaria caso se estivesse restrito a um campo de
estudo jurídico-político (Altavila, 2004) ou mesmo apenas voltado à
hermenêutica jurídica. Entretanto, em uma perspectiva de análise
histórico-política, o que se destaca ao indicar o estudo de Werner como
porta de entrada política da resiliência no pós-\versal{II} Guerra Mundial é o
fato de que ele trouxe para a análise a possibilidade de demarcar um
conjunto, que passará a ser alvo de investimento de direitos, formado
por crianças, mulheres e refugiados como efeitos da guerra e do próprio
funcionamento da política. Porém, o alvo de corte irradiador sobre a
qual ele investe deriva das devassas sobre os corpos de crianças e
jovens. E mais uma vez serão estes que servirão de balão de ensaio para
reafirmar a grandiloquência do humano verdadeiro por suas dignificações
via \versal{DUDH}; para projetar o futuro da paz a partir da tolerância que
encontra sua inflexão similar de suportabilidade na resiliência
acompanhada de cultura da paz; oferecer ao conceito de desenvolvimento
os contornos iniciais que trafegarão em suas construções posteriores de
sustentação (suportabilidade, sustentabilidade) ao que passará a ser
designado de desenvolvimento humano e desenvolvimento na primeira
infância voltados as melhorias do governo do vivo restaurado.

De outra feita, é possível demarcar que a atenção dada a este estudo
como referência na bibliografia sobre resiliência possibilita encontrar,
também, a reverência sistemática à valorização da situação de conjunto
de vulneráveis. Foi o estudo inicial de Werner a respeito da resiliência
que apontou as pistas da passagem, posterior ao deslocamento da situação
de risco para a condição de vulnerabilidade que exigirá mecanismos de
proteção pela continuidade mais da vida do direito e dos direitos do que
de gente de carne e osso. Não surpreende que gente se esfumace sob o
conceito de \emph{dignidade da pessoa humana}, conforme termo
introduzido pela própria \versal{DHU}. Será, também, pela resiliência em sua
articulação com o conceito de \emph{qualidade de vida} que se
redimensionará posteriormente a situação de risco em situação de
vulnerabilidade (conceito sedimentado na década de 1990) equacionada
pelos três conjuntos de investimentos iniciais derivados da \versal{DHU},
sinalizando o deslocamento dos direitos de minorias para a articulação
de conjuntos de vulneráveis e portadores de direitos inacabados.

O conceito de qualidade de vida firma-se de forma abrangente nos anos
1990, concomitante à disseminação do Programa de Tolerância Zero.
Ganhará uma gama ampla de definições onde cabe destacar a adotada pela
Organização Mundial da Saúde (\versal{OMS}), elaborada pelo Grupo de Qualidade de
Vida da Divisão de Saúde Mental da \versal{OMS}: "a percepção do indivíduo de sua
posição na vida no contexto da cultura e sistema de valores nos quais
ele vive e em relação aos seus objetivos, expectativas, padrões e
preocupações'' (\versal{WHOQOL} \versal{GROUP}, 1994). Entretanto, a literatura médica se
refere ao termo, pela primeira vez, na década de 1930, e atribui,
regularmente, o uso inicial da expressão, ao discurso político do
projeto de \emph{Grande Sociedade} do presidente dos \versal{EUA} Lindon Johnson,
em 1964 ao dizer que ``os objetivos não podem ser medidos através do
balanço dos bancos. Eles só podem ser medidos através da qualidade de
vida que proporcionam às pessoas'' (Seidl \& Zanon, 2004; Fleck, Louzada
et al, 1999).

A noção de portador de direitos não se confunde com o conceito
jurídico-político liberal, na esteira de Kant ou no escopo weberiano, do
direito positivo de Kelsen (1999) pela figura do ``sujeito portador de
direitos'' que opera pela distinção entre pessoa física e pessoa
jurídica. Não se dilui na abstração transcendental pleonástica de pessoa
humana (\versal{DUDH}). Não se remete também ao conceito jurídico-político
marxista de Pachoukanis (1988), e se distancia também do conceito atual
de cidadão pleno (definição jurídico-política institucional).
Entretanto, não ignora os efeitos políticos das condicionalidades de
certificação deste último, mas volta-se para o portador de direitos no
presente como a condição que efetiva o prolongamento da política, que a
resguarda e a assegura. Sob esta condição o portador de direitos
retraduz a vida da política sob a conduta do portador de direitos
inacabados orientada pela ampliação política contínua de gestões
compartilhadas de direitos por seus próprios portadores.

Assim o \emph{portador de direitos inacabados} decorre da noção de
\emph{pletora de direitos} e a de \emph{divíduo multifacetado por
direitos inacabados} ao demarcar a relação direta entre resiliência e
sustentabilidade na ecopolítica. ``Não se trata de um governo da
população como na biopolítica, mas de governo com cada população para
que viva agrupada, móvel, resiliente, participativa, em função de cada
um, de seu agrupamento e da conservação do planeta. Indivíduo,
redimensionado em divíduos por pletora de direitos e identidades,
compondo variadas subjetividades que possibilitam conexões temporárias,
paradoxalmente tênues e consolidadas, e que produzem sim a
dessubjetivação no indivíduo autônomo, mas que a torna irrelevante,
quando se considera que este indivíduo se metamorfoseia em divíduos com
variadas subjetividades. É preciso viver para fora e por dentro, do lado
de fora e conectado com vários ambientes resilientes, o Estado e
organizações transterritoriais: é preciso fazer parte de tecnologias
sociais, ser reconhecido e premiado, mas também saber fazer negócios
sociais sustentáveis e estar ocupado. (...) A sustentabilidade
encontra-se conectada à maneira pela qual se conserva o ecossistema, se
produz combatendo a degeneração anterior provocada pelo capitalismo e se
humaniza a política e a sociedade. Exige moderação, ou seja, o modo de
encontrar moderação aninha-se nas práticas da resiliência. A
sustentabilidade é uma prática que vai do divíduo multifacetado por
direitos inacabados e inexequíveis à economia verde, ao cálculo político
a respeito da democratização dos Estados, da sociedade civil e das
relações com a natureza, reformas nos usos de recursos naturais, e
confirma a prevalência das forças que defendem as práticas de
conservação diante das de preservação da natureza'' (Passetti, 2013:
13-27).

Kafka (1997) dizia que diante da lei há sempre um porteiro. Se na
biopolítica a lei se transformou em norma que passou a circular da vida
do corpo individual à vida da população, como espécie e ser vivo
(Foucault,1988), hoje na ecopolítica o porteiro se transfigurou em porta
e em portador. A conexão entre resiliência e o conceito de empoderamento
em suas várias facetas o confirma\footnote{Ver empoderamento no fluxo
  direitos do site ecopolítica:

  \emph{http://www.pucsp.br/ecopolitica/documentos/direitos/emponderamento\_direito.html}
  .}.

A recuperação da palavra da palavra empoderamento, sinalizou para
proveniências que numa primeira entrada histórico-política do termo
equalizam empoderamento à emancipação, remetendo inicialmente ao campo
liberal da liberdade religiosa e sugerindo uma relação intrínseca à
defesa da tolerância, que emergiu na política moderna como valor e
conceito que atravessam práticas de manutenção política de assimetrias,
e, portanto de suportabilidade, repercutindo em sua atualização no
âmbito dos direitos, também, pelo viés religioso e do castigo presente
na resiliência. Constata-se mais uma vez, de outra feita, que não há
Estado apartado da religião, por mais que os defensores da secularização
do Estado ou os militantes que clamam por um Estado verdadeiramente
laico afirmem o contrário.

No século \versal{XX}, o termo empoderamento continuou atrelado à ideia universal
de justiça ao desdobrar-se, num primeiro momento, em correlato de
emancipação social. Nos \versal{EUA}, na segunda metade do século \versal{XX} o termo
empoderamento teve um uso crescente, principalmente por movimentos civis
de minorias, com ênfase para pretos (Solomon, 1976) e mulheres (Costa,
2000). Sob repercussões no campo da psicologia (Zimmerman, 1990; 1995)
viria a se aproximar na década de 1970 da auto-ajuda e na década de 1980
da psicologia comunitária, para a ênfase recair na década de 1990 nos
movimentos em torno do direito de cidadania. Consolidou-se, por fim, no
interior do pluralismo democrático e foi absorvido pelo discurso do
desenvolvimento alternativo conjugando, sob a ideia de
auto-sustentabilidade, a liberdade política (Friedmann, 1996; Sen, G.
2000).

Entre os chamados ``alternativos'', o empoderamento é um meio para a
afirmação das chamadas políticas públicas e de combate à pobreza. Aos
liberais, interessa um empoderamento incisivo no âmbito privado, por
meio de ações voluntárias individuais que reduzem a intervenção estatal
direta, mas não prescinde de sua articulação. Contudo, é possível
afirmar que, em ambos os casos, o empoderamento aparece como chave para
o desenvolvimento e a constante e ininterrupta melhoria da democracia.
Passa a ser imprescindível a formulação de agendas, seja para pretos,
mulheres, jovens, indígenas, etc., o empoderamento, noutras palavras, é
ter poder de agenda (Horochovski, 2006).

O empoderamento aparece diretamente vinculado à democracia e à
participação. Constitui-se como uma forma de produzir novas
institucionalizações, mais participativas, estreitando as relações entre
as instituições estatais e a chamada sociedade civil, e alargando fóruns
representativos com base no pluralismo, no intuito de eliminar a pobreza
e as desigualdades formais e melhorar a qualidade de vida. Insere-se,
ainda, nos debates relativos à ecologia social (Horochovski, 2006) e,
diante das situações de desastres naturais, onde se tem um quadro
denominado de desempoderamento, é incentivado como meio mais eficaz para
se superar a situação catastrófica e possibilitar, ao indivíduo
empoderado, que ele reaja a esta situação. Portanto, nada mais comum do
que aproximar empoderamento à resiliência.

O tema do empoderamento mostra, ainda, articulação difusa com a
resiliência em inúmeros campos dos direitos. Destaca-se sua vinculação
emblemática no campo da saúde mental e, paradoxalmente, pelo discurso
que proveio do campo da luta antipsiquiátrica e que foi conformado no
interior do movimento antimanicomial em aberta defesa dos \versal{CAP}s e do
respectivo empoderamento de seus usuários em favor da melhoria das
políticas públicas (Vasconcelos, 2003). Talvez seja possível afirmar que
o ponto de atração entre resiliência e empoderamento articula-se pela
pró-atividade do reativo (resiliência) e a compensação do ressentido
(empoderamento), num duplo recíproco atravessado pela cultura do
castigo.

O atual investimento político na resiliência como traço inerente ao
portador de direitos inacabados em simultânea busca de empoderamento
sinaliza que a resiliência, também reveste fascismos (Passetti, 2013a);
explicita a vontade de poder ou da paixão tornada amor ao poder
(Foucault, 1996), vontade que se volta contra a vida (Nietzsche, 1998),
que não se confunde com vontade de potência e muito menos se aproxima do
gesto cruel de apetite de vida (Artaud, 1984).

A noção de portador de direitos inacabados aponta para uma forma de
governamentalidade específica da configuração atual dos direitos que não
traz nem dá forma a um direito novo antissoberania. Os vários portadores
variam e se mesclam, sequenciam-se e multiplicam-se e se justapõem na
mesma proporção equivalente da proliferação dos direitos inacabados: de
deficiência, portador de doenças graves, portador de transtornos
mentais, portador de câncer, portador de vírus da \versal{AIDS}, natureza
portadora de direitos, animais portadores de direitos, portador de
necessidades especiais por excelência em suas inúmeras variações,
portador de necessidades especiais no mercado de trabalho, portador de
necessidades especiais na escola, portador de necessidades especiais no
convívio social.

Como vimos, se diante da lei há sempre um porteiro, e não há dúvida que
hoje o direito continua sendo o do mais forte. A porta também se
multifacetou: da porta se transfigurou o porteiro e se retraduziu a
dívida eterna do portador em simultaneidade com a porta e o porteiro.
Urge subverter a paisagem e abrir passagem para percursos de vida livre.
Mas é preciso limar pacientemente a porta e o porteiro que se configuram
no portador de direitos inacabados.

Na biopolítica a teoria da degeneração foi o pano de fundo para a
contrução do anormal. O \emph{degenerado} era o \emph{portador de
perigo} para a sociedade (Foucault, 2001a), identificado como o anormal
por excelência. Ele era a expressão do perigo para a vida do corpo e da
espécie. Seus baixos começos se encontravam na criança indisciplinada,
insubmissa, indócil, incorrigível e seu ápice se configurou, a partir do
século \versal{XIX}, nos anarquistas que tomam a própria vida como objeto de luta
indisponível para ser governada e a cultura do castigo, com seus
desdobramentos punitivos como seu principal alvo de ataque.

Por sua vez, se a teoria da degenerescência encontrou na criança, em seu
corpo e em seu sexo, o campo oportuno para produzir a partir da
ingerência na família efeitos biopolíticos naquilo que emergia como
conceito de população, subjacente ao exercício da estatística e da
economia política que historicamente surgiram como saberes de Estado,
isso culminou no século \versal{XX} com os campos de concentração, matando
milhões e, simultaneamente, usando o próprio exercício de poder
biopolítico para fazer funcionar a gestão calculista da vida: não era
preciso esperar pelas bombas atômicas, para constatar que se estava
diante do reverso inerente e complementar da própria biopolítica, o de
matar a própria vida.

Qualquer tentativa de positivar o conceito de população, indispensável
ao de multidão em seu redimensionamento da massa, e a noção de
biopolítica enquanto resistência, como fazem Negri e Hardt, e tantos
outros em sua esteira, não passa de uma colonização grosseira nas noções
e análises de Michel Foucault em torno da genealogia do racismo e dos
anormais. E se a criança incorrigível, indócil, insubmissa foi, por sua
vez, aquela tomada também, como inassimilável no sistema de educação
normativa, como apontou Foucault (2001), mais uma vez não era preciso
esperar por isso, pois como atentaram os anarquistas é sobre o corpo
delas que se inicia o regime e a cultura do castigo.

Não é fortuito que uma das entradas significativas do conceito
degradação no século \versal{XX}, no pós \versal{II} Guerra Mundial, se encontra na \versal{DUDH};
concomitante a uma das proveniências da ecopolítica no campo dos
direitos universais. Destaque-se o seu artigo V, sob a forma de
degradante: ``Ninguém será submetido à tortura, nem a tratamento ou
castigo cruel, desumano ou degradante.'' O que permanece como
\emph{estado de fim da tortura}, volteia o enunciado: o que a lei
protege, na vida ela aniquila. É aquilo mesmo que permanece como
\emph{estado de fim da tortura} o mote para a aceitação da naturalização
do castigo na tão aclamada lei universal dos direitos humanos, pois o
expurgo reside não no castigo, caso contrário, para que uma lei?, mas no
``cruel'', no ``desumano'', enfim, no ``degradante''. Viriam outras e os
termos do artigo \versal{V} da \versal{DHDU}, passariam a nomear o próprio título de
normativas como a Declaração sobre a Proteção de Todas as Pessoas contra
a Tortura e outras Penas ou Tratamentos Cruéis, Desumanos ou
Degradantes\footnote{Aprovada pela Assembleia Geral das Nações Unidas em
  9 de Dezembro de 1975 (Resolução 3452 (\versal{XXX}).
  \emph{http://direitoshumanos.gddc.pt/3\_6/IIIPAG3\_6\_7.htm}}, os
Princípios de Ética Médica Relativos ao Papel do Pessoal de Saúde,
especialmente os Médicos, na Proteção de Prisioneiros e Detentos contra
Tortura e Outra Forma Cruel, Desumana ou Degradante de Tratamento ou
Punição\footnote{Resolução 31/85 de 13 de dezembro de 1976 da Assembleia
  Geral das Nações Unidas quando convidou a Organização Mundial da Saúde
  (\versal{OMS}) para que preparasse um código de ética médica com respeito à
  proteção de pessoas submetidas a qualquer forma de detenção ou prisão,
  contra a tortura e outros tratamentos ou penas ``cruéis'',
  ``desumanos'' ou ``degradantes''. Os princípios foram celebrados em
  janeiro de 1979 e ratificados pela \versal{ONU} em 1982. O Conselho Federal de
  Medicina (Brasil) os celebra em 1983.
  \emph{http://www.portalmedico.org.br/resolucoes/cfm/1983/1097\_1983.htm}
  .}; a Convenção Contra a Tortura e Outros Tratamentos e Penas Cruéis,
Desumanas ou Degradantes\footnote{Adotada pela resolução n. 39/46 da
  Assembleia Geral das Nações Unidas em 10 de dezembro de 1984 e
  ratificada pelo Brasil em 28 de setembro de 1989.

  \emph{http://www.dhnet.org.br/direitos/sip/onu/tortura/lex221.htm}},
etc.

E outras que não trariam os termos em seus títulos, mas em artigos de
referência para tantas outras anteriores e posteriores, como o Pacto
Internacional dos Direitos Civis e Políticos\footnote{Adotado pela
  Resolução n. 2.200-A (\versal{XXI}) da Assembleia Geral das Nações Unidas, em
  19 de dezembro de 1966. O Brasil o ratificou em 1992.

  \emph{http://www.planalto.gov.br/ccivil\_03/decreto/1990-1994/d0592.htm}},
que em seu artigo \versal{VII} destaca ``Ninguém poderá ser submetido à tortura,
nem a penas ou tratamento cruéis, desumanos ou degradantes. Será
proibido sobretudo, submeter uma pessoa, sem seu livre consentimento, a
experiências médias ou cientificas'' (\versal{ONU}, 1966); a Convenção sobre os
Direitos da Criança\footnote{Adotada em Assembleia Geral das Nações
  Unidas em 20 de novembro de 1989.

  \emph{http://www.unicef.org/brazil/pt/resources\_10120.htm}} em seu
artigo 37 afirma: `` Os Estados Partes zelarão para que: a) nenhuma
criança seja submetida à tortura nem a outros tratamentos ou penas
cruéis, desumanos ou degradantes. Não será imposta a pena de morte nem a
prisão perpétua sem possibilidade de livramento por delitos cometidos
por menores de 18 anos de idade; b) nenhuma criança seja privada de sua
liberdade de forma ilegal ou arbitrária. A detenção, a reclusão ou a
prisão de uma criança serão efetuadas em conformidade com a lei e apenas
com último recurso, e durante o mais breve período de tempo que for
apropriado; c) toda criança privada da liberdade seja tratada com a
humanidade e o respeito que merece a dignidade inerente à pessoa humana,
e levando-se em consideração as necessidades de uma pessoa de sua idade.
Em especial, toda criança privada de sua liberdade ficará separada dos
adultos, a não ser que tal fato seja considerado contrário aos melhores
interesses da criança, e terá direito de manter contato com sua família
por meio de correspondência ou de visitas, salvo em circunstâncias
excepcionais; d) toda criança privada de sua liberdade tenha direito a
rápido acesso à assistência jurídica e a qualquer outra assistência
adequada, bem como direito a impugnar a legalidade da privação de sua
liberdade perante um tribunal ou outra autoridade competente,
independente e imparcial e a uma rápida decisão a respeito de tal ação''
(\versal{ONU}, 1989).

É mais uma vez pelo corpo da criança que se inicia a naturalização do
castigo e permanece intocável seu aprisionamento assim como a tortura.
Promulgar leis, convenções, pactos, declarações, decretos, ou
ratificações de acordos imediatos ou tardios, não abolem torturas e
castigos. Os adultos são presos, torturados, mortos, nas ditaduras e nas
democracias. Mas os baixos começos já estavam lá exercidos como força da
autoridade que se pretende superior ao corpo tenro de uma criança: a
força do pai, da mãe, dos responsáveis de toda ordem, na casa, na
escola, na rua, no posto médico, no hospital, no projeto social, no
abrigo, na delegacia, na batida policial, nas triagens infindáveis, nas
anamineses e prontuários, nas ongs e institutos, nos relatórios de
direitos humanos, nos pareceres da comunidade científica, nas avaliações
contínuas, nos programas e projetos socioeducativos, arte-educativos,
psico-educativos, arteterapêuticos, ludorecreativos, nos tribunais de
toda ordem.

Tantas normativas em torno \emph{do estado de fim da tortura}
proporcionaram, deliberadamente, utilizar a tortura para reafirmar o
conceito de crime -- que simplesmente não existe, a não ser como uma
interpretação moral de um evento --, enquanto um dos baixos começos do
conceito de crime hediondo, ou seja, crime criminoso; possibilidade que
se deu ao institucionalizar novos itinerários para a ampliação de
tribunais em função de punir as impunidades, mais uma tautologia reversa
da linguagem penal. Mas foi, também, pelo corpo da criança que se
fomentou uma das proveniências do lastro da proteção que aniquila a vida
intensa, para produzir o governo do vivo restaurado como portador de
direitos inacabados, como \emph{algo} em desenvolvimento, como
\emph{algo} vulnerável, como \emph{algo} resiliente.

Se o degenerado na biopolítica é o portador de perigo, na ecopolítica o
degradado ou o passível de sê-lo é o portador de proteção, portanto
portador de direitos inacabados. Ele assume o equivalente ao vulnerável
e ao resiliente, também como portador de direitos inacabados.

Na biopolítica a construção do anormal antecedeu a construção do normal
(Canguilhem, 1995). Norma, normal, normalizador. Vocábulos concisos,
hirtos, eretos. Partem da base sólida em direção ao mais alto dos céus
almejados: aspiração fronteiriça que se afirma pela negação; variante
plana da dimensão tripartida projetada verticalmente rumo a seu teto
mais acima do chão; variação do direito a, do direito de, matéria
substancial da designação do direito e do avesso; dispositivo de
conveniência do lado direito para o que se pretende normal; dispositivo
de suspeita e exame para aquilo que é avesso e passa a ser designado
anormal; variações da providência; variantes providenciais. Há de se
estancar estes vocábulos conceitos, trabalhar suas proveniências,
percorrer seus desdobramentos que, por sua vez, tornam-se novas
proveniências específicas.

Na discussão tecida por Canguilhem em torno das palavras \emph{norma} e
\emph{normal}, ele retoma a diferenciação entre os conceitos
escolásticos e cosmológicos, em que os últimos são fundamentos dos
primeiros, para afirmar que \emph{norma} é um conceito escolástico e
\emph{normal} é cósmico ou popular. O argumento utilizado para a
associação de normal como uma categoria popular é denominado por ele
como sendo algo que o povo sente e reclama diante de uma situação
injusta. Contudo, ele se dedica a trabalhar as proveniências da
rotinização deste termo na língua cotidiana, para que se instituísse
como um termo naturalizado. Mostra que não o é, pois esta passagem
somente foi possível a partir dos vocabulários de duas instituições
específicas, a instituição pedagógica e a instituição sanitária;
passagem que correspondeu a um espaço imediato em torno da reforma
destas mesmas instituições que ocorreu simultaneamente à Revolução
Francesa.

O termo normal no século \versal{XIX} é o vocábulo preferencial para designar o
modelo escolar e `o estado de saúde orgânica'. Trata-se da reforma
enquanto uma racionalização que se desdobra na política e na economia
sob a égide também da dicotomia de um `maquinismo industrial', segundo
Canguilhem, que levará ao que se chamou, desde então, de normalização.
``Assim como uma escola normal é uma escola onde se ensina a ensinar,
isto é, onde instituem experimentalmente métodos pedagógicos, assim
também um conta-gotas normal, é aquele que está calibrado para dividir
um grama de água destilada em gotas, em queda livre, de modo que o poder
farmacodinâmico de uma substância em solução possa ser graduado segundo
as prescrições de uma receita médica. Da mesma forma, também uma via
férrea normal é, dentre as vinte e uma bitolas de uma via férrea,
praticadas em todas as épocas, a via definida pelo afastamento de 1,44m
entre as bordas internas dos trilhos, isto é, aquela que, em determinado
momento da história industrial e econômica da Europa, pareceu
corresponder melhor ao acordo que se procurava obter entre várias
exigências \emph{} antes de tudo não concorrentes \emph{} de ordem
mecânica, energética, comercial, militar e política. Enfim, também para
o fisiologista, o peso normal do homem, levando em conta o sexo, a idade
e a estatura, é o peso `que corresponde à maior longevidade previsível'
(...) Em todos os quatro casos, o que caracteriza um objeto ou um fato
dito normal, em referência a uma norma externa ou imanente, é poder ser
por sua vez, tomado como ponto de referência em relação a objetos ou
fatos ainda à espera de serem classificados como tal. Portanto, o normal
é, ao mesmo tempo, a extensão e a exibição da norma. Ele multiplica a
regra, ao mesmo tempo em que a indica. Ele requer, portanto, fora de si,
a seu lado e junto a si, tudo o que ainda lhe escapa. Uma norma tira seu
sentido, sua função e seu valor do fato de existir, fora dela, algo que
não corresponde à exigência a que ela obedece'' (Canguilhem, 1995:
210-211).

Canguilhem afirma que quando se sabe que `norma' é uma palavra latina
que designa esquadro e que normalis significa perpendicular, sabe-se o
suficiente sobre norma e normal, enquanto termos capazes de ser trazidos
para outros campos. Ele investiu de forma precisa nas palavras
estancadas nelas mesmas: ``Uma norma, uma regra, é aquilo que serve para
retificar, pôr de pé, endireitar. `Normar', normalizar, é impor uma
exigência a uma existência, a um dado, cuja variedade e disparidade se
apresentam, em relação à exigência, como um indeterminado hostil, mais
ainda do que estranho'' (Idem: 211).

Foucault (2001), em sua aula do dia 15 de janeiro de 1975 no `Collège de
France', discorrendo sobre a sua crítica à noção de repressão, exclusão
do leproso e inclusão do pestífero, invenção das tecnologias positivas
de poder, o normal e o patológico ele remete os ouvintes à leitura de
Canguilhem. ``Mais umas palavras, se vocês me derem alguns minutos. Eu
gostaria de dizer o seguinte. Gostaria de remeter a um texto que vocês
vão encontrar na segunda edição do livro de Canguilhem sobre \emph{O
normal e o patológico}. Nesse texto, que trata da norma e da
normalização, temos um certo lote de ideias que me parecem histórica e
metodologicamente fecundas. De um lado a referência a um processo geral
de normalização social, política e técnica, que vemos se desenvolver no
século \versal{XVIII} e que manifesta seus efeitos no domínio da educação, com
suas escolas normais; da medicina, com a organização hospitalar; e
também no domínio da produção industrial. E poderíamos sem dúvida
acrescentar: no domínio do exército. (...) Vocês também vão encontrar,
sempre no texto a que me refiro, a ideia que acho importante, de que a
norma não se define absolutamente como uma lei natural, mas pelo papel
de exigência de coerção que ela é capaz de exercer em relação aos
domínios a que se aplica. Por conseguinte, a norma é portadora de uma
pretensão ao poder. A norma não é simplesmente um princípio, não é nem
mesmo um princípio de inteligibilidade; é um elemento a partir do qual
certo exercício do poder se acha fundado e legitimado. (...) Parece-me
que é um erro ao mesmo tempo metodológico e histórico considerar que o
poder é essencialmente um mecanismo negativo de repressão; que o poder
tem essencialmente por função proteger, conservar ou reproduzir relações
de produção. E parece-me que é um erro considerar que o poder é algo que
se situa, em relação ao jogo de forças, num nível superestrutural. É um
erro enfim considerar que ele está essencialmente ligado a efeitos de
desconhecimento'' (Foucault, 2001a:61-63).

Canguilhem está empenhado, também, em liquidar o grande mito platônico
\emph{} parametrador da razão ocidental que dissocia saber e poder
\emph{} ao desconstruir a máxima socrática, `segundo a qual ninguém é
mau tendo consciência disto', ironizando que `ninguém é bom tendo
consciência de o ser'. ``Mas é no furor da culpabilidade, assim como o é
no grito do sofrimento que a inocência e a saúde surgem como os termos
de uma regressão tão impossível quanto desejada. O anormal, enquanto
a-normal, é posterior à definição do normal, é a negação lógica deste.
No entanto, é a anterioridade histórica do futuro anormal que provoca
uma intenção normativa. O normal é o efeito obtido pela execução do
projeto normativo, é a norma manifestada no fato. Do ponto de vista do
fato, há, portanto, uma relação de exclusão entre o normal e o anormal.
Esta negação, porém, está subordinada à correção reclamada pela
anormalidade. Não há, portanto, nenhum paradoxo em dizer que o anormal,
que logicamente é o segundo, é existencialmente o primeiro''
(Canguilhem, 1995: 215-216).

No exercício da ecopolítica, a relação da resiliência com a
vulnerabilidade se efetivou pelo seu contrário e encontrou seus baixos
começos primeiro sobre os corpos de crianças. Da mesma forma que, no pós
\versal{II} Guerra Mundial, os primeiros estudos sobre resiliência em torno de
crianças não se voltavam para este conceito e sim para crianças expostas
a riscos (Werner \& Smith, 1982), a criança resiliente foi construída
inicialmente como aquela que seria, por excelência, invulnerável
(Anthony, 1987), relação que persistiria na década de 1970. A inversão
ocorreu, posteriormente, como registra Klotiarenco ao enfatizar a
distinção entre resiliência e invulnerabilidade e ao mesmo tempo,
imiscuir o termo resistência no de resiliência. ``Durante a década de 70
ganhou popularidade o conceito de criança invulnerável, como aquele que
aludia a algumas crianças que pareciam constitucionalmentes tão fortes,
que não cediam frente às pressões de stress e da adversidade.
Entretanto, este conceito resultava confuso e equivocado por três
razões: a resistência ao stress é relativa, não absoluta, não é tão
estável ao longo do tempo e varia de acordo com o estágio de
desenvolvimento das crianças e a qualidade do estímulo. As raízes da
resistência de ambos provêm tanto do ambiente como do constitucional, o
grau de resistência não é estável, mas varia ao longo do tempo e de
acordo com as circunstâncias. Por estas razões, na atualidade se utiliza
preferencialmente o conceito de resiliência (...). Resulta,
imprescindível, também, neste plano conhecer o significado do vocábulo
vulnerabilidade, pois esta é uma característica básica para a gestação
de comportamentos resilientes'' (Klotiarenco et alli, 1997: 6-7).

A relação que se estabelece a partir daqui se situa em um oportuno duplo
reverso e recíproco entre vulnerabilidade e proteção, por meio da gestão
dos riscos. ``Os conceitos de vulnerabilidade e mecanismo protetor têm
sido definidos como a capacidade de modificar as respostas que as
pessoas têm frente às situações de risco. O conceito de vulnerabilidade
dá conta, de alguma forma, de uma intensificação da reação frente a
estímulos que em circunstâncias normais conduz a uma desadaptação. O
contrário ocorre nas circunstâncias nas quais atua um fator de atenuação
que é considerado como mecanismo protetor. Disto se desprende que
vulnerabilidade e mecanismo protetor, mais que conceitos diferentes
constituem o polo negativo ou o positivo de um mesmo. O essencial de
ambos os conceitos, é que só são evidentes em uma combinação com alguma
variável de risco'' (Idem, 1997: 11).

Passa-se, portanto, da equação biopolítica medo-risco-contágio para o
equacionamento ecopolítico medo-vulnerabilidade-proteção, no qual o
risco é apenas um fator, uma variável a ser gerida, na dimensão da
vulnerabilidade que se situa sempre como uma via de mão dupla, enquanto
algo que pode tanto cometer quanto sofrer um impacto, um abalo, um
choque, uma destruição, escalonados sempre como adversidades nocivas
diante das quais a resiliência mostra-se como o atributo, o processo, o
conjunto de capacidades de absorção, superação e adaptação como o
retorno ao mesmo por meio de adequações ajustáveis a uma condição de
governo do vivo restaurado (Coimbra \& Morais, 2015). A resiliência na
ecopolítica configura-se como um dispositivo.

Um dispositivo numa perspectiva genealógica histórico-política não
inventa nada, ele cria, possibilita e produz conformações e
deslocamentos de seu próprio funcionamento exaustivo remetido ao
arrefecimento do indomesticável da vida tomada em sua brutalidade
selvagem, crua e incontornável, seja na partícula mais ínfima do meu
corpo, do planeta terra ou do espaço sideral.

O Direito, os direitos, a proliferação de direitos, o deslocamento do
sujeito de direitos em portador de direitos inacabados nunca responderá
ao inqualificável da vida e a resiliência jamais pode estar situada como
uma prática de resistência ou potencializadora de práticas de
resistências, ainda que uma vasta produção variada em torno desta
palavra e deste conceito invista e insista, oportunamente, nesta
sinonímia.

A resiliência produz apaziguamentos, conformações e conformismos,
condutas e contracondutas, com ênfase no controle pela auto-regulação ou
gestão compartilhada de governos em ambientes consolidados ou em vistas
de se formar; vontade de poder que vai dos empoderamentos aos pequenos
ou grandes fascismos e mostra-se como campo preferencial da captura de
corpos e da tentativa de neutralização de práticas de
direito-antissoberania, de antigoverno, de antipolítica.

A resiliência é uma construção de verdade voltada à minimização e
supressão de resistências. Ela trafega em itinerários como fluxos
inteligentes de condutas e contracondutas; mostra-se hoje como uma
enzima que acelera reações, produzindo modulações catalizadoras. De
forma simultânea, apresenta-se como elemento modulador conectivo de
outros elementos, funcionando, talvez como a integrina conectada a
linfócitos-T específicos que disparam na totalidade a defesa e proteção
de um corpo diante da presença de uma bactéria, vírus ou congêneres,
assim como anticorpos indesejáveis. ``Integrinas são intermediadoras na
comunicação entre o citoesqueleto celular e proteínas plasmáticas ou da
matriz extracelular por meio da adesão célula-a-célula através de
interações com outras proteínas de membrana. Além da função de adesão,
as integrinas são reconhecidas moléculas de sinalização capazes de
realizar a tradução de mensagens por vias de sinalização clássica,
graças a sua estrutura molecular. Desse modo, as integrinas podem ser
consideradas moduladores chave da proliferação, migração, longevidade,
migração e ainda, a manutenção das funções específicas de diferenciação
celular. (...) As integrinas funcionam de modo complexo dentro de vias
de sinalização, de adesão, do citoesqueleto e desempenham papel
fundamental no sistema hematopoiético, especialmente no sistema imune''
(Brandão et al, 2012: 25). A integrina é uma proteína denominada
inteligente e conectiva pelos recentes estudos voltados tanto à
imunossupressão quanto à imunomodulação da proteção pontual e específica
de um corpo, por meio de medicamentos vivos derivados de ovários de
ratos para agir sobre aquilo que a medicina chama de sistema imunológico
de humanos. ``Humanização de anticorpos através de modificações
introduzidas em anticorpos recombinantes. Pode-se recorrer à fusão
genética da região variável do anticorpo original (de rato) com a região
constante de uma imunoglobulina humana, formando uma quimera. Apesar da
introdução de uma região constante humana, este tipo de construção ainda
mostra uma grande imunogenicidade devido à preservação da região
variável de rato. Para a construção de um anticorpo totalmente
humanizado, uma região constante é fundida com uma região variável
desenhada de forma a que a sua sequência seja a mais próxima possível de
uma região variável de anticorpo humano'' (Idem: 20-21).

O \emph{dispositivo resiliência} é um dispositivo de plasticidade
modulável de suportabilidade e reatividade, por excelência. Sua
modulação catalizadora atua, dependendo das relações de forças
envolvidas, produzindo totalizações e homogeneizações, pela aceleração
de fluxos e simultânea disputa e convergência de normativas
internacionais com portadores de direitos inacabados (novas
institucionalidades e governamentalidade planetária, com destaque para a
tolerância, cultura de paz, cultura resiliente e cidadão resiliente,
investimento na formação de ambientes resilientes e na sua consolidação
resiliente, restaurados de degradação e recuperados de degeneração).

Entretanto, sua modulação conectiva, produz a formação e avaliação
exaustiva-contínua-detalhada de individuações e dividuações pelos
próprios portadores de direitos inacabados. Condutas e contracondutas
resilientes, que vão do corpo de cada um, com atenção preferencial para
o corpo das crianças mostrando destaque às neurociências articuladas à
psiquiatria, desenvolvimento na primeira infância e à epigenética,
passando pela ênfase aos programas compartilhados pelo planeta e fora
dele, atravessando, dentre outros, os programas de software e seus
protocolos e contraprotocolos até as tecnologias espaciais de
monitoramento das mudanças climáticas do planeta Terra e dos robôs
enviados ao espaço sideral.

O \emph{dispositivo resiliência} na ecopolítica se forma, configura-se e
funciona pelas duas modulações específicas apontadas, atuando em
conjunto, e não por intersecção, mas pela simultaneidade dos itinerários
dos fluxos inteligentes em sua inerente plasticidade e suportabilidade
modulável.

Normalização do normal: ensinar a ensinar -- não mais por uma
instituição específica, mas por arranjos conectados por divíduos
portadores de direitos inacabados em suas atividades de trabalho, lazer,
e práticas políticas voltadas à resiliência; o normal não mais como
eixo, mas envolto em uma moralidade plástica flexível. O divíduo está
sujeito a medicalizáveis transtornos que não obstaculizam sua
produtividade e ocupação, e muito menos sua participação. Trata-se de um
deslocamento do eixo para os fluxos, um exercício em relações de poder.
Desdobrando-se a noção histórico-política de Canguilhem que o normal, na
ecopolítica, é existencialmente o primeiro passível de investimento em
sua condição de degradação e vulnerabilidade, portanto normalizável.

O indomesticável da vida não se qualifica, também não equivale a eras,
etapas ou escalonamentos do que se denomina por desenvolvimento,
tampouco se confunde com melhorias ou melhora no que chamam de estado de
saúde, de qualidade de vida, de redução de vulnerabilidades ou proteção
de vulneráveis. Afinal, é também em nome disso, que se sustenta também a
resiliência e permanece intocado o consenso em torno do regime do
castigo, da cultura do castigo, que sempre começa pelos corpos de
crianças e jovens.

Enquanto isso \emph{eles} seguem presos, encarcerados ou a céu aberto.
São a parte obscura e evidente, da prática do confinamento em nossa
cultura ocidental. São o ápice e a cara escancarada da aceitação de uma
cultura que se baseia no regime dos castigos. São a ponta aguda da
família que castiga, da escola que instrui e controla, do projeto que
inclui e produz participação, do programa que conecta, do Estado que
protege, pune e aniquila, do governo que segura e assegura, da polícia
que se forma, que forma e dá forma, da propriedade que se estampa sob a
força exercida sobre cada um de seus corpos. Eles já foram o caso de
polícia, o caso de correção, o caso de ressocialização, o caso de
abandono, o caso do menor e de segurança nacional, o caso do futuro
cidadão e da medida socioeducativa em suas aplicações variadas. Da mesma
maneira que a tortura assume no presente variações eufemistas, hoje eles
são o caso do adolescente em conflito com a lei, o caso preferencial da
medida, das medidas e das aferições neuropsiquiátricas com suas escalas
e limites para conter rebeldias, com a pretensão de antecipar
resistências. Eles são o subsídio ínfimo e imprescindível para validar o
consenso inquestionável do castigo que segue sobre corpos de crianças e
jovens, dos códigos e dos reformadores que não findam, do que é
arregimentado nas ondas de frequências da lógica da punição, dependendo
da época, e de forma mais recente sob os atuais abjetos acolhimentos,
recolhimentos e internações.

O abolicionismo penal libertário na lida com respostas-percurso, no
curso livre da vida, atenta para um acontecimento mínimo, que não tem
medida nem nunca terá: a abolição do castigo começa pelo corpo de cada
um num inclassificável apetite de vida. Não cessa com as nanométricas e
plásticas ingerências do \emph{dispositivo resiliência} e sua pretensão
em antecipar o surpreendente, as inegociáveis fissuras. Para Foucault a
coragem é sempre física. Diante do consensual regime dos castigos, da
transformação da vingança contra o corpo em direito e do direito ser o
dever moral diante do mais forte, nada é menos importante do que
desassossegar a liberdade instituída. Diante das atualizações
resilientes da cultura dos castigos, o abolicionismo penal libertário
segue sendo uma minúscula dissonância.

\chapter{Ecopolítica: poder em fluxo e política}

A ecopolítica emerge no final da \versal{II} Guerra Mundial, institui a
convocação à participação democrática impulsionada de modo surpreendente
pelo planeta e se consolida com a disseminação da racionalidade
neoliberal. Não se trata mais de obedecer somente ao governo soberano,
mas de produzir formas de governos da vida longeva e de direitos em que
se garanta a segurança, a liberdade e a propriedade (\emph{Declaração
Universal dos Direitos Humanos}, Art. \versal{XVII}). Sua forma atual se
consolida com a gestão do desenvolvimento sustentável em um planeta
resiliente.

Todos os esforços se voltam, segundo conceituações jurídico-políticas e
filosófico-políticas que pretendem firmar a existência inexorável do
Estado como referência para a continuidade da vida, para equacionar a
alegada \emph{anarquia} dos Estados ou a situação de Estados falidos,
elevando cada cidadão à condição de livres para a circulação (Ibidem:
Preâmbulo, Art. \versal{XIII}) com segurança social e desenvolvimento humano
(Ibidem: Art. \versal{III}, Art. \versal{XXII}, Art. \versal{XXV}), que permitirá instituir Missões
de Paz, a partir da década de 1990 e anunciar a tensão entre a nova
política e a antipolítica.

Mesmo sem referir-se diretamente à democracia, o preâmbulo da \versal{DUDH}
considera ``essencial que os direitos humanos sejam protegidos pelo
Estado de Direito, para que o homem não seja compelido, como último
recurso, à rebelião contra tirania e a opressão''. Por sua vez, o meio
ambiente, ausente na \versal{DUDH}, passa a ser contemplado com a
\emph{Declaração do Meio Ambiente Humano} ou \emph{Declaração de
Estocolmo}, resultado da Conferência da \versal{ONU} sobre o Meio Ambiente, em
1972, quando também se instituiu o \versal{PNUMA}\footnote{\versal{PNUMA}. Resumo para
  formuladores de políticas. Panorama ambiental global Geo5, 2012.
  \emph{http://www.pnuma.org.br/admin/publicacoes/texto/GEO5\_RESUMO\_FORMULADORES\_POLITICAS.pdf}.}
(Programa das Nações Unidas para o Meio Ambiente) com o objetivo de
coordenar a proteção ao meio ambiente, contando com a colaboração de
governos nacionais com a sociedade civil organizada. Na implementação do
desenvolvimento sustentável, o \versal{PNUMA} está conectado ao \versal{PNUD} (Programa
das Nações Unidas para o Desenvolvimento, criado em 1965, ligado ao
Conselho Econômico e Social das Nações Unidas -- \versal{ECOSOC}), que elaborou
em seu relatório de 1994 o conceito de \emph{segurança humana},
complementando o \versal{IDH} (Índice de Desenvolvimento Humano) criado em 1990,
ambos voltados para estudos sobre o desenvolvimento humano sustentável e
para a realização dos \versal{ODM} (Objetivos do Milênio), compondo
institucionalizações exigidas para a efetivação programática de \emph{um
futuro melhor}\footnote{Atlas do Desenvolvimento Humano Brasil 2013.
  \emph{http://www.atlasbrasil.org.br/2013/}.}. Enfim, como não há
direitos sem punições, as novas condições de presos encarcerados,
penalizações alternativas, legislações antiterrorismos condutas
criminalizáveis e judicializações conformam-se no fluxo de gestão de
riscos da violência (Waiselfisz, 2013), como situara Foucault (2008).

A ecopolítica, gradativamente, encontra maneiras de governar ambientes
de modo a inibir conturbações, levantes, desestabilizações de Estados e,
ao mesmo tempo, tornar sustentável o desenvolvimento capitalista com
segurança. As regulamentações que põem em funcionamento as regulações
seletivamente escolhidas exigem formalizações jurídico-políticas
relacionadas, direta ou indiretamente, aos governos do Estado e às
organizações a eles vinculadas de modo transterritorial. Produzem
maneiras de governar os ambientes nos quais a condição de obediente ao
superior é disseminada pelas variadas formas de governar a si e aos
outros, por meio de descentralizações, compartilhamentos e práticas de
governança com destaque especial para a formação e funcionamento das
elites secundárias. Trata-se de uma governamentalidade democrática
distendida, segundo a continuidade e prevalência do Estado de Direito e
as condições de efetivação do desenvolvimento sustentável.

A conservação do meio ambiente, a segurança humana e estatal, a proteção
e difusão de direitos e o controle da gestão dos chamados crimes
produzem um modo de se governar com moderação.

As \emph{Metas do Milênio} e a \emph{Carta da Terra} (não
desconsiderando a \emph{Agenda 21} e seus resultados globais) foram
documentos de referência e se mostraram condutores da Conferência
Rio+20, que incorporou em seus documentos o que estava previsto acerca
da sustentabilidade, qualidade de vida, vulnerabilidades, segurança,
ambiente e resiliência. A análise da ecopolítica parte desses arquivos
que situam a convergência entre as propostas da \versal{ONU}, Conferência das
Nações Unidas para o Desenvolvimento Sustentável e a Cúpula dos Povos,
situando as disputas democráticas entre \emph{adversários}, evitando
configurar o inimigo para além do terrorismo contemporâneo e práticas
sobejamente conhecidas e securitizadas em torno do tráfico de drogas
pela racionalidade neoliberal colonizando sociedade civil e Estado.

O discurso sobre a \emph{degeneração} do indivíduo e da espécie foi
suplantado pelo discurso sobre a \emph{degradação} dos ambientes que
inclui a recuperação, a restauração e a conservação que relaciona
\emph{divíduos} e populações governáveis por localizações e etnias em
torno da \emph{economia verde} (e mais recentemente \emph{azul,} em
referência aos recursos costeiros e dos oceanos), do
\emph{empreendedorismo social}, dos combates às \emph{vulnerabilidades},
com \emph{qualidade de vida}, \emph{sustentabilidade},
\emph{empoderamento} e construção de uma \emph{cultura de paz}. Importa
que este divíduo seja capaz de se metamorfosear constantemente segundo o
que dele se espera como capital humano. Ele deve ser equilibrado,
educado com carinho e atenção pelos pais, escolarizado para responder às
variadas exigências que o levem a ser uma criança estável, um jovem
amadurecido, um adulto responsável, um cidadão-polícia cercado de
equipamentos sociais. Um sujeito de múltiplas facetas econômicas,
pragmático em política, versado em cultura local, atento para as
condições de pobreza, cuidadoso com o meio ambiente, exigente em
segurança e flexível nas decisões, eventualmente destinatário de
\emph{cares} na \emph{feliz-idade}.

A produção inacabada de direitos consolida a noção de \emph{resiliência}
que, por sua vez, possibilita, viabiliza, visibiliza e orienta a gestão
da participação e faz aparecer o \emph{portador de direitos inacabados}.
As relações entre resiliência, governo das cidades e das pessoas estão
conectadas à redução de \emph{vulnerabilidades}, \emph{ao ambiente},
\emph{à segurança humana, alimentar e planetária}. A \emph{resiliência}
passa a ser um dispositivo que não só redimensiona a segurança, os
sistemas protetivos integrais, mas encontra-se conectado aos
\emph{dispositivos meio ambiente} e \emph{diplomático-policial} e
\emph{monitoramento}.

As práticas de \emph{resiliências} cada vez mais esclarecem suas
pertinências como contenção e inibição de resistências por educarem,
preferencialmente, para a formação de alguém (da criança ao idoso) como
capital humano voltado \emph{às melhorias das futuras} gerações, fazendo
sobressair não só práticas redutoras de resistências, mas a
\emph{conduta moderada} pelo pastorado horizontalizado. Procuram
obstruir a emergência de radicalidades que enunciem o insuportável (seja
a Conspiração das Células de Fogo, na Grécia, ou a presença
surpreendente da tática \emph{black bloc} em manifestações e protestos,
incluindo o Brasil) e explicitem o \emph{ingovernável}.

Sobre o consenso em torno do combate ao tráfico e à violência,
configura-se, com clareza, a relação entre \emph{saúde} e
\emph{segurança} como definidora das práticas de governo das condutas
por meio de programas governamentais e programas sociais e comunitários,
acoplando empreendimentos empresariais, bancários, comerciais e
principalmente culturais para \emph{cada um} que investe em um governo
\emph{pacífico} das comunidades (periferias), desdobrando as funções da
polícia repressiva em polícia de proximidade, dimensionando e refazendo
o policiamento das forças armadas e a da militarização da polícia.
Trata-se de recuperar da degradação, valorizar o pobre e ampliar
programas de erradicação da miséria, incluindo a controversa legalização
da maconha (caso Uruguai) e o vaivém acerca da continuidade efetiva do
tráfico de drogas e pessoas. A \emph{resiliência} investe na diferença
formalizada nos direitos com ênfase na combinação entre prevenção e
precaução.

A análise histórico-política da ecopolítica situa feixes de fluxos
inteligentes da racionalidade neoliberal que procriam dispositivos e que
configuram tanto uma nova política como a antipolítica. O
desenvolvimento sustentável em andamento dissipa as fronteiras entre
desenvolvidos e subdesenvolvidos/em desenvolvimento por tratar, agora,
de cuidados com a degradação que finalmente atingiu o planeta como um
todo, em superfície, mar e ar, e os viventes em qualquer lugar. A noção
de periferia também se desloca da relação anterior de centralidade para
ser compreendida como espaços imbricados nas cidades e também passando a
ser transterritorial.

O governo compartilhado de ambientes (\emph{dispositivo meio ambiente})
a partir da institucionalização de condutas constituídas em segurança,
dissolvendo a relação dentro-fora e redimensionando a polícia
(\emph{dispositivo diplomático-policial}), produzindo monitoramentos
materiais e de condutas de divíduos em função da convocação à
participação democrática nos regimes, empresas, comunidades
(\emph{dispositivo monitoramento}) pela pletora de direitos inacabados
conformando o portador de direitos inacabados, enquanto capacidade de
superação de adversidades (\emph{dispositivo resiliência}), conecta a
governamentalidade planetária transterritorial que atravessa
organizações internacionais, uniões de Estados, Estados nacionais,
programas, sociedade civil organizada e o divíduo.

Ambientes seguros monitorados e resilientes são o que produzem a
racionalidade neoliberal, a utopia capitalista e as práticas
democráticas como governo compartilhado dos humanos com os demais
viventes. A degradação em equacionamento depende dos investimentos
governamentais, empresariais e do capital humano em normalizações de
normais. Este caudaloso fluxo inteligente produz a paradoxal relação
oposta ideologicamente entre direita, centro e esquerda em relações
compartilhadas governadas pelo pluralismo.

Estamos diante de dois acontecimentos revisados, considerando-se as
constatações de Foucault em \emph{A ordem do discurso} a respeito das
interdições. Comecemos pela segunda interdição, a do sexo. Este
encontra-se agora comprimido nas relações entre o governo de si e o
governo dos outros, paradoxalmente redimensionado em sexo livre e
casamento efêmero, ambos policialmente supervisionados pelas religiões,
a ameaça de novas doenças e governados pela medicina da saúde, a
psiquiatria e o direito penal, redefinindo criminalizações e as
propagando, segundo as regras gerais do direito e as práticas do
portador de direitos inacabados, e principalmente pelo reordenamento na
família convencional.

O sexo desinterditado e dissolvendo paulatinamente a distinção de gênero
delineia para o portador de direitos inacabados as respectivas
criminalizações, segundo regulamentações e leis que \emph{protegem} as
minorias numéricas, os efeitos da liberdade individual neoliberal e seus
implícitos \emph{riscos} subjacentes às condutas. A utilidade produtiva
passou a incluir as variadas minorias, segundo empregos e ocupações;
dá-lhes visibilidades e garantias condicionadas à participação constante
e sob a formatação identitária variada e compósita. A desinterdição do
sexo, considerando os limites impostos pela religião, deve também
produzir uma conduta moderada, ajustada aos princípios da moralidade,
redimensionamentos de famílias convencionais, ajustes de direitos além
de penalizações e condutas criminalizáveis estratificadas, compondo uma
nova cultura universal pacificadora sobre o sexo e a família (violência
sobre corpos e mentes são penalizáveis, assim como se aprende a conviver
com a criação do mito dos preconceitos e suas violências próprias, a ser
combatido gerando uma contabilidade a ser reduzida, tanto pela
propagação da tolerância, quanto por medidas de tolerância zero).

Enfim, nem toda liberação inventa vida livre. O efeito principal da
liberação sexual, impulsionada pelo acontecimento \emph{1968},
metamorfoseou-se na renovação do casamento, que inclui os anteriormente
considerados normais, de modo que os degenerados normalizados pela nova
conduta sexual sejam socialmente aceitos produtivamente e incluídos pela
racionalidade neoliberal que, por se ocupar do capital humano inovador,
é, portanto, capaz de fundamentar a gestão de preconceitos ao que se
torna familiar.

Quanto à segunda interdição, a política, esta deixou de ser
eminentemente circunscrita ao Estado (e não se trata também de sua
designação genérica de práticas de ações societais, como definia Max
Weber em sua conhecida conferência \emph{A política como vocação}) e à
soberania. Ela se introduz pela prática democrática nas relações nas
empresas, comunidades, escolas, \versal{ONG}s, fundações, institutos... A antiga
oposição entre regime político democrático e estruturas autoritárias de
produção e administração foi suplantada, gradativamente, pela introdução
da racionalidade neoliberal em torno da produtividade inteligente.

As potências das resistências são o alvo. A convocação à participação
não trata mais de interdição (incluindo os efeitos-limites de direitos
de biopolítica: quem deve viver e quem deve morrer) ou simplesmente
sujeições, mas dissemina a \emph{esperança no amor} como uma nova
atualização do amor à humanidade (Ferry, 2009). É preciso amar: a
ocupação, os colegas, a empresa, a comunidade para que na competição se
obtenha a segurança necessária. Amar de modo compartilhado para que a
meritocracia se justifique e cresça, combinando a preservação do
interesse privado e público, e, por meio da fiscalização constante,
superar ou evitar os antagonismos pela programática compartilhada. É
recomendável amar tanto para que a humanidade e o planeta sejam
conservados de modo sustentável, assim como ocupações e empregos, vida
nas comunidades e condutas tolerantes que garantam o futuro das novas
gerações. Nesse sentido, é evidente a disseminação da programática de
condutas competitivas compartilhadas pela maior difusão de produtos
midiáticos, pelas redes de televisão sobre a produção de chefes e
alimentação, novas famílias, novos relacionamentos amorosos ou de
convivência em espaços fechados, relações amorosas entre corpos nus,
dificuldades em espaços abertos e na natureza etc.

A política também deixa de ser compreendida, como Foucault situara em
\emph{A vontade de saber} e \emph{Em defesa da sociedade}, como a guerra
prolongada por outros meios. A guerra convencional foi suplantada pelos
\emph{estados de violência} (Gros, 2009); o dispositivo
diplomático-militar acoplado ao \emph{dispositivo diplomático-policial}
nos leva a outras considerações relativas aos governos dos estados de
violência, dos \emph{perdedores radicais} e das guerras-fluxo. Não há
mais interdições políticas. Quem dela se apartar o fará por conta e
risco; quem pretender contornar ou enfrentar a nova política estará
obsoleto por recusar o protagonismo em voga e se excluir dos
empoderamentos, dispor-se à gestão do insuportável ou simplesmente se
conformar em ser deslocado para o ostracismo.

As relações de poder que se estabeleciam em redes e configuravam uma
situação estratégica estão em imediata discussão. Não se trata somente
de abordar como a literatura sociológica contemporânea considera as
relações em redes nas comunicações constantes, principalmente com a
expansão da internet e das chamadas redes sociais. Neste ponto a questão
passa a ser a seguinte: se as relações de poder estão em rede, como na
sociedade disciplinar, qual a distinção trazida pelas novas redes em
nossa sociedade? Trata-se do mesmo envolvimento individualizante e
totalizador?

Se na sociedade disciplinar as relações de poder eram em rede,
proporcionando às resistências atravessarem a estratificação social, na
sociedade de controle os fluxos darão mais elasticidades às relações de
poder pela intensidade, velocidade, capacidade de capturas e afluências.
Por manter a situação inacabada, convocar à participação, instituir
adversários móveis, compor gestões em empresas e em \emph{negócios
sociais}, ordenar intervenções externas em Estados como Missão de Paz,
enfim, as relações de poder em fluxos habilitam qualquer um a
cartografar as suas possíveis linhas de fuga e, mais do que isso,
orientam as resistências.

As relações de poder em fluxo não estão mais circunscritas formalmente à
estratificação social, esta divisão hierárquica da população por camadas
ssocioeconômicas, culturais e identitárias e, até certo ponto, herdada
ou adquirida pelos indivíduos em uma sociedade que estimula a mobilidade
social, segundo modos de conjugar as desigualdades. Hoje, a pirâmide
permanece como figura ilustrativa das mobilidades, todavia, as conexões
em fluxos produzem sensações compartilhadas e democráticas de que
ocorreu uma descentralização em círculos concêntricos, maleáveis e
moduláveis. Esta sociedade necessita da produção de direitos, da luta
organizada para eles, também estratificados em políticos, civis e
sociais. Quando Foucault situava as resistências na sociedade das
disciplinas ele o fazia para mostrar as articulações entre os indivíduos
em função de uma determinada relação de poder produzida pelas
disciplinas que repercutiam em qualquer espaço fechado e vigiado.
Produzia ressonâncias pelas condutas posicionadas e contraposicionadas.

O limite estava colocado pela indisciplina levando à paralização,
ocupação ou demolição daquele espaço. Mas as resistências nessas
sociedades produziam subjetivações que repercutiam no questionamento ao
soberano, simplesmente porque os indivíduos, temporariamente livres dos
espaços confinados, associavam-se ou se organizavam de modo a renovar
contraposicionamentos ou mesmo produzir antiposicionamentos. Foucault
(2011), ao analisar a trans-historicidade dos cínicos e, por conseguinte
sua atitude parresiástica, chamava a atenção para o \emph{militantismo}
do século \versal{XIX} capaz de colocar em xeque as formas de organização, o
direito de soberania, as hierarquias, os efeitos das disciplinas e os
paliativos biopolíticos, ou seja, como estas resistências ultrapassavam
os contraposicionamentos e se praticavam atravessando as estratificações
sociais para detonar a estrutura social.

As resistências na sociedade disciplinar mantinham esse fora relacional
e o dentro que se comunicavam de modo intenso, a ponto do poder soberano
enfrenta-las como inimigo perigoso a ser combatido, se necessário, em
uma situação de guerra civil declarada (em especial, a Revolução
Espanhola, na década de 1930, ou mesmo as Comunas na França, em 1871).
Em seus primeiros cursos, Foucault (2015) chamava atenção para a
política como guerra civil, diretamente relacionada à dominação e à
exploração por meio da profusão de ilegalismos; mais tarde (Foucault,
1999), irá redefini-la como guerra prolongada por outros meios,
abordando, em especial, a biopolítica em seu limite nazista. Isso se
deveu à sua análise inicial sobre os ilegalismos burgueses e populares
que se transformavam em leis e no discurso burguês. Entretanto, a
mobilização constante contra os ilegalismos populares (roubo, furto e
sublevações de poderes) encontrou instituições destinadas à exclusão e
ao asilamento modernos como a prisão ou o manicômio (em especial, a
colaboração de Césare Lombroso para a Antropologia criminal, saudando,
em seu livro de 1894 intitulado \emph{Os anarquistas}, a beleza dos
revolucionários e o ensandecimento dos anarquistas a serem confinados).

Se na Revolução Francesa, no primeiro momento, o rei e a rainha eram os
\emph{monstros}, imeditamente a seguir, com o terror de Estado, foi o
\emph{povo} sublevado que passou a ser o grande monstro, e, dentro dele
o anormal, o perigoso (Foucault, 2001a). No justo momento em que a razão
de Estado trouxe o pastorado para seu interior com a função biopolítica,
transformou o povo em população pela economia política e agregou o alvo
biopolítico à segurança. Conduziu o Estado para realizar seus interesses
no corpo-espécie de forma totalizadora, acoplado ao corpo disciplinado
individualizado, fazendo com que a imagem do monstro se ajustasse à do
anormal, o desajustado: o inimigo estava entre o povo e a forma de
combate \emph{pacífico} se daria por meio da biopolítica. Neste exato
momento, a política como guerra civil desdobra-se em guerra prolongada
por outros meios (educar e cuidar da saúde da população para a
produtividade interna e a defesa fortificada diante da ameaça externa)
conjugando soberania, polícia e biopolítica. Para tanto, foi preciso o
ajuste da defesa policial interna para a garantia de segurança externa.
Os adversários internos passaram ser alvo de uma gestão calculista da
vida, assim como os externos incorporados pela prática da diplomacia. Se
os inimigos externos do Estado podem surgir de situações inesperadas que
caracterizam uma possível e iminente guerra, os internos são sua ameaça
constante, e dos quais o Estado depende para o policiamento e a defesa
militar. Por isso se tornaram imprescindíveis os investimentos em
biopolítica (seja pelo próprio Estado ou pelas filantropias na sociedade
civil, ajustando, inclusive condutas à moralidade). No exterior, os
Estados se relacionavam com adversários competitivos, capazes de
submeter e governar colônias ou outro Estados \emph{fracos}
institucionalmente. No interior, o Estado moderno, diariamente, tinha de
conter a iminência da guerra civil, fazendo da política uma guerra
prolongada por outros meios que deve sobredeterminar a outra. A
biopolítica produz outras subjetividades fortalecedoras daquelas que o
contrato introduz nas relações de trabalho como garantias ao
trabalhador. A dominação produz sujeições e espera que essas sejam
pacíficas por meio dos assujeitamentos, o amor à obediência por meio da
segurança ao indivíduo, produzindo condutas que relacionavam sujeições e
assujeitamentos em função da contenção da guerra civil sempre eminente.

As resistências às disciplinas e à biopolítica nascente atingiam o
soberano e sua permanência sob eventuais reformas, atravessavam e
ultrapassavam a estratificação social ao delinear percursos
surpreendentes capazes de enunciar o inominável. Se as relações de poder
formavam redes, disciplinas e vigilâncias, o poder não se localizava em
um lugar específico ― o panóptico era um dispositivo de vigilância ―, e
essas redes variadas se combinavam pretendendo produzir cada vez menos
espaços de atravessamentos. Em certo sentido foi essa produção de poder
em redes que proporcionou uma conformação que, por mais segura que
fosse, ainda permitia atravessamentos, dissolução de nós e implosões.

Na sociedade de controles, as coisas já não são assim. A introdução de
um modo específico de segurança e defesa pela comunicação-informacional
durante e após a \versal{II} Guerra Mundial redesenhou as relações de poder por
meio da introdução da inovação motivada pela racionalidade neoliberal em
relação ao trabalhador metamorfoseado em capital humano, após o
acontecimento \emph{1968}. Esta metamorfose produziu controles pelos
monitoramentos da inteligência do corpo do indivíduo redimensionado como
divíduo, portanto permeável a variadas identidades, e, por conseguinte,
pleiteador de uma pletora de direitos que não mais se resumem aos três
direitos fundamentais das sociedades de disciplina, biopolítica e
soberania.

As lutas pelas liberações e libertações contra as sujeições e
assujeitamentos obtiveram não mais contornos nítidos como o espaço, as
fileiras, o exame, a administração do tempo da sociedade das
disciplinas, e tampouco as filantropias e políticas públicas eram
capazes de responder de imediato ao conjunto reivindicatório para
aplacar o perigo enunciado pelas classes populares. O sociólogo Max
Weber, no início do século passado, defendia frente às classes
perigosas, por ele assim nomeadas, com suas sujeiras, pestes e
contestações, a necessidade de políticas compensatórias urgentes pelo
Estado; recomendava um parlamento forte com políticas positivas que
disseminasse a ética da responsabilidade de cada um para conter o avanço
quase incontornável dos socialistas e comunistas; mostrava os limites do
bom governo liberal diante de uma \emph{massa} que se conformava e
trazia perigos ao liberalismo e à democracia. Alertava para a burocracia
estatal sendo procedimental o que elevava os custos do Estado ---
constatação projetada para o socialismo, e que se comunicava com a do
liberal austríaco Ludwig von Mises, que pretendia equacionar a situação
dos monopólios emperrando o liberalismo, fomentando a \emph{massa} e
constituindo um perigo às liberdades individuais ---, situação que o
socialismo levaria ao extremo. Cauteloso, von Mises mostrava que o
socialismo, como teoria e funcionando de modo fechado sem as dinâmicas
da história e do comércio exterior, era coerente. Assim sendo, pelo
governo burocrático e monopolista, o socialismo tenderia a fracassar
diante das demandas de mercado exterior (curiosamente a mesma crítica
que o marxista Eric Hobsbawm elaborará nos anos 1990, em seu festejado
\emph{A era dos extremos}, explicitando, dramaticamente, o revestimento
da racionalidade neoliberal à esquerda do Estado). As relações agônicas
de poder na sociedade de controles foram escapando da relação harmoniosa
ou conflitiva entre sociedade civil e Estado disputada entre liberais,
assim como da oposição contraditória entre Estado e sociedade civil
defendida pelos marxistas.

As máquinas cibernéticas precisam de outras velocidades, assim como as
relações sociais se produzirão também em velocidades surpreendentes.
Virilo e Deleuze com maiores detalhamentos --- visto que Foucault morreu
ainda na primeira metade da década de 1980, quando analisava uma
sociedade disciplinar em transformação, que, por vezes, chegou a chamar
de sociedade de segurança --- ou simples sugestões, alertaram para as
modificações nas relações de poder. Suas reflexões sublinham não só a
velocidade das máquinas como nos movem a constatar que as relações de
poder em fluxo não são meras metamorfoses de relações de poder em rede.
Elas produziram, de outro modo, a explosão nos nós das redes. Trata-se
agora, entre os vivos e como nas máquinas, de relações que exigem
protocolos diplomáticos, interfaces entre programas, geração variada de
aplicativos que fomentam a comunicação constante, contínua e em espaços
abertos, novas modulações na produtividade empresarial. Não há um tempo
calculado e gerenciado sobre o trabalho como no regime das disciplinas.
Espera-se mais, seja porque o capital humano deve inovar, empregado ou
não em um investimento (empresarial ou outro), assim como sua conduta
empreendedora não se restringe às formalidades contratuais traduzidas em
leis, posto que as normas são cada vez mais móveis e mais eficientes que
o formal edifício jurídico verticalizado, que também se descentraliza,
assim como as regras disciplinares.

Aquém da inovação, o divíduo neste fluxo pode produzir vírus, realizar
invasões a computadores, deletar arquivos e programas, divulgar
documentos secretos e ``balanços'' de empresas, espionar e interceptar
protocolos, enfim, tornar-se um perigo, pois sua maneira de resistir não
depende mais da estratificação social e de concentrações públicas. Como
divíduo, ele solitário, pode animar interceptações do modo pelo qual
funcionam as interfaces diplomáticas. E, num instante, ele pode passar
de inimigo criminoso a agente de segurança para o \emph{sistema}, e sua
atitude resistente (de contrapoder) em breve tempo pode se metamorfosear
em conduta de segurança, e é isso que se aprendeu a esperar desse
sujeito resiliente: obter reconhecimento de sua capacidade inovadora de
modo meritocrático. Trata-se de um livre exercício voluntário em função
de ser ocupado no interior de regras móveis do jogo. Em poucas palavras,
ainda que guarde certa familiaridade com o anarco-terrorista do século
\versal{XIX} e do início do século \versal{XX}, seu alvo é outro, não é a vida física do
soberano e o fim de sua metafísica.

Alguém poderia dizer que um tal mecanismo também opera na sociedade das
disciplinas. Todavia é preciso lembrar que tal façanha estava
relacionada à cooptação, de modo que o exército de reserva de poder,
funcionando ao lado e com a polícia, também recebia suas recompensas
segundo a moralidade. Na sociedade das disciplinas as resistências
indisciplinares radicais (anarquismo e comunismo) redundavam em
subjetivações voltadas a possíveis revoluções que repercutiam na
construção de uma nova moral a governar as práticas éticas de cada um.
Agora, um ponto reluzente que emana na sociedade de controles está na
relação agonônica de poder e resistências em fluxos: o do militantismo
livre das formas organizativas convencionais onde a produção da ética
está voltada para a interrupção do jogo político entre morais, seja
considerando-as a partir dos cuidados de si (governar menos ou não
governar), da evocação das práticas anarquistas nos movimentos
posteriores à antiglobalização (Newman, 2011) ou mesmo das práticas
autogestionárias da multidão em luta contra subjetividades capitalistas
atuais como a do endividado, do mediatizado, do securitizado, do
representado capaz de elevar-se como singularidade na multidão (Negri \&
Hardt, 2014). A presença da ética como princípio e fim da política que
começa no sujeito (divíduo ou indivíduo) se sobressai (Foucault, 2004),
seja para constituir uma faceta da nova política, seja para produzir
mais potências antipolíticas. As minorias potentes (Deleuze, 2010)
emergem, o militantismo (Foucault, 2011) ganha visibilidades e ambos
enunciam a problemática da ética de liberdades que enfrentam as dos
liberais e neoliberais.

As relações de poder em fluxo situam o verso e o reverso do capital
humano inovador e resiliente na imperativa convocação à participação
moderada em cada ambiente. A política não é mais a guerra prolongada por
outros meios na medida em que a guerra como tal se metamorfoseou em
guerra-fluxo e estados de violência monitoráveis. A política é
investimento contínuo em participação supondo a concretude do estado de
paz em construção pelo conjunto de condutas inovadoras e resilientes
permanentemente monitoradas. A política depende de uma
governamentalidade planetária que funciona por horizontalidades e
verticalidades, conduzindo condutas governadas em função do controle de
cada um e dos outros, educando para a construção da vida resiliente em
um planeta resiliente revestido de direitos e agilizados por seus
portadores.

A política não é mais a instância societal por excelência, nem a
superestrutura da sociedade com função de manter dominação e exploração.
A política pelas relações democráticas de poder em fluxo busca modular
as relações no sentido de garantir a vida e o planeta conservados no
futuro. Os resistentes antipolíticos, portanto, nesta energia maquínica
enfrentam as modulações que governam pela inovação e o empreendedorismo
do capital humano (onde se inclui a dosagem equilibrada entre interesses
privados e interesses coletivos) como os que buscam novas formas de
efetivar uma reforma urgente na sempre dinâmica relação entre Estado e
sociedade civil. Os exercícios de dominação e exploração passam a ser
equacionados a partir de supostos redutores meritocráticos e
democráticos, pretendendo dissuadir cada um da dominação e da exploração
pela inevitável naturalização das desigualdades fundadas pela economia
política. Apresenta-se uma nova materialidade de gestão de minorias que
funciona. Até quando? Entretanto, se os primeiros terão de se
desvencilhar das moralidades, para os demais a moral de Estado é o
parâmetro da continuidade da ordem e da produção de regras móveis dos
jogos que produzem protocolos diplomáticos. A segurança do Estado em
ambos os itinerários (de ordem e contra-ordem, ou de gestão da vida e de
nova gestão da vida), permanece fundamental, porém, a gestão calculista
da vida nos termos biopolíticos da sociedade das disciplinas cede lugar,
gradativamente, à gestão compartilhada dos viventes, como traça a
ecopolítica. São relações de poder em fluxos, de políticas e ecopolítica
que têm por meta inibir resistências, pois estas são inevitáveis às
relações de poder e redimensionadas pelos efeitos múltiplos que produzem
governança nos ambientes. Não se trata mais da indisciplina nos espaços
delimitados, mas da revolta em ambientes. Aos Estados-nacionais ou
federação de Estados cabe a segurança diante de terrorismos
transterritoriais, contenção de revoltas desterritorializadas e controle
específico de monitoramento de inteligências e condutas.

A produção de subjetivação neoliberal abre para a revisão constante do
que é a conduta perigosa ou criminosa, produzindo, pela disseminação de
direitos de minorias, variações penais de encarceramentos e penalizações
a céu aberto, \emph{melhorias} no governo dos infratores e gestões
compartilhadas das penalizações. Produz a relação compartilhada de
intimidades pelas \emph{redes sociais}, assim como busca interceptar a
convocação à participação, que lhe é cara e cuidadosamente zelada, em
movimentos que contestam a ordem. Também redimensiona os doentes mentais
em programas de saúde mental, produzindo novas gestões da loucura em
espaço aberto, e, com o auxílio da psiquiatria e das neurociências,
redefine a loucura como \emph{transtorno} (Passetti, 2013; Siqueira,
2010). Não há mais uma classificação sobre a loucura, mas índices e
indicadores de transtornos a que cada divíduo (obviamente, por ser um
divíduo em suas práticas, é arquivado em variados bancos de dados) é
passível e para os quais a bioquímica colabora decisivamente com a
medicação da criança ao idoso, do ocupado no trabalho ao desocupado
circunstancial. Uma sociedade que concebe o divíduo também o apanha em
suas fragilidades, dores e temores produzidos e medicados pela
racionalidade neoliberal inovadora que dele necessita extrair a
produtividade inteligente governada pelo \emph{dispositivo resiliência}.
Planeta, países, cidades, vilas, bairros, favelas, o divíduo, tudo deve
ser resiliente, segundo a sua família, convencional ou não, a comunidade
em que habita e que defende, o bairro que preserva, as vilas restauradas
para o turismo, assim como cidades e países sustentáveis, produzindo o
nivelamento no desenvolvimento que não mais opõe subdesenvolvidos,
desenvolvidos ou emergentes.

Planeta resiliente é a meta, durante e depois, dos Objetivos de
Desenvolvimento Sustentável (\versal{ODS}). As relações de poder em fluxos
modulam identidades, combinam direitos, convocam à participação,
produzem \emph{sensações} de segurança cada vez mais velozes, disseminam
empreendedorismos e reformam as velhas instituições. A escola, agora
para todos, deve ser governada por \emph{todos} (alunos, funcionários,
pais e comunidade) em função da moderação e do equilíbrio orçamentário a
ser alcançado. A escola ainda é o espaço disciplinar exemplar para
ensinar a obedecer, mas para formar o capital humano é imperativo que
ela se democratize, metamorfoseie as disciplinas, acelere as relações
monitoradas, aplaque a indisciplina, conecte empregos e talvez produza,
em definitivo, uma coleção de Gregor Samsa. Diferente da escola da
sociedade das disciplinas que ensinava a desempenhar uma função na
divisão técnica do trabalho, ela deve preparar o empreendedor e, por
conseguinte, se conforma como escola-empresa.

A utilidade da inteligência produtiva não exclui mais mulheres,
homossexuais, deficientes físicos, raças, etnias e toda a série de
minorias numéricas a ser contemplada pela empregabilidade empreendedora
de modo direto ou por meio de cotas. Os Estados as reconhecem como
justificativas reparadoras de injustiças passadas, regulamentando-as
segundo minorias identitárias empoderadas dentro e fora de
universidades, do mesmo modo que empresas, programas de saúde gerais,
produção e distribuição de medicamentos recomendados, exames médicos
agendados, simultaneamente a uma série interminável de \emph{gadgets}
(apetrechos tecnológicos e aplicativos) e \emph{widget} (programas que
funcionam como atalho tanto para sites como para equipamentos móveis)
destinados ao adequado monitoramento diário da sua saúde medicinal,
corpos ou contatos sexuais produzidos para o melhor controle de
rendimento de si e dos outros. O divíduo deve se sentir incluído de
qualquer maneira; a inclusão passa a ser a plavra-chave principal do
discurso da verdade na sociedade de controles.

Uma sociedade de energias inteligentes como essa necessita mais de
cérebro e menos de musculatura e físico como na mecânica das
disciplinas. Portanto, o corpo não estará mais visivelmente marcado pela
história que o arruinava como no passado. Ele, agora, passa a ser a tela
para inscrições de tatuagens, pequenas cirurgias corretivas e plásticas;
pode ser esculpido e deve transparecer jovialidade. O sucesso do
empreendedor está em sua reforma constante capaz de apresentá-lo
modificado, restaurado, remodelado, enfim, sempre jovem (na medida do
que ainda é possível: nada como encontrar uma senhora de 45 anos e lhe
dizer que aparenta 31, ou um senhor de 65 anos que deve aparentar 51, ou
seja, todos devem estampar um visual rejuvenescedor; e, obviamente, com
isso vêm as reformas de previdências e a proliferação de seguros e
organizações de aplicação das ``economias'' do capital humano, inclusive
para efetivar as transformações de sexo).

As relações de poder em fluxo se dão a céu aberto, em ambientes
monitorados, nas \emph{nuvens} eletrônicas, no espaço sideral e produzem
compartilhamentos. Desse modo, educando pela resiliência, não há mais
nada além de condenar tudo o que \emph{escapa} como vandalismo ou,
quando não, de anarquismo. O Estado aos poucos se vê em condições de
usar a violência mais esporadicamente contra os insurgentes (o que não é
sinônimo de matanças de suspeitos) e conta com a colaboração da
sociedade civil organizada para zelar com segurança nas manifestações de
ruas, quando acontecem, e também equipada de componentes eletrônicos. A
sociedade de controle, pelo compartilhamento, pretende inibir as
resistências.

Não se trata mais de implicar diretamente resistências e relações de
poder, mas situar neste deslocamento pelos controles como os fluxos
proporcionam resistências e resiliências. Talvez, assim, possa ficar
claro como se dá a incorporação do discurso sobre as redes sociais
digitais e como se distingue rede de fluxo na ecopolítica, onde corpos
são governados e se governam pelo uso da inteligência produzindo
governanças. A permanência dos chamados terrorismos aos moldes da
sociedade disciplinar já não mete medo; o medo advém do terrorismo e da
revolta transterritorial de prática antipolítica; o medo do Estado
permanece no campo das normalizadas condutas criminalizáveis. A
espionagem foi renomeada, principalmente a partir dos eventos
desdobrados pelas denúncias da Wikileaks (Serres \& Obrist, 2013) por
Julian Assange, em agosto de 2010 (Assange et al., 2013), e que
culminaram nas denúncias do colaborador terceirizado da \versal{NSA} (National
Security Agency), Edward Snowden (Greenwald, 2014), e demandam novas
regulamentações.

A liberdade democrática na internet ficou explicitada com o regime do
monitoramento contínuo e seguro. A democracia na internet não é um
simples exercício livre de direito, o que é compreensível como retórica
contra Estados não democráticos, porém, a comunicação constante
monitorada de cada um a cada Estado está preferencialmente relacionada
aos avanços desta tecnologia nos investimentos no espaço sideral e
respectiva segurança (Siqueira, 2015). Trata-se, portanto, de um
acontecimento que move a comunicação (segundo Deleuze, é sempre bom
lembrar que comunicar não é sinônimo de resistir), posto que subjacente
à convocação à participação encontra-se o monitoramento constante de
todos, haja vista a institucionalidade de governos de si e dos outros
nos mais variados ambientes conectados ou não eletronicamente pelas
tecnologias de inteligência. Cada divíduo deve corresponder a um
cidadão-polícia exponencial de sua vida e da vida dos demais viventes no
planeta pelo pastorado horizontal, com ou sem uso ainda da comunicação
eletrônica, assim como cada Estado corresponde a um Estado-polícia da
vida real, imaterial e virtual que é capturada. Por essa vida se
transita ou fecunda conectado às tecnologias de governo da inteligência.

A política não está interditada e não é mais a guerra prolongada por
outros meios. A guerra, quando acontece, é uma guerra-fluxo e tangencia
os estados de violência. Muito menos o homem é um animal de cuja
política depende sua vida. As políticas se complementam governadas por
um cidadão portador de direitos que exige segurança em seu ambiente. Por
mais de uma vez este cidadão se esquecerá do Estado, a não ser quando
obrigatoriamente convocado para as eleições quando o quer limpo de
corrupção, diante de modulações da participação em que se reivindica
algo compensatório, ou em um nada acidental cara a cara com a polícia
repressiva. Cada vez mais, para ele, a política se torna secundária ou
necessidade específica; sua esperança repousa na \emph{obrigação} do
Estado em lhe dar algo em troca de sua vida moderada\footnote{Os
  contrastes entre as \emph{jornadas de junho de 2013} e os chamados
  ``rolezinhos'' do verão 2014 --- quando o funk ostentação transitou
  pelos shoppings de periferias, explicitando o simulacro estético
  burguês, preconceitos, simulando protesto e dissimulando sua
  moderação, sob o efeito recorrente e óbvio da polícia repressora e da
  segurança privada--- provocam análises interessantes. Do ponto de
  vista da participação, os esforços intelectuais são arquivados para
  prever possíveis desdobramentos no futuro imediato, conectando os dois
  movimentos, o que seria positivo se excluídos os radicais; os
  intelectuais moduladores procuram em suas análises midiáticas
  argumentos para politizar o rolezinho a partir do déficit de
  equipamentos públicos e sociais; as análises se voltam para como
  incluir mais segundo a participação esperada com redutor de aversões a
  estas minorias e ao mesmo tempo educá-las ainda mais para a conduta
  resiliente, e eles experimentaram e vivenciaram seus fugazes momentos
  de celebridades (cf.
  \emph{http://www.pucsp.br/ecopolitica/galeria/galeria\_ed6.html}).}. E
para tal o Estado precisa estar limpo da corrupção. O divíduo entende
sua postura ética como fortalecedora da moral monitorando o Estado, os
políticos e suas conexões com empresarios. Por sua vez, os empresários
fortalecem sua nova filantropia financiando Institutos, Fundações e \versal{ONG}s
voltados para a gestão das condições de superação de pobreza e
concomitante formação do capital humano.

A relação consigo ― móvel e flexível ―, e com os outros encontrou
evidências radicais no Brasil com as \emph{jornadas de junho de 2013},
situando forças em luta, sob uma configuração específica da
\emph{apatheia} como a problematização da entrega ao governo de bem
público privatizado: os transportes. A predominância das análises com
base no conceito de multidão ganhou destaque a partir de análises mais
densas (Pélbart, 2013), rápidas e incisivas (Orlandi, 2013), descritivas
e favoráveis (Judensnaider et al., 2013; Vainer et al., 2013), e também
em estudos com base em um marxismo gramsciano (Nogueira, 2013). Em
comum, todas expressam a \emph{esperança} em mudança para melhor, vendo
positividades nos protestos como expressão da convocação à participação
por uma nova política, por vezes nomeada de biopolítica da multidão. A
ênfase na heterogeneidade, na dissimulação da identidade, sob o
anonimato mascarado ou o reconhecimento, na exposição de limites da
política partidária e institucional perde de vista a contribuição ao
\emph{status quo}, o arrefecimento posterior aguardando a ocasião, a
própria esperança de continuidade e, por vezes, negligencia o efeito
moral do juízo sobre os que são configurados como \emph{inimigo}, no
caso a força de fogo \emph{black bloc}\footnote{A presença constante da
  tática \emph{black bloc} desde os primeiros momentos em Seattle produz
  continuidades gradativamente disseminadas pelo planeta e chegou ao
  Brasil, assim nomeada, durante as chamadas \emph{jornadas de junho de
  2013}. A tática, como tal, para existir necessita da concentração
  previamente anunciada de um protesto político. Neste sentido, ela não
  é organizativa nos moldes tradicionais e por isso é surpreendente,
  seja por aglutinar anarquistas, seja por absorver segmentos
  pauperizados mendigando pelas cidades (Cf. Thompson, 2010; Deusen \&
  Massot, 2010; Bray, 2013, Passetti \& Augusto, 2014). A primeira
  manifestação \emph{black bloc} é atribuída ao grupo \emph{Love Lace},
  que atacou a sede do \versal{FMI}, em Washington, durante a Guerra do Golfo.}.

As resistências anarquistas, historicamente inquestionáveis, hoje passam
pela captura acadêmica e apresentam aspectos resilientes. As práticas
anarquistas ganham dimensão política quando acompanhadas da revolta,
ponto nodal dos anarquismos desde Proudhon, passando por Stirner, o
libertarismo do final do século \versal{XIX} e início do século \versal{XX} e que encontra
hoje certa disposição à resiliência a partir dos anarquistas
estadunidenses (Passetti, 2013b), em especial a relevância da relação
anarquismo/ecologia a partir do \versal{ISE} (Institute for Social Ecology),
fundado por Murray Boockchin, em 1974, que elaborou o conceito de
ecologia social relacionado ao municipalismo libertário (Augusto, 2012)
e que também tem por destino ser capturada pelo desenvolvimento
sustentável.

O monitoramento de ambientes resilientes e seguros produz uma nova forma
de comunicação instantânea que diz respeito a como conceitos e práticas
historicamente situadas como as anarquistas acabam se metamorfoseando em
palavras-chave. A destacar: ação direta, autogestão, ecologia social.
Outras, ou as mesmas, revestidas de significados repaginados, alimentam
o vocabulário dos ativistas em geral. Pela velocidade rápida, eles se
atêm muitas vezes à palavra ou ao \emph{novo} direcionamento sem acionar
a reflexão (por exemplo, o governo provisório de Michel Temer, em 2016,
no Brasil não incluiu negros e mulheres, e não foram poucos, senão a
grande maioria de contestadores de seu governo como ``golpe'', que
saíram às pressas exigindo inclusão).

O clima passou a ser o grande catalizador das mudanças anunciadas
conectando o humano (Rodrigues, 2012a). Cientistas alertam para os
riscos do planeta e as reuniões políticas caminham dentro dos parâmetros
capitalistas em função de um desenvolvimento sustentável\footnote{\emph{http://revistas.pucsp.br/index.php/ecopolitica/article/view/11385/8298}}
sintonizado com as Metas do Milênio\footnote{\emph{http://newsroom.unfccc.int/}}
e agora, com os \versal{ODS}. Iniciadas em Berlim, em 1995, as reuniões do \versal{COP}
(Conference of the Parties), referentes à Convenção Quadro das Mudanças
do Clima, estabelecida em 1992, e à criação do Intergovernamental Panel
of Climate Change ― \versal{IPCC}, situavam o desenvolvimento sustentável com
erradicação da pobreza, buscando encontrar uma situação estável para o
capitalismo e seus efeitos degradantes mediante adaptações das
atividades humanas em função das mudanças climáticas.

A produção do saber sobre a Terra, a partir dos efeitos atômicos e
constatação da degradação vivida e expressada por cientistas, movimentos
sociais, contestações localizadas, redesenho das relações
internacionais, a partir do final da chamada Guerra Fria, levaram a
noções como \emph{antropoceno} e à necessária busca de diplomacia
(Danowiski e Castro, 2014; Latour, 2013). A questão que se coloca situa
a noção de natureza ou ambiente como algo restrito ao saber das ciências
naturais. Danowiski e Castro salientam que a sustentabilidade tem
eficácia local, porém, em âmbito global ainda é uma ficção; mostram a
dimensão que a produção de saber sobre os efeitos do clima produz diante
das possibilidades diplomáticas de convergências possíveis entre a
degradação realizada por Humanos (os modernos) e as novas invenções
possíveis pelos Terranos (um povo em invenção), ou seja, é preciso que
os modernos se reconheçam como responsáveis para que uma diplomacia seja
possível. ``O que o antropoceno põe em cheque, justamente, é a própria
noção de antropos, de um sujeito universal (espécie, mas \textbf{também}
como multidão) capaz de agir como um \textbf{só povo}'' (Danowiski e
Castro, 2014: 121, grifos dos autores). A reviravolta produzida por
saberes que ultrapassam as ciências naturais também desemboca na crítica
ao sujeito universal, à nova feição do sujeito universal como multidão,
aos moldes de Negri e Hardt, ``pois é difícil conceber o povo de Gaia
como uma Maioria, como a universalização de uma boa consciência
`europeia'; os Terranos, podem ser um povo `irremediavelmente menor'
(por mais numerosos que venham a ser), um povo que jamais confundiria o
território com a Terra (...) \emph{o povo por vir}, capaz de opor uma
`resistência ao presente' e de assim criar `uma nova terra', \emph{o
povo por vir}'' (Ibidem: 125-126).

Ainda que, gradativamente, a produção de um novo saber atento aos moldes
nos quais povos não-modernos, os ameríndios, esteja adaptando a retórica
de ambientalistas a suas cosmologias e obtendo repercussões, a questão
principal ainda repousa com o que se passa ``nas gigantes metrópoles
técnicas'' (Ibidem, 127). Será por meio das técnicas que o embate entre
Humanos e Terranos se estabelecerá compondo técnicas antigas e recentes,
pois ``nem toda inovação técnica crucial para a \textbf{`resiliência'}
da espécie precisa passar pelos canais corporativos da \emph{Big
Science} ou pelas longuíssimas redes de humanos e não-humanos
mobilizadas pela implementação de `tecnologias de ponta''' (Ibidem: 131,
grifos meus).

Tomemos a plataforma Intergovernamental Plataform on Biodiversity and
Ecosystem Services ―\versal{IPBES}. A \versal{IPBES} tem por objetivo organizar o
conhecimento sobre biodiversidade para subsidiar decisões políticas em
âmbito mundial. Sua criação foi ratificada durante a \versal{COP}10, da
biodiversidade, em Nagoya. Segundo a \versal{IBPES}, os que tomam decisões
precisam de credibilidade científica e informação independente que levem
em consideração as relações entre biodiversidade, serviços
ecossistêmicos e populações. Fazem-se necessários métodos efetivos que
interpretem as informações científicas, para tomadas de decisões mais
bem fundamentadas. Ao mesmo tempo, a comunidade científica também
precisa saber das necessidades dos tomadores de decisões para provê-los
com informações relevantes. O diálogo entre a comunidade científica,
governos e outros \emph{stakeholders} em biodiversidades e serviços
ecossistêmicos deve ser fortalecido. A \versal{IPBES} possui um mecanismo
reconhecido, tanto pela comunidade científica, quanto pela comunidade
política internacional, para sintetizar, acessar e criticar informações
relevantes geradas por governos, academia, organizações científicas,
organizações não governamentais e até comunidades indígenas.

A primeira plenária da \versal{IPBES} aconteceu em Bonn, na Alemanha, em janeiro
de 2013. Durante a segunda plenária, em Antalaya, na Turquia, de 9 a 14
de dezembro de 2013, foi aprovado o marco conceitual da
plataforma\footnote{\emph{http://www.biotaneotropica.org.br/v14n1/pt/editorial}}
que ``fornecerá a referência teórica para o trabalho a ser desenvolvido
pelos especialistas nos próximos anos''\footnote{\emph{http://agencia.fapesp.br/ipbes\_aprova\_plano\_de\_trabalho\_e\_orcamento\_para\_os\_proximos\_5\_anos/18410/}.}.
O marco foi produzido pelo Painel Multidisciplinar de Especialistas da
\versal{IPBES}. Carlos Joly, coordenador do programa \versal{BIOTA}-\versal{FAPESP}, e Mark
Lonsdale, chefe de Serviços Ecossistêmicos da Commonwealth Scientific
and Industrial Research Organisation ― \versal{CSIRO}, da Austrália, foram
eleitos diretores do Painel Multidisciplinar de Especialistas da \versal{IPBES} ―
\versal{MEP}, na sigla em inglês. Segundo Joly, um dos principais objetivos do
\versal{MEP} é a ``integração do conhecimento científico com outros sistemas de
conhecimento, especialmente o de comunidades indígenas e
locais''\footnote{\emph{http://agencia.fapesp.br/carlos\_joly\_e\_eleito\_para\_direcao\_de\_painel\_do\_ipbes/17398/}}.

No interior do objetivo 1 (fortalecer as bases de capacidade e
conhecimento da interface ciência-política para implementar funções
chaves da plataforma), o item D diz: ``procedimentos, abordagens e
processos participativos para trabalhar com sistemas de conhecimento
indígenas e locais''\footnote{\emph{http://www.ipbes.net/work-programme/objective-1/45-work-programme/453-deliverable-1c.html}}.
A \versal{IBPES} tem por objetivo promover o engajamento de detentores de
conhecimentos indígenas e locais na construção da plataforma. Para isso,
formou-se uma força-tarefa para desenvolver uma lista e uma rede com
peritos; oficinas e diálogos globais com especialistas sobre
conhecimentos indígenas e locais; procedimentos e abordagens para
trabalhar com sistemas de conhecimento indígenas e locais, além de
estabelecer um mecanismo participativo para sistemas desses
conhecimentos na plataforma. Essa força-tarefa é composta por
especialistas, que compõem o quadro de membros de especialistas
multidisciplinares do painel (\versal{MEP}), por especialistas selecionados
especificamente para esse trabalho e por membros do \emph{bureau} do
painel. A força-tarefa sobre conhecimentos indígenas e locais elaborará
procedimentos, enfoques e processos participativos para trabalhar com
esses sistemas de conhecimentos.

Segundo o documento, o Marco Conceitual será ``um terreno comum básico,
geral e inclusivo'' entre os diferentes sistemas de conhecimento
reconhecidos pela plataforma ― o científico, o de comunidades locais e
indígenas ― para que se possa alcançar o objetivo principal da
plataforma: ``fortalecer a interface científico-normativo entre a
diversidade biológica e os serviços ecossistêmicos para a conservação e
utilização sustentável da diversidade biológica, o bem-estar dos seres
humanos a longo prazo e o desenvolvimento sustentável''\footnote{\emph{http://www.ipbes.net/images/documents/plenary/second/working/2\_4/IPBES\_2\_4\_ES.pdf}}.

O documento propõe uma paridade entre os diferentes sistemas de
conhecimento. Enquanto para alguns o objetivo é \emph{qualidade de vida
/ bem-estar}, para outros, procura-se \emph{viver em harmonia com a mãe
terra}. Esse ponto é constantemente salientado pelos que compõem o
quadro. Segundo Joly, ``é um marco inovador e muito moderno pois foi
escrito em três diferentes linguagens: a linguagem científica
consolidada (usando termos como `biodiversidade', `ecossistemas' e
`evolução'), a linguagem dos povos tradicionais andinos (`mãe terra',
`sistemas de vida' e `vivendo em harmonia com a mãe terra') e linguagem
transcultural consolidada na interface entre ciência e política dos
acordos internacionais (com termos como `qualidade de vida' e
`benefícios da natureza para as pessoas')''.\footnote{\emph{http://agencia.fapesp.br/ipbes\_aprova\_plano\_de\_trabalho\_e\_orcamento\_para\_os\_proximos\_5\_anos/18410/}.}

Em função da atualização do capitalismo, investe-se em governar
populações que habitam áreas com serviços ecossitêmicos ainda
abundantes. Ao mesmo tempo, elabora-se um discurso que incorpora outros
conhecimentos e fortalece princípios democráticos. A governamentalidade
planetária vai se consolidando ao produzir a interface definitiva entre
diferentes culturas, compartilhando modos de vida locais à
sustentabilidade. Nesse sentido, sobressai uma clara relação com a noção
de \emph{bem viver}, que tem como proveniência a cosmovisão indígena e
que em alguns países da América Latina, como Bolívia e Equador, ganhou
força no interior da política institucional como componente das práticas
democráticas de governo do Estado, consolidando a participação e
ampliando as resiliências. Se o \emph{Bem Viver} é a consolidação dessa
articulação entre Estado, capitalismo, conhecimento científico e
populações indígenas em âmbito nacional, a \versal{IPBES} procura desenvolver
essa relação no âmbito internacional da \versal{ONU}.

Na perspectiva genealógica está em questão a crítica local, ``algo que
seria uma espécie de produção teórica autônoma, não centralizada, ou
seja, que, para estabelecer sua validade, não necessita da chancela de
um regime comum'' (Foucault, 1999: 11). Trata-se, segundo Foucault, de
uma reviravolta de saber, uma insurreição de saberes sujeitados: ``os
conteúdos históricos que foram sepultados, mascarados em coerências
funcionais ou em sistematizações formais''; blocos de saberes históricos
presentes e disfarçados ``no interior de conjuntos funcionais e
sistemáticos'' (Ibidem). Mas os saberes sujeitados também são
compreendidos como não conceituais ou suficientemente elaborados. No
caso da crítica local nas pesquisas de Foucault, é o que está em
paralelo e marginal comparado ao saber médico (o psiquiatrizado, o
enfermeiro, o delinquente), um ``saber diferencial incapaz de
unanimidade e que deve sua força apenas à contundência que opõe a todos
que o rodeiam'' (Ibidem: 12), mas é desqualificado, e se explicita no
saber histórico das lutas. ``Chamemos, se quiserem, de `genealogia' o
acoplamento dos saberes eruditos e das memórias locais, acoplamento que
permite a constituição de um saber histórico das lutas e a utilização
desse saber nas táticas atuais'' (Ibidem: 13). Trata-se de fazê-los
intervir contra a instância teórica unitária ``em nome de um
conhecimento verdadeiro, em nome dos direitos de uma ciência que seria
possuída por alguns'' (Ibidem). As genealogias são anticiências,
insurreições, ``entre saberes centralizados de poder que são vinculados
à instituição e ao funcionamento de um discurso científico e organizado
no interior de uma sociedade como a nossa (...); é exatamente contra os
efeitos de poder próprios de um discurso considerado científico que a
genealogia deve travar o combate'' (Ibidem: 14). Trata-se de um
enfrentamento para desassujeitar os saberes históricos e torná-los
livres, uma reativação dos saberes ``menores''.

As práticas democráticas decentralizadas na convocação à participação
nas sociedades de controles operaram um redimensionamento inédito, como
expressa a plataforma \versal{IBPES}. Trata-se de reconhecer saberes outrora
sujeitados como componentes compartilhados para as novas plataformas de
saber que combinam o saber das ciências e esses insurretos que se
encontravam em paralelo ao marginal. Da mesma sorte, encontramos esses
saberes sujeitados no campo da saúde mental pela descentralização
derivada da luta antimanicomial, até mesmo na constituição do Equador,
fundada no \emph{Bem Viver} de populações indígenas tradicionais
sobreviventes. Trata-se, também, de reconhecer a colaboração
compartilhada necessária e suficiente ao desenvolvimento sustentável em
vigência ou após ele. É o marco inovador.

Por sua vez, a genealogia debruça-se sobre essa nova unificação
decentralizada de saberes, que vai da \versal{ONU} aos ameríndios, para traçar as
resistências que escapam das convencionais unificações teóricas ou de
sua mistura estratégica para realçar táticas expressas pelas revoltas. É
nesse sentido que a análise genealógica percorre a inclusão de minorias
guardando referência ao que seja menor como potência, numa convulsão
revoltada contra a maioridade, ou, nos termos de Foucault, ``ainda
devemos estar presos a maioridade kantiana e do Estado'' (Foucault,
2000).

A religião não é uma coisa natural. É a forma acabada de explicação do
sobrenatural. Ela atravessa culturas de maneiras mais ou menos
incisivas. Há deuses próximos dos humanos em várias culturas. Há deuses
distantes dos humanos fundadores da espécie, de uma vida original e do
juízo de valor. Portanto, se há muitas religiões entre povos
civilizadores, algumas delas se pretendem exclusivas. Produzem guerras e
atentados em nome do Superior. Traduzem o sobrenatural em escolha
preferencial e derradeira de deuses. Elas combatem umas às outras e cada
uma guerreia ou negocia o ecumenismo na disputa pela verdadeira
explicação do mundo. Produzem, por vezes, vertentes ramificadas mais ou
menos aceitas, negociáveis e tolerantes, principalmente depois do
Iluminismo. Quando matam e massacram em nome da sua pureza, veracidade e
cuidados com o rebanho, fazem de cada um em seu interior um soldado de
fé.

Ficaram mais complacentes depois de serem implacáveis na conversão dos
povos considerados selvagens. A partir do século \versal{XVIII}, passaram a
tolerar o ateu, e não foram poucas as expressões de modernidade e
laicização do Estado na criação da figura do cidadão livre com religião
própria ou sem religião. A separação formal do Estado da religião pelos
direitos foi somente um meio para reiterar a liberdade de pensamento e
credo privados garantidos pelo Estado. E do ateu, se nada mais é que
alguém descrente do ser superior transcendental, dele se espera a crença
no Estado laico. No limite, o rompimento da relação Estado-religião vai
além do ateísmo. Trata-se uma prática que ultrapassa o direito garantido
como segurança de Estado, ou seja, desconhecer religião como lugar
natural e intrínseco ao humano. Enfim, nenhuma criança nasce ateia ou
religiosa, ambos são efeitos de sociabilidades.

O debate cada vez mais concentrado sobre religião e Estado tem se
balizado pelos efeitos do terrorismo islâmico contra prédios,
escritórios internacionais, estações de transportes ocidentais. Situou
gradativamente a distinção do fundamentalismo, e depois de argumentações
relativas aos fundamentalismos ocidentais, inclusive o relativo ao
Estado laico, foi se deslocando para a distinção entre fundamentalismo
pacífico e violento. Cada vez mais religião e fundamentalismo, ou ainda,
Estado moderno e seus fundamentalismos, viram-se repaginados pelos
efeitos do terrorismo transterritorial acentuadamente definidos como
fundamentalistas violentos, propiciando com isso a isenção dos
muçulmanos pacíficos integrados ou parcialmente integrados ao
fundamentalismo democrático racional e legal do ocidente. Isso se deve à
presença crescente de islamitas na Europa a serem inocentados das
acusações xenófobas e fascistas da direita organizada.

Ao mesmo tempo, a acusação direta de relação entre o terrorismo e
Al-Qaeda, se é pouco nebulosa enquanto organização e vínculo direto, ao
mesmo tempo confirma a existência de uma programação Al-Qaeda sobre o
terrorismo contemporâneo que serve tanto aos dispositivos de segurança
quanto ao próprio terrorismo islâmico. Todavia, com o aparecimento do
Estado Islâmico no Iraque e na Síria, do Boko Haram na Nigéria, a
iminência de um califado governante em territórios fronteiriços de
Estados nacionais coloca novos rumos ao terrorismo islâmico. Ele é
também mais do que efeitos transterritoriais de atentados contra Estados
e sociedades ocidentalizadas. Trata-se de uma forma de organizar novas
formas de guerras-fluxos transterritoriais não mais visando, como no
século passado, ocupar ou destruir o Estado moderno, mas instituir um
combate entre Estados e por efeitos transterritoriais.

O islamismo assombra o ocidente do Estado laico e da sua respectiva
tolerância com as várias religiões. A reviravolta produzida em 1979 pela
chamada revolução iraniana contra o despotismo ocidental firmou o Estado
em uma república islâmica teocrática e os combates contra quem dele
descrê. Este islamismo político precisa tanto de Estado quanto de terror
transterritorial. Não está disponível a uma negociação possível, a
começar pela oposição entre xiitas e sunitas. Não insinua uma guerra
entre deuses, mas a guerra de homens para impor um deus exclusivo em
qualquer lugar. Pode se erguer em um espaço tradicional, como no Irã,
encontrar refúgios em outros Estados, como Al-Qaeda, ou ser uma variante
a partir de outros Estados, como no Iraque e na Síria.

Sobre os islâmicos europeus pacifistas e integrados repousam as
expectativas de condutas democráticas moderadas. Seus direitos são
observados segundo as leis do Estado laico e a eles devem se ajustar com
tolerância. Mesmo assim, por vezes ocorrem tensões com os pacifistas
quando estes pretendem obter reconhecimento de suas condutas ajustadas
aos preceitos religiosos. Estado e religião, mesmo depois do Iluminismo,
permanecem intrinsecamente relacionados. O ecumenismo procura encontrar
espaços para tréguas e negociações. O Vaticano produz a \emph{Laudato
Si'} (Papa Francisco, 2015; Carneiro, 2015) para sinalizar o destino
sustentável e como esse desenvolvimento preconiza, na fusão de saberes
científicos e tradicionais, a cultura de paz com tolerância e
resiliência que depende dos pastores dos deuses e de seu rebanho
bifurcado entre o fluxo à devoção e as variações proporcionadas pelo
multiculturalismo.

\chapter{Ecopolítica, governamentalidade planetária e resistências}

Compreender a ecopolítica produzida pela relação ascendente, descendente
e horizontal de fluxos de poderes a partir das análises genealógicas de
inspiração foucualtiana, não perdendo de vista a análise serial
proudhoniana sobre o agonismo constante e a ausência de absoluto,
efetivou a constatação da passagem da biopolítica, ainda que isso não
signifique a sua supressão nos efeitos internos. Passamos pelas
implicações dos usos de biopolítica como a extraída da noção de
raça\footnote{O termo biopolítica aparece incialmente no pensador,
  defensor do racismo científico e militante peronista argentino,
  Jacques de Mahieu (1969 {[}1968{]}).} por Foucault, às incursões de
Negri e Hardt, além daquelas de Agamben. Menos que disputar a verdade
sobre o uso da noção de biopolítica, ou mesmo da noção de ecopolítica na
atualidade (de nova ordem futura a política compensatória), a atenção
permaneceu retida em como se dá a produção de uma nova
governamentalidade planetária e suas respectivas institucionalizações.

O estudo regular da produção institucional internacional quanto às
proveniências da ecopolítica (em meio, ambiente, ecologia, meio
ambiente) desde o final da \versal{II} Guerra Mundial considera a emergência das
máquinas cibernéticas em ação ainda que rudimentar, mas voltadas ao
controle de populações (censo estadunidense, censo dos campos de
concentração, espionagem de guerra, produção de mísseis para a guerra, a
utilizações da bomba atômica contra civis e os precários testes
nucleares seguintes) e os efeitos tensos na divisão global entre
capitalismo e socialismo. Com a institucionalização da racionalidade
neoliberal a partir do último quartel do século passado, constata-se uma
mudança gradativa para as ampliações das responsabilidades da \versal{ONU} e suas
agências conectadas com organizações empresariais e da sociedade civil
diante da \emph{degradação do planeta} compondo o \emph{dispositivo meio
ambiente}. O capitalismo, diante do avanço soviético, produziu
contenções ao socialismo e surpreendentes adequações. A aceitação pela
\versal{URSS} da \emph{Declaração Universal dos Direitos Humano}s desde a sessão
decisiva da Conferência sobre Segurança e Cooperação na Europa, em
Helsinque, 1975, durante as negociações da \emph{détente}, abriu o
espaço para novas incursões \emph{globalizantes} democráticas e
capitalistas.

O fim da chamada Guerra Fria alavancou a situação da futura Europa, dos
mercados comuns, de específicas políticas de democratização (ainda que
sob as efêmeras e violentas ditaduras como na América Latina), mas, em
especial, inaugurou uma nova forma computo-informacional planetária de
uso econômico, social, cultural, político e pessoal destas máquinas com
a internet e os respectivos controles por seus provedores, combinando
poder centralizado com práticas descentralizadoras. Ao seu modo,
conjugou práticas arborecentes e rizomáticas. A comunicação ficou mais
rápida, constante, e perfurou os limites disciplinares, normalizadores,
as condutas do cidadão e do trabalhador, agora dimensionado em capital
humano e assim avaliado principalmente pelas empresas. Trata-se de um
movimento de convocação à participação nas disputas, no trabalho, no
lazer e nos limites traçados e sempre móveis entre interesses
particulares e coletivos, compondo procedimentos protocolares liberais
por interfaces entre os programas e diplomacia nas relações.

A democracia se acopla às disciplinas e produz novas normalizações,
reescrevendo a produção de poder da hierarquia como um todo, e empodera
cada um; metamorfoseia a doença mental em programas de saúde mental e
gestão dos transtornos, e dissolve a noção de anormal em
\emph{normalizações do normal}, entendendo haver potência produtiva e
inovadora em qualquer um, desde que adequável e disponível a
configurar-se como capital humano. As relações entre o capital e o
capital humano exigem que este seja inovador, parceiro e cúmplice para a
boa vida da empresa. O trabalhador passa a se constituir em empreendedor
de si, e também em possível empreendedor social em seus
\emph{ambientes}. Não se trata mais do indivíduo sob o hábito das
disciplinas, mas de divíduos polivalentes com acesso à produção,
tornando-se portadores de direitos inacabados nos mais variados
\emph{ambientes}. Isso repercute imediatamente em uma nova conduta que
se ajusta, gradativamente, aos imperativos da sustentabilidade (de cada
um e do planeta) e à formatação capitalista em curso do desenvolvimento
sustentável.

Conectando democracia na produção material e imaterial para a cultura de
paz, espera-se uma conduta resiliente voltada para as \emph{melhorias}
no futuro do \emph{planeta} e das novas \emph{gerações}. As repercussões
ocorrem na empresa, nas atividades da sociedade civil, nas articulações
com políticas nacionais e recomendações internacionais de modo
persuasivo e de adesão, fazendo emergir o \emph{dispositivo
resiliência}. As institucionalizações produzem certo redutor no elemento
omissão para a obtenção de consenso, o que coloca em xeque a noção de
massa, do antigo pastorado, e não faz da noção de multidão, mesmo
considerando as variadas singularidades em seu interior, uma noção
substituta imediata. A multidão está numa relação mais próxima ao
desenvolvimento sustentável \emph{alternativo} que busca em seu
enunciado a superação da condição atual capitalista, sem deixar de
reconhecer que as relações entre eles sejam compatíveis, como ficou
explicitado nas decisões da Rio+20. Em outras palavras, a situação do
\emph{antropoceno} exige soluções diplomáticas e a noção de multidão
apresenta uma possibilidade de compreensão ao vaivém em torno do
desenvolvimento sustentável, pois seu alvo é uma nova globalização
possível, em que o Estado permanece como a categoria do entendimento
conectado a uma atualização da noção de comunismo do século \versal{XIX} e início
do \versal{XX}.

A ecopolítica volta-se para mapear essas mobilidades e situar a
distinção entre as transgressões que repõem a ordem (no ambiente da
empresa, do empreendimento social, das relações internacionais, das
resiliências), mas também para o que restaura a ordem (nas cidades, nos
bairros, nas comunidades ou favelas), com os efeitos reativos, a ênfase
no protagonismo social, nos controles de vulnerabilidades e as
possibilidades efetivas impulsionadas pelos Objetivos do Milênio e nas
projeções dos Objetivos do Desenvolvimento Sustentável para esta e a
próxima década. De fato, não se necessita mais do que os resultados
divulgados anualmente para se concluir sobre o desdobramento dos \versal{ODM} em
\versal{ODS}, porém, esse novo trajeto enunciará outros pontos de efetivação das
institucionalizações inacabadas, da governamentalidade
planetária\footnote{A partir desse momento, o Nu-Sol formalizou na
  \versal{PUC}-\versal{SP} o Observatório Ecopolítica.
  \emph{http://www.pucsp.br/ecopolitica/observatorio-ecopolitica/n0.html}}.

A política é simultaneamente a guerra prolongada pelos efeitos internos
da ameaça de guerra civil e pela pacificação temporária e contínua pelas
relações diplomático-militares em âmbito externo. As relações de poder
em rede da sociedade disciplinar se metamorfosearam por meio de relações
de poder em fluxo características da sociedade de controle, na qual se
extrai a energia inteligente, a composição dos divíduos e a convocação à
participação. A política (biopolítica), de cuja existência dependia o
\emph{animal político} para viver nas relações de poder em rede, passou
e passa por \emph{inovações} que produzem novas relações de segurança
(seguros pessoais e securitizações) orientadas pelo \emph{dispositivo
diplomático-policial}, na medida em que também as guerras convencionais
foram sendo substituídas por estados de violência e guerras fluxo. A
política em um \emph{ambiente planetário} governado conecta as relações
entre meio ambiente, segurança, monitoramentos e resiliência para uma
governança global.

A apreensão da \emph{nova política} que se institui, assim como das
condutas \emph{antipolíticas}, foi complementada com a emergência do
\emph{dispositivo monitoramento}. Por meio do seu funcionamento,
compreende-se o governo das condutas como controle de si e dos outros
entre os fluxos em trânsito que o divíduo percorre, sejam eles
materiais, imateriais, traçados das cidades, espaços georeferenciados,
mapeamentos cognitivos, dimensões nanotecnológicas ou desafios
espaciais, incluindo a descoberta de exoplanetas. Normalizado, medicado,
também mapeado, arquivado, controlado em provedores ou nas suas
comunicações constantes, ultrapassando situações vulneráveis em seus
\emph{ambientes} cada vez mais seguros, penalizadores, normatizados e
participativos, o divíduo, assim como o planeta, encontra estabilidades
intermitentes para a racionalidade neoliberal, a sustentabilidade e o
desenvolvimento sustentável por meio do \emph{dispositivo
monitoramento}.

Os programas computo-informacionais, a diplomacia, os protocolos, os
equacionamentos de melhorias e o sexo liberado produzem certos conjuntos
móveis de populações, ambientes, produtividades e participação política.
O doente, o transtornado, o produtivo, a criança e o idoso, seus pais,
os Estados e os organismos internacionais também precisam do
monitoramento ascendente, descendente e horizontal. Exige-se uma nova
polícia composta pelos cidadãos-polícia, em exercício constante, e sem
prescindir da polícia repressiva, para que os protocolos e as inovações,
os empreendedorismos e as políticas compensatórias, os viventes em geral
tenham garantias de segurança climática, humana e suas variadas
flexibilidades. As condutas do cidadão-polícia monitorado e monitorando
compõem as metamorfoses do ambiente e as estabilidades intermitentes dos
fluxos de segurança.

A conexão dos fluxos direitos, segurança e meio ambiente também se
conecta com o de penalizações a céu aberto. Inexorável é o fim das
prisões para jovens, delinquentes e subversivos; os tribunais penais
nacionais e internacionais; entretanto, há uma interpenetração entre o
jurídico-político e o privado no governo das prisões, sem descuidos,
porque dependem das relações com os ilegalismos, as denúncias, a
seletividade penal, as condenações e as mortes. Pouco muda na relação
castigo-recompensa além da transformação do penalizado em agente de
controle de si e dos outros, por dentro e por fora das prisões, na
família, na escola, nas organizações. A norma que \emph{normaliza} na
sociedade de controles é governada pelo princípio democrático de
\emph{penalizar mais e melhor} a todos. A cultura do castigo se atualiza
e prolifera revestida das recomendações à tolerância, cujo ápice é a
tolerância zero como tentativa de suprimir o intolerável, o
insuportável, o ingovernável.

É no estranho movimento das resistências produzidas por relações de
poder em fluxo nesta política como algo a mais que guerra civil ou
somente guerra prolongada por outros meios que acontecimentos
surpreendem a respeito da vida dos viventes. Não se trata de nostalgia
do \emph{movimento 68}, nem tampouco de aderir à reativa extirpação dos
efeitos deste movimento, como a literatura histórica pretendeu fazer com
os anarquismos após a Revolução Espanhola. De fato, muitas das
contestações que emergiram em \emph{68} foram capturadas como direitos
pela racionalidade neoliberal. É uma evidência. Este é também o
\emph{ambiente} acolhedor dos teóricos que subestimam o acontecimento.
Entretanto, sobressai o ingovernável em \emph{1968}, em \emph{1999}. Do
ponto de vista estratégico, as resistências assumem outras dimensões e
performances. Usam, abusam, criam, sabotam e inventam no interior das
máquinas computo-informacionais; invertem os sinais da convocação à
participação mobilizando para protestos e ao mesmo tempo produzindo
relações horizontalizadas que vão tomando perfis claros a cada
acontecimento subsequente, orientados pelas práticas radicais
anarquistas de associação e ação direta.

Obviamente que são criminalizadas e estigmatizadas; algumas, por certo,
são capturadas, mas a cada novo momento uma produção silenciosa sobre os
embates anteriores e a instalação do ingovernável se torna visível e
dispensada da transparência burocrática das práticas
político-partidárias e estatais que acabam sempre enoveladas em suas
intrínsecas corrupções. Suas presenças constantes não as protegem de
infiltradados, de \emph{massas} patrióticas e fascistas que se imiscuem,
mas suas atitudes sinalizam para como o sistema parlamentar absorve,
paulatinamente, os movimentos de direita e/ou neonazistas como partidos
políticos e, consequentemente, explicita a \emph{verdade pluralista} do
Estado e as polícias repressivas à sua disposição, com ou sem o suporte
das milícias. As resistências \emph{antipolíticas} derivadas de
\emph{1999} sabem lidar com o apartidarismo de ocasião e a proliferação
dos partidos. Trata-se de resistências em fluxos, sem lugar específico
para atuar, cujas táticas são de explicitação dos efeitos da
propriedade, do Estado e do fim da política, não mais como um novo
recomeço, posto que este já se encontra consolidado com a convocação à
participação e a nova política, mas com ação direta. E para esta atitude
libertária, por mais que o Estado se proteja, equipe e monitore, sempre
haverá o surpreendente.

Por fim, o que tanta liberdade em fluxo produzida pela convocação à
participação trouxe para os ambientes que se formam, refazem, distendem
e retraem, ou seja, os ambientes resilientes, monitorados, seguros? As
convencionais distinções entre o centro e as periferias nas cidades, e
principalmente nas metrópoles, não dão mais conta das concentrações
populacionais ordeiras em suas habitações, deslocamentos para o
trabalho, espaços de lazer, escolas, saúde e demais equipamentos
sociais. A distinção não é somente geográfica e tampouco se reduz à
expulsão dos trabalhadores para lugares distantes. Importa buscar como
se vive em cidades a serem restauradas, muitas vezes sob os programas de
eventos internacionais, como se ocupa e governa o espaço, como devem se
ajustar aos controles de pacificação de ambientes e contenção de
nomadismos. É a política voltada para os fluxos que atravessam e
delimitam as possibilidades empreendedoras locais (o que também se
dissemina para populações ribeirinhas, indígenas, situações rurais
ordenadas principalmente por programas de sustentabilidade em função da
potencialidade do capital humano).

As relações paradoxais produzem relações democráticas no interior dos
interesses coletivos locais com o investimento constante em
\emph{melhorar} por meio da instauração de equipamentos sociais
governamentais e demais criados pela população local em parcerias
público-privadas. Abarcam e conectam desde o programa mais interessado,
como o de \emph{pacificação das favelas} relacionado com o tráfico de
drogas nas cidades como Rio de Janeiro e Medelín, como a \versal{MINUSTAH},
missão de reconstrução institucional democrática necessária diante das
intempéries da natureza conectadas à alegada situação de Estado falido
no Haiti (ainda que este seja o programa que mostre o limite, antes de
tudo situa a nova conformação geral da vida em espaços degradados
buscando \emph{melhorias} e que configura o que chamamos de \emph{campo
de concentração a céu aberto}).

Qualquer breve caracterização efetuada em vista à futura ordem ou
restrita às políticas ecológicas mostra que a produção de saber sobre a
degradação, produzida pelos Humanos, esteja ela disposta entre
capitalistas e socialistas, liberais e marxistas, situa o deslocamento
gradativo da biopolítica para a ecopolítica, em que o planeta passa a
ser o alvo do investimento produzido pela economia política e com as
respectivas tecnologias de segurança. Não se trata, portanto, de um
\emph{por vir} nem de uma redução da ecopolítica às políticas
governamentais, ainda que estas produções de saberes disputem o precioso
lugar da verdade, ou mesmo componham a mesma produção da verdade
enunciada pela sustentabilidade e pelo desenvolvimento sustentável em
curso. Em função da atualização do capitalismo, investe-se em governar
populações que habitam áreas com serviços ecossistêmicos ainda
abundantes e os escassos. Ao mesmo tempo, elabora-se um discurso que
incorpora outros conhecimentos e fortalece princípios democráticos.

A governamentalidade planetária vai se consolidando ao produzir a
interface definitiva entre diferentes culturas adequando modos de vida
locais à sustentabilidade com resiliência. Enquanto para alguns a
ecopolítica se constitui como a nova utopia capaz de transcender o
capitalismo ou mesmo como políticas ambientais, nossa hipótese confirma
que a ecopolítica, também por isso, é a gestão planetária em curso.

As relações entre os diversos saberes e a governança (Rodrigues, 2014),
a prática do governo como serviços público e privado por meio de
parcerias público-privadas, fortalecendo a relação entre os
\emph{stakeholders} e os \emph{stokeholders} (acionistas), funde cidadão
e consumidor e ultrapassa as relações meramente econômicas. A governança
emerge com o fim da Guerra Fria, o novo protagonismo da \versal{ONU} e a
constituição da União Europeia. Ela produz empoderamento, aumenta a
capacidade de controle (\emph{accountability}) e provoca a
auto-obediência pela recomendada conduta pró-ativa, criando novas
\emph{lideranças} em elites secundárias conectadas às elites
tradicionais e com isso reduzindo o uso da coerção pelo poder soberano.
Há, de fato, uma despolitização da política em seu sentido convencional,
deslocando-se para uma nova politização que ultrapassa uma suposta
consolidação do programa empresarial no Estado e que deveria produzir
uma nova forma de emancipação, adversa à convocação constante à
participação que se dá pelas relações de poder ascendentes, descendentes
e horizontais produzindo novos assujeitamentos, ainda que os combates
(agonismo) não cessem.

As relações entre os saberes e a política não se restringem a efeitos
sobre um suposto alvo principal, a degradação do planeta. Se o fosse, a
ecopolítica não passaria de uma nova ramificação das políticas públicas,
visando reparos, inclusões, ajustes, investimentos e vigilância, uma
nova \emph{ordem} que se constituiria no futuro, como supõem os
defensores da tese da ``singularidade'' e os ``aceleracionistas''. A
ecopolítica não é uma política ecológica ou ambiental de reparos ao meio
ambiente, visando melhorias por meio do desenvolvimento sustentável. Não
é a nova ordem que se anuncia a partir do combate ao capitalismo e o
novo desenvolvimento das forças produtivas, produzindo emancipação. A
governança é sim uma prática nova que se fortalece a partir do momento
em que, pelas variadas proveniências, emerge um novo problema, o
planeta, por meio das contínuas reflexões sobre a continuidade da
\emph{espécie} e da \emph{natureza} devastada, situadas e concentradas
em discussões, medidas e recomendações a respeito do aquecimento global.
Os sempre complementares, ainda que apresentados como opostos, programas
para o desenvolvimento sustentável, como ficou claro na Rio+20,
inscrevem o desenvolvimento sustentável como uma meta preferencial no
presente que é capaz de conectar capitalistas, governantes, organizações
internacionais, sociedade civil, enfim, os \emph{atores} que produzem o
teatro da governança e que, como tal, alimenta e retroalimenta a
perseverança da luta emancipatória. Trata-se de um teatro de empoderados
que tende a orquestrar protagonismos situando o antagonista num campo
metafísico; mas é justamente essa condição que dinamiza o teatro burguês
moderno.

Todavia, a ecopolítica, a vida dos viventes no planeta e no espaço
sideral, considera também as buscas de condições climáticas adequadas
nos exoplanetas. O clima, a segurança climática e humana, as novas
possibilidades de conservação da vida, a sustentabilidade e a
biodiversidade, enfim, não se resumem, incipientemente, ao \emph{meio}
como no século \versal{XIX}, ou do \emph{meio ambiente}, como a partir da segunda
metade do século passado. As relações são mais complexas do que supõe
lutas por medidas de melhorias de condições e garantias de um futuro
melhor para as novas gerações. A vida dos vivos redimensionada pela
racionalidade neoliberal tornou adversários os antigos inimigos, os
liberais e os marxistas; produziu o ecumenismo que distingue
fundamentalismos e posteriormente fundamentalismos pacíficos e
violentos; recompôs a relação ciência-religião recomendando moderação
nas palavras e práticas, exigindo autorização para se manifestar,
produzindo oscilações de governos à direita e à esquerda sempre imbuídos
do slogan que, por vezes, é necessário governar mais para não governar
demais. O pavor liberal ainda é o estatismo e, por conseguinte, sua
fobia ao Estado é sua política pró-ativa; do mesmo modo, taticamente as
esquerdas se voltaram para medidas ampliadas de governamentalização do
Estado para encontrar o caminho para a emancipação, atualizando o antigo
debate entre revolucionários e reformistas, e dando a estes últimos o
sentido do destino histórico. É como sempre produzir saber e práticas a
partir do Estado como categoria do \emph{entendimento}. Mas no entremeio
da governança, por vezes se consolidam micro-fascismos, assimilam-se
anarquistas, ou mesmo se deparam com surpreendentes táticas libertárias
voltadas para o imediato e apartadas de uma utopia emancipatória
(movimento antiglobalização, Occupy Wall Street, \emph{black bloc}, ou
seja, um conjunto múltiplo de práticas que acionam radicalidades e
capturas simultaneamente). Para o antropólogo David Graber, nenhuma
modificação em curso, em especial a vitória do Syriza na Grécia, seria
possível sem estes movimentos (Cf. David Graeber. ``A hora dos 99\%''. O
Estado de S. Paulo, 01/02/2015, p. E-2). Para ele, também, ``há certas
facções dentro da burocracia global dispostas a pensar a mudança
radical''. Outrossim, ele vê as mudanças à vista voltadas para a
ultrapassagem de políticas de austeridade para a de investimentos em
infraestrutura. Esta é também a perspectiva neoliberal para saídas da
crise, como se recomenda ao governo brasileiro, a partir de 2015. De
fato, Graeber indica com clareza que sem os movimentos derivados da
antiglobalização, com suas radicalidades anarquistas, também sublinhado
por Saul Newman (2011), nada disso estaria acontecendo; do mesmo modo
que sem a institucionalização do movimento a partir do Fórum Social
Mundial, também não haveria a sucessão de ajustes ao governo global.

As forças adversárias compõem para compartilhar (Grécia, eleições de
2015 com vitória da esquerda e arranjo com forças de direita com cunho
social para formar o governo; existência institucional da Frente
Nacional, na França, como terceira força parlamentar desde 2011;
coalizão governamental no Brasil desde o primeiro governo Lula), e
condutas reativas como a do \versal{PEGIDA} (Patriotische Europäer gegen die
Islamisierung des Abendlandes, organização alemão ontra imigração de
muçulmanos e islamização do ocidente, , a partir de Dresden na Alemanha,
principalmente após 2014, crescem na conduta xenófoba anti-islamista e
anti-imigratória. O inimigo principal não é mais a ideologia radical, as
drogas, os movimentos de libertação e liberação, mas o terrorismo
fundamentalista islâmico. Se na primeira metade do século \versal{XIX} a unidade
dos ilegalismos populares emergiu no combate direto à burguesia como
verdadeira responsável pelo crime de lesa-sociedade, na ecopolítica
busca-se encontrar a conduta resiliente adequada para reduzir
resistências, seja no âmbito institucional ou não, e até a conduta do
\versal{PEGIDA} tenderá à assimilação.

A violência organizada que há poucos anos se dava pelo terror
transterritorial, agora mostra sua dobra com o \versal{ISIS} ou Estado Islâmico
na Síria e no Iraque, pretendendo-se um califado organizador das ações
terroristas transterritoriais combinado com as táticas da Al-Qaeda. O
terrorismo islâmico começa a ganhar corpo de Estado. As políticas
internacionais visavam aplacar o terror por meio de missões capazes de
instituir democracias aos moldes do universal estadunidense, a partir da
revolução de Veludo na república Checa em 1989 (Ash, 2011); na mesma
década, por meio da metamorfose governamental na Polônia com a atuação
do sindicato Solidariedade e o assentimento do governo soviético de
Gorbatchev, ao final da doutrina Bréjniev em 1985, considerando cada
Partido Comunista como responsável por seu próprio governo (Skórzinski,
2012), e com o apoio do Vaticano (Carletti, 2012); ou ainda, com
políticas internacionais revestidas de um saber científico sobre Estados
Falidos, anarquia da sociedade e necessidade de superar algumas das
ditaduras tão condescendentemente suportadas pelo Ocidente, sob a forma
de missões (Haiti), novas intervenções de defesa humanitária (Irã),
efeitos de levantes populares (Argélia, Egito). Mesmo em Estados fortes
sob a racionalidade neoliberal, nem democracia e tampouco direitos
humanos e liberdade de pensamento entram em jogo (China, Arábia Saudita,
Emirados Árabes).

Assim sendo, o terrorismo é o inimigo enquanto as discordâncias
democráticas ou forças mais à direita ou à esquerda, conforme a ocasião,
compõem os compartilhamentos possíveis na governança global. Todavia,
como sublinha Graeber\footnote{``David Graeber narra revolução de
  Kobane'', 28/01/ 2015.

  \emph{http://outraspalavras.net/destaques/david-graeber-narra-a-revolucao-de-kobane-que-derrotou-isis/}},
as resistências ocorrem mesmo em um território visto e considerado como
em ampliação pelo Estado Islâmico a partir da tradição cultural curda
que atravessa Síria, Turquia e Iraque, um povo sem demarcação
territorial própria. Talvez por isso mesmo consiga enfrentar o \versal{EI} com
força e eficácia a partir da tática de guerrilhas, criando um nódulo
importante no interior de algo que pretende se apresentar como
homogêneo. Trata-se de uma singularidade a ser ouvida, capaz de provocar
uma insurreição diante de uma organização que também se declara
insurreta. Os embates localizados, mais uma vez, mostram a pertinência
da tese de Fréderic Gros a respeito dos estados de violência que
suplantam as guerras convencionais, mas também nos mostra como no
interior das insurreições, com suas dramaturgias religiosas,
desencadeiam-se outras que também procuram aninhar-se em Estado, como as
curdas, ou que talvez no decorrer das batalhas descubram a sua força
própria na dissolução de tal meta e as levem a se opor, para além dos
próprios Estados que as governam, ao Estado como princípio de paz. Do
mesmo modo que os anarquistas gregos, hoje, chamam o \versal{SYRIZA} para o
encontro nas ruas.

As resistências, portanto, também passam por modulações. Não mais
ocorrem somente no interior das situações disciplinares, nos embates
biopolíticos ou nos/contra governos da soberania, mas também se modulam
segundo os embates a partir da conformação horizontalizada das decisões
de suas forças com ausência de liderança identificável, para encontrarem
em cada situação as inventivas táticas de enfrentamento. Elas são também
antiestratégicas pela valorização das singularidades e as suas
multiplicidades. Não são mananciais de capturas e de inovação na reforma
das condutas pelo Estado ou sociedade civil organizada. Nelas, os
intelectuais moduladores não têm vez (apesar de ser impossível afirmar
que aqueles que as praticam não venham a se metamorfosear em
intelectuais moduladores), pois não estão em jogo novas formas de
governo das condutas, reformas govermamentais e programações globais,
mas os efeitos inevitáveis das insurreições. Estão em jogo, portanto
contenções imediatas, assim como a produção de lutas antipolíticas. São
olhos-d'água de contracondutas, na medida em que podem ser capturadas e
na desmedida de suas expansões. Há uma intelectualidade que emerge
condizente com as resistências na sociedade de controle fundada na
extração de energias inteligentes, participativa e democrática. Opera em
seus interiores para revirá-las e diferenciam-se da relação hierárquica
entre intelectuais de elite e de elite secundária.

Os intelectuais convencionais e inovadores (que compõem o que chamamos
de intelectual modulador) se ajustam às novas institucionalizações
inacabadas, próprias da ecopolítica, estando na burocracia nacional ou
global, estatal ou de \versal{ONG}s, compartilhando com os que fazem nos
movimentos da sociedade civil voltada para alcançar \emph{melhorias}, o
epicentro, a produção da verdade. Há o desparecimento gradual do
intelectual crítico e distanciado em função de uma conduta diretamente
engajada em específicas práticas de compartilhamentos. De um lado,
funcionam agendando os problemas levantados; de outro lado, no interior
de horizontalidades vinculadas à centralidade do poder, analisam, expõem
e sintetizam as novas práticas. Mas de ambos os lados funcionam como um
\emph{intelectual modulador}, compondo novas lideranças como
\emph{stakeholder}, ou capturando lideranças inovadoras nos regulares
movimentos de protesto e contestações. E se, como vimos, não se trata de
oposições institucionais e estruturais, mas de fluxos que avolumam
constantemente e escasseiam de tempos em tempos, a nova função do
intelectual exige, pelo menos do ângulo da governança global, que se
estabeleça uma demarcação clara sobre a permanência do intelectual de
elite (seja também de classe) e a entrada do intelectual produzido pelas
\emph{elites secundárias}, isto é, como se configura a fusão entre
intelectuais de formação acadêmica e intelectuais de formação
universitária gestados nas periferias das metrópoles para atuar no
próprio local.

Foucault situou de maneira própria a queda de prestígio do intelectual
desde o final dos anos 1970, ao mesmo tempo em que se colocava como um
antiestrategista: ``ser respeitoso quando uma singularidade se insurge;
intransigente quando o poder infringe o universal. Escolha simples, obra
penosa: pois é preciso ao mesmo tempo espreitar, por baixo da história,
o que a rompe e a agita, e vigiar um pouco por trás da política o que
deve incondicionalmente limitá-la'' (Foucault, 2004b: 81). Mas mudanças
aconteceram em relação ao intelectual. A suposta perda de prestígio
tanto do intelectual convencional quanto do intelectual profeta foi
redimensionada pelos efeitos de governança.

No âmbito da burocracia globalizada ou no interior dos Estados, os
intelectuais com suas distinções acadêmicas e os secundários portadores
de saberes locais específicos e/ou formados nas universidades de
periferias são destinados a se ocupar dos problemas locais, participar
das novas relações e conexões na cotidianeidade como portadores de
direitos inacabados voltados para encontrar o equilíbrio entre os
interesses individuais, como empreendedores, e os coletivos, como
comunidade. Conectam de maneira democrática e diplomática a ampliação de
direitos, as novas metas de segurança, as situações ambientais
amplificadas pelos problemas enunciados a partir da ecologia, a
condensação dos clamores por combate às impunidades, redimensionando a
cultura dos castigos, as penalizações, programas de assistências aos
corpos que vagueiam mendigos, sem habitação e drogados pelas metrópoles
ou \emph{confins}.

Monitora-se um ao outro, configurando a emergência de um cidadão polícia
conectado aos dispositivos diplomático-policiais planetários. Enfim, os
intelectuais \emph{moduladores} passaram a compreender e a participar
mais regularmente das conexões entre populações locais e \versal{ONG}s ,
institutos e fundações internacionais, Estados e governos na garantia de
direitos a seus portadores, desde populações indígenas e uma nova
ecogovernamentalidade (Ulloa, 2011; Malette, 2011), até as novas
caracterizações da vida nas favelas brasileiras, nas quais a fusão do
intelectual de elite e da elite secundária produzem verdades a respeito
de uma nova conformação da vida, da felicidade, rendimentos e das
melhorias crescentes almejadas pelo consumidor-cidadão.

O povo da favela, consumidor e cidadão, como institui a razão
neoliberal, expressa sua felicidade nas comunidades do Rio de Janeiro
(Meirelles \& Athayde, 2014; Alves \& Evanson, 2013; Scheinvar, 2014).
As conexões entre os dispositivos diplomático-policial (Rodrigues,
2013), meio ambiente (Carneiro, 2012), resiliência (Oliveira, 2011;
2014) e monitoramento (Augusto, 2012a) trazem o cidadão-polícia em seus
\emph{ambientes} resilientes, monitorados e monitorando. A obediência se
alarga de modo participativo, produzindo outros assujeitamentos, sem
deixar de contar com as medidas compensatórias governamentais nacionais
(Bourdieu, 2014) e mesmo globais (\versal{ODM}). A racionalidade neoliberal
(Foucault, 2008a) se afirma trazendo consigo a perspectiva do
\emph{desenvolvimento sustentável} redimensionando a noção de ambiente
para além do espaço natural degradado das áreas de conservação,
investimentos e preservação para ampliá-lo e amplificá-lo para
comunidades urbanas e rurais, espaços de vida e consolidação espacial de
populações (ainda que não se interrompam mas, ao contrário, dinamizem-se
os fluxos migratórios e imigratórios), além da expansão para o espaço
sideral (Passetti, 2013). Espera-se, sob as práticas democráticas de
governo dos Estados e governo dos humanos, uma conduta moderada capaz de
absorver a dinâmica das contracondutas, cujo limite é insurrecional e
traduz a revolta (Passetti, 2013a; Passetti \& Augusto, 2014; Passetti
\& Oliveira, 2006; Rodrigues 2013; 2014).

O campo de concentração a céu aberto é uma estratégia de política
moderada, de economia neoliberal, de cultura de paz, de controle de
vulnerabilidades em ambientes amplos como o planeta e restritos, indo
das favelas ao cotidiano estabelecido e seguro dos espaços de habitação
que contemplam moradia, empresa e equipamentos sociais às classes mais
abastadas. Deriva de saberes produzidos nos campos de concentração de
outrora em que o governo do espaço se dava pela imposição normativa e
repressiva do Estado conectado com a gestão dos prisioneiros por eles
próprios. A estes cabia o governo que decidia \emph{quem devia morrer} e
a esperança na preservação da espécie. Cabia-lhes organizar a vida para
a produção, as atividades culturais, um cotidiano \emph{pacificado},
estratificar os prisioneiros, delimitar campos de concentração segundo a
proveniência cultural e social da raça a ser governada encarcerada
(Agamben, 2008; Applebaum, 2009; Binet, 2012; Bitar, 1987; Brenner,
2014; Friedländer, 2012; Germain, 2014; Kassow, 2009; Kundera, 2014;
L'Amicale, 1982; Lower, 2014; Levi, 2015; Moreno, 1975; Peschel, 2009;
Power, 2004; Scholl, 2013; Sebald 2008; Sem-Sandberg, 2012; Sieminski,
2009; Soljenitzen, 2007; Wachsmann, 2015).

O convencional campo de concentração passou a ser uma estratégia de
gestão compartilhada que foi tomando contornos após a \versal{II} Guerra Mundial
e ajustou-se à razão neoliberal na gestão compartilhada nas comunidades,
prisões, condomínios, mas principalmente nas empresas. Não se trata,
porém, do direito de estipular quem deve morrer, nem tampouco de causar
a vida e devolver à morte, como na biopolítica. Trata-se de \emph{como
viver} para alcançar longevidade por meio da proliferação de direitos
inacabados que coloca cada um como portador de direitos, potente capital
humano normalizado, moderada e eticamente responsável pelo seu
empreendimento e pelo eventual empreendedorismo social conectado aos
negócios sociais. O campo de concentração a céu aberto mostra também as
novas formas do encarceramento dos sujeitos livres, não mais segundo a
sua nacionalidade e soberania territorial, mas livre para trânsitos
regulados segundo acordos internacionais relativos às nacionalidades de
origem, e que também produzem os contingentes de refugiados, sob
governos também combinados. Mais do que isso, delimita a possibilidade
de vida nas \emph{bordas} dos variados \emph{ambientes} e rompimentos
com as possibilidades da revolta. O campo de concentração a céu aberto
dá forma à gestão compartilhada. Tudo é possível de solução para
melhorias quando regulados em seu interior para equacionar interesses
coletivos em conexão com Estados e também com organizações
transterritoriais. Trata-se, por fim, de consolidar a sustentabilidade,
para a qual o divíduo resiliente controla a si e aos outros.

O governo da população é a apropriação do povo pela economia política. O
povo revolucionário expandido na Revolução Francesa exige mais que
mudança na soberania. Não se resume à sua incorporação como
representado, com direitos universais. Se o fosse, não ocorreria o
terror de Estado e sua prática comum de segurança diante da ameaça da
guerra civil iminente. Seja pelos seus primórdios, no século \versal{XVIII}, em
decorrência da Revolução Russa, ou pela conservação da democracia, a
iminência da guerra civil situa o efeito contínuo da revolução moderna
que é o de repor a soberania, a dominação política.

Os liberais expressarão sempre sua \emph{fobia ao Estado},
principalmente pela sua própria ampliação no governo da vida e na defesa
de suas regulamentações, geralmente pautadas na liberdade de circulação
de mercadorias e indivíduos pelo mercado e sociedade civil. Em
decorrência da sua ocupação ampliada por mais forças políticas, o Estado
visa regular a participação com o intuito de minar as ameaças à sua
continuidade sob a forma de políticas compensatórias. Desta maneira,
depende do funcionamento de seu aparelho repressivo, garantido por
dispositivos de segurança que atuam tanto interna como externamente. Seu
alvo é conter os inimigos capazes de provocar riscos à sua continuidade,
e, portanto, governar é administrar riscos. Depende do modo como se
organizam direitos, estrutura-se a segurança, normalizam-se as condutas,
como o governo do Estado produz intervenções no \emph{meio}, em especial
a urbanização das cidades e suas políticas de saúde e educação, mas é
imprescindível fincar os limites das relações de castigo e recompensa.

Não se governa somente de cima para baixo, como supuseram os
revolucionários modernos, no sentido de realizar a justiça no Estado,
mas agora é de modo ascendente, descendente e horizontal que se efetivam
relações de poder em fluxos. Os liberais sabem que o Estado não é
somente uma instância de regulação e regulamentações legais e normativas
de governo. Para o mercado seguir livre e conjugar exploração e
dominação em um patamar de convivência e conivência entre a maioria das
forças políticas é necessária a cooperação de interesses.

Os revolucionários, e porque não dizer reformistas em geral, procuram
não só levar adiante a guerra civil como também encontrar uma
perspectiva de \emph{paz} internacional, por meio da ocupação e
dilatação do Estado. Seja por meio de políticas compensatórias
ampliadas, cuja expressão maior talvez tenha sido o
\emph{welfare-state}, ou por meio de uma programática da justiça social
conduzida, de um lado, por vanguardas planificadoras e, de outro lado,
por composições majoritárias de forças reformistas, na qual se incluem
junto aos socialistas reformistas europeus os \emph{liberais de
esquerda} estadunidenses no século passado, a contenção da guerra civil
como desdobramento da revolução que substitui o soberano permanece
inalterada. Muda a biopolítica, a gestão da vida segura, no seu
\emph{meio}, com seus direitos e suas penalizações.

Desta perspectiva, pode-se dizer que o socialismo não foi uma
alternativa ao liberalismo, mas um de seus desdobramentos, ou melhor,
outra produção de gestões da vida. No âmbito externo, a propagação
violenta ou pacífica da ocupação do Estado produzindo socializações
também é considerada como condição \emph{sine qua non} para a construção
da \emph{paz entre os povos.} Os socialistas revolucionários e os
reformistas sempre souberam que o Estado é a instância de regulação e
regulamentações legais e normativas de governo. Em nome de conjurar as
injustiças de mercado e as ilusões democráticas, suprimem forças
políticas ou produzem composições em função da justiça social e da
circunscrição da guerra.

Liberais e socialistas sabem que para governar dependem da participação
contínua e incisiva da sociedade civil, seja intensificando essas
relações ascensionais e descensionais, como os liberais, seja
construindo essa relação por meio da planificação geral, como os
socialistas, o que levaria num determinado momento à extinção do Estado
e à consolidação do novo processo de realização da justiça social com
paz internacional, na justa medida em que não haveria mais o Estado.
Para os liberais, somente a existência do Estado é capaz de pacificar o
\emph{estado de guerra} internacional moderno desenhado filosoficamente
por Thomas Hobbes, enunciado federativamente por John Locke e
dimensionado, finalmente, por Immanuel Kant. E, na perspectiva
histórica, somente as decisões provenientes dos encontros de Westfalia,
no século \versal{XVII}, consolidado pelo \emph{Tratado de Viena}, no início do
século \versal{XIX}, dão sentido ao equilíbrio almejado entre a razão de Estado e
uma balança de poderes europeia que fortifica a competição entre os
próprios Estados. Ainda que o conjunto não tenha colocado um fim na
guerra e, ao contrário, tenha levado a duas grandes guerras mundiais no
século \versal{XX}, demarca os pontos necessários e suficientes para a emergência
de uma política de paz da segunda metade do século passado aos nossos
dias a partir da \versal{DUDH}.

Filosofia e história compõem, hoje, o que se denomina como \emph{cultura
de paz}, esse esforço institucional para se encontrar, a partir de um
centro imantado de forças democratizantes e difusor de programações
pacificadoras, as condições adequadas para programações que governem a
gestão compartilhada da exploração e da dominação. Esse centro
necessário e suficiente foi construído no interior da \versal{ONU} desde sua
fundação, quando se reorganizaram os princípios de segurança, direitos,
punições e de gestão da vida em um \emph{meio} não mais restrito à
circulação de mercadorias e indivíduos, mas pelas superfícies,
profundidades, poros, ou seja, \emph{ambientes}. O trajeto da \versal{ONU} não
deve ser compreendido como linear, na medida em que ela, incialmente, é
predominantemente governada pelo seu Conselho de Segurança, agrupando os
vencedores da \versal{II} Guerra Mundial, delineando os direitos dos vencedores,
os explicitamente ocidentais, como registra a \emph{Declaração Universal
dos Direitos do Homem}, de 1948. Diante do horror produzido pelo
nazi-fascismo, com seu direito conspurcado de decidir quem devia viver e
quem devia morrer, e dos temores pelo socialismo propagado pela \versal{URSS}, o
Conselho de Segurança acabou sendo composto por \versal{EUA}, Grã-Bretanha,
França e \versal{URSS}. A entrada da China como membro permanente ocorrerá
somente em 1971. A história do socialismo chinês construído a partir do
\emph{campo} em direção à \emph{cidade} contrariou a política soviética
desde os efeitos iniciais resultantes da expulsão dos japoneses e a
tentativa de consolidar o governo dos nacionalistas, como desejavam os
estadunidenses, ou de possível composição, como ansiavam os soviéticos.
Ainda que fosse uma força socialista, a China estava equidistante da
\versal{URSS} e dos \versal{EUA}. A China \emph{revolucionária} era também força de
combate direto com o socialismo soviético, na medida em que
impulsionaria possibilidades de \emph{guerra civil} a partir do
\emph{campo}. A luta entre essas táticas não só marcou as lutas no
oriente como no ocidente \emph{terceiro-mundista}, como situavam os
analistas do \emph{subdesenvolvimento}. Em poucas palavras, a China,
para o ocidente, era chave porque enfrentava duas frentes: o capitalismo
democrático ocidental e o socialismo imperial soviético. Rompido com a
\versal{URSS}, o governo chinês inicia sua aproximação com os Estados Unidos
ainda na década de 1960. Isso provocará investimentos gradativos do
capitalismo na China, com acordos desde o fim da guerra do Vietnã e
identificação de interesses entre as duas potências. A \versal{URSS} a partir de
então entrará em processo de \emph{isolamento} que levará à derrocada do
socialismo na década de 1990, e à projeção da China como segunda grande
economia mundial.

Aos poucos, os dispositivos policiais de gestão da segurança interna e
dos dispositivos diplomático-militares de relações internacionais
passarão por redefinições diretamente relacionadas à retomada
capitalista globalizadora, produzindo a retração do socialismo
internacional. A nova composição política dos membros permanentes do
Conselho de Segurança da \versal{ONU} também facilitará programas de saúde,
educação, governo da pobreza, pacificações necessárias que ocorrerão
principalmente após os levantes do acontecimento \emph{1968}, e que
forçarão redimensionamentos sociopolíticos que prepararam a introdução
da \emph{racionalidade neoliberal}.

Não se tratava mais de governar territórios e populações em seu
\emph{meio}. O meio ganhou outra dimensão pelos efeitos de
radioatividades, desgaste de recursos naturais pela exploração
desmesurada das forças produtivas, incompatibilidades generalizadas
produzidas pela ampliação de políticas compensatórias, levantes,
protestos e guerras locais de efeito mundial no Oriente Médio e
asiático, e insurgências e guerrilhas na América Latina e África, neste
caso radicalizando o processo de descolonização. Não foi só a chamada
guerra fria que funcionou como causalidade múltipla. Os efeitos da \versal{II}
Guerra Mundial extrapolam o mero embate entre Estados, produzindo a
\emph{corrida espacial}, e com ela investimentos que não mais se davam,
ou darão apenas pela oposição capitalismo-socialismo. O capitalismo, que
outrora soubera conjugar os mais diferentes regimes políticos às suas
conquistas, coloca em prática a continuidade do mesmo princípio, apenas
sinalizando o deslocamento do universal francês revolucionário para um
novo universal, como situou Pierre Bourdieu (2003), o democrático
estadunidense.

O \emph{ambiente} internacional ocorre pela ultrapassagem da gestão
pelos Estados-Nação para o de novos blocos multilaterais econômicos e
políticos com a emergência da União Europeia, na década de 1990. Não é
mais internacional no sentido convencional, mas \emph{transterritorial}.
Os acordos, negociações e vetos não se restringem à bilateralidade ou
multilateralidade das relações entre Estados. Estas relações se dão em
escala \emph{global}, conectando sociedade civil, Estados e \versal{ONU},
produzidas por uma infinidade de institutos, organizações, fundações e
seus simpósios, seminários, colóquios, fóruns, encontros preparatórios
para responderem a uma agenda e introduzirem novos temas. As regulações
e regulamentações dos Estados-nacionais passam a ser governadas também
pelas decisões e recomendações produzidas pelo governo \emph{comum} na
\versal{ONU}.

O \emph{meio ambiente} resultará de uma produção tensa e conectada entre
efeitos de \emph{1968}, condições de propagação da racionalidade
neoliberal, desejo de democracia com contenção de autoritarismos e para
expelir totalitarismos. O campo para a nova dimensão da \emph{paz}, a
segurança planetária, e os melhores usos dos recursos naturais com
contenção de variáveis que calcinam os recursos naturais está aberto. O
capitalismo precisa encontrar sua nova faceta de \emph{desenvolvimento}
que conjugue gestão dos recursos naturais, expansão de obtenção de
recursos para além da Terra, gestão da situação de pobreza, ampliação de
direitos, condições para a sedimentação da \emph{cultura de paz}, para,
com isso, produzir condições para o ecumenismo, extirpar os terrorismos
ou iminentes guerras civis, redimensionar o perigoso e potencial útil,
investir em inteligências diplomáticas, não mais restritas às relações
internacionais, assim como fazer da democracia mais do que um regime
político.

É preciso encontrar \emph{como compartilhar} tudo ou quase tudo isso. É
necessária uma nova governamentalidade desterritorializante que assimile
as condições transterritoriais dos Estados, os fluxos de populações
(migrantes, emigrantes e de refugiados), elevando expectativas de saúde
e educação na formação do novo parceiro do capital, não mais em situação
antagônica, mas de cooperação material e imaterial, ou seja, a força de
trabalho redimensionada como capital humano. Esse \emph{novo
desenvolvimento}, capaz de compor no interior de sua racionalidade
neoliberal as antigas forças à direita e à esquerda do Estado,
chamar-se-á \emph{desenvolvimento sustentável}. A essa
governamentalidade que o sustenta chamamos de governamentalidade
planetária.

Esse novo jeito de governar não prescinde da biopolítica. Ainda que
pretenda expurgar o seu direito de causar a morte e decidir quem deva
viver, causar a vida e devolver a morte não é mais tarefa do
Estado-nação, ainda que ele seja o responsável pela vida dos viventes em
seus espaços. Não se trata mais de governar a população em seu
\emph{meio}. E muito menos a população é o alvo do governo da
biopolítica no sentido da economia política clássica, destituída de sua
força política como povo. A população é o conjunto de povos arranjado em
certos espaços, de superfície e de atmosfera. As populações estão
disponíveis e dispostas a serem governadas com democracia e se governar
como democráticas. Elas são governadas pelo aparato repressivo, mas
agora, sob o universal democrático estadunidense, também são educadas
para se governar, produzir seu estilo de vida segundo os riscos
inerentes ao seu empreendimento. Cada um deve conhecer explicitamente os
efeitos punitivos penais e normativos, e, por conseguinte, cada um deve
avaliar o risco racional de qualquer conduta \emph{criminosa}, seja ela
contra a ordem política, a disposição e acesso aos bens materiais,
atentados contra a vida de terceiros ou de espaços da natureza passíveis
de legislação.

O \emph{dispositivo meio ambiente} e o redimensionamento dos
dispositivos policiais e diplomático-militares provocam e são também
provocados por outros dispositivos gerados pela pletora de direitos de
minorias e pelas condutas penalizáveis não mais restritas ao
funcionamento de instituições austeras como a prisão, os asilos e os
manicômios. Do ponto de vista da segurança, podemos dizer que os
dispositivos diplomáticos ganham amplificações capazes de compor o
policial ampliado e o militar. Na medida em que a ameaça de uma guerra
mundial foi dissipada pelo funcionamento da \versal{ONU}, os grandes problemas de
segurança giram em torno dos chamados estados de violência e
guerra-fluxo cada vez mais vistos como alvos de investimentos
pacificadores.

\section{mundo}

A análise de declarações, cartas, acordos, tratados, convenções e planos
internacionais em interfaces com programas de governos, situaram o
governo das condutas pelos Estados nacionais, organismos e instituições
internacionais, empresas, \versal{ONG}s, institutos e fundações em função da
continuidade capitalista, democrática (representativa e participativa),
com base na conservação do planeta, incluindo o divíduo inovador,
empreendedor, empoderado, resiliente e responsável diante do possível no
desenvolvimento sustentável.

A democracia liberal passou a ser o regime preconizado de governo do
Estado e as práticas democráticas foram introduzidas pela racionalidade
neoliberal na gestão de empresas a comunidades por meio de condutas
diplomaticamente ampliadas por meio de interfaces. Abriram-se fluxos
conectados e inacabados de responsabilidade social e negócios sociais
vinculados à sustentabilidade, envolvendo comunidades na gestão entre
interesses dos indivíduos e os coletivos. Desse modo, a configuração da
\emph{convocação à participação} por interfaces resultou na emergência
do pastorado horizontal, compondo o \emph{cidadão-polícia}, um sujeito
portador de direitos inacabados e, preferencialmente, resiliente, que
monitora e é monitorado. Os efeitos, saberes e práticas sobre o meio
ambiente, os direitos, a segurança e as penalizações a céu aberto
configuraram, enfim, os \emph{dispositivos meio ambiente},
\emph{diplomático-policial, resiliência} e \emph{monitoramento}, que
atravessam as relações de \emph{governo dos outros} e o \emph{governo de
si}, dando ao governo entre os súditos uma nova função de governança que
ultrapassa o que era esperado do pastorado moderno fundado na
biopolítica. Na era da ecopolítica, o pastorado exercita-se
simultaneamente pelas políticas de Estado, as práticas individuais e
coletivas na resiliente sociedade civil organizada e monitorada,
conectando a extração de fluxo inteligente de cada um com cuidados com
as variadas populações em um ambiente, o planeta e as condições
siderais. A liberdade com segurança do liberalismo e do neoliberalismo é
assim enfatizada em função das melhorias para as futuras gerações.

A dilatação transterritorial contínua atingiu o planeta e sua vida no
universo em expansão, proporcionando outros negócios inteligentes e
rentáveis, consolidando-se a polícia das condutas. Alcançados os
Objetivos de Desenvolvimento Milênio (\versal{ODM}) da \versal{ONU} para 2000-2015, a
partir de agora são redimensionados e ampliados em Objetivos do
Desenvolvimento Sustentável a serem atingidos até 2030. A confiança nas
práticas democráticas institucionais e empresariais e a tolerância
recomendada nas condutas, ainda que a penalização seja cada mais
ampliada e acoplada a programas de intervenções de segurança, que vai da
comunicação eletrônica à conduta cidadão-policial, fazem com que a
ecopolítica se consolide desprendendo-se dos alvos e metas da
biopolítica.

Se os acontecimentos desde 2011 expuseram, definitivamente, a
configuração democrática eletrônica, com seu governo verticalizado
combinado ao governo rizomático e com explicitação dos monitoramentos,
espionagens, regulamentações e programáticas voltadas ao governo da
inteligência próprio da sociedade de controles, abriram-se também, para
o campo mais ampliado das insurreições, situando o terrorismo
transterritorial com pretensões de constituir uma nova forma de Estado
conectando teologia e racionalidade laica (do Talebã ao Estado
Islâmico).

Expõe-se, insistentemente, a necessidade da aderência à convocação a
participar como complemento à representação partidária, à gestão das
empresas e aos negócios sociais, e confirma-se a emergência do
\emph{intelectual} \emph{modulador}. Equacionam-se as relações entre
adversários, próprias à democracia contemporânea, situando os novos
inimigos, não mais segundo a economia ou os regimes políticos, mas
investindo contra as práticas radicais intoleráveis (sejam elas
derivadas do movimento antiglobalização ou dos variados terrorismos).
Configura-se o deslocamento da defesa da sociedade para a defesa do
planeta, segundo efeitos jurídico-políticos nacionais diante das
recomendações internacionais, regulação e regulamentação de programas de
\emph{segurança}, contenção de \emph{vulnerabilidades},
\emph{conservação ambiental}, \emph{cultura de paz} para a situação
atual conformada em \emph{ambientes}, muitas vezes sintetizada nos
enunciados sobre o clima no planeta e conformando um \emph{ambiente
planetário} governado pela polícia das condutas. O mundo melhor a ser
atingido, sempre em um futuro próximo e ao mesmo tempo longínquo, exige
tolerância (que exclui radicalismos políticos e fundamentalismos
religiosos violentos), confiança nas instituições, em cada divíduo como
portador de direitos em uma programática sustentável, e segurança à vida
de cada um, aos demais viventes conectados à segurança de cada um, do
capital e do Estado.

São institucionalizações governadas com \emph{moderação} com os
monitoramentos garantindo a segurança para consolidar o controle de si e
dos outros, acomodando uma polícia das condutas entre cidadãos, Estados
e entre Estado e cidadãos e futuros cidadãos. A vida deve ser governada
pelos portadores de direitos inacabados com sustentabilidade,
esperando-se produzir \emph{harmonia} pela conciliação desejada pelos
divergentes (os adversários), a gestão da anomia identificada, os
investimentos no corpo e na inteligência pelas tecnologias de governo e
o ajustamento do governo de cada um como \emph{governado resiliente}.
Espera-se com isso fortalecer a governança do local ao global.

Ganham destaques as relações estabelecidas no âmbito das comunidades
como modulações de governos. A noção de \emph{campo de concentração a
céu aberto} procura dar conta destas novas relações com base nos saberes
metamorfoseados a partir do nacional-socialismo e do socialismo
soviético, combinando campos de transição (refugiados), campos de
concentração (vida das populações com gestão compartilhada), economias
próprias (incluindo os gulags), formação de comunidades xenófobas (atual
anti-islamismo europeu, como o \versal{PEGIDA} alemão) com direitos e moderação
(ecumenismo e distinções entre fundamentalismos violentos e pacíficos) e
as comunidades locais.

Uma institucionalização, enfim, que procura plasticamente elastificar um
novo momento da cultura do castigo por meio do cidadão-polícia voltado
para o embate contra as impunidades por meio do \emph{punir mais e
melhor}. Este é convocado a participar por meio de conduta moderada, ou
seja, diplomaticamente policial, exercitando-se de modo complementar às
práticas repressivas de Estado; ao mesmo tempo, ele monitora as condutas
violentas contra minorias numéricas e faz dos espaços de vida vivente os
devidos \emph{ambientes} a serem melhorados.

As resistências a serem inibidas chegaram a ser propositalmente
identificadas como resiliências como recomenda, pretende e insufla a
governança global e local. Porém, há um novo acontecimento nas relações
de poder governadas pela inteligência que convoca a conduta resiliente a
expor os questionamentos voltados para \emph{melhorias futuras}: cada
indivíduo no seu local de moradia, trabalho, circulação, cidades,
Estados e planeta deve governar a si e aos outros, segundo adversidades
a serem ultrapassadas no futuro, exercitando o pastorado horizontal. A
emergência das \emph{elites secundárias} se tornou imperativa para
reforçar a presença vibrante dos desdobramentos das culturas específicas
de \emph{divíduos} e grupos portadores de direitos inacabados, em uma
convivência social pluralista que mescla minorias numéricas segundo o
evento ou a urgência do momento. As \emph{elites secundárias} funcionam
em modulação recrutando e seguindo o recrutamento pelas elites
principais, compondo o esperado fluxo para o trânsito democrático,
educando simultaneamente os setores pauperizados por meio de programas e
assistências a se conduzirem com moderação como \emph{cidadãos-polícia}
(sua proveniência) e a partir de um \emph{quantum} de acesso a bens de
consumo capaz de ampliar seus empoderamentos.

Escapam deste caudaloso fluxo de captura de inteligências algumas
práticas radicais que expõem os limites da segurança, do bem viver, do
bem-estar, do bem comum, perfurando as ideologias e a racionalidade
neoliberal, anunciando uma atualização da virtualidade cínica.
Entretanto, majoritariamente se fortalece a tese por uma nova/outra
globalização: o vaivém de uma maior ou menor intervenção do Estado não
altera o funcionamento da racionalidade neoliberal (vide os acordos para
a gestão do \versal{SYRIZA} na Grécia no início de 2015 e suas conexões
programáticas com experiências latino-americanas como Venezuela e
Brasil; ou se preferirem, o encantamento europeu com as práticas
latino-americanas como no passado dos grupos guerrilheiros de lá na
segunda metade do século \versal{XX}, com as práticas dos de cá e do norte da
África). Porém, vê-se diante das novas resistências que cada vez mais se
intensificam com base na política radical de proveniência anarquista.

Trata-se de uma nova governamentalidade planetária que produz e é
produzida incessantemente em função da superação da degradação e em
função da restauração e da boa governança. Somos convocados a ser
parceiros nesta tarefa que vai da casa ao espaço sideral e que retorna
aos nossos miúdos movimentos e sensações compartilhadas compondo uma
subjetividade voltada para responder à participação e nos acolher com
autoajuda. É assim que devemos ser conservacionistas, inovadores,
democratas, amorosos defensores da humanidade do planeta.

A tradição que vivemos em nossa cultura sempre remete às reformas da
sociedade e, no limite, à construção de uma nova sociedade. Para os
democratas liberais a sociedade é realidade e utopia, devendo-se
estancar em seu interior as eventuais distopias, ou seja, aquelas
construções de mundo projetadas nas lutas entre as forças que produzem
no capitalismo as estagnações temporárias, os retrocessos entendidos
como eventuais disfunções (autoritaritarismos, fascismos, nazismo), mas
com funções estratégicas de combate ao inimigo; ou ainda a fabricação de
um novo mundo e uma nova sociedade, como almejou o socialismo no século
\versal{XX}. Entretanto, a produção dessas verdades que sustentam e
retroalimentam os sistemas está em relação direta com a ideia de mundo
contraposto a um novo mundo, de uma sociedade contraposta a outra
sociedade.

Mundo e sociedade são duas verdades que se sustentam desde a modernidade
na crença do Estado como categoria do entendimento. Da filosofia à
física, o mundo sempre foi algo que abarcou a imensidão externa aos
territórios como o espaço celestial, seus astros, estrelas e verdades
míticas. O mundo sempre esteve relacionado a uma subjetividade capaz de
afirmar em seu centro o cidadão ateniense, o senhor, o rei, o pastor, o
indivíduo livre e autônomo, a classe como o sujeito da história. O mundo
interior, composto de sonhos, inconsciente, desrazão, desejos e faltas
foi relacionado à imediata realidade material ou ao encontro de paz e
tranquilidade após a morte; ajustou-se gradativamente às similares
buscas exteriores por Estados, empresas, classes, sujeitos considerados
perigosos, rebeliões, insurreições, revoluções. Esta dinâmica
reformadora do mundo parte da relação indivíduo-sociedade como algo a
ser alterado ou renovado. A subjetividade produzida pelas relações de
domínio, sujeição e assujeitamentos a deixa inalterada, por dentro e por
fora, apesar de procurar manter a distinção entre dentro e fora
(interesses privados e coletivos, o privado e o público, o Estado
construindo a sociedade), para que as práticas reformadoras configurem a
constante permanência da existência. Se somos passageiros da vida
humana, somos componentes decisivos da humanidade. Perseguimos, ou
melhor, devemos perseguir a melhor humanidade a ser institucionalizada,
fazendo dos mundos previstos e do mundo em que vivemos o ponto crucial
das lutas que desembocarão ou não em revoluções. Contra o que somos?

Foram os anarquistas que primeiro romperam com o círculo
comunidade-sociedade, anunciando um presente-futuro como Anarquia. A
análise serial de Proudhon procurou dar um fim ao círculo vicioso
comunidade-sociedade dos liberais ou ao comunidade-sociedade-comunismo
dos comunistas. Mas mesmo entre os anarquistas permanecia a construção
de novo mundo, a presença da sociedade livre de Estado, mas sociedade
como Stirner (2004) indicou a respeito da seriação proudhoniana. Enfim,
no limite a sociedade viria se tornar a categoria do entendimento, o que
os comunistas também incorporaram a partir de Marx e Engels e dos
debates com Proudhon nos anos 1840. Permanecemos ainda nas conformações
evolutivas de conservação do humano biológico, visto como sujeito de sua
vida, na gestão governamental real ou transcendente de sua vida, e na
qual pela sua inexorável morte deveria deixar sua marca para uma melhor
e mais justa humanidade. O sujeito (indivíduo), o biológico (espécie) e
a humanidade (em sociedade) governam por fora e por dentro, enquanto,
moderna e contemporaneamente, as instituições e a psicologia, a educação
e a cultura governam os divíduos.

A relação entre mundo e sociedade, mais recentemente, tem entrado em
discussão (Danowiski \& Castro, 2014) devido justamente à absorção dos
saberes da física e da química na compreensão da tensão entre o Humano
(de fundamentação europeia) e o Terrano (recuperado dos povos
ameríndios). A noção de mundo não parece mais dar conta, quer da
programática sustentável pelos arranjos internacionais, quer pelas
práticas tranterritoriais. Não se trata mais de encontrar meios pelos
quais a vida biológica do humano seja garantida diante do direito de
fazer viver e deixar morrer. O mundo governado pelos humanos e seu
respectivo humanismo pretende ainda ser suficiente para explicar a
imperativa disputa pela defesa do planeta e sua projeção no espaço
sideral e o triangulo equilátero herdado da Revolução Francesa se
metamorfoseou em um quadrado cujo lado suplementar é o amor. É
inquestionável a expansão contínua do universo, do mesmo modo que a
ampliação dos saberes sobre o corpo e a mente humana, cada vez mais
milimetricamente mapeados e decompostos em minúcias. O direito de fazer
viver e deixar morrer agora deixa de ser um exercício nacional sobre a
população em um território ou com territórios anexados como inaugurou o
nacional-socialismo, já em si uma proveniência transterritorial sobre
populações. Não se trata mais, em seu limite, de um direito que se
aplica a quem deve viver e a quem deve morrer. Não estamos mais nos
limites do racismo de Estado da biopolítica e muito menos no âmbito
restrito da vida biológica e do meio.

Hoje, convenções e tratados e novas condutas humanitárias não admitem
mais o racismo e o penalizam, procuram extinguir preconceitos com
medidas de condutas criminalizáveis, gerar zonas de assimilação de
refugiados, atender populações de rua, acolher os migrantes e
imigrantes. Manter vivo é fundamental. Manter vivo o humano e todas as
formas de vida no planeta de modo sustentável. Reconhecer,
paulatinamente, que os efeitos da degradação foram energizados desde a
revolução industrial e produzir diplomacias programáticas que mantenham
produtividades capitalísticas e ao mesmo tempo conservem a vida (pelo
menos, enquanto se pesquisam os exoplanetas). É preciso reconhecer os
\emph{terranos} para que as noções de humano e mundo sejam também
reformadas, afinal, aos poucos, deve-se reconhecer que o planeta é de
todos, do capital e do capital humano. De sorte que Terranos, neste
diapasão, poderão ter acesso à nova sustentabilidade em termos do
pluralismo que ela esteia. Não se trata de opor Terrano a Humano, mas de
amalgamá-los pelo pluralismo em uma sociedade reformada. Isso abre a
possibilidade, inclusive, de se pensar que no futuro a sociedade possa
se reconhecer como pós-capitalista, ainda que nesse itinerário nebuloso
em que a sociedade prevalece sobre o Estado encontremo-nos na
ambivalência do entendimento pelas categorias de Estado e sociedade
revisitadas. Isso situa a invariante mudança em que nos encontramos, na
qual a ecopolítica aparece como conexão de saberes e verdades que, na
dinâmica das relações de poder em fluxo, produz essa nova situação em
que o Estado por si só não é a categoria do entendimento e tampouco a
sociedade será a nova categoria do entendimento. Abre-se uma fissura
para o contra o que somos.

A ecopolítica nos mostra a vida sob a forma desse entendimento ambíguo e
ao mesmo tempo parceiro, forma pela qual se prevê a sustentabilidade,
mas não se antecipa a superação do capitalismo para além das ideologias,
ou melhor, pouca atenção se dá às novas formas associativas que essa
nova configuração da reforma pode acionar em função da superação da
evolução comunidade-sociedade. Se estudiosos do presente, como Negri e
Hardt (2014), ao identificarem subjetividades a serem rompidas em nossa
época como o endividado, o midiatizado, o securitizado e o representado,
procuram firmar o conceito de \emph{multidão}, e outros, como Newman
(2011), dissecam a auto-dominação, os assujeitamentos, e encontram
disposições táticas na dinâmica dos discursos, as atenções para a
superação de mundo segundo os humanos talvez não esteja somente na
possibilidade de manter vínculos diplomáticos a partir do reconhecimento
da degradação pela parte capitalista e humana no embate. A
sustentabilidade também se volta cada vez mais para as populações
indígenas, ribeirinhas, nômades procurando assimilá-las, e o
capitalismo, sabemos, possui força física e ardilosa para levar isso
adiante da mesma forma que a racionalidade neoliberal é exímia em
produzir diplomacias.

Pela ecopolítica compreendemos que a relação Estado-sociedade civil é
mais do que mero enunciado liberal, posto que tal relação depende do
Estado para que a economia política tradicional ou reformada aconteça. A
ecopolítica situa a conexão entre Estado e sociedade civil procedente
dos efeitos da \versal{II} Guerra Mundial que busca efeitos totalizadores
relacionados com os dividualizantes pela produtividade das energias
inteligentes que geram condições para implementações de convocações à
participação de cada um. Não há mais exclusões, todos são passíveis de
ser ocupados produtivamente de forma direta ou indireta, para o que
colaboram os variados monitoramentos e a arquivística. Uma conduta
resiliente é a mais adequada para se levar adiante o planeta resiliente,
governando-se em conjunto efeitos climáticos, mobilizações de gente e
populações, delimitando claramente os adversários e os inimigos.

Como em todo pluralismo, os que não estiverem no interior da unidade são
imediatamente identificados como inimigos. E aos inimigos, como o
terrorismo e o tráfico de drogas transterritoriais, cabe o uso da força
(física e de agências de inteligência) para matar e conter
guerras-fluxo; aos que escapam como perdedores radicais, o combate pela
força de morte; aos mobilizadores de protestos, suas inserções nas
condutas criminalizáveis e penalizações; aos convencionais delinquentes,
a prisão ou o monitoramento calculado; às zonas vulneráveis e populações
vulneráveis, programas de inclusão com a comunidade, combinando força
física e inovações nas garantias de interesses coletivos. A análise
ecopolítica procura e encontra a produção de dispositivos que consolidam
o desenvolvimento sustentável, ao mesmo tempo em que atiça para a nova
dinâmica das resistências. Trata-se, portanto, de uma área de saber que
busca enfrentar as relações de poder em fluxo, ininterruptos, e suas
surpreendentes resistências. A ecopolítica é, antes de tudo, a forma de
governo do vivente, do animal ao vírus, pela conexão entre Estado e
sociedade.

As análises relativas às ultrapassagens da biopolítica pela ecopolítica
expuseram a assimilação de populações em \emph{ambientes} e a situação
relacionada ao \emph{ingovernável}, a partir dos dispositivos \emph{meio
ambiente}, \emph{diplomático-policial}, \emph{resiliência} e
\emph{monitoramento}: a produção da relação \emph{governo-verdade} na
geração de moderações, considerando a pertinência da conduta resiliente;
o direito como conservação da vida restaurada e a emergência do
\emph{cidadão-polícia} portador de direitos inacabados; as novas
configurações dos espaços monitorados; a situação sideral; as novas
conformações da \emph{segurança} humana e transterritorial. O direito de
causar a vida e devolver a morte se transfigura em direito a como viver.

O governo da vida do planeta incorpora as práticas de biopolítica e de
controle da população com vistas à produtividade, mas as transforma pelo
incentivo e a convocação à participação concomitante à relação
sustentável entre humano e natureza, à dinâmica das seguranças, às
modulações dos monitoramentos. O governo da vida biológica passou a ser
atravessado pelos mapeamentos sobre as inteligências, as proteínas e o
espaço sideral. As referências dos saberes biológicos, ampliados pelos
ecológicos, encontram referências cada vez mais próximas da física (como
resiliência) e da química (como antropoceno), que produzem as inovações
no discurso estendido produzido pela relação governo-verdade. As
relações individuais (disciplinas/vigilância e punições) e as
totalizantes (biopolítica como tecnologia de governo), produzindo
normalizações, são ainda interfaces no redimensionamento de relações
dividuais (direitos/monitoramentos e variedades penalizadoras) e
totalizantes (ecopolítica como tecnologia de governos de si, dos outros
e da conservação possível do planeta para um futuro melhor; a união de
todos diante dos efeitos do antropoceno; a busca por uma nova ordem no
sentido de superação do estado das coisas e que, para tal, exige uma
formalização responsável dos adversários para que uma diplomacia
política seja possível). Diante dos anormais, a produção dos normais, a
produção normalizante; da degeneração, a superação da degradação; e
assim as punições amplificadas a céu aberto combinam com os
encarceramentos de segurança máxima pautados na defesa e monitoramento
dos direitos humanos. Quanto às resistências, pretende-se incorporá-las,
suplantá-las, retê-las, modificá-las e capturá-las sob a forma
resiliência.

A combinação dos dispositivos em funcionamento explicita a metamorfose
da biopolítica em ecopolítica. Esta emerge não apenas por meio da
confluência do saber da economia política com os dispositivos de
segurança, como balizava a biopolítica. Exige outros dispositivos,
talvez uma profusão de dispositivos segundo os fluxos que são
produzidos; delimita os inimigos móveis transterritoriais (terrorismos
islâmicos, movimentos de contestação e ilegalismos transterritoriais),
produz a confluência de forças como jogo democrático pluralista de
reconhecimento dos adversários distintos dos inimigos; amplia o
pastorado horizontalizado e é capaz de conjugar as novas relações de
poder horizontalizadas com a verticalidade convencional do poder
soberano; internacionaliza e transterritorializa as relações com o
governo da vida, procura absorver o \emph{ingovernável}, introduz a
figura do \emph{stakeholder}, produz elites secundárias e as conecta à
principal; dinamiza as relações entre fluxos de saberes, práticas
institucionais, a variedade de seguranças, a formação educacional de
crianças e jovens para a conduta resiliente, bem como de comunidades; o
governo das prisões e das penalizações absorve comunidades e cria outras
voltadas para a gestão compartilhada da prisão (privatizada ou não) como
no caso do \versal{PCC}-Brasil; ocorre a metamorfose da doença mental em saúde
mental, da velhice em gerontologia, com \emph{sujeitos} protagonistas e
empoderados, em esperados ambientes cada vez menos vulneráveis e com
melhor qualidade de vida, formatando uma \emph{cultura de paz}
constatada pelos resultados dos diversos índices que indicam ascensões
esperadas pelos países, em função da utopia do 1 ou do 100, segundo os
graus de variabilidades.

A ecopolítica não governa mais a população pela estatística, mas por
meio dos índices e dos mapas georreferenciados que aferem situações
reais e de melhorias a serem ampliadas ou conquistadas pelas populações
e pela totalidade do planeta dentro e fora da atmosfera. Do corpo se
extrai a inteligência fundamental para ultrapassar a condição de miséria
como empreendedor de si resiliente que governa e é governado com
segurança em seus ambientes pelos monitoramentos. A céu aberto estão
comunidades, museus, trânsitos, concentrações no planeta e a incansável
busca pelos exoplanetas. O corpo inteligente está na Terra e no espaço e
os \emph{spin-offs} produzidos ali reverberam na vida terrena (Siqueira,
2015). Espécies desaparecem ou se transformam no ciclo vida e morte,
incapaz de ser controlado, apesar dos esforços dos gestores. A
expectativa de vida aumenta, a gestão da miséria se desmembra, a água
seca, o calor aumenta. Há um fogo a ser adensado. Conhece-se cada vez
mais o corpo (seus tecidos, órgãos, células, proteínas) pelas
nanotecnologias e as inteligências pelos avanços das neurociências,
bioinformática e controle pela psiquiatria e a farmacologia.

Diante das drogas ilícitas (as que não são legalizadas por meio da
farmacologia e as sintéticas que alimentam o fluxo econômico capitalista
sem distinguir legal de ilegal), produz-se uma quantidade cada vez maior
de medicamentos para o controle nervoso em função da estabilidade para a
moderação, incluindo-se a medicalização gradativa de crianças e da
pobreza, segundo a programática de benefícios que atua sobre divíduos e
grupos. O sexo encontrou na liberação sua função diversificadora da
sexualidade. Além do incesto, nada é proibido, todavia, o espetáculo da
sexualidade reitera a prostituição para torná-la mais segura que
delitiva. A conduta dos normais normalizados e liberados (com ou sem
pecados religiosos a administrar, a conquistar pela ensandecida conduta
do mártir ou encenadas nos refúgios ermos) medicaliza e judicializa o
sexo (de produtos, profissionais e autoajuda ― os manuais para todas as
condutas, com casamentos hetero, homo, transexuais). A céu aberto, eles
estão revestidos de armaduras que pretendem blindar a iminência de uma
experimentação radical e asseguram a liberdade liberal ou
fundamentalista de qualquer matiz. Estão sob a governança global.

\section{diminutas}

\emph{eu não espero pelo dia em que todos os homens}

\emph{concordem}

\emph{apenas sei de diversas harmonias bonitas possíveis sem}

\emph{juízo final}

\emph{alguma coisa está fora da ordem}

\emph{fora da nova ordem mundial...}

Caetano Veloso, Fora da ordem

A pesquisa que realizamos entre 2010 e 2015, rubricada como Projeto
Temático Fapesp \emph{Ecopolítica. Governamentalização planetária, novas
institucionalizações e resistências na sociedade de controle}, procurou
dar conta desses deslocamentos ocorridos desde a \versal{II} Guerra Mundial,
sinalizando para o funcionamento de dispositivos específicos
relacionados ao \emph{meio ambiente}, à \emph{segurança}, aos
\emph{direitos} e às \emph{penalizações a céu aberto.}

A pesquisa em arquivos eletrônicos das máquinas cibernéticas mostrou a
relevância do acompanhamento das resistências em suas manifestações
imediatas, as práticas reativas que elas instigaram e as formas de
capturas, os ajustes entre a politização democrático-representativa e a
participativa, não só no Estado, mas na produção de produtos, serviços,
empreendedorismos sociais, negócios sociais, mobilizações em torno do
clima, catástrofes naturais, terrorismos, condutas de controle. Os
arquivos eletrônicos de documentos, escritos e imagens de comunicação
contínua, como afirmou Deleuze, compuseram os mapas necessários e
suficientes (ainda que sempre inacabados, abrindo-se a novos, às devidas
complementações e aos adendos). A cartografia também se tornou mais
completa, pois estes mapas foram sempre lidos, segundo regulações e
regulamentações internacionais e seus efeitos nos Estados acompanhados
de estudos e pesquisas científicas realizadas pelas humanidades e pela
difusão de práticas científicas multidisciplinares.

O \emph{meio} não é mais o mesmo; a \emph{paz internacional} não depende
somente de dispositivos diplomático-militares e a \emph{ordem} interna
dos Estados não se adequa apenas aos dispositivos policiais e à
biopolítica; os direitos, desde a \emph{Declaração dos Direitos
Humanos}, de 1948, ajustaram-se pelos efeitos de revoltas e protestos e
situaram uma região de atração entre as convencionais \emph{direita} e
\emph{esquerda} políticas do Estado; os castigos passaram a ser
governados de outro modo, mas não deixaram de ser punições, agora
regulamentados com tolerância (zero) como \emph{mais e melhores}. Sob um
cenário construído com \emph{tolerância} para a encenação da
\emph{cultura de paz} neste novo teatro da vida, o \emph{desenvolvimento
sustentável} apresenta-se como a nova peça para um velho personagem, o
capitalismo, que desta vez atrai para o palco como um enorme
\emph{chorus line} os cidadãos e os seus filhos, como se já não houvesse
mais plateia e qualquer um pudesse ser um protagonista em uma breve
cena. Não se trata da tragédia grega, mas seguramente de um drama que
pode se metamorfosear em tragicomédia.

Distante das análises ideológicas, a genealogia da ecopolítica encontra
na racionalidade neoliberal transformando a força de trabalho em capital
humano o elemento discursivo fundamental para sua consolidação, na
medida em que procura fazer de cada um o parceiro inovador, empreendedor
e ativo na gestão de um planeta situado no \emph{antropoceno} e que cada
vez mais depende de uma diplomacia efetiva. Protocolos e programas se
combinam assim como a gestão computo-informacional cada vez mais busca
definir a presença de cada um e de suas condutas \emph{aplicativas}.
Trata-se de uma sociedade de controles, como indicou Gilles Deleuze, de
comunicação contínua, que é governada pela obsessão por segurança, como
indicaram as pesquisas de Michel Foucault. Resistir é um pouco mais e
menos diante das resiliências. O poder não se relaciona mais tão somente
em redes, as resistências não se movem apenas pela estratificação social
nesta relação complementar indissolúvel entre poder e resistências. A
política não é mais a guerra civil prolongada por outros meios, olhada
pelos seus governos interiores, nem a guerra prolongada por outros meios
na intersecção com outros governos de Estado.

O poder em fluxo e a política transterritorial dão as novas eficácias e
eficiências para a ecopolítica. O terror que vinha do Estado ou que o
atacava para reformá-lo ou revolucioná-lo adentra os meandros
computo-informacionais, produz fortalezas dogmáticas que conjugam
racionalidades e religiões, e permanece como a parte constitutiva do
governo do Estado, seja pelas forças de defesa que procuram assegurar a
liberdade liberal, pelos ainda vestígios do socialismo revolucionário,
seja pelas forças que pretendem uma uniformidade teológico-estatal.
Entre tantas diplomacias, não há como conter os terrorismos, do mesmo
modo que entre tamanho esforço resiliente não há como estancar
resistências \emph{antipolíticas}.

Uma pesquisa nunca é conclusiva. Esta pretendeu apenas situar a
emergência da ecopolítica, seus desdobramentos, sua presença, no campo
estrito da produção de verdades que se configuram como saberes na medida
em que os enunciados emanam e situam forças em luta na sociedade. A
genealogia da ecopolítica é um método e um convite para outras
incursões.

A ecopolítica é uma situação da ordem atual, mas é também princípio e
fim dessa mesma ordem. Do mesmo modo que a biopolítica foi capaz de dar
conta de uma situação nova advinda do final do século \versal{XVIII} e que
mostrou seus limites durante a \versal{II} Guerra Mundial, a ecopolítica é uma
noção analítica que busca situar os efeitos decorrentes de uma situação
inédita: ao mesmo tempo em que forças oponentes de Estados ameaçaram a
vida no planeta, passaram a buscar outras condições de vida fora do
planeta e salvaguardá-lo. Claro está que todas as conquistas siderais
não apontam para deslocamentos próximos de todo o povo da Terra para
melhores \emph{ambientes}, mesmo porque a história civilizatória deste
planeta sempre tratou de equacionar as \emph{melhores condições} para os
superiores, fossem eles reis e príncipes, aristocratas, burgueses,
elites ou vanguardas. Todavia, a composição atual democratiza, por meio
da participação e da configuração do capital humano, quais são e serão
os beneficiados de hoje no futuro próximo. É preciso melhorar o
\emph{ambiente} no planeta do mesmo modo que se exige encontrar outros
ambientes siderais ― e se estes ainda não são os exoplanetas, muito se
deve aprender para se ocupar ambientes similares ao da Terra.

Uma pesquisa como essa não deixou de recorrer à nova biblioteca
instalada nas nuvens eletrônicas da internet. Esta absorve as mais
variadas expressões da vida, registradas em arquivos, sites,
enciclopédias livres, iconografias, mapotecas, uma infinidade de
informações acessadas por meio de palavras-chave. Sim, as mesmas que
orientam a busca por teses, dissertações, artigos, ensaios, resultados
de pesquisas que compõem o grande banco de dados científico, este
preservado bem do saber sobre a humanidade. Mas não são apenas os bancos
de dados que democratizam comunicações por meio de servidores e
provedores hierarquizados que governam o fluxo das informações, conexões
e especulações. Eles também estão sob o governo dos provedores
supervisionados por Estados e forças militares.

A democracia convoca a participar e depois seleciona, exclui e pune quem
ferir os regulamentos e a moral, criptograficamente ou não. De outra
sorte estão os programas de relacionamentos pelos quais se adquire
amizades, amores, amantes, seguidores, financiamentos e também se
agenciam contestações, protestos e revoltas. São estas diferenças na
gestão democrática centralizada de poder e exercitada de modo rizomático
que produzem também provedores que escapam do governo central dos
saberes. Até quando? Ou melhor, até quando os dispositivos de segurança
serão capazes de conter revoltas, protestos e reviravoltas? A pesquisa
sobre a ecopolítica não deixou de acompanhar a produção de verdades
nestes diferentes fluxos, posto que o pensamento científico e crítico
também dela não se apartou, ou ao contrário, neles se filiou ou pelo
menos se instalou transitoriamente.

\emph{Ecopolítica} é um livro posterior aos relatórios de pesquisa
produzidos e arquivados eletronicamente em
\emph{http://www.ecopolitica/pucsp.br}. Está
dividido seguindo os fluxos \emph{meio ambiente}, \emph{diretos},
\emph{segurança} e \emph{penalização a céu aberto}. Situa a ecopolítica
para depois traçar percursos relativos a essa governamentalidade
planetária sob novas institucionalizações e resistências.

\emph{Ecopolítica} é um livro que não conseguiria ser escrito por uma
autoria. É resultante do trabalho constante de uma equipe com
atribuições horizontalizadas segundo os fluxos e que, no decorrer da
pesquisa, foi consolidando análises a respeito de caudalosos fluxos
afluentes. Sabe-se que em nosso pensamento tudo se relaciona,
principalmente no jeito libertário de tocar a vida, e que consultar um
mestre mais experimentado é sempre imprescindível. Assim funcionamos uns
para os outros, do mesmo modo que buscamos ouvir de interlocutores
disponíveis que nos acompanharam com interesse nesses anos, para deles
receber leituras, sugestões e novas inquietações. Transitamos pelo
planeta falando a pequenos públicos em encontros internacionais e
nacionais sobre a pesquisa. Alguns se mostraram atentos, outros pouco
preocupados com a temática ou voltados para outras inquietações dentro
de eventuais similaridades. Em todo canto ficava a constatação que
lidamos com algo que acontece e que de variados modos está sendo
apreendido por teorias que buscam se reciclar, atualizar, modificar,
ainda que circunscritas nas fronteiras do \emph{dentro} e do
\emph{fora}. Nossa atenção para os fluxos deslocou-nos imediatamente da
fronteira, das margens e das bordas que são preocupações pertinentes de
Estados, mas nem sempre dos resistentes que não desconhecem o fora
relacional. São imprescindíveis para resilientes, estão na mira daqueles
que almejam uma definitiva governança global, dos credores e
financiadores do capitalismo. Os fluxos nos mostram as conexões entre
direita e esquerda para além das ideologias, da política, da
configuração dos intelectuais \emph{moduladores} própria da
sustentabilidade, também delineada em ordem e contraordem conectadas em
\versal{ONU} ou Cúpula dos Povos, acontecendo nos aparelhos de verificação.

O \emph{mundo} não é mais o mesmo traçado e distinto na Grécia, na
Europa ou como sinônimo de Terra. Não é mais parte do Universo criador.
Não é mais \emph{mundo das ideias}. Não há mais \emph{mundo}. Não é mais
o outro nome de Terra-Gaia. Não há mais a crença no domínio do homem
sobre a natureza, mas ainda se crê em Deus e no Homem. Se o filósofo
Friedrich Nietzsche certa vez insinuou irmos para além do bem e do mal e
para tal mostrou os fundamentos de cada valor, a fronteira entre eles se
dissipa na medida em que se dissolve a moral, em que éticas governam
liberdades, não como conduta derivada da moral, mas como atitudes que
inventam éticas outras livres da sujeição e assujeitamentos.

A ecopolítica faz funcionar o ponto perdido no panorama e o planeta, com
suas paisagens preservadas e conservadas, seus humanos, animais, pedras,
florestas, rios, mares, oásis, desertos, elevações e depressões, cujos
céus são monitorados por potentes telescópios, suas superfícies e
profundidades cada vez mais conhecidas pelos satélites no decurso dos
cataclismos e modificações humanas e o universo em expansão navegável no
rumor do incerto.

A ecopolítica, mais do que uma específica política ambiental ou
ecológica, procura dar conta das multiplicidades enredadas sob o signo
de segurança planetária que encontra no desenvolvimento sustentável a
forma contemporânea de produzir a sinonímia
capitalismo-democracia-utopia. A ecopolítica se interessa pelas lutas
diante da conservação moderada da vida humana dentro e fora do planeta,
para a qual todo cidadão e toda criança deve se educar para governar-se
democraticamente em favor das melhorias das condições de existência, o
compromisso de cada um com a futura geração.

A ecopolítica não trata de modelos, mas justamente da produção constante
de maneiras conformes e alternativas para garantir a sustentabilidade,
pretendendo fazer de cada instante um momento de participação modulável
de cada um elevando-se à categoria de protagonista.

A ecopolítica emerge dos acontecimentos derivados da \versal{II} Guerra Mundial
que colocou, pela primeira vez, a possibilidade de extermínio do humano
e do planeta; absorve a biopolítica, situa as intrínsecas relações entre
Estados, sociedades civis e \versal{ONU}, que visam estabelecer a governança
possível, cujo objetivo nuclear é suprimir a relação
governante-governados.

A ecopolítcia situa as novas relações políticas formais que ultrapassam
a representatividade para nela agregar a convocação à participação de
todos, organizados e mobilizados em função de melhorias,
fundamentalmente quanto à miséria, saúde e educação dos \emph{povos}.

A ecopolítica produz novas institucionalizações, por conseguinte abertas
e inacabadas, e resistências surpreendentes diante da expectativa de
formação de pessoas, povos e planeta resilientes.

A análise da ecopolítica dirige-se ao interesse dos resistentes. Na aula
inaugural de \emph{Em defesa da sociedade}, Foucault em poucas palavras
resume o que a genealogia espera enquanto repercussões: ``o silêncio do
adversário ― {[}e{]} é este um princípio metodológico ou um princípio
tático que se deve sempre ter em mente ― talvez seja da mesma forma o
sinal que não lhe metemos medo algum. E devemos agir, acho eu, como se
justamente não lhe metêssemos medo'' (Foucault, 1999: 19).

Ao discurso político basta colocar questões relativas às formas e
estruturas de governo; ao científico, afirmar as regras em torno das
quais se diz a verdade, condições e estruturas; o discurso moral ``se
limita a prescrever os princípios e as normas de conduta'' (Foucault,
2011: 59). O discurso filosófico se interessa pela verdade e as
condições para se dizer a verdade.

A genealogia nos instiga a uma atitude parresiasta, da coragem de dizer
a verdade sem medos diante do dizer-a-verdade (\emph{alétheia}) do
governo, e da \emph{polithéia} e da \emph{ethopoiésis} que geraram na
cultura grega a formação do sujeito. Trata-se de seguir e produzir
resistências. ``Neste Ocidente que inventou tantas verdades diversas e
moldou artes da existência tão múltiplas, o cinismo não para de lembrar
o seguinte: que muita pouca verdade é indispensável para quem quer viver
verdadeiramente e que muito pouca vida é necessária quando se é
verdadeiramente apegado à verdade'' (Ibidem: 166).

A vida como obra de arte antiplatonismo é irrupção do elementar,
desnudamento da experiência. Não há como discordar de Foucault, tampouco
de suas sugestões acerca do militantismo livre de organização e das
uniformidades. É preciso uma prática antiaristotélica diante de toda
arte adquirida e suas convenções, como recusa e agressão. Não há como
abdicar da violência e as práticas da revolta o confirmam, são
indomesticáveis e acionam a potência da vida. E se entendemos o que
virá, lido ou não com o nome de porvir, é pelo que rompe, foge, escapa,
nomadiza. O que escapava em \emph{1968} e \emph{1999} foi capturável por
conservadores, pela racionalidade neoliberal e pelas práticas
democráticas. O que escapa de 2011 a 2013 é o que também escapou de
\emph{1999}. Pouco importa se preferem dar o nome de multidão, pelo viés
do mundo novo, a um mundo outro que escapa. E nesse fluxo, obviamente,
tudo pode acontecer. Que seja a multiplicidade da transistoricidade do
cinismo, reiterando que o poder é da ordem do governo, do enfrentamento
entre adversários e entre inimigos.

Em \emph{Nascimento da biopolítica}, Foucault perguntou ``quais são os
efeitos reais da governamentalidade ao cabo de seu exercício?''
(Foucault, 2008a: 21). É nada relativo à economia política que apenas
descobre a naturalidade da prática de Estado, seu correlato perpétuo,
sua verdade. Ela requer uma governamentalidade mínima e, como liberal, é
consumidora de liberdades (do mercado, do exercício do direito de
propriedade, de discussão, de expressão...). E se é consumida, é também
produtora e a organiza, não enquanto o imperativo \emph{seja livre}, mas
como produzir para que você seja livre, portanto condições para ser
livre, que necessariamente repercutem em produção de limites, controles,
temores, repressões, medidas de punição e disseminação do medo (Ibidem:
86-87). A análise da governamentalidade se volta para ``estatizações
progressivas, fragmentadas mas contínua de certas práticas, de maneira
de fazer, e se quiserem de governamentalidade'' (Ibidem: 105). A
governamentalidade liberal se revê e se programa nesta época neoliberal.

A governamentalidade planetária situa dilatações inexoráveis do que se
conhece como o sujeito no ocidente de suas práticas de subjetivação em
relações de poder como combate. Trata-se de como nos governamos e como
somos governados enquanto condutas esperadas como técnicas de si e
governamentalidade política, que nos permite acessar estratégias e
táticas de resistências (Foucault, 1977a). Se nas sociedades
disciplinares o alvo era a população com seus dispositivos de segurança
sob o dinamismo das relações soberania-disciplina-biopolítica, na
sociedade de controles o alvo é o planeta com seus variados dispositivos
(dentre eles, conectados à segurança estão o meio ambiente, o
diplomático-policial, o monitoramento e o resiliência) que dinamizam as
relações soberania-controles-ecopolítica.

Para Foucault é possível tratar o Estado Moderno e sua
governamentalidade como correlato do que foram a segregação para a
psiquiatria, as técnicas de disciplina para o sistema penal, e a
biopolítica para as instituições médicas (Foucault, 2008: 162), a partir
de uma nova tecnologia geral do governo dos homens, pois o que se
governa são homens pela incorporação do pastorado cristão, as relações
diplomático-militares como apoio e a polícia interna como suporte
(Ibidem: 164). Trata-se de governar condutas, conduzir e conduzir-se,
ser conduzido de outra maneira, uma contraconduta, resistência de
conduta, enfim, técnicas de sujeição sobe práticas de subjetivação. É
sempre bom lembrar que o pastorado emergiu como relação de
``enfrentamentos, hostilidade, guerra como algo que é difícil chamar de
rebelião de conduta'' (Ibidem: 226). Rebeliões de conduta se distinguem
de revoltas políticas em sua forma e objetivo (Ibidem: 227). Foucault se
pergunta como designar resistências a formas de poder que não exercem
soberania ou exploram, mas conduzem? (Ibidem: 263-266). Rebelião não
distingue resistências difusas e moderadas, desobediência é um termo
demasiado débil, insubmissão está relacionada à rebeldia militar, e
dissidência é pensar de outra maneira, recordando os russos contra o
governo do \versal{PC} na \versal{URSS}, contra a pastoral da obediência. Para todas essas
designações, Foucault propõe contraconduta, luta contra os procedimentos
na condução de outro que é encontrável nos loucos, delinquentes e
enfermos.

Os conteúdos histórico-políticos das lutas, sepultados pelo discurso
filosófico jurídico-político e pelos saberes em coerências funcionais ou
sistematizações formais, surpreendem e reaparecem. Esses saberes
sujeitados reaparecem no saber histórico das lutas, acoplam
conhecimentos eruditos e das memórias locais e escancaram efeitos
centralizadores de poder vinculados às instituições (Foucault, 1999). No
passado recente, a conjugação da teoria da soberania (de proveniência
liberal fundada na troca contratual ou marxista relacionada à
funcionalidade econômica do poder) com a mecânica das disciplinas
combinou, de um lado, lei e ciência do direito e, de outro lado, norma,
normalização e ciências humanas. No presente dos viventes, a teoria da
soberania estabelece relações mais estreitas com os fluxos de poder sob
o comando da racionalidade neoliberal que democratiza não mais o
soberano (o rei, a democracia parlamentar), mas as relações de poder:
não mais se visa normalizar, posto que todos somos normais sob
transtornos, porém, normaliza-se os normais pelos dispositivos
\emph{meio ambiente}, \emph{diplomático-policial}, \emph{monitoramentos}
e \emph{resiliência}. A relação entre o corpo-máquina (disciplinas) e o
corpo espécie (biopolítica) agora é dimensionada pela governança que
democracaticamente produz a gestão das inteligências de cada um em
variados ambientes, revestidos de direitos e práticas
diplomáticas-policiais que dimensionam as novas funções de
monitoramentos do pastorado resiliente. Em lugar das massas, os variados
e híbridos conjuntos de minorias exercitando-se como portadores de
direitos inacabados, convocados a participar em função do melhor mundo
futuro, não só na Terra, mas no espaço sideral, na busca incessante pelo
\emph{melhor} exoplaneta. O corpo-planeta é o alvo. ``O corpo ― e tudo o
que diz respeito ao corpo, a alimentação, o clima, o solo ― é o lugar da
\emph{Herkunft}: sobre o corpo se encontra o estigma dos conhecimentos
passados do mesmo modo que dele nascem os desejos, os desfalecimentos e
os erros; nele também eles se atam e de repente se exprimem, mas nele
também eles se desatam, encontram em luta, se apagam uns aos outros e
continuam seu insuperável conflito. (...) A genealogia, como análise da
proveniência, está portanto no ponto de articulação do corpo com a
história e a história arruinando o corpo'' (Foucault, 1979: 22). O corpo
é a Terra!

A análise de Foucault permaneceu pertinente quando nos deslocamos para a
ecopolítica; estamos numa dinâmica que faz progredir condutas e
contracondutas em função da inibição de resistências. Mas são estas que
situam \emph{antipolíticas} por meio da revolta, inventam o mundo outro
e fazem reaparecer os saberes sujeitados provocando novas relações na
produção da verdade.

17 de dezembro de 2015. Inauguração do Museu do Amanhã (\versal{MA})\footnote{\emph{https://www.museudoamanha.org.br/}.
  O \versal{MA} é resultado de parceria entre a prefeitura do Rio de Janeiro,
  Fundação Roberto Marinho, Banco Santander, como provedor máster,
  Shell, como mantenedora, com apoio da Secretaria do Meio Ambiente do
  estado do Rio de Janeiro e da Finep, e a parceria tecnológica com a
  Cisco. A gestão do \versal{MA} é responsabilidade do Instituto de
  Desenvolvimento e Gestão (\versal{IDG}) \emph{http://www.idg.org.br/quem-somos/}
  e \emph{http://www.idg.org.br/o-que-fazemos/}.}, na cidade do Rio de
Janeiro, 23 anos após a \emph{Rio 92}. A arquitetura arrojada lembra uma
aeronave que decola em direção ao espaço. Localizado no centro
restaurado da cidade, o Porto Maravilha, é considerado mais um legado
dos megaeventos Copa do Mundo e Olimpíadas. Segundo o arquiteto Santiago
Calatrava, ``a ideia é que o edifício fosse o mais etéreo possível,
quase flutuando sobre o mar, como um barco, um pássaro ou uma planta'',
resultante de ``um diálogo muito consistente para que o edifício se alie
à intenção de ser um museu para o futuro, como uma unidade educativa''.
Ainda segundo o arquiteto espanhol, ``um passeio ao redor do Museu é uma
lição de sustentabilidade, de botânica, uma aula do que significa
energia solar'', e segundo a gerente geral de Patrimônio e Cultura da
Fundação Roberto Marinho, ``sempre procuramos criar algo que agregue
valor e destaque a preocupação com o meio ambiente e o ser humano. Nossa
intenção, com o museu, era criar algo que instigasse o visitante a
entender a época em que vivemos''\footnote{\emph{http://museudoamanha.org.br/pt-br/content/arquitetura-de-santiago-calatrava}}.

O \versal{MA} pretende justificar uma nova concepção de museu, não mais voltado
para o passado e a coleção de objetos, mas para o presente computacional
em direção ao futuro. Um museu voltado à sustentabilidade e ao
desenvolvimento sustentável: ``hoje vivemos na Era do Antropoceno: a
ação humana, seja individual ou coletiva, gera impactos e dimensões
geológicas sobre o planeta'' (Oliveira, 2015: 5). O \versal{MA} conta com o
Observatório do Amanhã, capaz de ordenar informações coletadas por
tecnologias de informação, mas principalmente com a \versal{NASA} (National
Aeronautics and Space Administration), \versal{INPE} (Instituto Nacional de
Pesquisas Espaciais), o \versal{IPCC} (Intergovernmental Panel of Climate
Change), o \versal{WRI} (World Resources Institut), dentre as mais de oitenta
instituições com as quais mantém colaboração permanente. A tarefa deverá
ser compartilhada por todos, principalmente pelas ciências que se
deslocam para ``saberes transitórios, sempre sujeitos à superação e à
renovação'' (Ibidem: 11) em direção a futuros possíveis para os próximos
cinquenta anos. Portanto, amanhã não é futuro, mas está no presente. É
um museu que pretende ordenar informações atualizadas a cada dia, a cada
instante, orientando o presente em direção ao futuro próximo, melhor,
tangível e possível.

O \versal{MA} é também parte do arquivo da humanidade voltado à gestão do meio
ambiente e das desigualdades e se propõe a levar cada visitante a pensar
sobre si e sobre \emph{nós}, ``sobre como queremos viver no mundo ---
com sustentabilidade --- e com os outros --- pela convivência'' (Ibidem:
17). Para tal, o \versal{MA} conta com o Laboratório de Atividade do Amanhã
(\versal{LAA}), espaço de encontros transdisciplinares voltado para ``exploração
de cenários futuros'', impactos de ``tecnologias exponenciais'' e
``empreendedorismo'' (Ibidem). O espaço interior do \versal{MA} procura dar conta
das cinco estações: do Cosmos, da Terra, do Antropoceno, dos Amanhãs e
de Nós. Ao final da exposição o visitante deverá estar convencido que
tudo é possível com sustentabilidade, tudo pode ser melhorado, tudo
depende de nossa tolerância e de \emph{nossos} objetivos comuns. O \versal{MA} é
o convincente espaço miniaturizado da governamentalidade planetária.

1º de janeiro

\emph{Hoje percebo que o que escrevi ontem na verdade escrevi hoje: tudo
o que correspondia a 31 de dezembro escrevi no dia 1º de janeiro, isto
é, hoje, e o que escrevi dia 30 de dezembro é o que escrevi dia 31, isto
é, ontem. Na realidade, o que estou escrevendo hoje escrevo amanhã, que
para mim será hoje e ontem, e também de certo modo amanhã: um dia
invisível. Mas sem enxergar.} (\versal{BOLAÑO}, 2006: 571).

Certa vez, em 1891, um jovem escreveu uma carta ao escritor Oscar Wilde,
conhecido por suas transgressões e que incluía entre elas a anarquia,
pedindo uma explicação sobre a inutilidade da arte após ler \emph{O
retrato de Dorian Gray.} Wilde respondeu de pronto: ``A arte é inútil
porque seu objetivo se resume em criar um estado de espírito. Ela não
pretende instruir nem influenciar qualquer tipo de ação. Ela é
esplendidamente estéril, e a característica de seu prazer é a
esterilidade. Se à contemplação de uma obra de arte se segue qualquer
espécie de atividade, ou a obra é medíocre, ou o espectador não
conseguiu ter a completa dimensão artística'' (Wilde in Usher, 2014:
347). Em conversa com Franck Maubert, o pintor Francis Bacon dizia: ``se
você não tem um tema que o obceca e atormenta interiormente, você cai na
decoração. Você pode até procurar, beber em todos os livros e naquilo
que o cerca, mas isso não basta. (...). Eu preciso de coisas que me
toquem profundamente. E isso nem sempre funciona''. É preciso ```se
deixar levar', `ficar à deriva'. (...) Passei toda a minha vida assim, à
deriva. (...) Gosto daqueles que pesquisam, desmontam, desossam,
inventam. Trabalho sobre mim mesmo. (...) Fui sempre um otimista, mesmo
não acreditando em nada. Quando morremos, não prestamos mais para nada.
Só nos resta ser enfiados num saco plástico e jogados no lixo,
compreende?'' (Bacon in Maubert, 2010: 53).

Gilles Deleuze (2015), em curso sobre Foucault, logo após sua morte,
ressalta haver no filósofo-historiador francês uma forte presença de
Maurice Blanchot. Ao comentar as relações de saber relacionadas à
exterioridade, situa as formas de interioridade com comodidades,
aparências e meios subordinados, pois ver não é falar, falar não é ver,
havendo, portanto, uma disjunção entre falar e ver, de modo que os
enunciados se dispersam na linguagem. A interioridade está sempre
subordinada às formas da exterioridade. Quando Foucault se volta para a
subjetivação está em questão o fora mais distante, exterior e indomável:
a paixão. E é ela que a nossa sociedade quer enclausurar, como a nau dos
loucos, as prisões, os asilos, os hospícios e suas metamorfoses,
diríamos, a céu aberto na contemporaneidade. Porém, é neste fora que ao
mesmo tempo \emph{eu morro} (imediato vivente) e \emph{se} morre, o
começado que nunca termina (a entrada do vivo), o coexistente à vida. O
\emph{se} do poder é o \emph{se} luta, \emph{se} como estratégia e o
\emph{se} choca. As resistências são pontos nessa linha e sua navegação
se dá na linha infinita que \emph{se morre}. São e serão sempre
surpreendentes, pois nelas há uma paixão indomável, ingovernável, para
além das contracondutas, e que demole o amor ao \emph{amor}, ao tanto
amor que pretende governar as subjetividades e obstruir novas
subjetivações. Como sublinha outro filósofo, Max Stirner, ``o amor é
decerto a mais bela e derradeira repressão de si, a forma mais gloriosa
de se aniquilar e sacrificar, a vitória sobre o egoísmo mais culminante
em delícias; mas ao despedaçar a vontade própria obstaculiza ao mesmo
tempo a própria vontade que é, para o homem, a fonte primeira da sua
dignidade de ser livre'' (Stirner, 2002: 20).

Exercícios de poder devem ser vistos como governamentalidades, não pela
história do conhecimento, mas pelas formas de veridicção; não pela
história da dominação, mas pela história dos procedimentos de
governamentalidade; não pela história do sujeito (das subjetividades),
mas pela história da pragmática de si e das formas que adquire. Trata-se
de uma história das experiências. A noção de modernidade é compreendida
pela genealogia da modernidade como questão. Portanto, não há em
Foucault ou nas análises genealógicas algo que diz respeito à
pós-modernidade, mas somente a questão da modernidade e o que ela
suscita, incita, produz, ou seja, relações de poder e resistências,
produção de saberes (inclusive de saberes insurgentes), jogos da
verdade, efeitos de soberania.

Governar a si e aos outros é a questão de como o (in)divíduo governa a
si e aos outros, de como se constitui como sujeito na relação consigo e
com os outros. É por isso que, ao lado da isonomia e da isegoria
herdadas dos gregos, Foucault deu tanta atenção à parrésia enquanto
manifestação da verdade que não teme a morte, que é revolta, é cinismo
trans-histórico. Governar a si e aos outros na ecopolítica situa o
agonismo entre a \emph{nova política} e a \emph{antipolítica}, os fluxos
da governamentalidade planetária, a programática inacabada das sensações
e das institucionalizações.

Não estamos mais nos \emph{baixos inícios} com os \emph{pogroms} russos
ou em seu \emph{ápice} nos campos de extermínio nazistas. Não se
\emph{pensa} mais em exclusões violentas em massa ou em extermínios, mas
em como governar populações em seus ambientes dando-lhes, assim, uma
nova dimensão ao campo de concentração (favelas-comunidades, espaços
incrustados em centros restaurados de cidades, condomínios,
empresas...). Também não há mais como pretender equacionar o dilema
público-massa próprio dos liberais, porque não há mais massa seguidora
de líderes ou esvaziada de subjetividades; ao contrário, as minorias
numéricas assumem suas reais \emph{identidades} requeridas e
reconhecidas, organizadas em \emph{elites secundárias}, funcionando para
o melhor e a ideal sustentabilidade, que se apresenta capaz de
redimensionar as desigualdades capitalistas em empreendedorismos e
pletora de direitos que habilita seus portadores. Como meio de inibir
resistências radicais, por fim, reconhece-se nesse itinerário que cada
um é um \emph{sujeito resiliente} adequável e adequado ao \emph{planeta
resiliente} de hoje e do amanhã.

\chapter{Bibliografia}

\versal{ACOT}, Pascal. \emph{História da Ecologia}. Tradução de Carlota Gomes.
Rio de Janeiro: Campus, 1990.

\versal{ACSELRAD}, Henri et al. \emph{O que é Justiça Ambiental}. Rio de Janeiro:
Garamond, 2009.

\versal{AGAMBEN}, Giorgio. \emph{O que resta de Auschwitz}. Tradução de Jeanne
Marie Gagnebin. São Paulo: Boitempo, 2008.

\_\_\_\_\_\_\_\_\_\_ \emph{O que é o contemporâneo? E outros ensaios}.
Tradução de Vinicius Nicastro Honesko. Chapecó: Argos, 2009.

\_\_\_\_\_\_\_\_\_\_ \emph{O reino e a glória: uma genealogia teológica
da economia e do governo: homo saccer \versal{II}, 2.} Tradução de Selvino J.
Assmann. São Paulo: Boitenmpo, 2011.

\_\_\_\_\_\_\_\_\_\_. \emph{Profanações}. Tradução de Selvina J. Assman.
São Paulo: Boitempo, 2007.

\versal{ALEKSIÉVITCH}, Svetlana. \emph{Vozes de Tchernóbil}. \emph{A história
oral do desastre nuclear}. Tradução de Sonia Branco. São Paulo:
Companhia das Letras, 2016.

\versal{ALKER} Jr., Hayward e \versal{HAAS}, Peter. ``The rise of global Ecopolitics''.
In: \versal{CHOUCRI}, Nazli (Ed.). \emph{Global Accord: Environmental Challenges
and International Responses}. Cambridge/Massachusetts: The \versal{MIT} Press,
1993.

\versal{ALTAVILA}, Jaime de. \emph{Origem dos direitos dos povos}. São Paulo:
Ícone, 2004.

\versal{ALVES}, Maria Helena Moreira; \versal{EVANSON}, Philip. \emph{Vivendo no fogo
cruzado: moradores da favela, traficantes de droga e violência policial
no Rio de Janeiro}. Tradução de Fernando Moura. São Paulo: Editora
\versal{UNESP}, 2013.

\versal{ALVIM}, Francisco. \emph{Poemas: {[}1968-2000{]}}. São Paulo/Rio de
Janeiro: Cosac Naify/7 Letras, 2004.

\versal{AMORIM}, João Alberto A. \emph{A \versal{ONU} e o Meio Ambiente: Direitos Humanos,
Mudanças Climáticas e Segurança Internacional no Século \versal{XXI}.} São Paulo:
Atlas, 2015.

\versal{ANDERSON}, Benedict. \emph{Imagined Communities}. London/New York: Verso,
1991.

\versal{ANTHONY}, Edwyn James (Org.). \emph{The invunerable child}. Washington
D.C.: Library of Congress, 1987.

\versal{ANTONIO} Neto, Adib. \emph{As influências dos tratados internacionais
ambientais celebrados pelo Brasil no ordenamento jurídico brasileiro.}
In: \emph{Jusbrasil.} Brasil: Jusbrasil, 2009. Disponível em:
\emph{http://www.lfg.jusbrasil.com.br/noticias/971596/as-influencias-dos-tratados-internacionais-ambientais-celebrados-pelo-brasil-no-ordenamento-juridico-brasileiro-adib-antonio-neto}.

\versal{APPLEBAUM}, Anne. \emph{Gulag. Uma história dos campos de prisioneiros
soviéticos}. Tradução de Maria Vilela e Ibraíma Dafonte. Rio de Janeiro:
Ediouro, 2009.

\versal{ARRHENIUS}, S. \emph{On the influence of Carbonic Acid upon the
Temperature of the Ground}. In: \emph{Philosophical Magazine and Journal
of Science}. London, v. 41, s. 5, 1986, pp. 237-276. Disponível em:
\emph{http://www.rsc.org/images/Arrhenius1896\_tcm18-173546.pdf}.

\versal{ASH}, Timothy Garton. \emph{Os fatos são subversivos. Escritos políticos
de uma década sem nome.} Traducão de Pedro Maia Soares. São Paulo:
Companhia das Letras, 2011.

\versal{AUGUSTO}, Acácio. "Terrorismo anarquista e luta contra a prisão". In
Edson Passetti \& Salete Oliveira (orgs). \emph{Terrorismos}. São Paulo:
\versal{EDUC}, 2006, pp. 139-148.

\_\_\_\_\_\_\_\_\_\_.Municipalismo libertário, ecologia e
resistências''. In: \emph{Revista Ecopolítica}. São Paulo: \versal{PUCSP}, v. 2,
2012, pp. 01-35. Disponível em:
\emph{http://revistas.pucsp.br/index.php/ecopolitica/article/view/9076/6684}.

\_\_\_\_\_\_\_\_\_\_. ``Penalizações a céu aberto, uma política
planetária''. In: \emph{Revista Ecopolítica}. São Paulo: \versal{PUCSP}, v. 4,
2012a, pp. 84-101. Disponível em:
\emph{http://revistas.pucsp.br/index.php/ecopolitica/article/view/13062}.

\_\_\_\_\_\_\_\_\_\_. \emph{Política e antipolítica: anarquia
contemporânea, revolta e cultura libertária}. Tese de Doutorado. São
Paulo: \versal{PUC}-\versal{SP}, 2013.

\versal{ASSANGE}, Julian et al. \emph{Cypherpunks: liberdade e o futuro da
internet}. Tradução de Cristina Yamagami. São Paulo: Boitempo, 2013.

\versal{AZEVEDO}, Reinaldo. ``Há quem queira a Bomba brasileira e esse debate tem
de sair do armário''\emph{.} In: \emph{Revista Veja}. São Paulo, 2010.
Disponível em:
\emph{http://veja.abril.com.br/blog/reinaldo/geral/ha-quem-queira-a-bomba-brasileira-e-esse-debatem-tem-de-sair-do-armario/}.

\versal{BALKO}, Radley. \emph{Rise of the Warrior Cop: the militarization of
America's Police Forces}. New York: PublicAffairs, 2013.

\versal{BARREIRA}, Marcos e \versal{BOTELHO}, Maurilio Lima. ``O Exército nas ruas: da
Operação Rio à ocupação do Complexo do Alemão -- notas para um
reconstituição da exceção urbana''. In: \versal{BRITO}, Felipe e \versal{OLIVEIRA}, Pedro
Rocha de (Orgs.). \emph{Até o último homem.} São Paulo: Boitempo, 2013,
pp. 115-128.

\versal{BARRETT}, Daniel. \emph{Los sediciosos despertares de la anarquia}.
Buenos Aires: Los Sediciosos despertares de La anarquia, 2011.

\versal{BATH}, C. ``The New Context of International Relations: Global
Ecopolitics''. In: \emph{Political Research Quaterly}. Utah, v. 33, n.
1, 1980, pp. 132-133. Disponível em:
\emph{http://prq.sagepub.com/content/33/1/132.extract}.

\versal{BAYLEY}, David e \versal{PERITO}, Robert. \emph{The Police at war: fighting
insurgency, terrorism, and violent crime}. Boulder/London: Lynne Rienner
Publishers, 2010.

\versal{BECKETT}, Samuel. \emph{Primeiro amor}. São Paulo: Cosac Naify, 2004.

\versal{BENTHAM}, Jeremy et al. \emph{O Panóptico}. Tradução de Guacira Lopes
Louro; M. D. Magno; Tomaz Tadeu. Belo Horizonte: Autêntica, 2008.

\versal{BERLIN}, Isaiah. \emph{As raízes do Romantismo.} Tradução de Isa Mara
Lando. São Paulo: Três Estrelas, 2015.

\versal{BEUYS}, Joseph. ``Conclamação à Alternativa''. In: \versal{FARKAS}, Solange e
\versal{D'AVOSSA}, Antonio (Curadores). \emph{A revolução somos nós. 2010-2011}.
São Paulo: \versal{SESC}, 2010.

\versal{BIGO}, Didier. ``Guerras, conflitos, o internacional e o território''.
In: \versal{MILANI}, Carlos (org.). \emph{Relações Internacionais}: perspectivas
francesas. Salvador: Editora \versal{UFBA}, 2010, pp. 333-348.

\versal{BIN} \versal{LADEN}, Osama. ``Terror for Terror (October 2001)''. In: \versal{LAWRENCE},
Bruce (org.). \emph{Messages to the World}: \emph{the statements of
Osama bin Laden}. London/New York: Verso, 2005.

\versal{BINET}, Laurent. \emph{\versal{HH}h\versal{H}}. Tradução de Paulo Neves. São Paulo:
Companhia das Letras, 2012.

\versal{BITAR}, Sergio. \emph{Isla 10}. Santiago: Pehuén Editores, 1987.
Disponível em:
\emph{https://historiadetodos.files.wordpress.com/2013/02/Dawson-isla-10.pdf}.

\versal{BOFF}, Leonardo. ``Um ethos para salvar a Terra''. In: \versal{CAMARGO}, Aspásia
et al. (Orgs.). \emph{Meio Ambiente Brasil: avanços e obstáculos
pós-Rio-92}. São Paulo: Estação Liberdade e Instituto Socioambiental;
Rio de Janeiro: Fundação Getúlio Vargas, 2004.

\versal{BOLA}ÑO, Roberto. \emph{Os detetives selvagens}. Tradução de Eduardo
Brandão. São Paulo: Companhia das Letras, 2006.

\versal{BOURDIEU}, Pierre. ``Dois imperialismos do universal''. Tradução de
Rachel Gutiérrez. In: Daniel Lins e Loïc Wacquant. \emph{Repensar os
Estados Unidos.} Campinas, Papirus, 2003, pp. 13-19.

\_\_\_\_\_\_\_\_\_\_. \emph{Sobre o Estado}. Tradução de Rosa Freire
D'Aguiar. São Paulo: Companhia das Letras, 2014.

\versal{BRANCOLI}, Fernando. \emph{Primavera Árabe: praças, ruas e revoltas}. São
Paulo: Desatino, 2013.

\versal{BRAND}ÃO, Elsa; \versal{MELO}, Joana; \versal{AMORIM}, Ricardo. \emph{Farmacologia da
Imunossupressão}. Dissertação de Mestrado. Porto: Departamento de
Bioquímica e Biologia Celular da Universidade do Porto, 2012.

\versal{BRAY}, Mark (2013). \emph{Translating anarchism. The anarchism of Occupy
Wall Street}. Winchester/Washington: Zero Books.

\versal{BRENNER}, Hannelore. \emph{As meninas do quarto 28}. \emph{Amizade,
esperança e sobrevivência em Theresienstadt}. Tradução de Renate Müller.
São Paulo: Leya, 2014.

\versal{BRINGEL}, Breno e \versal{MUÑOZ}, Enara. ``Dez anos de Seattle, o movimento
antiglobalização e a ação coletiva transnacional''. In: \emph{Ciências
Sociais Unisinos}. São Leopoldo: Unisinos, v. 46, n. 1, 2010, pp. 28-36.
Disponível em:
\emph{http://revistas.unisinos.br/index.php/ciencias\_sociais/article/viewFile/168/38}.

\versal{BRÜGGEMEIER}, F. J. et al. (Eds.). \emph{How Green were the Nazis?
Nature, Environment and Third Reich}. Athens/Ohio: Ohio Univerity Press,
2005.

\versal{BURSZTYN}, Marcel e \versal{PERSEGONA}, Marcelo. \emph{A grande transformação
ambiental: uma cronologia da dialética homem-natureza.} Rio de Janeiro:
Garamond, 2008.

\versal{BUSH}, George H. W. (1991). Address Before a Joint Session of the
Congress on the State of the Union, 29 de janeiro de 1991. Disponível em
\emph{http://bushlibrary.tamu.edu/research/public\_papers.php?id=2656\&year=1991\&month=1}

\versal{BUZAN}, Barry. \emph{People, states and fear: an agenda for international
security studies in the post-cold war era}. Colchester/\versal{UK}: \versal{ECPR}, 2007.

\versal{BUZAN}, Barry e \versal{HANSEN}, Lene. \emph{A evolução dos Estudos de Segurança
Internacional}. São Paulo: Editora Unesp, 2012.

\versal{BUZAN}, Barry e \versal{WÆVER}, Ole. \emph{Regions and Powers: the structure of
International Security}. Cambridge: Cambridge University Press, 2003.

\versal{BUZAN}, Barry; \versal{WÆVER}, Ole; \versal{DE} \versal{WILDE}, Jaap. \emph{Security: a new
framework for analysis}. Bolder e Lodon: Lynne Rienner Publisher, 1998.

\versal{CABANES}, Robert; \versal{GEORGES}, Isabel; \versal{RIZEK}, Cibele S. e \versal{TELLES}, Vera da
Silva (Orgs.). \emph{Saídas de emergência: ganhar/perder a vida na
periferia de São Paulo}. São Paulo: Boitempo, 2011.

\versal{CAMARGO}, Aspásia, et al. (Orgs.). \emph{Meio Ambiente Brasil: avanços e
obstáculos pós-Rio-92}. São Paulo: Estação Liberdade e Instituto
Socioambiental; Rio de Janeiro: Fundação Getúlio Vargas, 2004.

\versal{COMISSÃO MUNDIAL SOBRE MEIO AMBIENTE E DESENVOLVIMENTO}. \emph{Nosso
futuro comum.} Rio de Janeiro: \versal{FGV}, 1988.

\versal{CAMPBELL}, Elaine. ``A vida emocional do poder governamental''. In:
\emph{Revista Ecopolítica}. São Paulo: \versal{PUCSP}, v. 1, 2011, pp. 35-53.
Disponível em:
\emph{http://revistas.pucsp.br/index.php/ecopolitica/article/view/7656}.

\versal{CANGUILHEM}, Georges. \emph{O normal e o patológico}. Tradução de Maria
Thereza R. C. Barrocas. Rio de Janeiro: Forense Universitária, 1995.

\versal{CARLETTI}, Ana. \emph{O internacionalismo vaticano e a nova ordem
mundial. A diplomacia pontifícia da Guerra Fria aos nossos dias}.
Brasília: Fundação Alexandre Gusmão, 2012.

\versal{CARNEIRO}, Beatriz Scigliano. Élisée Reclus: torrente libertária''. In:
\emph{Revista Ecopolítica}. São Paulo: \versal{PUCSP}, v. 1, 2011, pp. 104-113.
Disponível em:
\emph{http://revistas.pucsp.br/index.php/ecopolitica/article/view/7658/5606}

\_\_\_\_\_\_\_\_\_\_ ``A Construção do \emph{Dispositivo meio
ambiente}''. In: \emph{Revista Ecopolitica}. São Paulo: \versal{PUCSP}, v. 4,
2012, pp. 2-15. Disponível em:
\emph{http://revistas.pucsp.br/index.php/ecopolitica/article/view/13057/9560}.

\_\_\_\_\_\_\_\_\_\_. ``Ecopolítica e a Igreja Católica no terceiro
milênio: a conversão ecológica''. In: \emph{Revista Ecopolítica}. São
Paulo: \versal{PUCSP}, v. 12, 2015, pp. 13-69. Disponível em:
\emph{http://revistas.pucsp.br/index.php/ecopolitica/article/view/24725}.

\versal{CARSON}, Rachel. \emph{A Primavera Silenciosa.} Tradução de Raul de
Polillo. São Paulo: Melhoramentos, 1969. Disponível em:
\emph{https://biowit.files.wordpress.com/2010/11/primavera\_silenciosa\_-\_rachel\_carson\_-\_pt.pdf}.

\versal{CARVALHO}, José Carlos. ``A vocação democrática na gestão ambiental
brasileira e o papel do executivo''. In \versal{TRIGUERO}, Andre (coord)
\emph{Meio ambiente no século \versal{XXI}}. Campinas: Armazém do Ipê, 2008, pp.
258-273.

\versal{CATLIN}, G. \emph{Letters and notes on the manners, customs and condition
of the North American Indians.} London: Tosswill \& Miers, v. I, 1841.
Disponível em:
\emph{https://play.google.com/store/books/details?id=MA4TAAAAYAAJ\&rdid=book-MA4TAAAAYAAJ\&rdot=1}.

\versal{CEDI} et al. \emph{De Angra a Aramar: os militares a caminho da bomba}.
Rio de Janeiro: \versal{CEDI}, s/d.

\versal{CHAR}, Rene. ``Argumento \emph{- Fúria e mistério}''\emph{.} In: \emph{O
nu perdido e outros poemas.} Tradução de Augusto Contador Borges. São
Paulo: Iluminuras, 1995.

\versal{CHOUCRI}, Nazli (Ed.). \emph{Global Accord: Environmental Challenges and
International Responses}. Cambridge/Massachusetts: The \versal{MIT} Press, 1993.

\versal{CLAUSEWITZ}, Carl von. \emph{On War}. Tradução de Michael Howard e Peter
Paret. Princeton: Princeton University Press, 1976.

\versal{CHRISPINIANO}, José. \emph{A guerrilha surreal}. São Paulo: Conrad,
Com-Arte, 2002.

\versal{CHRISTIE}, Nils. \emph{A indústria do controle do crime -- a caminho dos
\versal{GULAG}s em estilo ocidental}. Tradução de Luis Leiria. Rio de Janeiro:
Editora Forense, 1998.

\versal{COIMBRA}, Cecília. \emph{Operação Rio}: \emph{O mito das classes
perigosas: um estudo sobre a violência urbana, a mídia impressa e os
discursos de segurança pública}. Niterói: Oficina do Autor/Intertexto,
2001.

\versal{COIMBRA}, Renata Maria e \versal{MORAIS}, Normanda Araujo de (Orgs.). \emph{A
resiliência em questão: perspectivas teóricas, pesquisa e intervenção}.
Porto Alegre: Artmed, 2015.

\versal{CONDE} Heliana. "Anarqueologizando Foucault". In \emph{verve}. São Paulo:
Nu-Sol, n. 28, 2015, pp. 91-123.

\versal{CONSPIRACIÓN DE CÉLULA DEL FUEGO} (\versal{CCF}). \emph{La vigência de la negación
y la sóbria sinceridad de nuestras intenciones}. Madri:
Nuestronuevoscomplots, 2011.

\versal{CORREA}, Paulo Gustavo Pellegrino. ``\versal{MINUSTAH} e Diplomacia Solidária:
criação de novo paradigma nas Operações de Paz?''. In: \versal{MATIJASCIC},
Vanessa Braga (org.). \emph{Operações de Manutenção de Paz das Nações
Unidas}: reflexões e debates. São Paulo: Editora Unesp, 2014, pp.
129-158.

\versal{COSTA}, Ana Alice. ``Gênero, poder e empoderamento das mulheres''. In:
\emph{Núcleo de Estudos Interdisciplinares sobre a Mulher}. Salvador:
\versal{NEIM}/\versal{UFBA}, 2000.

\versal{COURBET}, Gustave. ``Carta aos artistas de Paris''. In: \emph{verve.} São
Paulo: Nu-Sol, n. 15, 2009, pp. 123-124.

\versal{DANOWSKI}, Débora \& \versal{CASTRO}, Eduardo Viveiros de. \emph{Há um mundo por
vir? Ensaio sobre os medos e os fins}. Desterro (Florianópolis): Cultura
e Barbárie: Instituto Socioambiental, 2014.

\versal{DAY}, Richard J. F. \emph{Gramsci is dead: anarchist currents in the
newest social movements}. London: Pluto Press/Toronto: Between the
Lines, 2005.

\versal{DEGENSZAJN}, Andre. ``Terrorismos e invulnerabilidade''. In: \versal{PASSETTI},
Edson; \versal{OLIVEIRA}, Salete (orgs.). \emph{Terrorismos}. São Paulo: Educ,
2006, pp. 163-175.

\versal{DELEUZE}, Gilles. \emph{Foucault}. Tradução de Claudia Sant'Tanna
Martins. São Paulo: Brasiliense, 1988.

\_\_\_\_\_\_\_\_\_\_. \emph{Conversações}. Tradução de Peter Pal
Pélbart. São Paulo: 34, 1992.

\_\_\_\_\_\_\_\_\_\_. \emph{Sobre o teatro. Um manifesto de menos/ O
esgostado.} Tradução de Fátima Saadi e Ovídio de Abreu. Rio de Janeiro:
Jorge Zahar, 2010.

\versal{DENG}, Fancis; \versal{KIMARO}, Sadikiel; \versal{LYONS}, Terrence; \versal{ROTHCHILD}, Donald;
\versal{ZARTMAN}, William. \emph{Sovereignty as Responsibility}: \emph{conflict
management in Africa}. Washington: Brookings Institute, 1996.

\_\_\_\_\_\_\_\_\_\_. \emph{Curso sobre Foucault \versal{III} La subjetivación}.
Buenos Aires: Cactus, 2015.

\versal{DENEAULT}, Alain. \emph{Gouvernance: le management totalitaire}.
Montréal: Lux Éditeur, 2013.

\versal{DEUTSCH}, Karl (Ed.). \emph{Ecosocial systems and Ecopolitics: a reader
on human and social implications of environmental management in
developing countries.} Paris: \versal{UNESCO}, 1977.

\versal{DEUSEN}, David Van; \versal{MASSOT}, Xaxier (Orgs.) (2010). \emph{The Blac Block
papers}. Breaking Glass Papers.

\versal{DINSTEIN}, Yoram. \emph{Guerra, agressão e legítima defesa}. Tradução de
Mauro Raposo de Mello. Barueri/\versal{SP}: Manole, 2004.

\versal{DOYLE}, Michael. ``Kant, Liberal Legacies, and Foreign Affairs (Part
I)''. \emph{Philosophy and Public Affairs}, v. 12, n. 3, 1983, pp.
205-235.

\versal{DUARTE}, João Paulo Gusmão. \emph{Terrorismo}: caos, controle e
segurança. São Paulo: Desatino, 2014.

\versal{DOODS}, Felix \& \versal{STRAUSS}, Michael. \emph{Only one Earth: the long road
via Riofor sustainable development}. New York: Routledge, 2012.

\versal{DROUIN}, Jean Marc. \emph{L'Ecologie et son histoire: reinventer la
nature}. Paris: Flammarion, 1993.

\versal{DUPUIS-DÉRI}, Francis. \emph{Black Bloc}. Tradução de Guilherme Miranda.
São Paulo: Veneta, 2014.

\versal{DUPUY}, Jean-Pierre. \emph{Introdução à crítica da ecologia política.}
Rio de Janeiro: Civilização Brasileira, 1980,

\versal{ENZENSBERGER}, Hans Magnus. \emph{O curto verão da anarquia}. Tradução de
Márcio Suzuki. São Paulo: Companhia das Letras, 1987.

\_\_\_\_\_\_\_\_\_\_. \emph{El perdedor radical. Ensayo sobre los
hombres del terror}. Barcelona: Editorial Anagrama, 2007.

\versal{EVANS}, Gareth. \emph{Responsibility to Protect: ending mass atrocity
crimes once and for all}. New York: Brookings Institute, 2008.

\versal{FAO}. \emph{`Climate-Smart' Agriculture: Policies, Practices and
Financing for Food Security, Adaptation and Mitigation}. Rome: \versal{FAO},
2010.

\versal{FEITOSA}, Gustavo e \versal{PINHEIRO}, José Augusto. ``Lei do Abate, guerra às
drogas e defesa nacional''. \emph{Revista Brasileira de Política
Internacional}, Brasília, ano 55, n. 1, 2012, pp. 66-92.

\versal{FELICE}, Massimo Di e \versal{MUÑOZ}, Cristobal (Orgs.). \emph{A revolução
invencível: cartas e comunicados}. Tradução de Cláudia Schilling e
Valter Pomar. São Paulo: Boitempo, 1998.

\versal{FERRAZ}, Francielle Bonet e \versal{FERNANDEZ}, Jorge Hernandez. ``Integrinas na
adesão, migração e sinalização celular: associação com patologias e
estudos clínicos''. In: \emph{Revista Científica da \versal{FMC}}. Campos: \versal{UFNF},
v. 9, n. 2, 2014, pp. 25-34.

\versal{FERREIRA}, José Maria Carvalho. ``Da impossibilidade de superar a atual
crise do capitalismo''. In: \emph{verve}. São Paulo: Nu-Sol, n. 21,
2012, pp. 101-131.

\_\_\_\_\_\_\_\_\_\_\_. ``Globalização, \versal{TIC} e Trabalho Virtual''. In:
\emph{Revista Ecopolítica}. São Paulo: \versal{PUCSP}, v. 14, 2016, pp. 2-27.
Disponível em:
\emph{http://revistas.pucsp.br/index.php/ecopolitica/article/view/27850/19648}.

\versal{FERRY}, Luc. \emph{A nova ordem ecológica}. Tradução de Rejane
Janowitzer. Rio de Janeiro: Difel, 2009.

\versal{FIGUEIREDO}, Eurico de Lima. \emph{Os militares e a democracia}. Rio de
Janeiro: Graal, 1980.

\versal{FLECK}, \versal{MP}; \versal{LOUZADA}, S; \versal{XAVIER}, M, \versal{CACHAMOVIVH}. E, Vieira G, et al.
``Desenvolvimento da versão em português do instrumento de avaliação de
qualidade de vida da \versal{OMS} (\versal{WHOQOL} - 100)''. In: \emph{Revista Brasileira
de Psiquiatria}. São Paulo: \versal{ABP}, 21, 1999.

\versal{FOLKE}, C. et al. \emph{Resilience and Sustainable Development: Building
Adaptative Capacity in a World of Transformations}. Estocolmo: Norstedts
Tryckeri, 2002. Disponível em:
\emph{http://era-mx.org/biblio/resilience-sd.pdf}.

\versal{FOUCAULT}, Michel. \emph{Vigiar e punir}. Tradução de Lígia M. Pondé.
Petrópolis: Vozes, 1977.

\_\_\_\_\_\_\_\_\_\_. \emph{História da sexualidade I. A vontade de
saber}. Tradução de Maria Theresa C. Albuquerque e J. A. Guilhon de
Albuquerque. Rio de Janeiro: Graal, 1977a.

\_\_\_\_\_\_\_\_\_\_. \emph{Microfísica do poder}. Tradução e
organização de Roberto Machado. Rio de Janeiro: Edições Graal, 1979.

\_\_\_\_\_\_\_\_\_\_. ``Anti-Édipo: uma introdução à vida
não-fascista''. In: \versal{PELBART}, Peter Pál e \versal{ROLNIK}, Suely (Orgs.).
\emph{Cadernos de Subjetividade}. \emph{Gilles Deleuze.} Tradução de
Fernando José Fagundes Ribeiro. São Paulo: Núcleo de Pesquisa de
Subjetividade. Programa de Estudos Pós-Graduados em Psicologia Clínica
da \versal{PUCSP}, v. 1, 1993, pp. 197-200.

\_\_\_\_\_\_\_\_\_\_. ``O sujeito e o poder''. In: \versal{DREYFUS}, Hubert L. e
\versal{RABINOW}, Paul. (Orgs.). \emph{Michel Foucault -- uma trajetória
filosófica.} Tradução de Vera Porto Carrero. Rio de Janeiro: Forense
Universitária, 1995, pp. 231-249.

\_\_\_\_\_\_\_\_\_\_. \emph{A ordem do discurso.} Tradução de Laura
Fraga de Almeida Sampaio. São Paulo: Edições Loyola, 1996.

\_\_\_\_\_\_\_\_\_\_. \emph{Em defesa da sociedade}. Tradução de Maria
E. Galvão. São Paulo: Martins Fontes, 1999.

\_\_\_\_\_\_\_\_\_\_. ``O que são as Luzes?''. In: \versal{MOTTA}, Manoel B.
(Org.). \emph{Ditos e Escritos}. Tradução de Elisa Monteiro. Rio de
Janeiro: Forense Universitária, v. \versal{II}, 2000, pp. 335-351.

\_\_\_\_\_\_\_\_\_\_. ``Le jeu de Michel Foucault''. In: \emph{Dits et
Écrits \versal{II}.} Paris: Quarto Gallimard, 2001.

\_\_\_\_\_\_\_\_\_\_. \emph{Os anormais}. Tradução de Eduardo Brandão.
São Paulo: Martins Fontes, 2001a.

\_\_\_\_\_\_\_\_\_\_. ```Omnes et singulatim': uma crítica da razão
política''. In: \versal{MOTTA}, Manoel B. (Org.). \emph{Ditos \& Escritos}.
Tradução Manoel Barros da Motta. Rio de Janeiro: Forense Universitária,
v. \versal{IV}, 2003, pp. 355-385.

\_\_\_\_\_\_\_\_\_\_. ``A evolução do indivíduo perigoso na psiquiatria
legal do século \versal{XIX}''. In: \versal{MOTTA}, Manoel B. (Org.). \emph{Ditos e
escritos}. Rio de Janeiro: Forense Universitária, v. V, 2003a, pp. 1-25.

\_\_\_\_\_\_\_\_\_\_. \emph{A hermenêutica do sujeito}. Tradução de
Marcio A. Fonseca e Salma T. Muchail. São Paulo: Martins Fontes, 2004.

\_\_\_\_\_\_\_\_\_\_. (Maurice Florence). ``Foucault''. In: \versal{MOTTA},
Manoel B (Org.). \emph{Ditos e Escritos}. Tradução de Elisa Monteiro e
Inês A. D. Barbosa. Rio de Janeiro: Forense Universitária, v. V, 2004a,
pp. 234-239.

\_\_\_\_\_\_\_\_\_\_. ``É inútil revoltar-se?''. In: \versal{MOTTA}, Manoel B.
(Org.). \emph{Ditos e Escritos.} Tradução de Elisa Monteiro e Inês
Autran D. Barbosa. Rio de Janeiro: Forense Universitária, v. V, 2004b,
pp.77-81.

\_\_\_\_\_\_\_\_\_\_. \emph{Segurança, território e população}. Tradução
de Eduardo Brandão. São Paulo: Martins Fontes, 2008.

\_\_\_\_\_\_\_\_\_\_. \emph{Nascimento da biopolítica}. Tradução de
Eduardo Brandão. São Paulo: Martins Fontes, 2008a.

\_\_\_\_\_\_\_\_\_\_. \emph{O governo de si e dos outros}. Tradução de
Eduardo Brandão. São Paulo: Martins Fontes, 2010.

\_\_\_\_\_\_\_\_\_\_. \emph{A coragem da verdade}. Tradução de Eduardo
Brandão. São Paulo: \versal{WMW}-Martins Fontes, 2011.

\_\_\_\_\_\_\_\_\_\_. ``O saber como crime''. In: \versal{MOTTA}, Manoel B.
(Org.). \emph{Ditos e Escritos.} Tradução de Vera Lucia A. Ribeiro. Rio
de Janeiro: Forense Universitária, v. \versal{VII}, 2011a, pp. 62-69.

\_\_\_\_\_\_\_\_\_\_. \emph{Ditos \& Escritos}. \versal{MOTTA}, Manoel B. (Org.).
Tradução de Lucia Avellar Ribeiro. Rio de Janeiro: Forense
Universitária, v. V, 2012.

\_\_\_\_\_\_\_\_\_\_. ``O nascimento do mundo''. In: \versal{MOTTA}, Manoel B.
(Org.). \emph{Ditos e Escritos}. Tradução de Abner Chiquieri. Rio de
Janeiro: Forense Universitária, v. X, 2014, pp. 51-54.

\_\_\_\_\_\_\_\_\_\_. A sociedade punitiva. Tradução de Ivone C.
Benedetti. São Paulo: \versal{WMF}-Martins Fontes, 2015.

\_\_\_\_\_\_\_\_\_\_. ``O poder e a política de Michel Foucault''. In:
\emph{Revista Ecopolítica}. São Paulo: \versal{PUC}-\versal{SP}, v. 12, 2015a, pp. 93-107.
Disponível em:
\emph{http://revistas.pucsp.br/index.php/ecopolitica/article/view/24625}.

\_\_\_\_\_\_\_\_\_\_. \emph{La société punitive}. Paris:
\versal{EHESS}/Galimard/Seuil, 2013.

\versal{FOX}, Stephen. \emph{The American Conservation Movement: John Muir and
its Legacy}. Madison/Wisconsin: The University of Wisconsin Press, 1981.

\versal{FRANCO}, J. L. de Andrade. ``A primeira Conferência Brasileira de
Proteção à Natureza e a questão da identidade nacional''. In: \versal{HORTA},
Regina. \emph{Varia História}. Belo Horizonte: Departamento de História
da \versal{UFMG}, n. 26, 2002, pp. 77-96.

\versal{FRANCO}, J. L. de Andrade et al. (Orgs.). \emph{História Ambiental:
fronteiras, recursos naturais e conservação da natureza}. Rio de
Janeiro: Garamond, 2012.

\versal{FRANCO}, J. L. de Andrade \& \versal{DRUMMOND}, José Augusto. \emph{Proteção à
natureza e Identidade nacional no Brasil, anos 1920-1940}. Rio de
Janeiro: Fiocruz, 2009.

\versal{FRIEDLÄNDER}, Saul. \emph{A Alemanha nazista e os judeus} 2v. Tradução de
Fany Kon et al. São Paulo: Perspectiva, 2012.

\versal{FRIEDMANN}, John. \emph{Empowerment: uma política de desenvolvimento
alternativo}. Tradução de Carlos Silva Pereira. Oeiras: Celta Editora,
1996.

\versal{FUKUYAMA}, Francis. \emph{Construção de Estados: governo e organização no
século \versal{XXI}}. Tradução de Nivaldo Montigelli Jr. Rio de Janeiro: Rocco,
2005.

\versal{GERMAIN}, Marisa. ``Terror como gobernamentalidad obscena. Del saber
sobre la violência legal em la ditadura''. In: \emph{Revista
Ecopolítica}. São Paulo: \versal{PUCSP}, v. 9, 2014, pp. 47-61. Disponível em:
\emph{http://revistas.pucsp.br/index.php/ecopolitica/article/view/20508/15136}.

\versal{GIBBS}, David. ``How the Srebrenica Massacre redefined the \versal{US} Foreign
Policy''. \emph{Class, Race and Corporate Power}, vol. 03, n. 02, 2015.

\versal{GODWIN}, William. \emph{Investigacion acerca de la justicia politica}.
Tradução de J. Prince. Buenos Aires: Editrial Americalee, 1945. ``

\_\_\_\_\_\_\_\_\_\_. De crimes e punições''. In: \emph{verve.} Tradução
de Maria Abramo Caldeira Brant. São Paulo: Nu-Sol, n. 5, 2005, pp.11-86.

\versal{GOMES}, Maíra. \emph{A ``pacificação'' como prática de ``política
externa'' de (re)produção do self estatal}: reescrevendo o engajamento
do Brasil na Missão de Estabilização de Paz da \versal{ONU} para o Haiti
(\versal{MINUSTAH}). Rio de Janeiro: \versal{IRI}-\versal{PUC}/\versal{RJ}, Tese de Doutorado, 2014, 269p.

\versal{GRAEBER}, David. ``The new anarchists''. In: \emph{New Left Review}.
London: \versal{NLF}, v. 13, 2002. Disponível em:
\emph{http://newleftreview.org/II/13/david-graeber-the-new-anarchists}
\_\_\_\_\_\_\_\_\_\_.. ``O carnaval está em marcha''. In: \emph{Folha de
S. Paulo}. 14/10/2005. Disponível em:
\emph{http://www1.folha.uol.com.br/paywall/signup-colunista.shtml?http://www1.folha.uol.com.br/fsp/mais/fs1408200507.htm}

\_\_\_\_\_\_\_\_\_\_. . ``Concerning the Violent Peace-Police. An Open
Letter to Chris Hedges''. In: \emph{N+1}. New York: 501©3 nonprofit n+1
Foundation, 2012. Disponível em:
\emph{http://nplusonemag.com/concerning-the-violent-peace-police}.

\versal{GREENWALD}, Glenn. Sem lugar para se seconder. Tradução de Fernanda
Abreu. Rio de Janeiro: Sextante, 2014.

\_\_\_\_\_\_\_\_\_\_. ``Revele-se''. In: \emph{verve}. São Paulo:
Nu-Sol, n. 23, 2013.

\versal{GRAHAM}, Stephen. \emph{Cidades Sitiadas}: \emph{novo urbanismo militar}.
Tradução Alyne Azuma. São Paulo: Boitempo, 2016.

\versal{GROS}, Frédéric. \emph{Estados de violência: ensaio sobre o fim da
guerra}. Tradução de José Augusto da Silva. Aparecida/\versal{SP}: Idéias \&
Letras, 2009.

\_\_\_\_\_\_\_\_\_\_. \emph{Le príncipe sécurité}. Paris: Gallimard,
2012.

\versal{GUNDERSON}, L. e \versal{HOLLING}, C.S. (Eds.). \emph{Panarchy: understanding
transformations in human and natural system}s. Washington: Island
Presse, 2002.

\versal{HAMMAN}, Eduarda P. (org.). \emph{Brasil e Haiti}: reflexões sobre os 10
anos da Missão de Paz e o futuro da cooperação após 2016. Rio de
Janeiro: Instituto Igarapé, 2015.

\versal{HAGEN}, Joel, \emph{An entangled Bank: the origins of ecosystem ecology}.
New Jersey: Rutgers, 1992.

\versal{HARDIN}, Garrett. ``The tragedy of commons''. In: \emph{Sience Mag.}
Pennsylvania: Association of American. Geologists and Naturalists, 162,
v. 1968. Disponível em:
\emph{http://eesc.columbia.edu/courses/v1003/lectures/population/Tragedy\%20of\%20the\%20Commons.pdf}.

\versal{HARVEY}, David, Telles, Edson et al. \emph{Occupy. Movimentos de
protestos que tomaram as ruas}. São Paulo: Carta Maior/Boitempo, 2012.

\versal{HAYDTE}, Friedrich von der. \emph{A guerra irregular moderna, em
políticas de defesa e como fenômeno militar}. Tradução de Virginia
Bombeta. Rio de Janeiro: Bibliex, 1990.

\versal{HEHIR}, Aidan e \versal{MURRAY}, Robert (Orgs.). \emph{Libya: the Responsibility
to Protect and the Future of Humanitarian Intervention}. New York:
Palgrave McMillan, 2013.

\versal{HERNÁNDEZ GARCÍA}, Joel (1994). ``Nuevos âmbitos de la acción de las
operaciones de mantenimiento de la paz: implicaciones para el orden
mundial''. In: \versal{PELLICER}, Olga. Las Naciones Unidas Hoy: visión de
México. México: Fondo de Cultura Económica, pp. 150-173.

\versal{HERZ}, Mônica; \versal{HOFFMANN}, Andreia Ribeiro; \versal{TABAK}, Jana. \emph{Organizações
Internacionais}: história e práticas (2ª edição). Rio de Janeiro:
Campus/Elsevier, 2015.

\versal{HOFFMANN}, Florian. ``Mudança de paradigma? Sobre direitos humanos e
segurança humana no mundo pós-11 de Setembro''. In: \versal{HERZ}, Monica e
\versal{AMARAL}, Arthur Bernardes do (Orgs.). \emph{Terrorismo e Relações
Internacionais}. Rio de Janeiro: Editora \versal{PUC}-Rio/Edições Loyola, 2010.

\versal{HOLLING}, Crawford Stanley. ``Resilience and stability of ecological
systems''. In: \emph{Annual Review Ecology and Sistematics}. Vancouver:
University of British Columbia, v. 4, 1973, pp.1-23.

\versal{HOLLOWAY}, John. \emph{Mudar o mundo sem tomar o poder}. Tradução de Emir
Sader. São Paulo: Boitempo, 2003.

\_\_\_\_\_\_\_\_\_\_. \emph{Fissurar o capitalismo}. Tradução de Daniel
Cunha. São Paulo: Publisher Brasil, 2013.

\versal{HOROCHOVSKI}, Rodrigo. ``Empoderamento: definições e aplicações''. In:
\emph{30º encontro anual da \versal{ANPOCS}}. São Paulo: \versal{ANPOCS}, 2006, p. 3.
Disponível em:
\emph{http://www.anpocs.org/portal/index.php?option=com\_docman\&task=doc\_view\&gid=3405\&Itemid=232}.

\versal{HUBAC}, Olivier (org.). \emph{Mercenaires et polices privées}: \emph{la
privatisation de la violence armée}. Paris: Universalis, 2005.

\versal{HUGO MÃE}, Valter. \emph{A máquina de fazer espanhóis}. São Paulo: Cosac
Naify, 2011.

\versal{HUNTER}, Wendy. \emph{Eroding Military Influence in Brazil: politicians
against soldiers}. North Carolina: Chaper Hill, 1997.

\versal{IPCC}. \emph{Climate Change 2013: The Physical Science Basis.
Contribution of Working Group I to the Fifth Assessment Report of the
Intergovernmental Panel on Climate Change}. Cambridge: Cambridge
University Press, 2013. Disponível em:
\emph{http://www.climatechange2013.org/report/full-report/} .

\versal{JESSOP}, Ralph. Coinage of the Term \emph{Environmen}t: A Word Without
Authority and Carlyle's Displacement of the Mechanical Metaphor.
\emph{Literature Compass,} 9(11), 2012, pp. 708-720.

\versal{JOLL}, James. \emph{Anarquistas e anarquismo}. Tradução de Manuel
Vitorino Dias Duarte. Lisboa: Publicações Dom Quixote, 1964.

\versal{JUDENSNAIDER}, Elena et alii (2013). \emph{Vinte centavos}: \emph{a luta
contra o aumento}. São Paulo: Veneta.

\versal{KAFKA}, Franz. \emph{O processo}. Tradução de Modesto Carone. São Paulo:
Companhia das Letras, 1997.

\versal{KALDOR}, Mary. \emph{New and old wars: organized violence in a global
era}. Stanford: Stanford University Press, 2008.

\versal{KASSOW}, Samuel D. \emph{Quem escreverá nossa história? Os arquivos
secretos do Gueto de Varsóvia}. Tradução de Denise Bottman. São Paulo:
Companhia das Letras, 2009.

\versal{KEATING}, Tom. ``The \versal{UN} Security Council on Libya: legitimation or
dissimulation?'' In: \versal{HEHIR}, Aidan; \versal{MURRAY}, Robert (orgs.). \emph{Libya}:
\emph{the Responsibility to Protect and the Future of Humanitarian
Intervention}. New York: Palgrave McMillan, 2013, pp. 162-190.

\versal{KEEGAN}, John. \emph{Uma história da guerra}. Tradução de Pedro Maia
Soares. São Paulo: Companhia das Letras, 2002.

\versal{KELSEN}, Hans. \emph{Teoria pura do direito}. Tradução de João Baptista
Machado. São Paulo: Martins Fontes, 1999.

\versal{KENKEL}, Kai Michael. ``New Missions and Emerging Powers: Brazil, Peace
Operations and \versal{MINUSTAH}''. In: \versal{LEUPRECHT}, Christian; \versal{TROY}, Jodok; \versal{LAST},
David (Orgs.) \emph{Mission Critical: smaller democracies' role in
global stability operations}. Montreal: McGill-Queen's University Press,
2010, pp. 125-147.

\versal{KIRAS}, James. ``Irregular Warfare: terrorism and insurgency''. In:
\versal{BAYLIS}, John; \versal{WIRTZ}, James; \versal{GRAY}, Colin (orgs.). \emph{Strategy in the
contemporary world}. Oxford: Oxford Univeristy Press, 2010, pp. 185-207.

\versal{KOTLIARENCO}, María Angélica; \versal{CÁCERES}, Irma; \versal{FONTECILLA}, Marcelo.
\emph{Estado de arte en resiliencia}. Washington: \versal{OPAS}/\versal{OMS}, 1997.

\versal{KRASNER}, Stephen. \emph{Sovereignty}: \emph{organized hypocrisy}.
Princeton: Princeton University Press, 1999.

\versal{KRUG}, D. Krug e \versal{SIEGENTHALER}, J. ``Changing views about art and the
Earth''. In: \emph{Art \& Ecology: Photoessays: Art and the Earth}.
California: Green Museum, 2006. Disponível em:
\emph{http://greenmuseum.org/c/aen/Earth/Changing/artist.php}.

\versal{KUNDERA}, Milan. \emph{A festa da insignificância}. Tradução de Teresa
Bulhões C. da Fonseca. São Paulo: Companhia das Letras, 2014.

\versal{L'AMICALE D'ORANIENBURG-SACHSENHAUSEN}. \emph{Sachso}. \emph{Au coeur du
sustéme concentrationnaire nazi}. Paris: Plon, les Éditions de Minuit,
1982.

\versal{LACLAU}, Ernesto. \emph{A razão populista}. Tradução de Carlos Eugênio
Marcondes de Moura. São Paulo: Ed. Três Estrelas, 2013.

\versal{LAGO}, André Aranha Corrêa. \emph{O Brasil e as Três Conferências
Ambientais das Nações Unidas.} Brasília: Instituto Rio Branco e Fundação
Alexandre Gusmão, 2007.

\versal{LAPOUJADE}, David. \emph{Deleuze, os movimentos aberrantes}. Tradução de
Laymert Garcia dos Santos. São Paulo: n-1 Edições, 2015.

\versal{LAQUEUR}, Walter. ``Postmodern Terrorism: new rules for an old game''.
\emph{Foreign Affairs}, vol. 75, n. 05, September/October, 1996.
Disponível em

\emph{https://www.foreignaffairs.com/articles/1996-09-01/postmodern-terrorism-new-rules-old-game}

\versal{LATOUR}, Bruno. \emph{Facing Gaia --- Six lectures on the political
theology of nature}. New York: Macaulay, 2013. Disponível em:

\emph{http://macaulay.cuny.edu/eportfolios/wakefield15/files/2015/01/LATOUR-GIFFORD-SIX-LECTURES\_1.pdf}.

\versal{LAZZARATO}, M. \emph{Trabalho imaterial: formas de vida e produção de
subjetividade}. Tradução de Mônica Jesus. Rio de Janeiro: \versal{DP}\&A, 2001.

\_\_\_\_\_\_\_\_\_\_. \emph{O governo das desigualdades. Crítica da
insegurança neoliberal}. Tradução de Renato A. Santos. São Carlos:
\versal{E}d\versal{UFSCAR}, 2011.

\versal{LE} \versal{PRESTE}, Philippe. \emph{Ecopolitica Internacional.} Tradução de Jacob
Gorender. São Paulo: Senas, 2005.

\versal{LEAR}, Linda. \emph{Rachel Carson: witness for nature}. London: Penguin
Books, 1997.

\versal{LEAVITT}, David. \emph{O homem que sabia demais: Alan Turing e a invenção
do computador}. Tradução de Samuel Dirceu. São Paulo: Novo Conceito,
2007.

\versal{LEITE}, José Corrêa. \emph{Fórum Social Mundial. A história de uma
invenção política}. São Paulo: Ed. Fundação Perseu Abramo, 2003.

\versal{LEVI}, Primo. \emph{Assim foi Auschwitz. Testemunhos 1945-1986}. Tradução
de Fedeico Carotti. São Paulo: Companhia das Letras, 2015.

\versal{LIGHTMAN}, Alan. \emph{As descobertas. Os grandes avanços da ciência no
século \versal{XX}}. Tradução de George Schlesinger. São Paulo: Companhia das
Letras, 2015.

\versal{LIMA}, Carlo Alberto. \emph{Os 583 Dias da Pacificação no Alemão}. Rio de
Janeiro: Edição do Autor, 2012.

\versal{LINERA}, Alvaro Garcia. \emph{Potência plebeia. A ação coletiva e
identidades}. Tradução de Igor Ojeda e Mouzar Benedito. São Paulo:
Boitempo, 2011.

\versal{LOWER}, Andy. \emph{As mulheres do nazismo}. Tradução de Angela Lobo. Rio
de Janeiro: Rocco, 2014.

\versal{LUDD}, Ned. \emph{Urgência das Ruas: Black Bloc, Reclaim the Streets e os
dias de ação global}. Tradução de Leo Vinicius. São Paulo: Conrad, 2002.

\versal{MACHADO}, Paulo Affonso Leme. \emph{Direito Ambiental Brasileiro}. São
Paulo: Malheiros, 2014.

\versal{MAHIEU}, Jacques de. \emph{Précis de biopolitique}. Montreal: Ed. Celtic,
1969.

\versal{MALAGUTI} \versal{BATISTA}, Vera. ``O Alemão é muito mais complexo''. In: \versal{MALAGUTI}
\versal{BATISTA}, Vera (Org.). \emph{Paz Armada}. Rio de Janeiro: Revan, 2012,
pp. 55-102.

\versal{MALETTE}, Sébastien. ``Foucault para o próximo século:
ecogovernamentalidade''. In: \emph{Revista Ecopolítica}. São Paulo:
\versal{PUCSP}, v. 1, 2011, pp. 4-25. Disponível em:
\emph{http://revistas.pucsp.br/index.php/ecopolitica/article/view/7654/5602}.

\versal{MARSH}, George P. \emph{Man and Nature, or The Earth as Modified by Human
Action}. New York: Charles Scribner's Sons, 1907. Disponível em:
\emph{https://archive.org/stream/earthasmodifiedb07mars/earthasmodifiedb07mars\_djvu.txt}.

\versal{MARTÍNEZ-ALIER}, Joan. \emph{Ecologismo dos Pobres.} Tradução de Mauricio
Waldman. São Paulo: Contexto, 2012.

\versal{MAUBERT}, Franck. \emph{Conversas com Francis Bacon}. Tradução de André
Telles. Rio de Janeiro: Zahar, 2010.

\versal{M}c\versal{CORMICK}, J. \emph{Rumo ao paraíso: a história do movimento
ambientalista.} Tradução de Marco Antônio Esteves e Renato Aguiar. Rio
de Janeiro: Relume Dumará, 1992.

\versal{MEADOWS}, Donella et al. \emph{The limits to growth. a report for the
Club of Rome's project on the predicament of mankind}. New York:
Universe Books, 1972.

\_\_\_\_\_\_\_\_\_\_. \emph{Limits to Growth: the 30 years up-date.}
White River Junction/Vermont: Chelsea Green Publishing Company, 2004.

\versal{MEIRELLES}, Renato e \versal{ATHAYDE}, Celso. \emph{Um país chamado favela: a
maior pesquisa já feita sobre a favela brasileira}. São Paulo: Editora
Gente, 2014.

\versal{MILLER}, Jacques Alain. ``A máquina panóptica de Jeremy BenthaM''. In:
\versal{BENTHAM}, Jeremy et al. \emph{O Panóptico}. Tradução de Guacira Lopes
Louro; M. D. Magno; Tomaz Tadeu. Belo Horizonte: Autêntica, 2008.

\versal{MONGILLO}, J. e \versal{BOOTH}, B. \emph{Environmental Activists}. Westport,
Connecticut: Greenwood Publishing Group, 2001.

\versal{MORENO}, Giménez. \emph{Mauthauses, o campo de concentração e de
extermínio}. São Paulo: Ediciones Hispanoamericanas, 1975.

\versal{NEGRI}, Antonio e \versal{HARDT}, Michael. \emph{Império}. Tradução de Berilo
Vargas. Rio de Janeiro: Record, 2001.

\_\_\_\_\_\_\_\_\_\_. \emph{Multidão: guerra e democracia na era do
Império}. Tradução de Berilo Vargas. Rio de Janeiro: Record, 2004.

\_\_\_\_\_\_\_\_\_\_.Declaração. Isso não é um manifesto. Tradução de
Carlos Szlak. São Paulo: Editora n-1, 2014.

\versal{NEWMAN}, Saul. \emph{The Politics of Postanarchism}. Edinburgh: Edinburgh
University Press Ltd., 2010.

\_\_\_\_\_\_\_\_\_\_. ``A servidão voluntária revistada: a política
radical e o problema da auto-dominação''. In: \emph{verve}. Tradução de
Anamaria Salles. São Paulo: Nu-Sol, n. 20, 2011, pp. 23-48.

\versal{NOGUEIRA}, Marco Aurélio. \emph{As ruas e a democracia. Ensaios sobre o
Brasil contemporâneo}. Rio de Janeiro: Contraponto, 2013.

\versal{NOVAES}, Washington. \emph{Agenda 21: um novo modelo de civilização}.
Brasília: Ministério do Meio Ambiente, s/d.

\versal{NUENLIST}, Christian. \emph{At the Roots of the European Security System:
Thirty Years since the Helsinki Final Act. H-Soz-u-Kult, H-Net Reviews}.
Disponível em: \emph{http://www.h-net.org/reviews/showrev.php?id=27169}.
2005

\versal{OLIVEIRA}, Luiz Alberto (Org). \emph{Museu do amanhã}. Rio de Janeiro:
Edições de Janeiro, 2015.

\versal{OLIVEIRA}, Salete. ``Política e fissuras sobre crianças e jovens:
psiquiatria, neurociência e educação''. In: \emph{Revista Ecopolítica}.
São Paulo: \versal{PUCSP}, v. 1, 2011, pp. 77-103. Disponível em:
\emph{http://revistas.pucsp.br/index.php/ecopolitica/article/view/7657}.

\_\_\_\_\_\_\_\_\_\_. ``Política e psiquiatrização da ordem a céu
aberto''. In: \emph{Revista Ecopolítica}. São Paulo: \versal{PUCSP}, v. 10, 2014.
Disponível em:
\emph{http://www.pucsp.br/ecopolitica/galeria/galeria\_ed10.html}.

\versal{OPTIZ}, Sven. ``Governo não ilimitado -- o dispositivo de segurança da
governamentalidade não-liberal''. In: \emph{Revista Ecopolítica.} São
Paulo: \versal{PUCSP}, v. 2, 2012, pp. 3-36. Disponível em:
\emph{http://revistas.pucsp.br/index.php/ecopolitica/article/view/9075/6683}.

\versal{ORLANDI}, Luis B. (2013). ``Como sinto as manifestações no Brasil de
2013''. In: \emph{Revista} \emph{Alegrar}, n. 12. Disponível em:
\emph{http://www.alegrar.com.br/revista12/pdf/manifestacoes\_orlandi\_alegrar12.pdf}

\versal{OWEN}, Mark. \emph{Não há dia fácil}. Tradução de Donaldson M. Garschagen
e Berilo Vargas. São Paulo: Paralela, 2012.

\versal{PÁDUA}, J. A., (Org.). \emph{Ecologia e Política no Brasil.} Rio de
Janeiro: Espaço e Tempo: \versal{IUPERJ}, 1987.

\versal{PALEY}, Dawn. \emph{Drug War Capitalism}. Oakland/\versal{EUA}: \versal{AK} Press, 2014.

\versal{PAPA} \versal{FRANCISCO}. \emph{Carta Encíclica ``Laudato Si'''} \emph{sobre o
cuidado da casa comum}. Tradução da Editora do Vaticano. São Paulo:
Paulus/Loyola, 2015.

\versal{PARK}, Robert E. ``A cidade: sugestões para a investigação do
comportamento humano no meio urbano''. In: \versal{VELHO}, Octavio G. (Org.).
\emph{O fenômeno urbano}. Tradução de Sergio Santeiro. Rio de Janeiro:
Zahar, 1967.

\versal{PARTRIDGE}, Eric. \emph{A short etymological dictionary of modern
English}. New York: Routledge, 2006. Disponível em:
\emph{http://www.etymonline.com/index.php?allowed\_in\_frame=0\&search=ambient\&searchmode=none}.

\versal{PASHOUKANIS}, Evgeni Bronislávovich. \emph{Teoria geral do direito e
marxismo}. São Paulo: Acadêmica, 1988.

\versal{PASSETTI}, Edson. \emph{Ética dos amigos}. São Paulo: Imaginário, 2003.

\_\_\_\_\_\_\_\_\_\_. .\emph{Anarquismos e sociedade de controle}. São
Paulo: Cortez, 2003a.

\_\_\_\_\_\_\_\_\_\_. ``Vivendo e revirando-se: heterotopias libertárias
na sociedade de controle''. In: \emph{verve.} São Paulo: Nu-Sol, n. 4,
2003b, pp. 49-50.
\emph{http://www.nu-sol.org/verve/pdf/verve4.pdf}\_\_\_\_\_\_\_\_\_\_.
``Segurança, confiança e tolerância: comandos na sociedade de
controle''. In: \emph{Revista São Paulo Perspectiva}. São Paulo:
Fundação \versal{SEADE}, v. 18, n. 1, 2004.

Disponível em:
\emph{http://www.scielo.br/scielo.php?script=sci\_arttext\&pid=S0102-88392004000100018}.

\_\_\_\_\_\_\_\_\_\_. ``Poder e anarquia. Apontamentos libertários sobre
o atual conservadorismo moderado''. In v\emph{erve}. São Paulo: Nu-Sol,
n. 12, 2007.

\_\_\_\_\_\_\_\_\_\_. \emph{Anarquismo urgente}. Rio de Janeiro/São
Paulo: Achiamé/Centro de Cultura Social, 2007a.

\_\_\_\_\_\_\_\_\_\_. ``Ecopolítica e controle por elites''. In: \versal{PREVE},
Ana Maria e \versal{CORRÊA}, Guilherme (Orgs.). \emph{Ambientes da ecologia:
perspectivas em política e educação}. Santa Maria: Editora \versal{UFSM}, 2007b,
pp. 09-30.

\_\_\_\_\_\_\_\_\_\_. ``Terrorismos'' In: \emph{Anarquismo urgente}. Rio
de Janeiro/São Paulo: Achiamé/Centro de Cultura Social, 2007c, pp.
93-94.

\_\_\_\_\_\_\_\_\_\_. ``Foucault-antifascista, São Francisco de
Sales-guia e atitudes de \emph{parresiasta}''. In: \versal{RAGO}, Margareth e
\versal{VEIGA}-\versal{NETO}, Alfredo (Orgs.). \emph{Para uma vida não-fascista}. Belo
Horizonte: Autêntica Editora, 2009, pp. 117-134.

\_\_\_\_\_\_\_\_\_\_. ``Fascismo, pequenos fascismos, ou como designar
isso que vivemos na sociedade de controle?''. In: \versal{ABRAMOVAY}, Pedro
Vieira e \versal{MALAGUTI} \versal{BATISTA}, Vera (Orgs.). \emph{Depois do grande
encarceramento}. Rio de Janeiro: Revan/\versal{ICC}, 2010, pp. 273-292.

\_\_\_\_\_\_\_\_\_\_. ``Ecopolítica: proveniências e emergências''. In:
\versal{CASTELO} \versal{BRANCO}, Guilherme e \versal{VEIGA}-\versal{NETO}, Alfredo (Orgs.). \emph{Foucault,
filosofia \& política}. Belo Horizonte: Autêntica, 2011, pp. 127-141.

\_\_\_\_\_\_\_\_\_\_. "Foucault e a transformação". In Lúcia Bógus;
Simone Wolff; Vera Chaia (Orgs.). \emph{Pensamento e teoria nas Ciências
Sociais - Referências clássicas e contemporâneas}. São Paulo:
\versal{EDUC}/\versal{CAPES}, 2011a, pp. 205-220.

\_\_\_\_\_\_\_\_\_\_. ``Fluxos libertários e segurança''. In:
\emph{verve}. São Paulo: Nu-Sol, n. 20, 2011b, pp. 49-78.

\_\_\_\_\_\_\_\_\_\_. ``Governamentalidade e violência''. In:
\emph{Currículo sem fronteiras}. São Paulo, v. 11, n. 1, 2011c, pp.
42-53. Disponível em:
\emph{http://www.curriculosemfronteiras.org/vol11iss1articles/passetti.pdf}.

\_\_\_\_\_\_\_\_\_\_ ``Transformações da biopolítica e emergência da
biopolítica''. In: \emph{Revista Ecopolítica}. São Paulo: \versal{PUCSP}, v. 5,
2013, pp. 2-37. Disponível em:
\emph{http://revistas.pucsp.br/index.php/ecopolitica/article/view/15120/11292}

\_\_\_\_\_\_\_\_\_\_. ``O governo das condutas e das contracondutas do
terror''. In: \versal{CASTELO} \versal{BRANCO}, Guilherme (Org.). \emph{Terrorismo de
Estado}. Belo Horizonte: Autêntica, 2013a.

\_\_\_\_\_\_\_\_\_\_. ``Sobre a \emph{pequena} teoria anarquista''. In:
v\emph{erve}. São Paulo: Nu-Sol, n. 24, 2013b, pp. 211-219.

\_\_\_\_\_\_\_\_\_\_. ``Jornadas de junho: o insuportável''. In:
\emph{Revista Ecopolítica}. São Paulo: \versal{PUCSP}, v. 6, 2013c. Disponível
em:
\emph{http://www.pucsp.br/ecopolitica/galeria/galeria\_ed6.html}.

\versal{PASSETTI}, Edson e \versal{AUGUSTO}, Acácio. \emph{Anarquismos e educação}. Belo
Horizonte: Autêntica, 2008.

\_\_\_\_\_\_\_\_\_\_. ''O drama da \emph{multidão} e os trágicos
\emph{black bloc}: a busca do constituinte como destino e a \emph{ação
direta}''. In: \emph{Revista Ecopolítica}. São Paulo: \versal{PUCSP}, v. 9, 2014.
Disponível em:
\emph{http://www.pucsp.br/ecopolitica/galeria/galeria\_ed9.html} .

\versal{PASSETTI}, Edson e \versal{OLIVEIRA}, Salete (Orgs.). \emph{Terrorismos}. São
Paulo: Educ. 2006.

\versal{PÉLBART}, Peter Pál (2013). \emph{O avesso do niilismo. Cartografia do
esgotamento}. São Paulo: n-1 Edições.

\versal{PATTISON} James. ``The ethics of ``responsibility while protecting'':
Brazil, the Responsibility to Protect, and the Restrictive approach to
humanitarian interventions''. In: \versal{KENKEL}, Kai M.; \versal{CUNLIFFE}, Philip
(orgs.). \emph{Brazil as Rising Power}: intervention norms and the
contestation of global order. London/New York: Routledge, 2016, pp.
195-215.

\versal{PERROT}, Michelle. ``O inspetor Bentham''. In: \versal{BENTHAM}, Jeremy et al.
\emph{O Panóptico}. Tradução de Guacira Lopes Louro; M. D. Magno; Tomaz
Tadeu. Belo Horizonte: Autêntica, 2008.

\versal{PESCHEL}, Lisa A. \emph{The prosthetic life. Theatrical performance,
survivor testimony and the Therezin Guetto 194101963}. Minessota:
University of Minessota, 2009.

\versal{PETTMAN}, Ralph. ``Human security as global security: reconceptualising
strategic studies''. In: \emph{Cambridge Review of International
Affairs}. Cambridge: Cambridge University Press, v. 18, n. 1, 2005, pp.
137-150.

\versal{PFETSCH}, Frank. ``Why was the 20\textsuperscript{th} century warlike?''
Tradução de Estevão C.R. Martins In: \versal{MARTINS}, Estevão Chaves de Rezende
(org.). \emph{Relações Internacionais: visões do Brasil e da América
Latina}. Brasília: \versal{FUNAG}/\versal{IBRI}, 2003.

\versal{PIGLIA}, Ricardo. \emph{Respiração artificial}. Tradução de Heloisa Jahn.
São Paulo: Companhia das Letras, 2010.

\versal{PIRAGES}, Dennis Clark (Ed.). \emph{The sustainable society: implications
for limited growth}. New York: Praeger Publishers, 1977.

\versal{PIRAGES}, Dennis Clark. \emph{The New Context of International Relations:
Global Ecopolitics.} North Scituate, Massachusetts: Duxbury Press, 1978.

\versal{PRESS}, Daryl. The Mith of Air Power in the Persian Gulf War and the
Future of Warfare. \emph{International Security}, vol. 26, n. 2, 2001,
pp. 5--44.

\versal{PRÉVERT}, Jacques. \emph{Contos para crianças impossíveis}. Tradução de
Alexandre Barbosa de Souza. São Paulo: Cosac Naify, 2007.

\versal{POWER}, Samantha. \emph{Genocídio. A retórica americana em questão.}
Tradução de Laura T. Motta. São Paulo: Companhia das Letras, 2004.

\versal{PROUDHON}, Pierre-Joseph. \emph{Sistema das contradições econômicas ou
Filosofia da miséria}. Tradução de José Carlos Morel. São Paulo: Ícone,
tomo I, 2003.

\_\_\_\_\_\_\_\_\_\_. ``A Guerra e a paz''. In: \emph{verve}. Tradução
de Martha Gambini. São Paulo: Nu-Sol, n. 19, 2011, pp. 23-71.

\versal{RADKAU}, J., \emph{Nature and Power: a global history of Environment.}
Cambridge: Cambridge University Press, 2008.

\versal{RAYNAUT}, Claude et al. ``Pesquisa e formação na área de meio ambiente e
desenvolvimento: novos quadros de pensamento, novas forma de
avaliação''. In: \emph{Desenvolvimento e Meio Ambiente}. Curitiba: \versal{UFPR},
n. 1, 2000, pp. 71-81. Disponível em:
\emph{http://ojs.c3sl.ufpr.br/ojs2/index.php/made/article/viewFile/3058/2449}.

\versal{RECLUS}, Élisée. \emph{Du sentiment de la nature dans les societés
modernes}. Québec: Université Laval de Québec, 1866. Disponível em:
\emph{http://classiques.uqac.ca/classiques/reclus\_elisee/sentiment\_nature\_soc\_modernes/sentiment\_nature\_soc\_mod.pdf}.

\versal{RESENDE}, Paulo-Edgar A. e \versal{PASSETTI}, Edson. \emph{Proudhon. Política}.
Tradução de Célia Gambini e Eunice O. Setti. São Paulo: Ática, 1986.

\versal{RODRIGUES}, Thiago. Narcoterrorismo e o warfare state. In: \versal{PASSETTI},
Edson \& \versal{OLIVEIRA}, Salete (eds.). \emph{Terrorismos}. São Paulo: Educ,
2006, pp. 149-161.

\_\_\_\_\_\_\_\_\_\_\_.\emph{Guerra e política nas relações
internacionais}. São Paulo: Educ, 2010.

\_\_\_\_\_\_\_\_\_\_\_. ``As guerras do fim do mundo''. In:
\emph{Revista Ecopolítica}. São Paulo: \versal{PUCSP}, v. 1, 2011, pp. 114-124.
Disponível em:
\emph{http://revistas.pucsp.br/index.php/ecopolitica/article/view/7659}.

\_\_\_\_\_\_\_\_\_\_\_. ``Narcotráfico e militarização nas Américas:
vício de guerra''. In: \emph{Contexto Internacional}. Rio de Janeiro:
Instituto de Relações Internacionais da Pontifícia Universidade Católica
do Rio de Janeiro, v. 34, n. 1, 2012, pp. 9-41.

\_\_\_\_\_\_\_\_\_\_. ``Segurança planetária: entre o climático e
humano''. In: \emph{Revista Ecopolítica}. São Paulo: \versal{PUCSP}, v. 3, 2012a.
Disponível em:
\emph{http://revistas.pucsp.br/index.php/ecopolitica/article/view/11385}.

\_\_\_\_\_\_\_\_\_\_\_. ``Ecopolítica e segurança: a emergência do
\emph{dispositivo diplomático-policial}''. In: \emph{Revista
Ecopolítica}. São Paulo: \versal{PUCSP}, v. 05, 2013, pp. 117-158. Disponível em:
\emph{http://revistas.pucsp.br/index.php/ecopolitica/article/view/15217}.

\_\_\_\_\_\_\_\_\_\_\_. ``Agonismo y genealogía: hacia una analítica de
las Relaciones Internacionales''. In: \emph{Relaciones Internacionales}.
Madrid: Universidad Autonoma de Madrid, v. 24, 2013a, pp. 89-107.

\_\_\_\_\_\_\_\_\_\_\_. ``Guerra e terror''. In: \versal{CASTELO} \versal{BRANCO},
Guilherme (Org.). \emph{Terrorismo de Estado}. Belo Horizonte:
Autêntica, 2013b, pp. 203-219.

\_\_\_\_\_\_\_\_\_\_\_. ``Estados Unidos, América Latina e o combate ao
narcotráfico''. In: \versal{TOSTES}, Ana Paula; \versal{RESENDE}, Erica; \versal{TEIXEIRA},
Tatiana. (Orgs.). \emph{Estudos Americanos em Perspectiva: Relações
Internacionais, Política Externa e Ideologias Políticas}. Curitiba:
Appris, 2013c, pp. 119-136.

\_\_\_\_\_\_\_\_\_\_. ``Governar tudo, a todos e a si mesmo''. In:
\emph{Revista Ecopolítica}. São Paulo: \versal{PUCSP}, v. 9, 2014, pp. 62-70.
Disponível em:
\emph{http://revistas.pucsp.br/index.php/ecopolitica/article/view/20510/15137}.

\_\_\_\_\_\_\_\_\_\_. ``Drug-trafficking and Security in Contemporary
Brazil''. In: \versal{RYAN}, Gregory (org.). \emph{World Politics of Security}.
Rio de Janeiro: \versal{CEBRI}/\versal{KAS}, 2015, pp. 234-250.

\versal{RODRIGUES}, Thiago. \emph{Política e drogas nas Américas}: uma genealogia
do narcotráfico. São Paulo: Desatino, 2016.

\versal{RODRIGUES}, Thiago e \versal{AUGUSTO}, Acácio. ``Política, participação e
resistências na sociedade de controle: entre indignados e a
antipolítica''. In: \emph{Pensamiento Propio}. Buenos Aires:
\versal{CRIES}/\versal{CEGRE}, año 19, 2014, pp. 227-250. Disponível em:
\emph{http://www.cries.org/wp-content/uploads/2015/03/014-Thiago.pdf}.

\versal{SACHS}, Ignacy. \emph{Ecodesenvolvimento: crescer sem destruir}. Tradução
de Eneida Araújo. São Paulo: Vértice, 1986.

\versal{SAINT}-\versal{PIERRE}, Héctor. ``Breve Reflexión sobre el Empleo de las Fuerzas
Armadas''. \emph{Voces en el Fénix}, n. 48, 2015.

Disponível em:
\emph{http://www.vocesenelfenix.com/content/breve-reflexi\%C3\%B3n-sobre-el-empleo-de-las-fuerzas-armadas}

\versal{SANDS}, Phillipe. \emph{Principles of International Environment Law}.
Cambridge: Cambridge University Press, 2003.

\versal{SCAHILL}, Jeremy. \emph{Blackwater}. Tradução de Claudio Carina e Ivan
Weisz Kuck. São Paulo: Companhia das Letras, 2008.

\versal{SCHEINVAR}, Estela. ``Uma leitura `no fio da navalha'''. In:
\emph{Revista Ecopolítica}. São Paulo: \versal{PUCSP}, v. 8, 2014, pp.62-71.
Disponível em:
\emph{http://revistas.pucsp.br/index.php/ecopolitica/article/view/19464/14424}.

\versal{SCHOLL}, Inge. \emph{A rosa branca.} Tradução de Ana Carolina Scfäfer e
outros. São Paulo: Editora 34, 2013.

\versal{SEBALD}, W. G. \emph{Austerlitz}. Tradução de José Marcos Macedo. São
Paulo: Companhia das Letras, 2008.

\versal{SEIDL}, Eliane Maria Fleury e \versal{ZANNON}, Célia Maria Lana da Costa.
``Qualidade de vida e saúde: aspectos conceituais e metodológicos''. In:
\emph{Cadernos de Saúde Pública}. Rio de Janeiro: Fundação Oswaldo Cruz,
2004, pp. 580-588.

\versal{SEM}-\versal{SANDBERG}, Steve. \emph{Os destituídos de Lódz}. Tradução de Jaime
Bernardes. São Paulo: Companhia das Letras, 2012.

\versal{SEN}, Amartya. \emph{Desenvolvimento como liberdade}. Tradução de Laura
Teixeira Motta. São Paulo: Companhia das Letras, 2008.

\versal{SEN}, Gita. ``Empowerment as an Approach to Poverty''. In: \emph{Working
Paper Series}. Bangalore: Indian Institute of Management, n. 97.07,1997.

\versal{SERRANO}, Mónica. ``The Responsibility to Protect and its Critics:
explaining the consensus''. \emph{Global Responsibility to Protect}, n.
03, 2011, pp. 01-13.

\versal{SERRES}, Michel. ``Retour au contrat naturel''. In: \versal{KROKER}, Marilouise e
\versal{KROKER} Arthur (Eds.). \emph{Ctheory}. Victoria: University of Victoria,
2006. Disponível em: \emph{http://www.ctheory.net/articles.aspx?id=516}.

\versal{SERRES}, Michel; \versal{OBRIST}, Hans U. ``Quase-objetos''. In: \emph{Revista
Ecopolítica}. São Paulo: \versal{PUCSP}, v. 8, 2013. Disponível em:
\emph{http://www.pucsp.br/ecopolitica/galeria/galeria\_ed8-nomeslateral.html}.

\versal{SHIRKY}, Clay. \emph{A cultura da participação: criatividade e
generosidade no mundo conectado}. Tradução de Celina Portocarrero. Rio
de Janeiro: Zahar, 2011.

\versal{SCHULTZ}, Theodore W. \emph{O Capital humano. Investimentos em educação e
pesquisa}. Tradução de Marco Aurélio de Moura Matos. Rio de Janeiro,
Zahar Editores, 1973.

\versal{SIEMINSNKI}, Inga B. \emph{The Theresienstadt}. \emph{A study guide.}
Washington: American University, 2009.

\versal{SINGER}, P. W. \emph{Corporate Warriors: the rise of the privatized
military industry}. New York: Cornell University Press, 2008.

\versal{SIQUEIRA}, Leandro. ``Uma genealogia das compulsões''. In: \emph{verve}.
São Paulo: Nu-Sol, n. 18, 2010, pp. 149-166.

\_\_\_\_\_\_\_\_\_\_\_. \emph{Ecopolítica: derivas do espaço sideral}.
Tese de doutorado. São Paulo: \versal{PUCSP}, 2015.

\versal{SKÓRZINSKI}, Jan. ``Revolução do Solidariedade e o fim da União
Soviética''. In: \emph{Revista de Relações Internacionais.} Lisboa:
Instituto Português de Relações Internacionais, n. 33, 2012.

\versal{SOARES}, Guido. ``O meio ambiente global: de Estocolmo à Eco-92 e a
América Latina''. In: \versal{SOARES}, G. e \versal{REZENDE}, Paulo (Orgs.).
\emph{Ecologia, Sociedade, Estado}. São Paulo: Educ, 1995.

\versal{SOUZA}, Jessé. \emph{A} \emph{ralé brasileira.} Belo Horizonte: \versal{UFMG},
2009.

\_\_\_\_\_\_\_\_\_\_. \emph{Os batalhadores brasileiros}. Belo
Horizonte: \versal{UFMG}, 2010.

\versal{SOLOMON}, Barbara Bryant. \emph{Black empowerment: Social work in
oppressed communities}. New York: Columbia University Press, 1976.

\versal{SPITZER}, Leo. ``Milieu and Ambiance: an Essay in Historical Semantics''.
In: \emph{Philosophical and Phenomenological Research}. Providence:
Brown University, v. 3, n. 1, 1942, pp. 1-42. Disponível em:
\emph{http://www.jstor.org/stable/2108127\%20Acessado\%20em\%2004/09/2010}.

\versal{SOLJENITZEN}, Alexander. \emph{El archipiélago Gulag (1918-1956)}, 2007.
Disponível em:
\emph{http://www.pucsp.br/ecopolitica/documentos/docs\_especiais/docs/el\_archipelago.pdf}

\versal{SOUZA}, Jessé. \emph{Os batalhadores brasileiros}. Belo Horizonte: \versal{UFMG},
2012.

\versal{STACHELHAUS}, Heiner. \emph{Joseph Beuys, une biographie.} Tradução de
Xavier Carrèrre, Clémence Guibout e Jean-Yves Masson. Paris: Abbeville,
1994.

\versal{STIRNER}, Max. Algumas considerações sobre o Estado fundado no amor''.
In: \emph{verve}. São Paulo: Nu-Sol, n. 1, 2002, pp. 13-21.

\_\_\_\_\_\_\_\_\_\_. \emph{O único e a sua propriedade}. Tradução de
João Barrento. Lisboa: Antígona, 2004.

\versal{ST}. \versal{JOHN}, Ronald Bruce. \emph{Libya: from colony to revolution}. Oxford:
OneWorld, 2012.

\versal{STERN}-\versal{GILLET}, Suzanne. \emph{Aristotle's philosophy of friendship}. New
York: State University of New York Press, 1995.

\versal{SZYMBORSKA}, Wislawa. \emph{Um amor feliz.} Tradução de Regina Prybycien.
São Paulo: Companhia das Letras, 2016.

\versal{TAMANES}, Ramón. \emph{Crítica dos limites do crescimento: ecologia e
desenvolvimento.} Tradução de José Maria Brandão de Brito. Lisboa: Dom
Quixote, 1983.

\versal{THEODORO}, Suzi Hoff (Org.). \emph{Os 30 anos da Política Nacional do
meio Ambiente: conquistas e perspectivas}. Rio de Janeiro: Garamond,
2011.

\versal{THOMPSON}, A. K. \emph{Black Bloc, White riot}. \emph{Anti-Globalization
and the Genealogy of Dissent}. New York: \versal{AK} Press, 2010.

\versal{TILLY}, Charles. \emph{Capital, Coerção e Estados Europeus}. Tradução de
Geraldo Gerson de Souza. São Paulo: Edusp, 1996.

\versal{TODOROV}, Tzvetan. \emph{Os inimigos íntimos da democracia}. Tradução de
Joana Angélica d´Avila Melo. São Paulo: Companhia das Letras, 2012.

\versal{TORRES}, Alberto. \emph{A organização nacional}. Rio de Janeiro: Imprensa
Nacional, 1914.

\_\_\_\_\_\_\_\_\_\_. \emph{O problema nacional brasileiro}.
eBooksBrasil.org, 2002. Disponível em
\emph{http://www.ebooksbrasil.org/eLibris/torresb.html}.

\versal{TOYNBEE}, Arnold. \emph{Lectures on The Industrial Revolution in
England}. London/New York: Longmans, 1884.

\versal{TRIGUEIRO}, André (Coord.). \emph{Meio Ambiente no século \versal{XXI}.} Campinas:
Armazém do Ipê, 2008.

\versal{UEHARA}, Luíza. \emph{Política e modulações. Há vida libertária na
internet?}. Dissertação de Mestrado. São Paulo: \versal{PUC}-\versal{SP}, 2013.

\versal{UEKOETTER}, F. \emph{The Green and the Brown: a history of conservation
in Nazi Germany}. New York: Cambridge University Press, 2006.

\versal{ULLOA}, Astrid. ``A ecogovernamentalidade e suas contradições'' In:
\emph{Revista Ecopolítica}. São Paulo: \versal{PUCSP}, v. 1, 2011, pp. 26-42.
Disponível em:
\emph{http://revistas.pucsp.br/index.php/ecopolitica/article/view/7655/5603}.

\versal{USHER}, Shaun (Org.). \emph{Cartas extraordinárias}. Tradução de
Hildegard Fiest. São Paulo: Companhia das Letras, 2014.

\versal{VAINER}, Carlos et alii(2013). \emph{Cidades rebeldes. Passe livre e as
manifestações que tomaram as ruas do Brasil}. São Paulo: Carta
Maior/Boitempo.

\versal{VASCONCELOS}, Eduardo Mourão. \emph{O poder que brota da dor e da
opressão: empowerment, sua estória, teorias e estratégias}. São Paulo:
Paulus, 2003.

\versal{VEIGA}, José Eli da. \emph{A desgovernança global}. São Paulo: 34 Letras,
2013.

\versal{VILA}-\versal{MATAS}, Enrique. \emph{Dublinesca}. Tradução de Jose Rubens
Siqueira. São Paulo: Cosac Naify, 2011.

\versal{WAISELFISZ}, Julio J. \emph{Mapa da Violência 2013. Mortes matadas por
armas de fogo}. San Pedro/Costa Rica: \versal{CEBELA}/\versal{FLACSO}, 2013. Disponível
em:
\emph{http://www.mapadaviolencia.org.br/pdf2013/MapaViolencia2013\_armas.pdf}.

\versal{WALKER}, R.B.J. \emph{Inside/Outside: as Relações Internacionais como
Teoria Política}. Tradução de Luis Carlos Moreira da Silva. Rio de
Janeiro: Apicurí/\versal{PUC}-Rio, 2013.

\versal{WARD}, Barbara e \versal{DUBOS}, René. \emph{Uma terra somente: a preservação de
um pequeno planeta.} Tradução de Antônio Lamberti. São Paulo: Edusp,
1972.

\versal{WACHSMANN}, Nikolaus. \emph{kl. A História dos compôs de concentração
nazis}. Tradução de Miguel Matta. Publicações Dom Quixote: Alfragide,
2015.

\versal{WEIS}, Thomas G. \emph{Humanitarian Intervention}. Cambridge: Polity,
2007.

\versal{WELZER}, Harald. \emph{Guerras climáticas: por que mataremos e seremos
mortos no século 21}. Tradução de William Lagos. São Paulo: Geração
Editorial, 2010.

\versal{WERNER}, Emy E. e \versal{SMITH}, Ruth S. \emph{Vulnerable but invincible: a
longitudinal study of resilient children and youth}. New York: McGraw
Hill, 1982.

\versal{WOODCOCK}, George. \emph{Os grandes escritos anarquistas}. Tradução de
Júlia Tettamazi e Betina Becker. Porto Alegre: \versal{LP\&M}, 1981.

\_\_\_\_\_\_\_\_\_\_. \emph{História das ideias e movimentos anarquistas
2v}. Tradução de Júlia Tettamanzy. Porto Alegre: \versal{LP\&M}, 2002.

\versal{WORLD} \versal{COMISSION} \versal{ON} \versal{ENVIRONMENT} \versal{AND} \versal{DEVELOPMENT} --\versal{WCED}. \emph{Our common
future.} Oxford: Oxford University Press, 1987.

\versal{WORSTER}, Donald. \emph{Nature's Economy; a history of ecological ideas}.
New York: Cambridge University Press, 1994.

\versal{YOUNG}, George M. \emph{The Russian Cosmists.} New York: Oxford
University Press, 2012.

\versal{YOUNG}, Thomas. \emph{A course of lectures on natural philosophy and the
mechanical arts}. London: Printed for Taylor and Walton, v. I, 1845.

\versal{ZIMMERMAN}, M. A. ``Psychological empowerment: Issues and
illustrations''. In: \emph{American Journal of Community Psychology}.
Florida: Society for Community Research and Action -- Division of
Community Psychology,~ 1995, n. 23, pp. 581-599.

\_\_\_\_\_\_\_\_\_\_. ``Taking aim on empowerment research: On the
distinction between individual and psychological conceptions''. In:
\emph{American Journal of Community Psychology}. Florida: Society for
Community Research and Action -- Division of Community Psychology,~ n.
18, 1990, pp. 169-177.

\chapter{Agradecimentos}

Esta pesquisa-livro aconteceu pela presença da Fundação São Paulo/\versal{PUCSP},
da \versal{FAPESP}, de muitos colegas que nos visitaram pessoalmente e em
colóquio para conversar sobre o tema, de amigos que compareceram às
nossas exposições de resultados parciais e finais, de muitos jovens que
realizaram suas iniciações científicas conosco e que repercutiram em
mestrados, de mestrandos que se doutoraram e de projetos de pós-doc.

Entramos e saímos fortalecidos em vários encontros no Brasil e no
exterior; produzimos um site desde o início do projeto que hoje se
transformou em Observatório Ecopolítica
(\emph{http://www.pucsp.br/ecopolitica/}); realizamos aulas-teatro sobre
o tema da pesquisa (Revista \emph{\textbf{verve}} números 22 a 28,
\emph{http://www.nu-sol.org/verve/verve1.php}); fundamos em 2011 a
Revista Ecopolítica, com periodicidade quadrimestral
(\emph{http://revistas.pucsp.br/ecopolitica}) e lançamos uma série de
quatro documentários, exibidos no Canal Universitário-\versal{TV} \versal{PUC} e
distribuídos gratuitamente a universidades (disponíveis em:
\emph{https://www.youtube.com/channel/UCmrOPJTsO\_Y5wYnHUSAQgDg}).

Tudo foi possível porque existe o Nu-Sol (Núcleo de Sociabilidade
Libertária \emph{http://www.nu-sol.org}), com sua
coragem libertária e gosto pela pesquisa e pela vida. E quase tudo pode
ser possível porque existem leitores, gente que pesquisa, pessoas que
gostam da gente e que não gostam, mas principalmente toda a gente que
quer, produz e inventa liberdades.
