\chapter*{}

\vspace*{\fill}

\begin{flushright}
\begin{adjustwidth}{2.0cm}{}
\raggedleft\footnotesize\emph{Ousado e desafiador, esse livro enfrenta a difícil tarefa de
diagnosticar o presente, apontando para as rupturas profundas nas formas
pelas quais se manifesta o poder, no governo da vida e do planeta.
\emph{Ecopolítica} remete à ilimitada expansão e à progressiva
transformação do controle biopolítico das populações para o governo de
todo o sistema planetário, em nome do pluralismo democrático, da
inclusão social, da boa governança e da salvação das espécies. Como aqui
se mostra enfaticamente, a racionalidade neoliberal reconfigura todas as
dimensões da vida social em bases econômicas e o governo das condutas se
estende à própria produção da subjetividade do indivíduo tido como
``empresário de si mesmo''. Esse instigante livro também nos apresenta
uma cartografia das liberdades, das lutas e resistências que se abrem
com a ação direta, a antipolítica e as infinitas práticas da liberdade,
de que ele próprio é testemunho. Assim, é também da transformação
libertária que tratam essas pesquisas, das linhas de fuga e das
resistências possíveis na sociedade de controle, que coloca no centro
das atenções os temas da segurança, dos direitos e das penalizações,
ampliando consideravelmente seu domínio. Se as formas de controle e
sujeição na era planetária se tornam muito mais complexas, abrangentes,
sofisticadas e violentas, é preciso perceber por que emaranhados
irrompem fluxos, fissuras e devires em direção à construção de outros
espaços, heterotopias de si, do mundo e da vida, no aqui e agora.}

\medskip

\emph{Margareth Rago}
\end{adjustwidth}
\end{flushright}
\thispagestyle{empty}

\chapter*{}

\vspace{\fill}
\emph{Todos os endereços eletrônicos acessados e indicados em notas de rodapé
foram atualizados em junho de 2016.}

\medskip

\noindent\emph{Para o acompanhamento e a consulta destes documentos recomenda"-se que o
leitor navegue em \textless{}\emph{pucsp.br/ecopolitica/}\textgreater{}.}
\thispagestyle{empty}
