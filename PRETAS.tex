\textbf{Edson Passetti} é professor livre"-docente no Departamento de Política, Programa de Estudos Pós"-Graduados em Ciências Sociais e coordenador do Nu"-Sol (Núcleo de Sociabilidade Libertária, \textless{}\emph{www.nu-sol.org}\textgreater{}) da \versal{PUC"-SP}. Foi o coordenador do Projeto Temático Fapesp \emph{Ecopolítica. Governamentalidade planetária, novas institucionalizações e resistências}. Contato: passetti@matrix.com.br.

\textbf{Acácio Augusto} é professor no Departamento de Relações Internacionais da Universidade Federal de São Paulo (\versal{EPPEN"-U}nifesp) e pesquisador no Nu"-Sol. Contato: acacioaugusto1980@gmail.com.

\textbf{Beatriz Scigliano Carneiro} é consultora no  Núcleo de Estudos e Pesquisas de Politicas Socioambientais da Amazônia -- \versal{NEPPS}, da Universidade Estadual do Amazonas (\versal{UEA}) e pesquisadora no Nu"-Sol. Contato: bmscarneiro@uol.com.br.

\textbf{Salete Oliveira} é professora aposentada no Departamento de Política e Programa de Estudos Pós"-Graduados em Ciências Sociais e pesquisadora no Nu"-Sol. Contato: saletemagadadeoliveira@gmail.com.

\textbf{Thiago Rodrigues} é professor Associado no Instituto de Estudos Estratégicos (\versal{INEST}), da Universidade Federal Fluminense (\versal{UFF}), coordenador do Programa de Pós"-Graduação em Estudos Estratégicos do mesmo Instituto e pesquisador do Nu"-Sol. Contato: trodrigues@id.uff.br.

\textbf{Ecopolítica}: A emergência da ecopolítica está sintonizada com o fim da \versal{II} Guerra Mundial e com as institucionalizações internacionais subsequentes. O alvo principal dos governos é o planeta, visando recuperar sua vida degradada e a conservá"-lo de modo sustentável, em benefício das futuras gerações. A ecopolítica pressiona os regimes políticos para a democracia em sintonia com a racionalidade neoliberal. A ecopolítica pretende dar conta não só do governo da espécie humana, mas dos viventes na Terra e projetados para o espaço sideral. Novas resistências são produzidas. 

%Talvez repensar esse texto das pretas, está fugindo um pouco dos nossos padrões de texto das pretas, não?


%é um livro que não conseguiria ser escrito por uma autoria. É resultante do trabalho constante de uma equipe com atribuições horizontalizadas segundo os fluxos e que, no decorrer da pesquisa, foi consolidando análises a respeito de caudalosos afluentes.

%é um estudo que tenta mapear a passagem da biopolítica --- o controle da vida surgido em finais do séc \versal{XVIII}, como analisado por Foucault --- para a ecopolítica, a nova forma de governar que emerge pós-\versal{II} Guerra Mundial e se estende a todas as esferas da vida. É um poder que tenta codificar e controlar todos os fluxos,  do nível mais íntimo e subjetivo, como os \emph{cidadãos"-polícias} que assumem a posição de vigiar a si e aos outros, ao mais amplo, como as iniciativas de mapear e controlar a Terra e os espaços siderais. Nessa chave, os estudiosos do Nu"-Sol percorrem e analisam acontecimentos históricos e contemporâneos, atravessam os fluxos de poder para conclamar à urgente criação de formas de resistência mais libertárias e esquivas às emergentes e globalizantes linhas de controle.





